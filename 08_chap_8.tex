%%%% Débute page 166 dans le fichier PDF%%%%

\documentclass[a4paper,11pt]{book}
\usepackage[T1]{fontenc}
\usepackage[utf8]{inputenc}
\usepackage{lmodern}
\usepackage[french]{babel}
\usepackage{todonotes}

\newcommand{\asterism}{\bigskip\par\noindent\parbox{\linewidth}{\centering\large{*}\\[-4pt]{*}\hskip 0.75em{*}}\bigskip\par}%

\begin{document}
\title{Histoire de Talence}
\author{Maurice Ferrus}
\frontmatter
\maketitle

\mainmatter{}

\chapter{La vie intellectuelle}
\section{Le petit lycée de Talence}

Il fut créé en 1859, dans un domaine\footnote{Ce domaine est indiqué « Collége Royal » sur le plan cadastral de Talence (1846).} qui appartenait au lycée de Bordeaux depuis la Restauration. 

On l'appela pendant longtemps le « Petit Collège du Lycée impérial », lequel devint, après le 4 septembre 1870, le Lycée national.

En 1876, le Petit Collège — il avait encore ce nom — ne comptait pas moins de 300 élèves. L'enseignement qu'on y donnait ne dépassait pas la quatrième. 

Les bâtiments du petit lycée étaient, assure un chroniqueur, « d'une blancheur souriante ». En 1903, l'établissement fut fermé.

À l'occasion du centenaire du lycée de Bordeaux, le dimanche 25 juin 1905, un banquet fut organisé sous l'un des préaux du petit lycée de Talence. On nota parmi les convives, au nombre de cent trente : MM. Decrais, homme politique et diplomate; Thamin, recteur de l'Université; de La Ville de Mirmont, adjoint, représentant le maire de Bordeaux ; Charles Chaumet, devenu sénateur et ministre ; Paul Berthelot, aujourd'hui rédacteur en chef de la \textit{Petite Gironde}.

Dans un toast, M. Decrais déclara :

« Il est des heures où la trêve se fait, où les esprits et les coeurs sentent le besoin de se chercher, de se rejoindre. On peut dire qu'une de ces heures rêvées et fortunées est celle qui nous retrouve, réunis, dans ce lieu charmant. »

Il y eut ensuite une fête dans le parc. Des artistes se firent applaudir, entre autres le « caulègue » Despaux, dont on goûtait la truculente fantaisie.

À cette époque, M. Huc, adjoint au maire de Talence, avait exprimé le vœu que le petit lycée rouvrit ses portes, que l'on rendît à cet établissement « la vie universitaire ».

Ce vœu fut exaucé en octobre 1906.

Actuellement, les classes du lycée vont de la quatrième (A et B) à la sixième (A et B). Elles comprennent, en outre, des classes élémentaires (7\textsuperscript{e} et 8\textsuperscript{e}), des classes préparatoire et enfantine.

En 1925, la distribution des prix aux élèves des lycées de Longchamps, de Talence et des classes élémentaires du Grand lycée eut lieu le 11 juillet, à l'Alhambra, sous la présidence de M. Rochoux, président de l'Association des anciens élèves du lycée, conseiller à la cour d'appel de Bordeaux. Le discours d'usage fut prononcé par M. Pestourie, professeur de cinquième. Faisant allusion à certains élèves qui, sous prétexte de gagner plus tard beaucoup d'argent, ne voient pas l'utilité matérielle et immédiate de leurs études, M. Pestourie leur dit, en substance :

«\dots~Les chênes antiques du lycée de Talence, qui ont recueilli indiscrètement vos confidences, me les ont rapportées. \textit{Horresco referens} ! « À quoi sert le latin dans la vie ? \dots C'est une langue morte !\dots Je vais passer en B, il y a moins de travail !\dots A quoi bon travailler ? Ce ne sont pas toujours ceux-qui se trouvent les premiers au lycée qui occupent plus tard les meilleures places. » À cette dernière constatation, je répondrai tout dé suite que c'est très heureux que la vie ne règle pas les rangs d'après les places du lycée, car tout espoir est ainsi permis même au
plus célèbre de nos paresseux. Le plus souvent l'âpreté de la lutte pour la vie se chargera de stimuler plus tard son énergie défaillante.

» Soyez persuadés que le lycée, où l'on concourt pour la première place, est semblable à la vie sociale, où la concurrence est plus ardente encore et surtout moins loyale : concurrence commerciale ou professionnelle, de grades ou de titres, de vanité ou de snobisme... »

\section{Les écoles primaires supérieures}

Les écoles primaires supérieures de Talence sont particulièrement florissantes. Celle des garçons vient en tête des trois autres écoles analogues ouvertes à Bordeaux, à
Cadillac et à Sainte-Foy-la-Grande.

L'école primaire supérieure de garçons comptait, au
31 décembre 1921\footnote{\textit{Rapport sur la situation de l'enseignement primaire dans la Gironde} (année 1921)}, 294 élèves. Il y avait 69 élèves de Talence-Bègles. Le reste, soit 225, était fourni par Bordeaux, les communes suburbaines et quelques localités assez éloignées comme Pessac, Léognan, Gradignan, Cenon, Tresses, Saint-André-de-Cubzac, Ambarès, Camarsac, Carignan, Portets, Arbanats, etc.

Résultats des examens : Bourses d'enseignement primaire supérieur, 4 ; brevet d'enseignement primaire supérieur, section générale, 4 ; section commerciale, 5 ; section industrielle, 16; brevet élémentaire, 15 ; École normale de la Gironde, 3 ; École d'hydrographie de Bordeaux, 1 ; École supérieure de commerce, 3.

Au 31 décembre 1924\footnote{\textit{Rapport sur la situation de l'enseignement primaire dans la Gironde} (année 1924).}, l'école primaire supérieure comptait 365 élèves répartis en onze divisions distinctes, soit 71 élèves de plus qu'au 31 décembre 1921.

Les examens ont donné les résultats suivants : Brevet d'enseignement primaire supérieur, 54 ; brevet élémentaire, 11 ; École normale, 5 ; arts et métiers, 6 ; Instituts (agr., chimie, électr., techn.), 1 ; École de navigation, 1; travaux publics et ponts et chaussées, 1 ; postes, 4 ; C. A. professionnelle, 26.

Un concours a lieu chaque année pour limiter le nombre des entrées en 1re année. Il permet de choisir une proportion assez élevée de bons élèves et d'éliminer les non-valeurs. Les élèves non originaires de Talence et de Bègles sont admis à l'école moyennant le paiement d'une redevance de 200 francs\footnote{\textit{Ibid}. (année 1924).}.

Il y avait 202 élèves à l'école primaire supérieure de jeunes filles au 31 décembre 1921\footnote{\textit{Ibid}. (année 1921).}.

28 ont obtenu le brevet d'enseignement primaire supérieur, 10 le brevet élémentaire, 6 le brevet supérieur, 2 ont été admises au concours d'entrée à l'Ecole normale, 2 au concours des bourses, 22 au concours d'enseignement ménager (Chambre de métiers).

4 élèves avaient été reçues à leurs examens avec la mention « très bien ».

L'effectif (jeunes filles) comprenait, au 31 décembre 1924\footnote{\textit{Ibid}. (année 1924).}, 201 élèves réparties en huit divisions, soit une élève
de moins qu'au 31 décembre 1921.

Résultats des examens : Brevet d'enseignement primaire supérieur, 16 ; brevet élémentaire, 10 ; École normale, 5 ; brevet supérieur, 4; certificat d'aptitude professionnelle, 34; diplôme d'enseignement ménager, 9; puériculture, 12; infirmière, 2.

Les cours d'enseignement général, d'enseignement commercial et industriel sont donnés aux écoles primaires supérieures de Talence conformément aux dispositions du décret du 18 août 1920.

Sur le crédit de 1.000 francs alloué au département de la Gironde pour subventions aux caisses des écoles en 1924, Talence a reçu 100 francs; Caudéran arrive ensuite avec une allocation de 90 francs. Trente et une autres communes du département ont touché des sommes allant de 10 à 60 francs.

Enfin, en 1924, les deux écoles primaires supérieures de Talence (garçons et filles) réunissaient 566 élèves, alors que les mêmes écoles de Bordeaux (rue du Commandant-
Arnould et rue de Cheverus) n'avaient groupé que 496 élèves. Il est vrai de dire que les locaux des écoles primaires supérieures de Bordeaux sont insuffisants par rapport à
la population de la ville. 

\section{Le jardin botanique}

Des libéralités de Camille Godard ont permis de fonder, en 1892, le Jardin botanique ou plutôt les Jardin et Institut botaniques », car telle est l'inscription placée, en lettres dorées, au-dessus du portail d'entrée.

Ce jardin s'ouvre au numéro 356 du cours Gambetta ; il occupe un vaste terrain entre l'avenue d'Espeleta et le ruisseau des Malerettes.

De chaque côté du portail est encastrée dans le mur une «plaque en marbre rouge. On lit sur la plaque de gauche : « Université de Bordeaux » ; sur la plaque de droite : « Faculté de médecine et de pharmacie ».

Au-dessous de chaque plaque de marbre, il y a un casier en bois, avec treillis métallique, où l'on placarde des avis. Une petite affiche — dont le texte suit — garnit en partie l'un des casiers :

\begin{small}
\begin{center}
\textsc{UNIVERSITÉ DE BORDEAUX\\
FACULTÉ DE MÉDECINE ET DE PHARMACIE}
\end{center}

L'entrée du jardin botanique (annexe du laboratoire de botanique et de matière médicale) est réservée à MM. les Étudiants en cours d'études.

Le public n'y est pas admis.
\end{small}


On se demande pourquoi les promeneurs n'ont pas le droit de visiter le Jardin botanique ?\dots

Cet établissement est pourvu de deux serres, d'une orangerie, d'un laboratoire de recherches et de champs de culture des plantes appartenant aux diverses familles végétales, et, particulièrement, des plantes médicinales.

C'est le professeur Guillaud\footnote{Jean-Alexandre Guillaud était né à Aumagne (Charente-Inférieure), le 12 février 1849.}, décédé à Saintes le 18 novembre 1923, après avoir occupé, pendant plus de quarante ans, la chaire d'histoire naturelle à la Faculté de médecine et de pharmacie, qui a « créé et magistralement ordonné le beau Jardin botanique de l'Université, à Talence »\footnote{\textit{Rapport du Conseil de l'Université, année scolaire 1923-1924}.}. Cette oeuvre lui était chère et il voulut la continuer par delà la tombe en instituant la Faculté bénéficiaire d'un legs important affecté spécialement à son entretien.

Le Jardin a maintenant pour directeur M. Beille, professeur de botanique et matière médicale à la Faculté de Bordeaux.

M. Georges Petit assure les fonctions de jardinier-chef du domaine.

Le terrain du Jardin botanique était, au dix-septième siècle, recouvert de vignes, appartenant au sieur Béchon\footnote{\textit{Notre-Dame de Talence dans la chapelle des monges au XVIII\textsuperscript{e} siècle}.}. Il se trouva ensuite englobé dans le domaine Peixotto. Le fameux banquier transforma tout son bien en un lieu des plus séduisants.

Le Jardin botanique garde quelques vestiges de sa splendeur d'antan. On y voit, outre des parterres rectilignes, une pièce d'eau magnifique, au centre de laquelle est une île de forme circulaire à laquelle on accède par un pont rustique. Au fond, exactement dans l'axe du bassin, se dresse une élégante colonne de pierre sur socle, haute de 5 mètres, et dont le chapiteau ionique supporte une boule. Les arbres forment charmilles en bordure du cours Gambetta.

L'étang est alimenté par le ruisseau de Talence, qui suit une direction parallèle. Par une canalisation souterraine, l'étang rejoint le ruisseau à l'endroit où il y a la première des chutes d'eau qui faisaient tourner jadis le moulin de la Lande.

L'île disparaît sous les feuilles énormes de plantes appelées « Ferdinanda eminens ». Ces feuilles ne constituent pas seulement un agrément, elles sont utiles : leur moelle remplace, en effet, « la moelle du sureau pour la micrographie du laboratoire de pharmacie et de matière médicale ». C'est là une innovation due à M. le professeur Beille. Depuis la guerre, la moelle de sureau coûte cher. M. Beille l'a remplacée par le suc d'une autre plante.

M. Petit exerce la profession de jardinier-chef au Jardin botanique depuis plus de trente-huit ans. Autrefois, en bêchant la terre, il a mis à jour de nombreuses pièces de monnaie, notamment des pièces de 1 et 2 centimes. Il ne les a pas conservées et ne se souvient pas très bien de leur effigie.

Comment ces pièces de monnaie se trouvaient-elles éparpillées dans le sol ? Voilà un point que nous avons cherché à élucider, sans y parvenir du reste.

Notons que le legs de Camille Godard, pour la fondation du Jardin botanique, était de 100.000 francs.

\section{L'observatoire de Talence}

Situé rue des Visitandines, 24, il a été fondé, en 1900, par un savant trop modeste, M. Henri Mémery, de la Société astronomique de Bordeaux.

Il s'agit d'une installation d'amateur qui a pour but, en premier lieu, de permettre à tous ceux qui désirent acquérir les éléments de la science astronomique d'effectuer les observations susceptibles de retenir leur attention (soleil, lune, planètes, etc.).

D'autre part, M. Mémery a commencé à Talence, il y a vingt-cinq ans, une série d'observations journalières du soleil en vue de la recherche des relations qui paraissent exister entre les phénomènes solaires (taches, facules, éruptions diverses) et les variations atmosphériques terrestres. Les résultats de ces travaux sont communiqués aux établissements et sociétés scientifiques, ainsi qu'aux
personnes s'intéressant, pour des raisons diverses, à la question si importante de la prévision du temps. 

La méthode en vigueur pour l'étude des phénomènes météorologiques est celle dont les bases ont été établies par Leverrier en 1855. Elle a subi des modifications portant sur l'emploi de la T. S. F., au lieu du télégraphe ordinaire pour la transmission des dépêches, et sur un nombre plus élevé des stations météorologiques de l'Europequi envoient, chaque jour, au Bureau central de Paris, les renseignements sur l'état de l'atmosphère.

Toutefois, la méthode en usage depuis soixante-dix ans apparaît insuffisante, car elle ne permet la prévision locale du temps que vingt-quatre heures à l'avance, et encore pas avec une certitude absolue. Le progrès réalisé dans ce domaine de la science est peu sensible ; de plus, la météorologie avoue son ignorance en ce qui concerne les causes des variations brusques et anormales du temps.

Pour ses études touchant la météorologie générale, M. Mémery utilise les méthodes nouvelles. Il a complété sa série de vingt-cinq années d'observations journalières du soleil par une série de vingt-cinq années d'observations analogues faites en Angleterre à partir de 1874. Il possède ainsi \textit{cinquante années d'observations quotidiennes du soleil}.

La comparaison entre les variations notées sur la surface du soleil et celles des éléments atmosphériques sur l'ouest de l'Europe conduit à des résultats excessivement
curieux.

L'observatoire de Talence apporte donc une précieuse contribution à l'étude de la prévision du temps, et c'est tout à l'honneur de M. Henri Mémery.

Lors du dernier passage de Mars à proximité\footnote{En août 1924.} de la Terre, nombre d'amateurs ont défilé à l'observatoire de Talence. On leur a fait remarquer « l'extrême difficulté qu'offre l'observation de la fameuse planète, l'impossibilité matérielle de limiter sur un dessin les détails aperçus à la lunette, et l'embarras où se trouve l'observateur pour rendre fidèlement la forme des taches sombres visibles sur ce monde voisin »\footnote{\textit{L'Information astronomique} n° 1, septembre 1924.}.

\section{Les établissements hospitaliers}

Durant la guerre, afin de former rapidement de bons aviateurs, on était arrivé progressivement à imposer aux candidats pilotes un programme d'examen physiologique et médical, qui était à peu près au point au moment de l'armistice. Le centre militaire de Bordeaux, où l'on fait passer ledit examen, a son siège à l'hôpital temporaire de Talence ; il est placé sous la direction du 18\textsuperscript{e} corps d'armée. On y visite trois ordres de candidats : les candidats civils (dix-huit ans) aux bourses de pilotage offertes par le ministère de la Guerre pour s'assurer, à l'avance, un recrutement aérien militaire ; les candidats provenant de la préparation militaire (futurs officiers de réserve); les candidats militaires déjà incorporés. En 1921 il y a eu, au total, 113 candidats; ce chiffre a été dépassé en 1922.

On soigne particulièrement à l'hôpital de Talence les pilotes ou les observateurs victimes d'accidents survenus au centre d'instruction de tir et de bombardement aérien de Cazaux. Le blessé est transporté par avion sanitaire au camp de Beau-Désert où va le prendre une ambulance automobile qui le conduit à l'hôpital temporaire.

On a parlé de bâtir un hôpital militaire à Talence.

Dans une de ses séances, en août 1924, le Conseil général de la Gironde a adopté, à l'unanimité, deux voeux demandant que le projet de construction d'un hôpital militaire à Talence soit définitivement abandonné ; le second, que l'hôpital temporaire n° 4 de Talence soit supprimé dans le plus bref délai. Ces voeux étaient présentés par
MM. Borderie, Teyssier, Boyer et Costedoat.

Les parlementaires, membres du Conseil général, s'engagèrent à intervenir auprès du ministre de la Guerre pour empêcher l'édification de l'hôpital projeté, et dont la réalisation, suivant les explications que fournit M. Costedoat, coûterait une trentaine de millions.

La question est toujours pendante.

L'hôpital temporaire est formé d'une suite de baraquements uniformes. Il s'ouvre cours Gambetta. Le vaste domaine dans lequel il est installé a aussi une entrée sur le chemin Frédéric-Sévène, entrée marquée par une grille qu'encadrent deux piliers circulaires sur lesquels on lisait, il y a peu de temps encore, ce nom en lettres noires :
« Crespy ».

Une magnifique allée d'arbres part de la grille qui, après être restée longtemps ouverte nuit et jour, est maintenant constamment fermée et cadenassée. Aussi ne s'explique-t-on plus l'inscription ci-après peinte sur une planchette fixée sur le premier arbre de l'allée, à droite : « Défendu à toute personne étrangère à l'hôpital de pénétrer dans le parc ».

\section{L'école Florence-Nightingale}

Deux écoles d'infirmières de Bordeaux sont reconnues par les pouvoirs publics.

L'une d'elles fonctionne à Talence, dans le superbe domaine de Bagatelle, limité par la route de Toulouse, la rue Robespierre et le chemin Frédéric-Sévène : c'est l'École Florence-Nightingale.

Cette école fondée en 1919, grâce aux libéralités de Rockfeller, est placée « sous l'invocation des infirmières américaines mortes pour le droit et pour l'humanité au cours de la guerre de 1914-1918 » ; elle est dirigée par M\textsuperscript{lle} le docteur Hamilton et M\textsuperscript{lles} Mignot, sous-directrices.

Les infirmières de la maison de santé protestante rue Cassignol sont formées à Bagatelle, où elles trouvent, à l'occasion, une maison de repos.

Le dimanche 16 mars 1924, un homme de science et de dévouement, un des maîtres de la chirurgie française, le docteur Jean-Louis Faure\footnote{Jean-Louis Faure a été élu membre titulaire de l'Académie de médecine, en décembre 1924. Il est né, en 1863, à Sainte-Foy-la-Grande (Gironde). Cette petite commune avait eu déjà un de ses enfants membre de l'Académie de médecine : le célèbre chirurgien Paul Broca, fondateur de l'École d'anthropologie.}, professeur à la Faculté de médecine de Paris, a visité les oeuvres sociales de Bordeaux et de la banlieue, entre autres l'École Florence-Nightingale et le dispensaire aménagé dans le même établissement.

Jean-Louis Faure, dans une belle allocution, a souligné que l'école des gardes-malades de Bagatelle pouvait être considérée comme un modèle, et il a exalté le rôle à la fois social et moral des infirmières qui donnent aujourd'hui — comme elles l'ont fait durant le conflit mondial — « d'admirables exemples de courage, d'abnégation et de savoir »\footnote{\textit{La Petite Gironde}, n° du 17 mars 1924.}.

Ce même dimanche 16 mars 1924, on a posé la première pierre d'un hôpital qui doit compléter les magnifiques installations prévues sur le terrain de Bagatelle. La pierre recouvre un coffret métallique contenant le procès-verbal relatif à cette cérémonie. Des pièces de monnaie ont été jointes au procès-verbal qui est revêtu des signatures du professeur Jean-Louis Faure et des hautes personnalités qui accompagnaient l'éminent  chirurgien dans sa visite à Florence-Nightingale.

L'hôpital de Bagatelle comprendra six pavillons. On compte y centraliser tous les services de la maison de santé protestante de la rue Cassignol, à l'exception, toutefois, de la clinique de jour.

\section{La pouponnière de Cholet}

La Fédération des oeuvres girondines de protection de l'enfance, fondée le 11 avril 1918, avait, peu après, émis le voeu qu'il fût créé dans les environs de Bordeaux une
pouponnière avec asile d'allaitement et asile pour femmes enceintes. La réalisation intégrale de ce voeu, nous la trouvons dans la pouponnière de Cholet\footnote{Voir page 43. \textbf{TODO : mettre à jour le renvoi}}. Cet établissement, ouvert en 1920, constitue, en effet, une véritable école de puériculture. Il dépend des hospices de Bordeaux. En deux années de fonctionnement, il a donné asile « à 48 femmes enceintes, à 352 mères nourrices et à leurs enfants ; il a reçu, en outre, 245 enfants sans mère »\footnote{\textit{Journal de médecine de Bordeaux}, n° du 20 septembre 1923.}.

Grâce aux bonnes conditions hygiéniques et à la surveillance médicale constante, la mortalité y a été presque nulle.

Si, actuellement, par suite d'une organisation encore insuffisante, tous les nourrissons de Bordeaux ne reçoivent pas un lait excellent, on a la satisfaction de pouvoir donner à un grand nombre un bon lait de vache provenant d'une laiterie de la banlieue. Ce lait est distribué non pasteurisé en des récipients cachetés, dans l'heure qui suit la traite. Les nourrissons de Cholet, notamment, sont alimentés par ce moyen.

À l'opposé de l'hospice du cours de l'Argonne, la pouponnière de Cholet. ne reçoit que des nourrissons sains. 

M. le docteur Rocaz, médecin des hôpitaux, chef du service médical de la pouponnière de Cholet, a souligné l'utilité de cette institution et le rôle efficace qu'elle doit jouer dans la lutte contre l'avortement volontaire. Il y a intérêt selon lui, à réunir dans le même local deux des trois sections de l'établissement : l'asile des femmes enceintes et l'asile d'allaitement.

« La future mère qui aura vécu pendant plusieurs mois au milieu de mères et de bébés sentira, écrivait-il, éclore en elle des sentiments d'amour maternel qui ne se seraient peut-être pas développés dans une autre ambiance ; elle ne songera plus à abandonner l'enfant qu'elle attend\dots

» De plus, la femme enceinte, sortant d'un asile d'allaitement, sera pour cet asile le meilleur agent de propagande auprès de ses compagnes de la Maternité ; elle y recrutera des adhérentes et évitera ainsi bien des abandons\footnote{\textit{Le Sud-Ouest économique}, n° du 28 mars 1922.}. »

\section{La Lyre talençaise}

Elle a été fondée en 1902. Présidée par M. Renard, elle a pour directeur M. Raoul Auguste. Les cours de solfège et d'instruments organisés par la Lyre talençaise sont régulièrement assurés par M. Estivaux.

En 1924, la Lyre talençaise a donné une grande fête à l'occasion du vingt-deuxième anniversaire de sa fondation et en l'honneur de Sainte-Cécile. Un don important de M. Faget, conseiller général, permit aux dirigeants de la Société de gratifier tous les lauréats d'une récompense en espèces.

\section{Talence et les Lettres}

Talence a vu naître, le 22 août 1872\footnote{Registre des naissances (commune de Talence).}, l'un des plus brillants universitaires de la fin du XIX\textsuperscript{e} siècle : Paul Couvreur, fils de Joseph-Auguste-Edmond Couvreur, censeur au lycée de Bordeaux.

La déclaration de naissance eut pour témoins : Gabriel Rousseau, surveillant général au lycée de Talence, et Édouard Clara, professeur au même lycée.

Paul Couvreur figure dans la liste des lauréats du Concours général (concours de Paris). Il remporta : en 1888 en rhétorique, le 1\textsuperscript{er} prix (nouveaux) de géographie; en 1889, en philosophie, le 2\textsuperscript{e} prix de physique, et en 1890, en rhétorique, le 1\textsuperscript{er} prix (vétérans) d'histoire.

Helléniste de l'école de Tournier, Couvreur fut l'auteur, avant d'être agrégé, d'une excellente édition du Phédon, préparée à l'école. Il est mort le 25 janvier 1898, à Lille, où il était maître de conférences de philologie classique.

\asterism

L'Académie française a attribué, en 1926, le grand-prix de roman à un écrivain de talent, originaire de Bordeaux : M. François Mauriac. Cet écrivain a situé à Talence l'action d'un de ses romans, et l'un des meilleurs : Le Désert de l'Amour, paru en 1925. On voit son héroïne descendre du tramway, à l'arrêt de l'église de Talence, car elle habite tout près de cet édifice. Un autre des principaux personnages — un potache — quitte le tram à un arrêt un peu plus loin.

Un courant de sympathie s'est établi entre les deux voyageurs, et l'on est empoigné littéralement par les dispositions qu'ils prennent pour se retrouver le soir, à la même
heure, dans la même voiture.

M. François Mauriac promène le lecteur dans les règes de vignes. Pouvait-il oublier que Talence est dans cette admirable région des Graves, où l'on récolte un vin fameux ?

Il parle aussi de la croix de Saint-Genès, quand elle se dressait encore au carrefour, de cette croix « qu'adorèrent écrit-il, en passant, les pèlerins de Saint-Jacques-de-Compostelle, et où ne s'appuyaient plus que les contrôleurs des omnibus »\footnote{\textit{Le Désert de l'Amour}, p. 23.}.

\asterism

Deux jeunes littérateurs de nos amis : MM. Louis Palauqui et Henri Bouffard, ont publié, en 1924, à Bordeaux, sous le titre de \textit{Plume-la-Poule}, un très amusant recueil de contes. L'idée de ce titre original leur vint en apprenant qu'il y avait, à Talence, un quartier dénommé Plume-la-Poule.

\section{La mairie}

La mairie de Talence a été construite en 1881, par l'architecte Marius Faget. Les divers services sont actuellement un peu à l'étroit dans cet édifice qui s'ouvre sur la belle avenue d'Espeleta.

La mairie présente un bâtiment (un étage avec mansarde) flanqué de deux pavillons à fronton triangulaire, élevés d'un étage seulement et formant avant-corps. Au centre de la construction, à hauteur du premier étage, règne une galerie reliant les deux pavillons. Cette galerie ou tribune, protégée par une balustrade, est supportée par quatre colonnes d'ordre ionique.

\subsection{Liste des maires de Talence}

\textbf{TODO : Tableau à mettre en forme}

18. Liste des maires de Talence
NOMS ET PRÉNOMS
DATES D'ENTRÉE EN FONCTION
7 février 1790.
14 novembre 1790.
CAUDERES.
BRUN.
BONNEFOUS (Louis).
MATERRE-FERRAND (Félix).
26 prairial, an VII
27 mars 1806.
DE15 mai 1808.
25 janvier 1813.
13 octobre 1815.
8 novembre 1816
17 septembre 1822.
22 avril 1825.
28 mars 1848.
22 décembre 1851.
20 juin 1855.
9 janvier 1863.
18 février 1866.
29 mai 1868.
9 septembre 1870.
11 mai 1871.
3 septembre 1871.
5 mai 1872.
11 février 1874.
12 juin 1876.
30 septembre 1877.
Huc (Adolphe).
IRIQUIN (Jean).
LASSERRE (Georges).31 décembre 1905.
17 mai 1925.
FLORET.
DAVRIN (Jean).
MAILLÈRE père.
DE MONTESQUIOU (Jean).
BLUMEREL (Joseph).
ROUL.
COLLINEAU aîné.
DEVÈZE (Antoine).
ROUSSEAU (L.).
MÉGRET (Jean-Jacques).
CUGINAUD.
DANGUILHEM (A.).
MÉGRET (Jean-Jacques).
LATÉRRADE (Charles).
MÉGRET.
LATERRADE.
DE VILLE-SUZANNE (Clément).
SÉVÈNE (Frédéric).
VILLE-SuZANNE (Clément).
SÉVÈNE (Frédéric).
JUILLAC (Jacques).
30 décembre 1877.
24 avril 1881.
27 janvier 1901.

\subsection{Municipalité élue en 1919.}

Iriquin, \textit{maire}; Boucanus (Joseph-Charles) et Gardères, \textit{adjoints}; Balès, Barraud, Burlot, Calmel, Cammas, Counord, Courcelles, Dublanc, Dupuy, Gaillard, Imbert, Jabouin, Joyaux, Laborde, Lacoste, Lasserre, D\textsuperscript{r} Mandoul, Nardoux, Renout, Saint-Aubin, Teillaud, Vergne, Vernis, Cornet.

\subsection{Municipalité élue le 17 mai 1925.}

Lasserre (Georges), \textit{maire}; Courcelles (Louis), Mounier (Laurent), Cabot (Pierre), Rochebayard (Albert), \textit{adjoints}; Barboussat, Bréfeil, Bujac, Cammas, Capdevielle, Dublanc, Escath, Gaillard, Granger, Jabouin, Marzin, Merlet, Pinaud, Pourtau, Rispal, Saint-Paul, Serizier, Stancill, Suberbielle, Teillaud, Terriou, Vien.

M. Georges Lasserre est également conseiller général du 4\textsuperscript{e} canton de Bordeaux, dans lequel est englobée la commune de Talence.


CHAPITRE IX
VIOGRAPHIE - ARCHÉOLOGIE
L'ancien chemin de fer de La Teste
La grande ligne ferrée Bordeaux-Irun, avec embran-
chement à Lamothe sur Arcachon, traverse la commune
de Talence.
En 1921 a été mise en exploitation la ligne du chemin
de fer de ceinture, reliant la gare du Midi à celle de Saint-
Louis (Médoc). L'établissement de cette voie ferrée a amené
la création de la gare « La Médoquine-Talence
Il y avait eu déjà une station de « La Médoquine » au
temps du premier chemin de fer de Bordeaux à La Teste.
La gare de départ de ce chemin de fer se trouvait entre
les rues de Pessac, des Treuils et de Ségur, sur l'emplace-
ment occupé par le Conseil de guerre, la caserne Boudet
et le château d'eau.
Talence n'était pas tête de ligne, mais il ne s'en fallait
que de quelques mètres — tout au plus l'intervalle de la
rue de Ségur, laquelle, rappelons-le, séparait autrefois
une partie du territoire de Bordeaux de celui de Talence
1. Plan du projet d'annexion à la ville de Bordeaux du territoire des
communes limitrophes. (Archives municipales. Dossier 30 D. 2.) Voir
page 21.La première pierre de la gare du chemin de fer de La
Teste — le quatrième construit en France
fut
posée so-
—
lennellement, en août 1839, par le duc d'Orléans, fils aîné de
Louis-Philippe, qui se trouvait à Bordeaux avec la du-
chesse, son épouse 2. L'inauguration de la ligne eut lieu
le 6 juillet 1841. On comptait dix-neuf stations ; la pre-
mière était celle de « La Médoquine ».
Un ami élu Parnasse —J.-B. Couvé — composa un petit
poème en effectuant le voyage de Bordeaux à La Teste.
Voici un échantillon de sa lyre :
Nous marchons sur un viaduc
Après la Médoquine.
De Monsieur de Vergez3 le truc
Est fameux, j'imagine.
C'est un beau morceau,
Il me place haut.
Je règne sur l'espace,
Je me fais l'effet
D'un roi, d'un préfet
Qu'on laisse encore eu place...
Couvé faisait allusion au passage du train sur un viaduc
de 920 mètres établi après le Haut-Brion.
Le chemin de fer de La Teste devint, le 27 juillet 1853,
propriété de la « Compagnie des chemins de fer dû Midi
et du canal latéral à la Garonne 4 ». Le 8 mars 1854 fut
proposée la suppression des « garages ou ports secs » de
la ligne dé La Teste, entre autres La Médoquine. Celte
proposition fut adoptée.
2. Article de Gustave LABAT inséré dans l'ouvrage Le Centenaire du
Lycée de Bordeaux.
3. M. de Vergez, ingénieur, était le constructeur de la ligne ferrée de
La Teste ; il eut comme collaborateur Alphand, alors aspirant ingénieur.
Paris dut plus tard à Alphand des transformations remarquables.
1. Tel était le nom primitif de la Compagnie du Midi.La Compagnie du Midi s'empressa de construire une
nouvelle gare de La Médoquine en créant le chemin de fer
de ceinture.
La gare « La Médoquine-Talence
» — pour l'appeler
par son véritable nom — est dotée d'un service de factage
et de camionnage qui, desservant les quartiers très éloi-
gnés de Bordeaux-Saint-Jean, présente pour le public le
plus grand intérêt.
De février à décembre 1921, ladite gare n'a pas expédié
moins de 5.500 tonnes.
Les nonagénaires de La Médoquine qui, dans leurs jeu-
nes années, regardaient passer le petit chemin de fer de
La Teste, voient circuler maintenant
sur le même point, à
90 à l'heure et dans
un fracas de tonnerre, les grands ex-
press européens.
La ligne du chemin de fer de ceinture a été ouverte :
1° au service normal des marchandises le 18 février 1921
;
2° au service des
voyageurs le 14 mars de la même année.
Les voies romaines
Plusieurs routes romaines parlaient de Bordeaux allant
dans différentes directions. Deux d'entre elles conduisaient
l'une à La Teste
par Croix d'Hins, l'autre à Bayonne par
Salles. Ces longues chaussées aux robustes empierrements
et dont on voit encore, par endroits, quelques vestiges,
furent suivies
par les armées de l'empire, les fonction-
naires, les voyageurs.
Les chemins de Saint-Jacques-de-Compostelle furent
plus tard, et pour la plupart, semble-t-il, établis
sur des
routes romaines. Les pèlerins s'étant arrêtés à Bordeaux,
rue du Mirail, à l'hôpital Saint-Jacques, et, reprenant leur
voyage, tournaient rue des Augustins, gagnaient la rueSainte-Catherine 5 et s'engageaient route de Bayonne,
(cours de l'Argonne) ; arrivés au carrefour Saint-Genès,
ils prenaient « le grant chemyn romyeu » (cours Gambet-
ta) 6, le suivaient jusqu'au chemin Roul, pénétraient dans
ce chemin qu'ils parcouraient dans son entier, laissant à
droite le village du Grand Courneau de Ruhan, et ils attei-
gnaient une route désignée encore sur les plans « chemin
de la voie romaine ».
La rue Sainte-Catherine, le cours de l'Argonne, le cours
Gambetta et le chemin Roul constituaient apparemment
un tronçon de la route Bordeaux-Bayonne, par Salles, tra-
cée par les vainqueurs de la Gaule, et cette route passait
donc sur le sol où devait se former le coeur même de Ta-
lence.
L'autre voie romaine Bordeaux-La Teste par Croix-
d'Hins, plus au nord, sillonnait également le futur territoire
de Talence, car elle suivait à peu près le tracé de la ligne
ferrée Bordeaux-Arcachon.
A travers les rues de Talence
Les nouvelles municipalités, à quelque parti qu'elles ap-
partiennent, n'ont, en général, dès leur installation, qu'un
souci : glorifier tout de suite leurs idoles ou leurs amis dé-
funts. On change alors les plaques de rues. On supprime
des noms sentant le terroir ou rappelant des traditions ou
des faits d'histoire locale et on les remplace par des noms
de personnages politiques.
5. Il y a trois ou quatre ans, des ouvriers occupés à des travaux de
terrassements, près de la porte d'Aquitaine, mirent à jour dans l'axe de la
rue Sainte-Catherine, à environ 1 m. 50 de profondeur, un sol pavé. On
se trouvait en présence — ce fut l'avis des gens compétents — d'un frag-
ment de route romaine.
6. Voir page 123 (note 72) et page 133.C'est là un très mauvais système. Certes, nous n'entre-
rons pas dans le détail des inconvénients multiples que
provoquent les perturbations viographiques. Nous ferons
seulement observer que lorsqu'on s'engage dans cette voie
— le mot est de circonstance — on ne peut prévoir où l'on
s'arrêtera.
Supposons une municipalité socialiste arrivant au pou-
voir : elle modifie les plaques indicatrices dans le sens de ses
opinions. C'est un sentiment très naturel.
Qu'une municipalité modérée survienne : elle s'empres-
sera de faire disparaître les appellations socialistes. C'est
encore humain.
Qu'une municipalité royaliste soit élue : elle fera revivre
les noms qui lui sont chers au détriment de tous les autres.
Il importe donc d'agir avec beaucoup de circonspection
quand on touche aux noms de rues consacrés par l'usage.
D'autre part, on ne devrait donner à celles-ci que les noms
d'hommes politiques dont la valeur personnelle, l'autorité
morale et surtout les services rendus au pays imposent un
respect unanime.
De cette manière, on éviterait l'engrenage. On ne mécon-
tenterait personne. On ne froisserait aucune susceptibilité.
On n'éveillerait aucune jalousie.
Les municipalités de couleurs différentes qui se succéde-
raient n'auraient pas à user de représailles vis-à-vis les
unes des autres.
Il y a, au surplus, un moyen de tout concilier, du moins
dans une certaine mesure. C'est de laisser une fois pour
toutes les noms existants, quels qu'ils soient, et de réserver
les noms nouveaux pour les voies nouvelles. C'est ce qu'a
compris, d'ailleurs, la municipalité de Talence qui, dans sa
séance du 12 juillet 1926, a baptisé comme suit les rues et
places créées dans les lotissements du Haut-Brion, des
Templiers, de Toulouse et Dunoyer :de
« Lotissement du Haut-Brion : Rue A, portera le nom
rue Marcel-Sembat ; rue B, rue Louis-Blanc ; rue C, rue
Ader ; rue D, rue Etienne-Dolet; la place, place du 11-No-
vembre.
» Lotissement des Templiers : La rue sera dénommée
rue Vergniaud ; la place, place Camille-Desmoulins.
» Lotissement de Toulouse : La voie n° 1 prendra le nom
de rue Général-André ; la voie 2, rue Docteur-Dupeux; la
voie 3, rue Mattéotti ; la voie 4, rue Ernest-Renan ; la voie
6, rue Jules-Guesde.
» Lotissement Dunoyer : La voie n° 1 sera dénommée
rue Calixte-Camelle; la voie 2, rue Paul-Louis-Courrier;
la voie 3, rue Léon-Bourgeois ; la place, place du 1er-Mai ».
La place du 11-Novembre est la plus heureuse de ces
dénominations, car c'est la date qui vit la fin du plus grand
cataclysme que l'histoire ait jamais enregistré.
Et maintenant, passons à une petite étude sur la viogra-
phie talençaise qui avait naturellement sa place toute mar-
quée dans cette monographie.
Bontemps (chemin)
Ce nom est apparemment celui de Bontemps-Dubarry
qui, pour raison de santé, donna sa démission de membre
du Conseil municipal de Talence le 15 avril 18157.
Cauderès (rue de)
Elle perpétue la mémoire d'un enfant de Talence, Jean
Baptiste Cauderès, prêtre du diocèse de Bordeaux.
En 1790, Cauderès avait trente-six ans ; des textes de
cette époque font suivre son nom de la mention : « ci-
7. Registre des procès-verbaux de la municipalité de Talence (Archi-
ves de la commune.)devant curé de Canéjan depuis 1780. » Il fut déclaré émigré
le 24 nivôse an II (13 janvier 1794), et rentra en France
après la Terreur. Mgr d'Aviau, archevêque de Bordeaux,
le nomma chanoine honoraire le 27 juin 1803.
Cauderès avait publié en 1783 un Eloge du comte d'Es-
taing 8.
Seul, le côté est de la rue de Cauderès appartient à Ta-
lence ; le côté ouest fait partie de Bordeaux.
Charles-Laterrade (rue)
Né à Bordeaux en 1818, Charles Laterrade fut agricul-
teur, littérateur et homme politique. Membre du Conseil
municipal de Talence le 11 mai 1871, il fut élu deux fois
maire dé cette commune, où il créa une bibliothèque popu-
laire 9 et organisa des cours d'adultes. Il fut révoqué, après
le 24 mai, par l'Ordre moral.
Laterrade fut conseiller général du quatrième canton ;
il mourut le 22 juin
1876.
Clément-Thomas (rue)
Né à Libourne le 31 décembre 1809, Clément Thomas fut
nommé représentant du peuple en 1848.
Déporté après le coup d'Etat de 1852, il rentra à Paris
lors de la proclamation de la troisième République. Le
4 novembre 1870, il devint commandant en chef de la pre-
mière armée, des gardes nationales de la Seine. Il démis-
8. Charles-Hector d'Estaing, amiral français, se signala par des suc-
cès contre les Anglais sur terre et sur mer pendant la guerre d'Amérique,
prit Saint-Vincent et l'île de la Grenade ; monta sur l'échafaud, parce que
noble, en 1794.
9. Il y a une dizaine d'années, la Bibliothèque populaire de Talence
fut supprimée; les livres en furent répartis entre les différentes écoles de
la localité. (Notes de la mairie.)sionna le 14 février 1871. Après l'insurrection du 18 mars,
il fut arrêté et fusillé, séance tenante, sans le moindre simu-
lacre de jugement. L'Etat a élevé au Père-Lachaise, à Pa-
ris, un monument à la mémoire de Clément Thomas et du
général Lecomte, fusillé, dans les mêmes conditions que
lui, quelques jours plus tard.
Colonel-Moll (rue du)
Le lieutenant-colonel Moll commandait le territoire du
Tchad quand il fut tué, le 9 novembre 1910, au cours d'une
action contre les forces du sultan du Massalit. Cet officier
supérieur, doublé d'un administrateur de premier ordre,
avait pour devise : « Je commande, donc je suis responsa-
ble. »
Cronstadt (avenue de)
Cronstadt, port militaire et commercial de la Russie, au
fond du golfe de Finlande. Le tsar Alexandre III y reçut,
en 1891, une escadre française. Ce fut le point de départ
de l'ancienne alliance franco-russe.
Le 13 octobre 1893 eut lieu, à Toulon, la réception d'une
escadre russe placée sous les ordres du contre-amiral Avel-
lan. Cet événement ne fit que développer les sentiments
d'amitié qui unissaient les peuples français et russe.
De Mons (chemin)
Cette voie reliant le cours Gambetta au chemin de Gra-
digna10 porte le nom d'une famille noble. Il est fait men-
tion sur le plan11 d'un quartier de Talence, au début du
du XVIIIe siècle, de la « vigne de Monsieur de Mons ».
10. Plan de Bordeaux el de sa banlieue par Louis LONGUEVILLE.
11. Ce plan n° 3097 est aux Archives municipales de Bordeaux.D'autre part, on relève sur une liste des moulins établis
sur le ruisseau de Talence « le moulin d'Arz à Mons »12.
Le 11 juin 1594, il y eut deux reconnaissances de fonds
situés à Talence et relevant des maisons nobles de Langon
et de Mons13.
Le 3 décembre 1692, le sieur Rivière fit l'acquisition
« d'un bois de haute futaye, appartenant à M. de Mons,
conseiller, situé en la paroisse de Talence, en graves 14.
Il s'agissait de M. Démons, conseiller au Parlement, de-
venu propriétaire de la seigneurie de Thouars, le 22 mars
1692.
D'Espeleta (avenue)
A l'origine, le domaine de Peixotto s'étendait du chemin
Erédéric-Sévène au chemin de Suzon15, qui est la dernière
voie que l'on trouve, à gauche, avant d'arriver avenue
d'Espeleta, en partant de l'église de Talence.
Le baron Espeleta, propriétaire de ce domaine, fit don
à la commune du terrain sur lequel on a construit la mairie
et créé la belle esplanade y conduisant. Le baron avait con-
senti cette donation à la condition que son nom fût donné
à ladite esplanade 16. Voilà pourquoi il y a, à Talence, une
avenue d'Espeleta.
12. Registre du Clerc de ville. (Archives municipales de Bordeaux.)
13. Inventaire sommaire. Registre de la Jurade, t. XI.
Le même ouvrage nous apprend que les 25 juin et 26 juillet 1536, il y
avait eu cinq reconnaissances, « en faveur du seigneur de la maison
noble de Langon, de fonds situés à Talence ».
14. Inventaire sommaire. Registre de la Jurade, t. IX.
15. Plan 138 (Archives départementales). On lit sur ce plan, levé en
juin 1782 : « Emplacement de la maison, appent, chay, cuvier et écuries
qui existaient lors de la reconnaissance consentie par M. de Jegun, le
12 avril 1758, devant Laville, notaire à Bordeaux. Le tout démoli par le
sr Pexotte ».
;
16. Notes de la mairie.Edison (rue)
Cette, voie a pris le nom du grand physicien américain,
inventeur de nombreux appareils électriques, notamment
du phonographe. Précédemment, c'était la rue du Prince-
Noir.
Le souvenir du fils célèbre d'Edouard III est rappelé,
en dehors du château, par l'enseigne Usine du Prince noir,
peinte — en lettres noires, dans flèche blanche, sur fond
noir — sur le mur, à l'angle du chemin de Roustaing et du
cours Gambetta.
Émile-Zola (rue)
Cette rue était dénommée antérieurement « chemin des
Montagnes », par rapport à une élévation de terre qui
l'avoisinait et sur laquelle passait la ligne du chemin de fer
de Bordeaux à La Teste17.
François-Coppée et de l'Union (rues)
Ces rues ont remplacé la « rue du Bois-de-Boulogne »,
et la « petite rue du Bois-de-Boulogne », qui rappelaient
la salle de danse dont il a été question dans un précédent
chapitre. Il est regrettable, à notre avis, que cette jolie dési-
gnation de Bois-de-Boulogne ait complètement disparu de
Talence.
Gambetta (cours)
Ce cours constitue l'artère principale de Talence, les deux
autres grandes voies, le cours du Maréchal-Galliéni et la
route de Toulouse, n'appartenant que par un côté à la com-
mune.
17. Nous avons vu dans notre enfance des talus qui étaient des ves-
tiges de ces « montagnes ».Le cours Gambetta est la voie la plus animée, la plus
commerçante. On y trouve les écoles primaires, la poste,
des établissements divers. Il passe devant l'église et devant
la mairie; il est ensuite borde par les magnifiques domaines
qui font l'incomparable charme de Talence.
A partir du chemin Banquey jusqu'à l'église de Talence,
une ligne de beaux peupliers se dressait de chaque côté
du cours Gambetta. Un petit fossé courait devant les mai-
sons. Sur les instances des propriétaires, les peupliers fu-
rent abattus, les fossés comblés, et l'on créa les trottoirs.
Maréchal-Galliéni (cours du)
C'est l'ancien chemin de Pessac. Le côté des numéros
pairs appartient à Talence ; celui des numéros impairs dé-
pend de Bordeaux.
Le Conseil municipal de Talence eut, le premier, la pen-
sée de baptiser du nom de Galliéni le chemin de Pessac.
Les édiles bordelais approuvèrent cette désignation le 12
novembre 1920. Par suite, le chemin de Pessac devint le
cours Général-Galliéni.
Le 12 avril 1921, la Chambre des députés adoptait sans
débat, à mains levées et à l'unanimité, une proposition de
loi tendant à conférer, à titre posthume, la dignité de maré-
chal de France au général Galliéni.
Du fait de ce vote, les plaques indicatrices « cours Géné-
ral-Galliéni » furent remplacées par celles de « cours du
Maréchal-Galliéni ».
Galliéni mourut le 27 mai 1916, dans la maison de re-
traite et de convalescence des soeurs franciscaines, rue
Maurepas, 29, à Versailles. On l'a justement appelé « l'ani-
mateur de la victoire de la Marne ».Pierre-Curie (rue)
Pierre Curie naquit à Paris le 15 mai 1859. Chef des
travaux de physique à l'Ecole de physique et de chimie
industrielle de Paris en 1882, il fut ensuite professeur de
physique générale à l'Ecole de physique et de chimie, puis
à la Sorbonne, en 1904. On lui doit, entre autres décou-
vertes, celle du radium. Sa femme collabora activement à
ses nombreux travaux. La moitié du prix Nobel pour les
sciences fut attribuée à M. et Mme Curie en 1904.
L'année suivante, Pierre Curie entrait à l'Académie des
sciences.
Ce grand savant fut écrasé par une charrette alors qu'il
traversait la place Dauphine, à Paris, le 18 avril 1906.
Rocambole (impasse)
Il y a eu un conseiller municipal de Talence appelé
Gilbert et surnommé Rocambole18. Ce conseiller habitait
dans la voie à laquelle son sobriquet a été donné.
Roul (chemin)
Il commence cours Gambetta et aboutit à la limite de
la commune, vers Pessac.
Propriétaire du château de Monadey, Roul fut élu mai-
re de Talence le 22 avril 1825. Député de la Gironde, il
obtint du Parlement, en 1845, une somme de 5 millions
pour achever le pavage de la route allant de Bordeaux à
Bayonne par les grandes landes.
Roul finit ses jours à Monadey à l'âge de quatre-vingt-
quatre ans. Il avait exprimé le désir d'être inhumé dans
18. Rocambole est le héros d'un des romans du fameux conteur Pon-
son du Terrail, qui mourut à Bordeaux, rue de Pessac.une chapelle qu'il avait fait édifier dans son domaine;
mais on ne tint pas compte de ce désir.
La dépouille mortelle de l'ancien député fut donc por-
tée au cimetière de la commune.
le
Le plan cadastral de Talence fut terminé en 1847
baron Sers étant préfet de la Gironde, et Roul, maire de
Talence.
Roustaing (chemin de)
Ce chemin conserve le souvenir des deux familles qui
ont tant marqué dans l'histoire de Talence : les Roustaing,
seigneurs de Brama, et les Rostaing, seigneurs de La
Tour.
Taillade (chemin de la)
Ce chemin porte le nom d'une famille qui avait des
propriétés dans le quartier. Il est appelé « ruelle Tail-
lard » sur le tableau d'assemblage du plan cadastral 1847,
et « ruelle Taillade » sur une des divisions du même plan.
Vieille-Tour (chemin de la)
En 1813, c'était le « chemin n° 5 de 3e classe »26. Partant
du lieu appelé La Médoquine, il passait « devant le do-
maine de M. le maréchal Perignon, Latour, Clamageran.
Sandré » et allait « jusqu'au village du Grand-Cournau ».
Il était long de 900 mètres.
Ce chemin s'appelait « chemin du Courneau à la Médo
quine » en 184721. On le baptisa, par la suite, chemin de
la Vieille-Tour, peut-être parce qu'il restait encore quel-
ques vestiges de l'ancienne Tour des Rostaing ?
19. Depuis lors, il n'a pas été établi d'autre plan cadastral de Talence.
20. Registre des procès-verbaux de la municipalité de Talence. (Archi-
ves de la commune.)
21. Plan cadastral.Visitandines (rue des)
On appelle Visitandines un ordre de femmes institué
en 1610, à Annecy, par saint François de Sales et la ba-
ronne de Chantal, en mémoire « de la visite que la Vierge
fit à sainte Elisabeth quelques jours après l'Annonciation ».
Cet ordre, dont la règle est peu sévère, fut approuvé par
Urbain VIII, en 1626, et se répandit bientôt en France, en
Italie, en Allemagne, en Pologne.
Les Visitandines étaient aussi dénommées « Religieuses
de la Visitation ».
Un groupe de ces religieuses s'établit à Talence. Il de-
vait occuper un emplacement situé dans l'angle formé par
la rencontre du cours Gambetta et du chemin de la Tail-
lade. Cet endroit est, d'ailleurs, ainsi indiqué sur le plan ca-
dastral de Talence : « La Visitendine. »
A la Révolution, le domaine des Visitandines devint
propriété de la Nation, et fut vendu, le 13 janvier 1791,
pour la somme de 31.400 livres, à M. Peynado fils, négo-
ciant, Fossés de ville 22. Il comprenait « un bien de campa-
gne, maison, chai, cuvier, 20 j. 17 r. vigne et terre » 23.
La rue des Visitandines est devenue depuis peu rue
Emile-Combes.
Les noms d'hommes politiques tiennent une place im-
portante dans la viographie talençaise. On y trouve, le
cours Gambetta et les rues Adolphe-Thiers, Jules-Simon,
Carnot, Jules-Ferry, Ludovic-Trarieux, Emile-Combes
Paul-Bert, Charles-Floquet, Maurice-Berteaux, de Freyci-
22. Cours Victor-Hugo, à Bordeaux.
23. M. MARION, J. BENZACAR, CAUDRILLIER. Documents relatifs à
vente des biens nationaux, t. I.
lanet, Jean-Jaurès, Waldeck-Rousseau, Félix-Faure, Camille
Pelletan. Deluns-Montaud, Emile-Loubet, René-Goblet.
Les écrivains sont représentés par les rues Voltaire,
Victor-Hugo, Emile-Zola, de Balzac, Elisée-Reclus, Au-
guste-Comte, François-Coppée.
Les rues Pasteur, Marcelin-Berthelot, Pierre-Curie et
Edison symbolisent la science.
La Révolution est rappelée par les rues Mirabeau, Dan-
ton et Robespierre.
Deux voies portent des dates célèbres : la rue du XIV-
Juillet (commémoration de la prise de la Bastille en 1789),
et la rue du IV-Septembre (proclamation de la troisième
République en 1870).
Les militaires sont aussi à l'honneur. Outre le cours du
Maréchal-Galliéni, il y a les rues Hoche, Marceau, Chanzy,
Margueritte, Bourbaki, Bordas.
Le chemin Bayard évoque sans doute la mémoire du
Chevalier sans peur et sans reproche ?
La rue Denfert-Rochereau glorifie le colonel qui défen-
dit Belfort, et la rue Rouget-de-l'Isle perpétue le nom de
l'officier du génie auteur de La Marseillaise.
L'éminent économiste Léon Say a également son nom
sur une plaque de rue.
Et voici des dénominations d'inspiration républicaine :
rues de l'Egalité, de la Liberté, de la Fraternité, de la
Vérité, de la Paix, de l'Union, de l'Espérance ; place de la
Concorde. Enregistrons aussi la rue de la République.
Les rues de Coulmiers et de Bazeilles portent des noms
de batailles. Celles d'Alsace et de Lorraine ont été aussi
baptisées en souvenir des deux provinces arrachées à la
France en 1871 et qui ont été réintégrées dans notre terri-
toire en 1918.
Des sentiments poétiques ont fait naître ces charmantes
appellations : rues de la Fauvette, de la Charmille, de laPrairie, du Porte-Bonheur (dans celte dernière artère de-
vait fleurir le muguet).
Des voies tiennent leur nom du quartier qu'elles traver-
sent : chemins de Banquey, de la Médoquine ; rues de Pey-
davant et du Haut-Carré. D'autres ont pris le nom d'anciens
châteaux auxquels elles conduisent : chemin de Thouars
de Raba, etc.
La rue Pey-Bouquey se rapporte à un bien dont il est
question dans un acte de 1532, et qui était situé « esgra-
bes de bourdeaux, au plantier de pey Coucquey, autrement
au Haumont.
Dans une esporle de 1618, dont il existe une copie datée
de 1691, le même plantier est ainsi désigné puch bouc-
quey
Plusieurs artères rappellent les noms d'hommes qui ont
présidé, en tant que maires, aux destinées de la commune :
Roul, Charles-Laterrade, de Mégret, Frédéric-Sévène, Jac-
ques Juillac.
En donnant à une rue le nom de Eugène-Olibet 25, on
a rendu un hommage légitime au commerce et à l'industrie.
Certaines voies portent des noms de propriétaires. Une
autre a un nom bizarre : c'est la rue du Pont-projeté.
Trois lignes de tramways desservent Talence par la
route de Toulouse, le cours Gambetta et le cours du Maré-
chal-Galliéni. Depuis un an environ, les trams de la ban-
lieue empruntent le réseau urbain. En sorte que les voya-
geurs venant de Léognan, Gradignan ou Gazinet sont
24. Arch. départ., G, 1001. (Renseignements communiqués par M. Trial,
membre de la Société archéologique).
Dans le langage du pays, pey a le sens de Pierre (Pey Berland), et
puch ou puy celui d'élévation du sol (Puy Paulin).
25. Honoré-Jean Olibet et son fils Eugène ont fondé, après 1862,
l'usine de Talence.directement conduits, s'ils le désirent, au centre même de
Bordeaux; avant cette pénétration, les voyageurs devaient
changer de voitures aux barrières, ce qui était parfois fort
désagréable.
Un coup d'oeil dans un Annuaire de 1840 nous a permis
de savoir comment on allait, il y a 86 ans, de Bordeaux
à Talence. Des voitures omnibus dites « les Bordelaises »
partaient de la place de la Comédie et se rendaient à Ta-
lence en passant par la place d'Aquitaine (de la Victoire).
Le prix du voyage était de 45 centimes, soit 15 centimes
de la Comédie à la place d'Aquitaine, et 30 centimes
d'Aquitaine à Talence.
D'autres voitures omnibus dites « les Françaises » se
rendaient de la rue Gobineau à la barrière du Moulin-
d'Ars, en passant par la place des Capucins. Le prix du
trajet était de 30 centimes : soit 15 centimes de la rue Go-
bineau aux Capucins, et 15 centimes des Capucins à la
barrière du Moulin-d'Ars.
Une entreprise de voitures brédoises assurait le service
de Gradignan. On s'embarquait place d'Aquitaine. On lit,
à ce propos, dans l'Annuaire de 1840 :
« Il part tous les jours à 7 heures, 11 heures du matin
et 3 heures de l'après-midi une voiture pour Talence et
Gradignan, et revenant à 9 heures, 2 heures et 4 heu-
res25 bis. »
Sous la rubrique « Entreprises diverses » on trouve le
service de Pessac. Trois départs avaient lieu tous les jours
pour cette destination : à 9 heures du matin, à midi et à
4 heures du soir, en hiver; à 8 heures du matin, à midi
et à 5 heures du soir en été. Dans la belle saison, un autre
départ avait lieu le dimanche, à 2 heures de l'après-midi.
La voiture stationnait au coin de la rue de Berry, près
de l'hôpital.
25 bis. Cet Annuaire est aux archives municipales de Bordeaux.En 1865, des voitures de la Compagnie générale des
Omnibus effectuaient le transport des voyageurs de la
place extérieure d'Aquitaine à Talence. Le service durait
de 7 heures du matin à 7 heures du soir 26. Le tarif était
toujours de 30 centimes.
En 1876, même point de départ et même prix. Il existait
alors une autre ligne de voitures reliant le cours du XXX-
Juillet (devant la maison n° 1) à la barrière Saint-Genès.
Le coût du voyage était de 20 centimes.
DÉCOUVERTES ARCHÉOLOGIQUES
Deux tombes Jumelles
Un coffre à double cuve — ou deux tombes jumelles —
paraissant appartenir à la fin de l'ère mérovingienne, fut
découvert, en 1875, par M. Delfortriedans le jardin de
la maison 59, cours Gambetta. Ce tombeau servait d'auge !
Les gens qui lui avaient donné cette destination ignoraient
évidemment tout l'intérêt qu'il présente au point de vue
de l'archéologie.
Suivant les renseignements recueillis, le propriétaire
de l'immeuble avait trouvé le coffre à la place qu'il occu-
pait. Les habitants du quartier « l'avaient toujours vu là
où il était ». On estima que le tombeau avait dû être trou-
vé sur les lieux ou à une petite distance. Comme il a un
poids énorme, il est peu probable, en effet, que l'on se
soit donné la peine de le transporter de loin. La pierre ne
présente aucune inscription.
26. Charles COCKS. Guide de l'étranger à Bordeaux
27. Société arehéologique de Bordeaux, t. II, année 1875.Au cours d'une de ses séances en 1878, la Société d'ar-
chéologie adressa des remerciements à M. Juillac, proprié-
taire à Talence, qui avait offert gracieuisement, à celte so-
ciété, le coffre à double cuve22.
Le tombeau est déposé au Musée des Antiques, rue
Mably.
Une stèle funéraire
Il y a eu au château de Thouars une stèle funéraire atti-
que20, en marbre pentélique, dans un bel état de conserva-
tion. Pendant longtemps, cette stèle resta reléguée en un
hangar du château. Elle se dressa ensuite sous un ber-
ceau de verdure, contre la margelle d'un puits couvert au
ras du sol.
Voici le sujet sculpté sur la face principale, la seule or-
nementée
Deux femmes s'abordent et se donnent la main. Elles
se présentent de profil. Toutes deux portent une sorte
de manteau passé sous le bras droit, et dégageant l'épaule.
Le pan du vêtement s'enroule autour de l'autre bras, fait
bourrelet à la ceinture, puis retombe.
Les deux femmes se serrent la main en signe d'adieu,
vraisemblablement.
Cette scène est encadrée par deux pilastres terminés par
des chapiteaux plats supportant une voûte faite de trois
arcs. De chaque côté de l'archivolte, il y a une petite ro-
sace en relief.
Au-dessus du cintre, dans la frise, apparaît une inscription
grecque que l'on peut traduire ainsi :
:
Zozimè, fille de Kallinikos, milésienne,
Femme de Phocion d'Otrynè.
28. Société archéologique de Bordeaux, t. V, année 1878.
29. Le monument a été découvert par le père Royer qui l'a signalé,
en février 1913, à M. le comte Aurélien de Sarrau, lequel l'a visité en
compagnie de M. l'abbé Moureau, curé de Talence.Cette inscription révèle le mariage d'un Athénien avec
une étrangère. De telles unions n'étaient pas admises au
temps d'Athènes libre, ou elles étaient très rares. Elles se
multiplièrent quand la Grèce devint province romaine. On
en conclut que la stèle de Talence remonte à la fin du
second siècle ou du premier siècle avant Jésus-Christ.
Le monument est couronné d'un fronton triangulaire
montrant, dans le tympan, une rosace de même style que
les deux autres, mais de plus grande dimension. Il a
1 m. 20 de hauteur, 0 m. 50 de largeur au fronton, 48 cm. 5
au corps, et 12 cm. 5 d'épaisseur 50.
A propos de cette stèle, M. Paul Fournier, professeur à
la Faculté des lettres de Bordeaux, a écrit :
« À la banalité du sujet, des visages, des poses, on recon-
naît le produit d'un atelier sur le chemin du cimetière, où
les parents de la morte ont fait leur choix, le lendemain
de l'enterrement 31. »
Comment cette pierre funéraire se trouvait-elle à Ta-
lence ? M. Paul Fournier assure qu'elle ne « peut passer
pour un témoin de notre histoire locale ». A-t-elle été
apportée sur nos quais par un bateau qui l'aurait reçue
d'une barque à laquelle elle aurait servi de lest ?
L'armateur Balguerie junior a été propriétaire du châ-
teau de Thouars. Il est possible qu'un de ses navires ait
ramené la pierre à Bordeaux dans les conditions indiquées.
L'armateur, ayant eu son attention attirée sur la stèle,
se serait empressé de faire transporter celle-ci dans son
domaine à Talence. Il n'y aurait plus songé par la suite,
puisqu'elle fut retrouvée, dans un hangar, par Mme la mar-
quise du Vivier, qui la fit elle-même placer contre la mar-
gelle du puits.
30. L'Aquitaine, n° du 21 février 1913,
31. Revue des Etudes anciennes, n° de juillet-septembre 1913.Pour M. le comte Aurélien de Sarrau, la stèle a dû être
exécutée et sculptée sur place ; « elle devait, dit-il, surmon-
ter quelque sarcophage ». A l'appui de son opinion, M. de
Sarrau souligne que, dans le même lieu, se trouve un grand
sarcophage 32 de marbre blanc veiné de bleu — marbre des
Pyrénées
— dans lequel fleurissent, actuellement, grami-
nées et plantes diverses. Il fait observer, en outre, qu'au
temps d'Ausone, la plupart des serviteurs étaient Grecs.
Quoi qu'il en soit, la stèle du château de Thouars « porte
à trois le nombre des inscriptions grecques de Bordeaux
ou de sa banlieue » 33. Elle a été emportée par le proprié-
taire qui a vendu le domaine à M. d'Ornellas.
Curieuses margelles de puits
Il y a à Talence trois margelles de puits en pierre avec
écussons des XVIIe et XVIIIe siècles.
La première de ces margelles se trouve dans le domaine
de Capdaurat, entre le chemin de Pey-Bouquey et celui
de la Vieille-Tour. Elle est formée par un renflement cir-
culaire portant un écusson de forme ancienne, avec une
croix et la date 1731.
La seconde margelle est à l'extrémité de la vigne de Cap-
daurat, près du chemin de la Vieille-Tour. De même for-
me que la précédente, elle est cependant d'une facture plus
soignée ; sur son renflement se détache un écusson portant,
gravés, le monogramme du Christ avec croix et le millé-
sime 1699.
La troisième margelle orne la cour de la propriété du
Castel, qui est en bordure du chemin Roul et appartient
32. Ce cercueil de pierre a les dimensions suivantes : longueur,
2 m. 10; largeur, 0 m. 74; hauteur, 0 m. 56, profondeur (intérieur), 0m. 44;
épaisseur, 0 m. 12.
33. Paul COURTEAULT. Revue historique de Bordeaux. 1913.à M. Holagray. Elle présente « deux moulures bombées,
en forme de doucine droite et renversée, séparées par une
bande ornée d'un écusson contourné, de forme mo-
derne » ".
Monnaies romaines
En février 1857, on fit, au Noviciat des frères des Ecoles
chrétiennes, à Talence, cours Gambetta, 124 35 une impor-
tante trouvaille touchant la science numismatique. En creu-
sant un trou dans une allée du jardin pour en extraire
du sable, on découvrit, à 70 centimètres de profondeur,
un lot de pièces romaines en bronze, de moyen et petit
module, à l'effigie des derniers empereurs romains. Il n'y
avait autour aucun vestige de bois pouvant faire présumer
que les pièces avaient été placées dans une caisse. On
s'était borné sans doute, avant de les enfouir, à les enve-
lopper dans un petit sac qui aura été détruit par les siè-
cles. L'absence complète de monnaies françaises parmi
celles qui ont été mises à jour « autoriserait à croire que
le dépôt remonte aux invasions barbares » ".
Toutes les pièces, parfaitement conservées, « étaient
réunies et collées ensemble au moyen d'oxide »
La plus vieille est à l'effigie de Postumus, qui fut pro-
clamé empereur dans les Gaules au début de l'an 261. En
267, après avoir vaincu le tyran Lélien, près de Mayence,
Postumus fut massacré par ses soldats pour n'avoir pas
voulu leur livrer le pillage de cette ville
La moins ancienne des pièces de monnaie rappelle Cons-
34. Ces margelles ont fait l'objet d'une communication do M. l'abbé
Royer (séance du 9 juillet 1920 de la Société Archéologique).
35.Le Noviciat est indiqué sur l'Atlas départemental du Conseil géné-
ral, cours Gambetta, à gauche, un peu après le pont du chemin de fer.
36. La Guienne, nos des 15 et 22 février et 1er mars 1857.
37. Ibid.
38. L'art de vérifier les dates.tantin Il dit « le Jeune né à Arles le 1re mars 316, et pro-
»,
clamé auguste et empereur l'an 337, après la mort de son
père, le grand Constantin.
Constantin II périt, l'an 340, dans une embuscade que
lui dressèrent les généraux de
son frère Constant, près
d'Aquilée39.
L'aqueduc romain
En 1826, l'attention de Billaudel, ingénieur des ponts et
chaussées, fut attirée par la découverte d'une portion
d' aqueduc,
au Pont d'Ars, dans une sablière appartenant
à un pharmacien nommé Cazenave.
Une commission fut chargée d'aller reconnaître cette
portion d'aqueduc. Elle comprenait MM. Blanc-Dutrouilh,
secrétaire général; Billaudel, l'ingénieur ; Durand, archi-
tecte; Lartigue, pharmacien, et Jouannet, conservateur des
antiquités du département.
Cette commission fit un rapport 40 très intéressant.
L'aqueduc, rencontré à six pieds de profondeur, était
de forme rectangulaire, construit en béton et présentait
tous les caractères des constructions romaines de ce genre.
On en découvrit d'autres fragments qui permirent d'établir
que cet ouvrage était tantôt rampant, tantôt souterrain,
tantôt porté au-dessus du sol, suivant les ondulations du
terrain.
Quand l'aqueduc courait au-dessus du sol, il était porté
ou sur un mur ou sur des arcades. C'est sur des arcades
qu'il franchissait le ruisseau dé Talence ou des Maleret-
tes. La commission estima que « les noms de Pont d'Ars
et de Courneau d'Ars que conservent, sur la rive droite
30. L'art de vérifier les dates.
40. Ce rapport a été publié dans les Actes de l'Académie des sciences,
bettes-lettres et arts de Bordeaux (années 1825-1887).des Malerettes, deux endroits éloignés l'un de l'autre de
cinq cents toises, semblent rappeler le souvenir de ces
arcades. »
MM. Blanc-Dutrouilh, Billaudel et leurs collègues s'oc-
cupèrent de rechercher la direction de l'aqueduc. De la
sablière du pharmacien Cazenave, par conséquent du Pont
d'Ars, il allait en droite ligne sur Bordeaux.
Cet aqueduc n'était autre que celui dont il a été question
au premier chapitre et qui conduisait l'Eau Blanche au
coeur de Burdigala. Elie Vinet, on s'en souvient, en
avait, le premier, reconnu les vestiges en 1552.
L'ouvrage avait été édifié sous Tibère. On peut en
voir des morceaux au musée des Antiques, rue Mably, en-
tre autres, une pierre ayant 43 centimètres de largeur à
l'intérieur et 67 centimètres de hauteur.
Des restes peu connus du même aqueduc subsistent
près du village de Sarcignan 41.
De sa promenade au Moulin d'Ars en 1552, Elie Vinet
avait rapporté un fragment de l'aqueduc consistant en
« un tuïau de terre cuite d'environ demi pié de diamètre,
rompu par les deux bouts, et aiant encore de longueur
bien près de pié et demi » 42. Il avait fait présent de
ce tuyau « à maistre Joseph de La Chassagne, conseiller
du roi en la court du Parlement de Bourdeaus, homme
fort studieux et grand admirateur d'antiquité » 48.
On lit dans un article intitulé « Talence » et publié dans
le Musée d'Aquitaine (année 1823) :
« Nous avons reconnu dans le sud de Talence, du côté
du château de Salle, quelques pieds d'un aqueduc ram-
pant, que nous nous proposons de suivre pour reconnaî-
tre sa direction ».
41. J.-A. BRUTAILS. Guide illustré dans Bordeaux et les environs, 1906.
42. L'antiquité de Bourdeaus.
43. Ibid.L'auteur de cet article a mal situé le château de Salles
ou plutôt le Bien de Saltes 44, qui est en quelque sorte
au centre de Talence, exactement à gauche du cours Gam-
betta, un peu avant le quartier de Banquey.
D'autre part, on ne sait s'il donna suite à son projet
de s'assurer de la direction de l'aqueduc rampant. En tout
cas, la commission nommée en 1826 paraît avoir mis les
choses au point en ce qui concerne l'aqueduc de Tibère
qui passait, au Pont d'Ars, sur le territoire de Talence.
Une inscription latine
Dans le sol où fut ouverte la rue Duluc — laquelle sépa-
rait autrefois Bordeaux de Talence — on a trouvé « une
pierre funéraire, celle d'un Espagnol, avec inscription la-
tine ». Celte découverte, suivant le comte de Sarrau, indi-
que que peut-être il y a eu là un domaine gallo-romain.
Peintures décoratives au château du Prince-Noir
M. Clavé, étant propriétaire du château du Prince Noir,
fit faire, un peu avant l'année 1888, des réparations
intérieures à la partie du château reconstruite vers le XVIIe
siècle. En effectuant le travail, les ouvriers trouvèrent,
sous du papier de tenture, une décoration qui était desti-
née à un petit oratoire du premier étage.
M. Augier a cru pouvoir attribuer ces restes de peintures
décoratives à la fin du XVIIe siècle 45. Elle consistent dans
l'ornementation d'un contre-rétable constitué par deux pi-
lastres cannelés supportant une corniche; la peinture a
44. Plan cadastral de Talence.
45. Séance de la Société archéologique de
1888.
Bordeaux du 13 janviersuppléé à la sculpture dans toutes les moulures. Sur les
côtés formant la saillie des pilastres, il y a de jolies ara-
besques avec des têtes d'anges, se détachant sur un fond
noir. Des filets dorés venaient égayer les tons rouges,
bruns, verts et bleus dont sont couverts les fonds et les
surfaces planes. Le plafond est un lambris, sur lequel le
peintre décorateur a tracé des divisions pour des caissons
dans lesquels un ornement varié a été peint de différentes
couleurs sur un fond. Au centre, le Saint Esprit dans des
nuages avec des têtes d'anges.
Des réparations antérieures avaient fait disparaître le
reste des décorations sur le mur.
Ces peintures, quoique n'offrant pas beaucoup d'intérêt,
devaient être néanmoins, estimait-on, conservées et restau-
rées.
Une médaille aragonaise
Quand on creusa les fondations de la maison 199, rue de
Saint-Genès, on trouva beaucoup d'ossements, des crânes
dolichocéphales, des pièces de monnaie d'Henri, roi d'An-
gleterre ; de Philippe - Auguste, des doubles - tournois
Louis XIII, une médaille aragonaise46. La maison en ques-
tion s'élève sur une partie de l'emplacement de l'ancien ci-
metière de Saint-Genès, où devaient être inhumés, non
seulement les habitants de la paroisse, mais encore les pè-
lerins décédés en arrivant aux portes de Bordeaux.
La médaille aragonaise indique bien le retour de voya-
geurs venant d'Espagne.
46. Notes de M. le comte de Sarrau.CHAPITRE X
EN L'HONNEUR
DES MORTS GLORIEUX
Le cimetière américain
Un cimetière américain a été créé sur le territoire de
Talence. On y a inhumé les restes des soldats du général
Pershing morts, dans les formations sanitaires, des suites
de blessures reçues à l'ennemi.
Nous avons assisté, le 30 mai 1920, à la manifestation
patriotique organisée, dans ce cimetière, à l'occasion du
Mémorial Day. A côté des autorités françaises étaient réu-
nis MM. Théodore Jaeckel, consul des Etats-Unis; Atlee,
consul d'Angleterre; le chapelain Hewetson; Mathews, du
service des cimetières américains; Smith Port, secrétaire
de l'Y. M. C. A. La colonie américaine de Bordeaux avait
fait déposer dans la nécropole une gerbe de roses avec
cette inscription en anglais : « A la mémoire des soldats
américains tombés pour la patrie. »
Chaque année, le 30 mai, une pieuse cérémonie se dé-
roule, avec le même, rite, dans l'American cimetery.
Le cimetière communal
La ville de Talence a été fortement éprouvée par la guerre
1914-1918. Combien de ses enfants sont morts au champd'honneur! L'âme de ces braves plane sur la commune; elle
y entretient la notion du devoir, l'amour de la liberté, le
culte du patriotisme.
Talence a élevé au cimetière un monument à ses fils tom-
bés pour la plus noble des causes.
Le monument, d'une sobre architecture, présente une
pyramide sur piédestal, entouré de gradins. Il porte cette
inscription : « Aux enfants de Talence morts pour la pa-
trie
La face antérieure de la pyramide montre une palme;
sur la face opposée, il y a un glaive, la lame tournée vers
le sol. Cette arme, dans sa position, doit symboliser la
paix. Un flambeau renversé orne chacune des deux autres
faces de la pyramide. C'est sans doute l'image de la vie
éteinte en pleine ardeur ?
Les noms des héros garnissent le piédestal, formant
comme un palmarès sublime. Quatre urnes, que supportent
des socles, encadrent le mausolée commémoratif.
L'inauguration de ce monument a eu lieu le dimanche
1er juillet 1923, au cours d'une émouvante solennité.
Le cortège officiel se forma devant la mairie et se ren-
dit à la nécropole par le chemin de Suzon. Marchaient en
tête la batterie La Talençaise et la Lyre talençaise, jouant
des marches funèbres. Venaient ensuite les sociétés loca-
les avec leurs drapeaux ou bannières : Les Enfants de Ta-
lence (gymnastes), le bataillon féminin des Libellules, Les
Camarades de combat, La Sauvegarde du Poilu, Les Vé-
térans de 1870-1871, l'orphéon l'Avenir.
Les élèves du lycée de Talence et ceux des écoles com-
munales fermaient le cortège avec M. Jean Iriquin, maire
de Talence, qu'accompagnaient M. Boucanus, adjoint, et
plusieurs autres membres de la municipalité.
Au cimetière étaient réunis, au milieu d'une affluence
recueillie, les représentants du gouvernement, ceux de l'ar¬
».mée, les délégués des corps élus, les consuls des puissan-
ces amies et alliées. Les morts glorieux de Talence reçu-
rent, ce jour-là, l'hommage suprême de la France et des
nations qui luttèrent à nos côtés pour le triomphe du droit,
de la justice et de la civilisation.
M. Iriquin salua les braves disparus, s'inclina devant
les familles endeuillées, puis, s'adressant à la jeunesse, il
dit :
« N'oubliez pas que notre nation était condamnée à dis-
paraître à tout jamais, si les héros dont le nom est mainte-
nant gravé sur ce monument n'avaient constitué de leurs
poitrines une barrière à l'envahisseur. »
Le maire, d'une voix émue, donna lecture de la liste des
fils de Talence tués à l'ennemi.
A l'appel de chaque nom, les élèves des écoles répon-
daient en choeur : « Mort pour la France ! »
Le secrétaire général de la préfecture évoqua les pre-
mières heures du conflit mondial :
« La France, souligna-t-il, a manifesté son sincère désir
de paix quand, en août 1914, elle faisait reculer ses trou
pes de 10 kilomètres et laissait ses frontières ouvertes à
l'ennemi ; mais lorsque la guerre nous fut imposée, ce fut
avec un mâle courage que tous les Français volèrent au
salut de la patrie 1.»
Et le représentant du gouvernement conclut en faisant
appel à l'union indispensable de tous les citoyens pour
permettre à la France de se relever, de panser ses blessures
et de poursuivre ses glorieuses destinées.
1. La Petite Gironde, n° du 2 juillet 1923.Liste des enfants de Talence morts pour la patrie
et dont les noms figurent sur le monument commémorant.
NOMS ET PRÉNOMSRÉGIMENT
Aso Louis-Manuel ......10e huss.
7e colon.
144e R. I.
AGO Gabriel……
Albert-Victor ....
ALICOT Gaston .......... 220e R. I.
AVEZ André-François . . 160e R. I.
ADAM
AMEAU Jean……
142e R. I.
ANDIRAN Henri……
19e corps.
AYMAHD Joseph
53e R. I.
ARMELLA Henri-André . . 155e R. I.
AUDIGNON Pascal
1er R.A.C.
AUGER Louis-Paul
56e R. I.
ARTIGUEBIELLE Emile . . . 35e R. I.
AGUERRE Jean……
43e R. I.
BERNARD
Jean-Emman. 20e R. I.
BERNARD Raymond
417e R. I.
BOURBON Marcel ......... 108e R. I.
BOULNOIS Auguste ....... 7e colon.
BARRÈRE Félicien
58e R.A.C.
BARBÉ Jean-Marie
...........
......
......
......
...... 16e ch.àp.
BARANDON
Pierre ......
16e
ch.àp.
R. I.
BARRÉRE Maurice ....... 212e R. I.
BARRIÈRE Jules
119e R. I.
BAUEH Jean-Georges
127e R. I.
..
BEAUMONT Simon
299e R. I.
.......
......
BÉCÈDE Jacques-Urbain. 140e
terr.
BELHADE Gaston-Louis
18e sect
BELLEGARDE Frédéric . . 272e R.
BELLOCQ Antoine
124e R. I.
RENARD Charles............ 90e R.A.L.
BELHADE Louis
230e R. I.
BÉRAUD Edouard
6e R. I.
BÉRANGER Antoine
10e huss.
BÉNÉTIER Justin-Jean . . 4e b. chas
BERGEREAU Fernand
20» train
BERGERET Pierre-Albert. 14e R. I.
BERJU Pierre-Auguste
18e R. I.
.
BERJU Jean……
232e R. A.
BERIU Armand……
413e R.A.I.
BERJU Ferdinand . . . . 3e R. I.
......
............
......
.....
...
.
I.
NOMS ET PRÉNOMS
RÉGIMENT
...........
BERNARD André
88e R I.
BERTRAND Constant
r. m. Sén.
BETBEDAT Fernand
1er zouave
BETBEDAT André
20e R. I.
BAUMET Pierre-Emile . . 220e R. I.
BERNARD Jean-Léonce . . 144e terr.
.....
......
......
BESSIÈRE Paul-Théodore 135e R. I.
BEYS Jean……
2e tiraill.
BILLAUT……
5e tiraill.
BLONDEL Henri ............ 7e colon.
BOISDRON Alfred
31e R. I.
Du BOIS DE MAQUILLÉ G. 144e R. I.
BOLLAYUSON Raymond
344e R. I.
BONANGE Guillaume .... . 7° colon.
BONFILS Georges-Elie . . 108e R. I.
BORDESSOULE Jean-Louis 208e R. I.
......
Bosc Jean . . . . . . . . . . 5e d. flotte
BOUCANUS Louis-Jean . .. 214e R. I.
BOUCHON Robert ..... 49e R. I.
BOUFFANAIS Armand
144e R. I.
..
BOULOU Marcel............. 18e sect. I.
BOUOUET Georges-Jean. 12e R. I.
BOURBON Emile-Pierre . . 67e colon.
BOURDON Marcel-Jean . . 108e R. I.
BOUSQUET Charles ..... 170e R. I.
BOUSQUET Martin-Henri. 144e R. I.
BLAQUIÈRE Raoul ........ 7e colon.
144e R. I.
BOSCQ Pierre……
BOUDOT Hippolyte ...... 24e chass.
BONOTUIL Daniel ....... 58e R.A.C.
BIDEGARRAY HENRI
U A UI.A1A 11 nn 1
AA A/Ui A
. 1er
1er
]nr AAiinua
zouave
3e colon.
BOUY Jean-Maurice
6e R. I.
..
BRÉSIL Marie-Joseph . . 205e R. I.
BREUIL Henri-Noël ..... 110e R. I.
BROCHAIN Jean ........... 144e R. I.
BROCHARD Jean-Henri
344e R. I.
..
BROUSSIGNAC Ernest
123e R. I.
.
9e R.A.L.
..
.......... 344e R. I.
BROUEL Louis-André
BRU GuillaumeNOMS ET PRÉNOMS
RÉGIMENT
101e R.A.L
BRUN Jean……
BOURBON Jean……
BOUCHARD Gabriel
BOISSIÈRE Louis
144e terr.
144e R. I.
20e R. I.
BRUBALLA Armand ...... 18e R.A.C.
BRÛLÉ Léon……
142e terr.
BULEY Labrunière ...... 7e chass.
BUFFET Bernard-Eugène 8e chass.
BRUNET Jean……
12e R. I.
BRUNEAULT Paul-René
100e R. I.
BRANEYRE Jacques
20e R. I.
BARADERIE Emile ......
9e zouave
...
.....
...
Emmanuel
.
CANTOU Maximilien ..... 20e R. I.
CAPDEVILLE Pierre ...... 20e R. I.
CARBONNEL Alphonse
201e R. I.
..
CASABONNE Julien ........ 201e R. I.
CASTERA Jean……
18e R. I.
CASTERA Pierre-René . . 57e R. I.
CASTAGNOS Jean-Octave 100e R. I.
CAUMÈS Gabriel ......
2e gr. av.
CAUQUIL Henri
4e zouave
CAZAUX Henri……
344e R. I.
CAZENAVE André . . . .
18e R. I.
.
CESTAC Marceau-Didier. 7e Maroc.
CHADAINE Fernand ..... 49e R. I.
CHAMBRON Louis ...... 7e colon.
CHANCEAULME Marcel
418e R. I.
..
CHAMOULEAU Jean-Henri 20e R. I.
CHANLOU Alexandre ..... 96e R. I.
CHAUMENY Lucien
342e R. I.
CIRCO Jacques-Marie . . 327e R. I.
CLAOUI Pierre ............... 18e R. I.
CLARRACQ Léon-Jean
génie.
. 4e
CLAVERIE Louis ....... 112e terr.
CLÉMENT Charles ....... 18e R. I.
CLOUZÉT Pierre............. 418e R. I.
COLLIN André……
57e R. I.
COLOMÈS Henri ............ 169e R. I.
BARANDON
......
Pierre ..... 24e R. A.
COSTE Léonard .......
144e R. I.
COMMAGÈRE
.... 412e R. I.
COURANT Guillaume
CRISTOFOLI Victor ......
CROUAN Jean-Pierre
CLESS Jacques
CHANLOU Jean-Louis
...
..........
167e R. I.
109e R. I.
2e étrang.
63e R. I.
Léopold ......... .. 7e colon.
CHEVALIER Désiré . . . . . . 32e R. I.
...... 168e R. I.
DARMAYAN Henri
CATALA
NOMS ET PRÉNOMS
RÉGIMENT
Jean-Paul . . 234e R. I.
DARBON Louis……
57e R. A.
DANCHOTTE Jean-Etienne 17e chass.
DALBE Joseph-Abel .... 121e chas.
DANIAUD Albert ....... 176e R. I.
DARTIGUE Louis-Cyprien 14e Artill.
DAZET Jean-Marie-René. 6e R. I.
DECOUVELAÈRE Alfred . . 4e cuirass.
DELFIEUX Jean ............ 167e R. I.
DETY Adrien-Louis . . . . 27e chass.
DAUGREILH
DELON Henri……
53e R. A.
DELON Dominique ....... 417e R. I.
DÉSIRÉ Jean……
11e R. I.
DELRIEU Alexandre ..... G. art. d'Afriq
DEYZIEU Antoine
20e R. I.
DERIDET Jean-Jules ..... 18e C.O.A.
DENIGÈS Fernand
5e dépôt.
DEGAT Adolphe
96e R. I.
DELMAS Jean-Pierre
21e R.A.C.
DELUGAT Henri
112e RAC
DESPLATS Marc ............. 251e R.
DICHON Marcel
142e terr.
.......
.......
...........
.....
...........
............
I.
DIOT Pierre-Roger ..... 103e R. I.
DOUX-GAYAT Jean
123e R. I.
DOUSSET Xavier
147e R. I.
DURAND Pierre .............. 78e R. I.
.......
......
DUPLAND André ............. 2e génie.
DUPHIL Henri……
7e colon.
DUMENIL Louis-Célestin. 18e R. I.
DUDOUSSAT Pierre ...... 57e R. I.
DUCASSE Henri-Eugène. 1er R. D. P. G.
DUBEAU Gabriel
44e R. I.
DUBAYLE Jean……
138e R. I.
DUCASSOU Pierre
269e R. I.
DUPRAT Pierre
23e R. I.
DUBET Claverie-Georges 7e R. I.
DUBARRY Paul……
123e R. I.
DUPORT Lucien
418e R. I.
DUFAZA Henri-Maurice. 123e R. I.
DUBOURG Pierre
37e colon.
DUPERRIER Ferrand
371e R. I.
DUPONT Pierre .............. 6e R. I. C.
DURAC Amédée……
12e R. I.
........
.......
.............
..........
.......
....
DURRIEU Antoine
...... 9e R. I.
DRUJON Pierre-Emile . . 57e R. I.
DASSÉ Léon……
133e R. I.
Albert ............. 37e colon.
DEXRERT Clément ......
125e R. I.
EXPERT Joseph-Albert.. 123e R. I.
DANSANNOMS ET PRÉNOMS
RÉGIMENT
140e terr.
2e aviat.
ETCHEBARNE Félix ......... 2e génie
EYCHENNE François . . . . 87e R. I.
ELGUEZABAL Roger ....... 2e aviat.
FAUVEL Fernand
7e colon.
FAUVEL René-Xavier . . . 123e R. I.
FAUX Benoît……
91e R. I.
FOULQUIÉ Jean-Georges. 61e R. I.
FOURTON Jean……
IIe R. I.
FRÈCHES Antoine-Gaston 83e R. I.
FAYE Julien……
FORTIN Pierre
18e R. I.
ELIE Pierre……
ESCOFIÉ Marcel
......
......
........
Abel.
GAUTIER Jean, en famille
3e colon.
Justin-Henri. 7e colon.
GARÈS Henri……
20e R. I.
GAILLARD Henri-Paul . . 220e R. I.
GALÈS Raoul ............ 31e colon.
GARAT Lucien-Joseph . . 41e R. I.
GARDÈRES Firmin ....... 22e colon.
GARREAU Gaston ........
34e colon.
GABILLARD
146e R. I.
GARDÈRES Louis-Hippol. 37e colon.
GARDIA Roger-Marcel . . 57e R. I.
GAUJAC Jean-Marie ..... 123e R. I.
GAUSSENS Marcel-Pierre. 1er zouave
GAUTHIER Louis
65e R. I.
GAUTHIER Antoine ...... 208e R. I.
GAUTHIER François ...... 7e colon.
GAUDIN Pierre ............ 60e R. I.
GAY
160e R. I.
GAYE Louis-Narcisse . . 12e R. I.
GHINZONE Fernand
123e R. I.
GIBILY Henri……
315e R. I.
GIMEL Jean……
60e R. I.
GIMEL Charles ........... 72e R. I.
GENTIL Emile-Céleslin . . 135e R. I.
GAGEAC Jean……
........
Jean……
....
GOUGET de Casteras… 18e sect.
GERAUD Jean-Marcel . . 10e huss.
GENDREAU Louis-Joseph. 44e R. I.
GERGOÙIL Jean
20e R. I.
GONSARD Florentin
51e tiraill.
GORGUES Henri-Jérome. 73° R !
GOUJON Jean……
126e R. I.
GOULIÉ Jean-Pierre . . . 10e R. I.
GRAS Georges……
50e R.
GRANGE André-Jean
3e colon.
GRANERO Jean-Henri . . 3e colon.
..........
....
....
I.
NOMS ET PRÉNOMS
RÉGIMENT
......
201e R. I.
GRÉGOIRE Pierre
330e R. I.
GREL Robert……
GROS Pierre-Gabriel . . 44e R. I.
49e R. I.
GUÉRINEAU Moïse
147e R. I.
GUICHARD Charles
83e R. I.
GUINARD Albéric
GUILLEMOTONIA François 1er génie
17e chass.
GUITARD Jean-Louis
GUSTIN Gustave-Joseph. 132e R. I.
ILALLEY Henri ............. 11e R. I.:
HAURET Jean-Baptiste . . 18e R. I.
144e R. I.
HENRY Joseph……
16e R. I.
HERAIL Marcel..............
HUSSE Pierre-Norbert 203e A. L.
HOUSSAY Emile ............ 62e R. I.
206e R. I.
IMBERT Joseph..........
31e ch.àp.
JACOT Pierre……
34e R. I.
JAMAY Pierre……
.......
....
.......
...
2e b. Afr.
20e R. I.
JAUBERT Théodore
JEAN-MARIE Etienne ..... 52e colon.
JOLLY Alexandre ........ 13e R. A.
JAMMAUD André ...........
....
3e colon.
JOYAUX Henri……
LABORDE Roger-Etienne 87e R. I.
37e colon.
LABAT Jean……
LABBÉ Jean-Baptiste . . . 31e R. I.
LABORDE Aurélien-Pierre 3e colon.
83e R. I.
LABORIE Blaise
24e colon.
LABORIE Laurent
95e terr.
LABOUR Abel……
LACROISADE Joseph et
LACROISADE Jacques
234e R. I.
..
LACHAISE Joseph
255e A. L.
LACROIX Etienne-Jean. . 3e colon.
LACROTTE Jean
15e drag.
LADIGUE Marcel .........
3e colon.
LAFON Louis-Georges . . 137e R. I.
LAFONT André-Frédéric. 34e R. I.
LAGIÈRE Jean-Mathieu . . 34e R. I.
LAHILLE Auguste
83e R. I.
..........
......
....
..........
......
22e colon.
LALANDE Jean……
LALANNE Elie-Jean ....... 6e R. I.
LALANNE François ....... 418e R. I.
LALLEMAND Isidore
......
150e R. I.
LALOT Gabriel
144e R. I.
7° colon.
LAMONERIE Etienne
LAMOTHE Paul-Gabriel.. 50e R. I.
........
.....
LAMOTHE Marcel ......
LABEYRIE Jean..............
ch.àp.
3e chass.
27eNOMS ET PRÉNOMS
RÉGIMENT
LAMOU Jean-Camille . . . 50e R. I.
LANNELONGUE Pierre
2e colon.
LANTRÈS Henri ............. 73e R. I.
LAPEIGNE Emile-Jean . . 144e R. I.
LAROCHE Léon……
144e R. I.
LAHRIEU SANS Louis . . 11e R.
LASFARGEAS Emile ...... 37e colon.
LAUILHÉ Louis-Jean
62e R. I.
LABEYRIE Marcel-Lucien 53e colon.
LAUSSUT Jean-Léopold. 257e R. I.
LAUSSUT Gabriel
18e R. I.
LAVAL Henri-Pierre . . . 20e R. I.
250e R. I.
LAVAL Jean……
LAVEAUX Henri
—
LAVEUVE Louis
—
LAVILLE Georges-Pierre. 18e R. I
LAVOUX Jean……
418e R. I.
LEBRUN Fernand
146e R. I.
LECOULEUR Louis
145e R. I.
LECUYER Victor............ 1er colon.
LÉON Aaron-Roger
57e R. I.
LEDUC André-Pierre . . . . 20e R.
I.
.....
......
............
..........
.......
....
....
I.
...... At. Réun.
LEFRIEC Anatole
LÉGLISE Jean-Maurice
33° colon.
.
LESPINASSE Georges . . . 162e R. I.
LIÈS Elie……
344e R. I.
LLADOS Gabriel-Frédéric 212e R. I.
LOUSTAUD Victor
20e R. I.
LUBET Martial.........
49e R. I.
LUCEAU Pierre
6e R. I.
LABORDE Etienne-Roger. 87e R. I.
MADRIGNAC Maurice .... 57e R. I.
MALET Emile……
37e colon.
MANCEAU Pierre-Max . . 12e R. I.
MARCHAIS Joseph
34e R. I.
MAREUGE Eugène
7e colon.
MARCHAND Charles ..... 369e R. I.
MARTIN Léon……
18e train,
MARTIN Edouard-Pierre. 144e R.
MARTIN Emile……
18e train
MARTY Jules……
14e R. I.
MASSAT Jules……
143e terr.
MASSIEUX Edouard
93e R. I.
MAURICE Jean-Louis . . . 58e R.A.C.
MAYSONNAVE Justin
49e R. I.
MAZURIER Raymond
11e infant.
MÉDARD Jacques
3e b.dem.
MERCÉ Bernard ............... 3e colon.
MERCIER Jean……
12e R. I.
......
........
.......
.......
I.
....
.....
...
......
NOMS ET PRÉNOMS
.......
MEUNIER Auguste
MEURIC Louis ........
MEYRIGNAC Pierre
RÉGIMENT
I.
323e R.
24e R. A.
3e zouave
MIAILLES Andre-Albert. 34e R.
MILLAC Jean……
6e R. I.
MILLAC
161e R.
161e
R. I.
MILLAC Guillaume…
MIQUAD André……
214e R. I.
MOLINIER Justin-Jean . . 7e génie
MONTIBUS Eugène
234e R.
MONTRIGNAC Jacques . . 344e R. I.
MORIN Maxime……
144e R.
MORLAAS Jean-Baptiste. 144e R. I.
MORLANNE Pierre
68e R. I.
MORRON Gaston..............
401e R. I.
MOULIN Marcel.............
37e col.
MOUNIER Alphonse ..... 7e col.
.....
I.
I.
......
I.
I.
.......
MURAT Edouard ............... 37e col.
NAVARRI José……
1er r. étr.
NEUGEBAUER Albert ....
344e R.
OLIVE Raoul……
6e col.
OLIVIER Roger-Ernest. . 49e R. I.
PAUNOM Camille
7e R. I. C.
PARICAUD Maurice ........ 42e ch.àp.
PAUDIÈRE Louis
19e chass.
PAGÈS-MENET Daniel
37e colon.
PAUILLAC Pierre
106e R. I.
PAQUETTE Jean-Marie . . 12e R. I.
PASQUET Jean……
56e chass.
PALLIER Jules-Emmanuel 18e C.O.A.
PALLIER Joseph-Henri. . 18e C.O.A.
PETIT Paul-Joseph
9e R. I.
PERSAT Gilbert-Maurice. 402e R. I.
PERROTIN Arnaud
143e R. I.
PENIN Jean-Joseph
100e R. I.
PELLETIER Germain
157e R.
PÉCASTAING Charles . . . . 69e R. I.
PERDRIEL René-Pierre. . 123e R. I.
PHILIPPE René-Marcel. . 6e R. I.
PIGNON Edmond ........... 3e colon.
PIRON Louis……
matelot.
PIQUEMAL Antoine ...... 6e génie.
108e R.
PICART Maurice
108e
R. I.
PICAPER Laurent
340e terr.
POURQUERY Louis
170e R. I.
58e R.A.L.
POITEVIN Mathieu
POMMARÈS Georges
15e R. I.
I.
......
........
....
........
....
......
....
....I.
......I.
......
......
......
.....
POUJOS Elie-André .... 144e R. I.
42e colon.
POULIER Louis ........
POUBLAN Jean……
112e R. I.NOMS ET PRÉNOMS
RÉGIMENT
PRÉVOST Marcel ....... 21e colon.
PRIOU François-Paul . . 123e R. I.
PUJOL Maurice-Georges 49e R. I.
PUJADE Lucien-Henri . . . 172e R. I.
PUCHOIS Alexandre . . . . 120e R. I.
QUEYREIX Pierre
....... 6e R. I.
RABAUD Yvan……
RANCINAN
Léonce
Emile.
RAUFAST Henri
...
RAYMOND Louis.
RÉGNIER Henri ..........
RAPAUD
RICAUD
.
Pierre.
150e R. I.
6e R I.
123e R. I.
22e colon.
22e colon.
7e colon.
144e R. I.
37e colon.
12e R. I.
RIEUX Jules……
ROBERT
ROCHE Raoul……
ROCHEFORT Adolphe . . . 88e R. I.
ROCTOU Jean……
139e terr.
ROGALLE DE PEY Jean . . 418e R. I.
ROBIN Gabriel
18e G. aut
ROQUES Edmond
128e R. I.
ROMAIN Jean
142e col.
ROQUES
7e colon.
ROUCHAUD Henri . ...
54e R. I.
ROULIÈRE Pierre
344e R. I.
ROUMÉGOUX Jean ........ 24e R. A.
ROUSSEAU Dominique . . 30e R. I.
Roger...
...........
...
Jean……
........
Roux Joseph……
ROUZIÈS Léon.
ROYER Emile……
RUA Louis……
SABATIER Albert
SABOURDIN Jean
18e R. I.
.......
......
DE SANDOVAL André
.....
........
SAINTOUT Jean
SAINT-MACARY Jean
SCHWERER Eugène
.....
.....
Baptiste .....
SERNIN Antoine ........
SENTENAC
20e R. I.
9e R. col.
57e R. I.
144e R. I.
2e R. A C
176e R. I.
9e R. I.
49e R. I.
83e R. I.
144e col.
58e R. A.
NOMS ET PRÉNOMS
RÉGIMENT
4e R.I.C.
Pierre.
............ 9e tirail.
SERNIN
SIRVEN Pierre
SISSOKO Pierre
SOUPÈNE Adolphe
SOUM Dominique
......... 53e R. I.
........ 6e R. I.
....... 60e R. I.
1er zouav.
158e R. I.
SABATIÉ Théodore
THIBAUDEAU Paul ......... 2e génie.
344e R. I.
TAILLACOT Jean
346e R. I.
TALUT Pierre ...........
7e R. C.
TARRIDE Roger ............ 1er zouav.
150e R. I.
TEMPLIER Mathieu . . . . 71e R. I.
84e R. I.
TESSIER Henri
TEXIER Maurice.............. 37e colon.
TEYSSIER Pierre ........ 162e R. I
220e R. I.
THÉBAUD Victor
138e R. I.
TISNÉS Jean……
224e RAC.
TRÉMOULET Pierre . . . . 30e b. alp.
....
........
......
......
6e R. I.
2e tirail.
57e R. I.
7e colon.
1er étrang.
3e zouav.
96e terrill.
220e R. I.
. . . 88e R. I.
. . . 71e R. I.
144e R. I.
122e R. I.
140e col.
265e R. I.
7e colonial
76e col.
68e R. I.
TSCHEK Charles
TURBET DELOFF Jean
.
VALLET Georges
......
VAUDUN Adolphe .......
.......
VAREILLE Pierre
VERGINE Eloi.
VERGNAUD Louis
VERGNOLLE Georges
VERNAUDON Georges
VERNET Antoine
VIDEAU Fernand
......
......
.......
VERGINE Eugène
VIGUIER-Elie.
VERNIS
.......
-—TABLE DES GRAVURES
18
34
60
84
Le château La Mission Haut-Brion.
Le Jardin botanique
La mairie de
Talence……
L'aqueduc romain
100
130
158
174
180
192
215TABLE DES MATIÈRES
LISTE DES SOUSCRIPTEURS.. . . . . . . . . . . .. . . . . . . . . . . .. . . . . . .
AVANT-PROPOS
CHAPITRE
jours
.................................................................................
I. — Des origines de Talence jusqu'à nos
II. — Châteaux historiques
CHAPITRE III.
— Les villas célèbres
CHAPITRE IV.
— Autres domaines remarquables
CHAPITRE
V. — Les édifices religieux
CHAPITRE VI.
— Le vin. — Événements divers. — Lieux
de plaisir
CHAPITRE VII.
— Talence de 1789 à 1823
CHAPITRE VIII. — La Vie intellectuelle
CHAPITRE IX.
Archéologie……
Viographie.
—
—
CHAPITRE
X. — En l'honneur des morts glorieux
CHAPITRE
TABLE DES GRAVURES. . . . . . . . . . . . . . . . .
.
9
15
19
35
61
85
105
131
159
175
193
219
229
ERRATA
Page 115, 4e alinéa. — Au lieu de : chemin ouvert par des marais,
lire : chemin ouvert dans des marais.
Page 125, note 80. — Au lieu de : voir page 51, lire :
10.467. — Bordeaux. — Impr. GOUNOUILHOU, rue Guiraude, 9-11,
—
1926.


« Guilhaumes Marcelha s'en anaba enta Talamon » (s'en
alla vers Talmont), est-il écrit dans les registres de la ju-
rade (séance du 19 août 1420)
En 1451, le sire d'Orval, fils aîné du sire d'Albret, « en-
nemy capital des Anglais », se trouvant dans le Médoc avec
500 hommes, apprit que le maire de Bordeaux et ses gens
(dont 9.000 hommes de pied, Bordelais et Anglais) vou-
laient venir le charger. Le sire d'Orval accourut avec sa pe-
tite armée et livra combat aux troupes nombreuses du
maire de Bordeaux. Cette bataille se déroula « ou près de
Talence, ou de Bègles ou dans les landes voisines du Hail-
lan » 30.
Le sire d'Orval remporta la victoire. Ses hommes tue-
rent 1.600 Anglais, firent 1.200 prisonniers et contraigni-
rent les autres à s'enfuir. Le maire de Bordeaux était par-
mi les fugitifs.
Avaient collaboré, aux côtés du sire d'Orval, au succès
27. T. IV, p. 435.
28. Registre de la Jurade, t. IV, p. 627.
29. Ibid.
30. Bernard DE GIRARD, seigneur du Haillan. Histoire générale des
roys de France, p. 1074 et 1075.
de la journée : Etienne de Vignolles, Robin Petit Escossois
et de l'Espinasse « brave et ancien capitaine » ".
Le 27 novembre 1499, Petro de Puilha, de Saint-Genès,
signa dans un contrat « pour la façon d'une cloche pour
l'église de Saint-Ciers-la-Lande » 32.
Dans la paroisse Saint-Genès de Talence habitait un
apothicaire nommé Pierroton de Sabarots, qui fit son tes-
tament le 12 mai 1552. Résumons les principales clauses
du document notarié :
Pierroton de Sabarots ordonne que, après son décès,
son corps demeure sur la terre vingt-quatre heures ; il
lègue 25 livres aux Augustins pour la réparation de leur
église et 4 francs «aux pauvres ladres de Goullée ». Il
laisse à sa femme divers avantages « pourveu qu'elle ne se
marie avecques homme de court ne de robbe longue » ".
\
Jean II du Bernet, seigneur de Talence et de Savignac,
fut nommé « conseiller à la Cour de Parlement de Bor-
deaux » par lettres patentes datées du 25 juin 1585, « si-
gnées HENRY, et par le roi FORGET »54.
Sur le chemin Pey-Davant, en face de l'allée qui menait
à l'ancienne chapelle Notre-Dame de Rama, était un bien
31. Bernard DE GIRARD, seigneur du Haillan. Histoire générale des
roys de Franee, p. 1074 et 1075.
32. Archives historiques de la Gironde, t. LII.
33. Inventaire sommaire des Archives historiques de la Gironde, sé-
rie G, t. II.
34. BOURROUSSE DE LAFFORE. Nobiliaire de Guienne et de Gascogne,
t. III, p. 348.
qui a appartenu à Mme de Lartigue, belle-mère de Montes-
quieu 35.
Il s'agit du domaine de Momplaisir. M. de Lartigue était,
en outre, propriétaire d'une vigne qui entourait la chapelle
Notre-Dame de Rama.
De leur côté, les Montesquieu avaient une maison noble
à Talence.
Bourrousse de Laffore nous apprend que « Jean-Bap-
tiste de Secondat, écuyer, baron de Montesquieu, seigneur
de Castelnouvel, Talence et Raymond », président à mor-
tier au Parlement de Bordeaux, se démit de sa charge en
faveur de son neveu, l'illustre Montesquieu, qu'il institua
son héritier universel 36. Vraisemblablement le célèbre pu-
bliciste est allé à Talence où il avait des intérêts. Peut-être
même y a-t-il médité les pages d'un de ses admirables livres
dont on a dit qu'ils vivront autant que la langue française?
Le maréchal duc de Richelieu, nommé gouverneur de
la Guienne, fit son entrée à Bordeaux le 4 juin 1758. Le
26 septembre de la même année, revenant de faire un
voyage à Bayonne, il s'arrêta sur le chemin de Talence
et se rendit « dans la maison de Messieurs Agard, négo-
ciants de Bordeaux »37, où une fête des plus brillantes
fut donnée en son honneur.
Le maréchal n'arriva chez lui que « le lendemain matin
35. Notre-Dame de Talence dans la chapelle des Monges au XVIII°
siècle.
36. Nobiliaire de Guienne et de Gascogne, t. II, p. 258.
L'oncle de Montesquieu mourut le 29 juin 1716.
37. François DE LAMONTAIGNE. Chronique bordelaise.
Le 6 novembre 1762, les deux frères Pierre et Pierre-Jacques Agard
représentèrent les lettres de. bourgeoisie du sieur Jacques Agard, leur
père, du 17 août 1720. (Livre des bourgeois de Bordeaux.)
à quatre heures » 38. Il habitait l'hôtel du Gouvernement 39,
rue Porte-Dijeaux. Comme il n'y a pas une grande distance
entre Talence et cet hôtel, on peut en conclure que la
soirée dut se prolonger singulièrement dans la maison de
MM. Agard.
Le 12 février 1765 eut lieu, dans l'église Saint-Genès de
Talence, le mariage de Pierre Agard, ancien consul de
la Bourse et ancien directeur de la Chambre de commerce
de Guienne, avec Marie-Rose Drouet. Etaient présents :
Pierre Ozée Dublan, procureur au bureau des finances,
jurat de Bordeaux ; Isaac Couturier, écuyer ; Jacques Le-
grix fils, trésorier de France ; Charles Agard, chanoine10
On peut lire dans une Chronique bordelaise, rédigée de
1735 à 1759 par un personnage anonyme du collège des
Jésuites de Bordeaux, les lignes suivantes consacrées à
un jeune Talençais qui ne pouvait évidemment s'attendre
à ce qu'on s'occupât de lui après son trépas :
« Le 21 septembre 1759, décéda à l'hôpital Saint-Jacques
de Bordeaux, Helies, domestique infirmier du collège,
âgé d'environ 20 ans, natif de la parroisse de Talance, du
village de Peylane. Le landemain, on l'enterra dans le
cimetière dudit hôpital, le R. P. Dupin offician, en pré-
sence des Frères Coutin, Brisseau, Latapy, Martial et Les-
parre, des domestiques du collège et de tous ceux qui
voulurent venir à l'enterement. On ne sonna la petite cloche
de nostre église que pendant l'office et l'entairement 41. »
38. François DE LAMONTAIGNE. Chronique bordelaise.
39. Cet hôtel a servi d'archevêché; c'est maintenant la résidence par-
ticulière du préfet.
40. G G, 22. (Archives communales.)
41. Fonds des Jésuites, carton 39. Cette chronique a été transcrite
pour les Archives historiques par M. Maurice Birot.
En nivôse an IX, un double crime fut commis à Talence.
Ce forfait est ainsi rapporté dans la Gazette nationale 42.
« Bordeaux, 16 nivôse an IX.
» Il est douloureux d'avoir à faire connaître des crimes
qui font frémir l'humanité ; mais la sûreté publique nous
fait un devoir d'annoncer que dans la nuit du 13 au 14
de ce mois, un vigneron de la commune de Talence a tué
dans leurs lits sa mêre et son frère plongés dans le som-
meil. Le citoyen Bonnefou, maire de cette commune, qui
nous transmet cet événement, nous invite à publier le signa-
lement du meurtrier afin qu'il puisse être reconnu et
arrêté partout où on le rencontrera. Il se nomme Jean sur-
nommé Jambon, âgé d'environ trente-trois ans ; il est bos-
su ; sa taille est au-dessous de cinq pieds ; il est assez mal
vêtu ; il porte une veste bleue rapiécée et une culotte de
cotonille rayée43. »
On ne dit pas si ce bossu fut retrouvé.
C'est sur le territoire de Talence que furent capturés, le
11 juillet 1925, trois des bandits espagnols auteurs de l'atta-
que à main armée de la fabrique de meubles Harribey,
cours du Maréchal-Galliéni. Cette affaire eut un tel reten-
tissement qu'il importe d'en rappeler les principales phases.
42. C'était le Journal officiel d'autrefois.
43. Gazette nationale ou le Moniteur universel, n° du 24 nivôse an IX,
p. 464.
Les bandits espagnols étaient quatre : Aznar, dit le « Ne-
gro », Recassens, Benito Castro et Cazals.
Le 11 juillet 1925, à onze heures du matin, les quatre
malfaiteurs pénètrent dans la cour de la maison Harribey.
Revolver au poing, ils montent au premier étage, où se
trouvent les bureaux et où ils pensent pouvoir s'emparer
d'une somme considérable représentant le montant des sa-
laires du personnel ouvrier. Arrivés dans les bureaux, ils
font usage de leurs armes, tirant à tort et à travers. Ils
blessent d'abord grièvement un contremaître, M. Lahousse
— qui fut par la suite amputé d'une jambe —; ils criblent
de balles le caissier M. Gimont père et M. Roger Gimont
fils, lequel tombe pour ne plus se relever. Mais l'alarme est
donnée. Des ouvriers accourent. Les assassins se retirent
sans butin. L'argent qu'ils avaient convoité ne devait
être apporté du reste qu'un peu plus tard dans les bureaux.
Alors c'est la retraite précipitée des bandits par le chemin
de la Médoquine. Ils font feu sur le garde champêtre Lie-
bot, de Talence, et deux employés de la Compagnie du
Midi qui voulaient leur barrer la route. Un de ces employés,
M. Bordier, est tué par un des projectiles; le garde cham-
pêtre est blessé au flanc gauche.
Poursuivant leur course, les misérables traversent la
ligne du chemin de fer, à 150 mètres de la station de la
Médoquine, et trois d'entre eux sont capturés peu après :
Recassens, Benito Castro et Cazals. Le quatrième, « le Ne-
gro », est resté introuvable.
Les renseignements fournis sur Recassens et Benito Cas-
tro attestèrent que ces deux individus étaient des plus dan-
gereux. Recassens, âgé de vingt-sept ans, originaire
d'Omillo, province de Lérida, avait exercé dans son pays
la profession de boulanger. Il avait été interné pendant deux
ans aux Baléares comme terroriste. Depuis, il était recher-
ché par la police espagnole pour avoir pris part à Barce¬
lone, et sur divers autres points du territoire, à de nom-
breux attentats. Il avait notamment dirigé l'attaque à main
armée d'un train transportant de l'argent et avait abattu
d'un coup de revolver un employé qui essayait de lui ré-
sister.
Castro Benito était lui aussi un repris de justice. Né le
1er octobre 1892 à Villafranca (Espagne), il était boulanger
de son état. A vingt ans, il quitta sa famille. Venu en Fran-
ce, il fut condamné pour vol, d'abord à Bordeaux, ensuite à
Reims. Lors de sa première condamnation, il avait été ex-
pulsé, mais était revenu dans notre pays, après s'être
muni de faux papiers.
Les trois bandits comparurent devant la cour d'assises
de la Gironde les 30 et 31 octobre 1925. Recassens et Benito
Castro furent condamnés à mort; Cazals s'en tira avec les
travaux forcés à perpétuité.
Les deux « pistoleros » ont subi la peine capitale le
14 janvier 1926, dans la cour du fort du Hâ. Benito Castro
à été exécuté le premier ; le cou dans la lunette, il a crié en
espagnol : « Vive l'anarchie ! » Trois minutes plus tard,
Recassens était amené devant la guillotine et décapité à son
tour. Il n'eut pas le temps de pousser le même cri que son
complice.
Les obsèques de M. Bordier, victime de son dévouement,
puisqu'il était tombé en voulant aider à capturer les cri-
minels espagnols, furent l'occasion d'une émouvante so-
lennité. M. Bordier habitait Talence. Toute la population de
la commune s'associa à l'hommage suprême qui fut rendu
à sa dépouille mortelle.
Devant la tombe, M. Georges Lasserre, maire de Ta-
lence, prononça l'éloge du défunt. M. Bordier avait donné
un noble exemple du devoir civique. Un coeur d'élite battait
dans la poitrine de cet enfant du peuple. Nul ne devait
l'oublier.




\chapter{LES ÉDIFICES RELIGIEUX}

\section{L'ancienne église Saint-Genès de Talence}
La paroisse Saint-Genès-de-Talence dépendait de l'archiprêtré de Cernès. Le chef-lieu de cet archiprêtré était la paroisse de Saint-Pierre-de-Gradignan\footnote{Le curé de Gradignan avait le titre d'archiprêtre de Cernès.}.

La paroisse Saint-Genès-de-Talence comprenait, outre l'église matrice : la chapelle Notre-Dame-de-la-Rame, le prieuré de Bardenac et la chapelle Saint-Pierre.

L'église matrice Saint-Genès-de-Talence — Gleysa Sent Genest — était située « à l'ouest du chemin de Saint-Genès\footnote{Rue de Saint-Genès.}, entre la rue des Treuils et les maisons qui sont en face de la rue Duluc, autrefois rue Vigneron »\footnote{Léo DROUYN. \textit{Bordeaux vers 1450}.\\
	Selon Charles Chauliac, l'église Saint-Genès, dont on voyait encore les ruines au commencement du XIXe siècle, était construite « sur l'emplacement actuel de la maison de la rue Saint-Genès portant le numéro 203, entre les rues Solférino et de Ségur ». (\textit{Étude sur les croix des carrefours de Bordeaux}, 1881.)} . Sa position à cet endroit est certifiée par un « plan géométrique de partie du plantier de Saint-Genez »\footnote{Plan n° 1250. (Archives départementales.)}. Pierrugues, sur son plan de 1819, la place sur le même point, après la rue de Ségur, en face de la « rue du Luc ». Un chemin la séparait du territoire de Sainte-Eulalie.

L'église Saint-Genès existait au XIII\ieme{} siècle, comme l'atteste un acte de 1287, par lequel Amanieu d'Escure donna aux religieuses de Sainte-Claire « une rente sur des vignes situées entre Saint-Nicolas et Saint-Genès\footnote{Léo DROUYN. \textit{Bordeaux vers 1450}.} ». Elle était « d'une passable grandeur, mais pas assés considérable pour le nombre des parroissiens »\footnote{Abbé BAUREIN. \textit{Questionnaire}.}. Sa structure n'offrait rien de remarquable.

Il y avait deux autels collatéraux : l'un était dédié à la Sainte Vierge, l'autre à Saint Barthelemi. Il y avait deux clochers, un petit, à côté du choeur\footnote{Ce petit clocher menaçait « ruine prochaine » en 1788. (Cahier de visite, G. 646. (Archives départementales.)}, et un plus grand.

L'église avait naturellement son cimetière, dont il est fait mention dans des actes de 1341\footnote{\textit{Société archéologique de Bordeaux}, t. XXXIX.}.

À la fin de l'année 1545, ou tout au début de 1546, le Parlement avait rendu un arrêt condamnant plusieurs locataires d'une maison située rue « Bouau » à payer à Guilhem de la Ville et Pierre du Hazera, « scindics des fabriqueurs de l'église parroissiale de Saint-Genès-de-Talence, la somme de soixante dix livres dix sols cinq deniers parisis ». Les locataires, parmi lesquels était Gratienne Mosson, devaient, en outre, évacuer l'immeuble.

Le 12 février 1546, un huissier fut chargé de faire exécuter l'arrêt de la cour. Cet huissier, accompagné de Guilhem de la Ville et Pierre du Hazera, se transporta le mercredi 6 avril 1546 « avant Pasques », dans la maison en question et parlant aux locataires leur demanda « pourquoi ils n'avaient pas vuidé »\footnote{Archives départementales, G, 3146.}.

En 1677, Sainct-Aignan, prêtre, vicaire perpétuel de Saint-Genès-de-Talence, enregistra un bref portant concession d'indulgences à ladite paroisse et daté du 19 octobre 1676\footnote{Actes pontificaux. Série G, 525. (Archives départementales.)}.

Suivant une commission de Louis d'Anglure de Bourlemont, archevêque de Bordeaux, primat d'Aquitaine, en date du 19 juin 1691, Henry Chapotel, prêtre et archiprêtre de Cernès, curé de Gradignan, visita, le 23 septembre 1691, l'église Saint-Genès-de-Talence et la chapelle SaintPierre; il rendit compte de l'état de ces deux édifices, ainsi que du mobilier qu'ils contenaient\footnote{G. 646. Archives départementales.}.

Ayant lu le procès-verbal de visite de M. Chapotel, l'archevêque de Bordeaux décida, le 1er juin 1693, en ce qui concerne Saint-Genès-de-Talence, que les « pierres sacrées des autels de Notre-Dame et de Saint-Barthelemy seraient avancées de quatre doigts sur le devant desdits autels ».

D'autre part, le cimetière « tourné du costé du nord » serait fermé d'une muraille, et l'on placerait des grilles aux deux entrées\footnote{\textit{Ibid}.}.

Du 1\ier{} décembre 1766 au 6 décembre 1767 « le grand ouvrié de l'église Saint-Genès-de-Talance » — ou trésorier de la fabrique — fut André Dubos, dit « Baron », qui eut pour successeur Pierre Prévôt\footnote{Livre des comptes de la paroisse Saint-Genès de Talance 1730 à 1793, folio 153. (Archives départementales.)}.

André Dubos rendit compte « de la recette et depance par luy faite en laditte calitté, en présence et du consentement de messire de Gourgue de Touars, présidant en la première chambre des enquêtes du Parlement de Bordeaux, et syndic honoraire de la paroisse » ; de M. Fourtin, curé de Saint-Genès; Jean Perrens, syndic honoraire, et des propriétaires et habitants de la commune « convoqués, et assemblés au son de la cloche en la manière accoutumée »\footnote{\textit{Ibid}.}.

Le 13 janvier 1788, Simon Langoiran, vicaire général du diocèse, visita l'église Saint-Genès. On relève sur son procès-verbal\footnote{Cahier de visite G. 646. (Archives départementales.)} les observations suivantes :

Registre des baptêmes, mariages et sépultures : « Ils sont très mal écrits parce que M. le curé veut transcrire tous les actes, et que sa main tremblante ne lui permet pas de le faire comme il serait nécessaire\footnote{Il s'agissait de l'abbé Fortin, « écrivain de plusieurs documents paroissiaux si difficiles à déchiffrer, même de son temps, paraît-il, que l'archevêque de Bordeaux fil paraître une ordonnance pour lui interdire la rédaction des registres de sa paroisse... ce qui fut d'ailleurs la cause qu'ils ne furent plus tenus depuis cette époque ».\\C'est la \textit{Société archéologique de Bordeaux} (t. XXXIX), qui souligne ce détail, inexact du reste, car les archives communales possèdent le registre des mariages, naissances et décès de la paroisse Saint-Genès-de-Talence allant de 1786 à 1792 (G G, 26, in-4°, 184 feuillets).}. »

Cimetière : « En bon état, à la réserve de vingt brasses de mur du côté du couchant qu'il faudrait, faire. »

Reste net, charges acquittées : « Les charges excèdent les revenus ».

Le même procès-verbal fait savoir qu'il n'y avait point de maître d'école, de maîtresse d'école, de sage-femme, d'hôpitaux, de monastères d'hommes et de filles, de « commanderies et lieux en dependans ».

Enfin, on lit sur le même document :

« Chapelles publiques hors de l'église : il y en a une à Bardanac où se rendent le jour de Saint-Marc les processions de Talence, Gradignan et Pessac ».

« Objets et fondations : Il y a trois fondations pour les pauvres : la première de 12 livres de rente laissée par M. Tolan, ancien curé de Talence ; la seconde de 1.200 l. de capital, laissée par M. Dufau, grand chantre, laquelle somme produit une rente de 60 livres, exactement payée ; la troisième de 1.200 l. de capital, donnée par Mme la Présidente Fossier ; cette rente est « arriérée », en sorte qu'il est dû 400 l.

» Il y a aussi : 1° une messe haute pour M. Tolland, ancien curé de Talence ; 2e six messes basses dont l'honoraire n'a pas été payé depuis plus de cinquante ans, quoiqu'il y ait des fonds existants sur lesquels ledit honoraire est établi ; 3° une messe pour Mme de Chauvignac ; 4° une pour Me Petit, bourgeois ; 5° une autre dont l'honoraire est payé par Me Bobérat. »

Un historien, l'abbé Baurein, fut vicaire à Talence pendant deux années. Les actes du registre des baptêmes, mariages et sépultures de la paroisse Saint-Genès-de-Talence, durant les années 1738 et 1739 sont, en effet, signés « Baurein, vicaire »\footnote{L'abbé Baurein, originaire de Bordeaux, y est mort en mai 1790, rue du Hâ. On lui doit les Variétés bordeloises, ouvrage en plusieurs volumes publiés de 1784 à 1786.}.

Le 28 pluviôse an II, la municipalité et les membres du Conseil de la commune de Talence arrêtent, « en conséquence d'une lettre des administrateurs de Bordeaux du 22 du même mois, qu'il sera fait état des biens nationaux provenant de diverses suppressions des corps et communautés, et qui demeurent invendus »\footnote{Registre des procès-verbaux de la municipalité de Talence. (Archives de la commune.)}. En tête de la liste de ces biens figure le lot suivant : « Eglise Saint-Genès, cimetière, maison attenante, cour, jardin et vignes, estimé 20.000 livres »\footnote{\textit{Ibid}.}.

Il y eut une autre estimation du même lot, le 18 pluviôse an III ; elle atteignait celle-ci 20.250 livres\footnote{M. MARION, J. BENZACAR, CAUDRILLIER : \textit{Documents relatifs à la vente des biens nationaux}, t. I\ier{}.}.

Le 12 messidor an III, M. Rolland, négociant rue Baut, 273, fut déclaré adjudicataire, pour la somme de 110.000 livres, de l'église Saint-Genès, du cimetière, d'une maison attenante au cimetière et de 3 journaux, 27 règes, le tout appartenant à la fabrique\footnote{M. MARION, J. BENZACAR, CAUDRILLIER : \textit{Documents relatifs à la vente des biens nationaux}, t. I\ier{}.}. M. Rolland reçut les clés de l'église le 15 thermidor an III.

Un autre bien de la même fabrique (28 r. vigne à Talence) mis aux enchères le 12 messidor an III et estimé 1.000 livres, fut adjugé à M. Sandre, rentier, 44, place Nationale, pour 1.600 livres.

Le vocable de l'église Saint-Genès rappelait soit saint Genès qui fut histrion à Rome ; soit saint Genès d'Arles\footnote{Le Bordelais Paulin, évêque de Nole, auteur de \textit{Lettres} et \textit{poésies latines}, a écrit aussi une \textit{Histoire du martyre de saint Genès d'Arles}.}, greffier public à Arles, au quatrième siècle; soit encore saint Genès qui fut évêque de Clermont.

Le premier, l'histrion, jouait un jour devant Dioclétien une parodie indécente des cérémonies du christianisme, lorsqu'il fut frappé d'une vision intérieure et se convertit à la foi nouvelle. Il subit le supplice du chevalet en 286 ou 303.

Saint Genès, le greffier, avait été chargé de transcrire un édit de persécution contre les chrétiens, signé à Arles par Maximien. Il s'y était refusé, en sa qualité de catéchumène et avait pris la fuite. Arrêté, il fut décapité. L'église célèbre le 25 août la fête de saint Genès.

Cette fête concerne les deux martyrs : l'histrion romain et le greffier d'Arles.

Quant au troisième saint Genès, l'évêque de Clermont, qui mourut vers 662, sa fête a lieu le 3 juin.

\section{La croix de Saint-Genès}

Elle se dressait au carrefour formé par la jonction de la route de Bayonne (cours de l'Argonne) et de la rue de Saint-Genès\footnote{Ces deux voies s'appelaient, vers le milieu du dix-huitième siècle : « Chemin royal de Bordeaux à Bayonne » et « chemin de la porte SainteEulalie à Saint-Genez ». Plan n° 1250. (Archives départementales.)}.

M. Chauliac pense que la première croix placée au carrefour était probablement celle du cimetière de Saint-Genès\footnote{\textit{Étude sur les croix des carrefours de Bordeaux}.}.

À notre avis, cette croix n'avait rien de commun avec le cimetière qui, bordant l'église Saint-Genès, se trouvait, par conséquent, à une petite distance du carrefour. La croix de Saint-Genès était une des croix qui jalonnaient le chemin de Saint-Jacques (aujourd'hui cours de l'Argonne dans une partie de sa traversée de Bordeaux, et cours Gambetta dans sa traversée de Talence).

Dans un texte de l'an 1400, il est fait mention d'une nouvelle croix à Saint-Genès, sur le grand chemin de Saint-Jacques\footnote{Léo DROUYN. \textit{Bordeaux vers 1450}.}. On observera qu'il s'agit d'une « nouvelle croix » ; ainsi il y en avait eu une autre, précédemment, sur le même chemin.

La croix de Saint-Genès menaçait ruine à la fin du XVIe siècle ; on la reconstruisit en 1606.

Un contrat fut passé le 22 mai 1606 entre, d'une part. Nicolas Dupuy et Arnaud de Lagrave, laboureurs et tous deux fabriciens de la paroisse Saint-Genès, et, d'autre part, Pierre de Coustances, maître-maçon. Le contrat stipulait que la croix devait être élevée sur cinq degrés « de bonne pierre dure de Bouchet ou de Rauzan ».

Pendant la Révolution, le croisillon fut détruit et le chapiteau mutilé ; une croix en fonte ouvragée remplaça plus tard la croix de pierre.

Cette réparation fut peut-être l'oeuvre du citoyen Babot, qui écrivait le 23 nivôse an IX : 

« Je déclare avoir racommodés la Crois de St-jenès par ordre du citoyen Bonnefous et qu'il m'a payer ledit ouvrage dont je te tiens quitte\footnote{Registre des délibérations de la municipalité de Talence. (Archives de la commune.)}. »

En 1902, la croix de Saint-Genès fut portée dans un petit enclos voisin, aménagé dans le domaine de la Solitude de Nazareth ; on substitua alors à la croix en fonte une croix en pierre.

Entouré de murs sur trois côtés, le monument est protégé sur le quatrième côté — celui regardant la place SaintGenès — par une grille en fer. Il présente une colonne hexagonale. A mi-hauteur est sculptée une statuette de saint, la tête nimbée. Cette figure est apparemment celle de l'histrion de Rome, ou celle du greffier d'Arles, ou bien encore celle de saint Genès, l'évêque de Clermont.

Ce pieux personnage porte une longue barbe et est coiffé d'un chaperon. Il tient à la main une banderole avec inscription devenue illisible.

La partie supérieure du fut est ornée de quatre têtes d'anges formant chapiteau. Entre les têtes existent de petits cartouches sur deux desquels on lit le monogramme du Christ. Le millésime 1606 est également gravé sur un des cartouches; il surplombe la statuette de saint Genès\footnote{Pour certains, celle figure serait tout simplement celle de saint Jacques.}.

On conserve aux Archives municipales de Bordeaux une remarquable aquarelle d'un artiste bordelais, M. Edmond Fontan, représentant la croix de Saint-Genès en 1900, par conséquent à l'époque où elle se dressait encore sur son emplacement primitif.

\section{Notre-Dame de Rama}

%%%% la note de cette page renvoie à une page à mettre à jour
Suivant une tradition, il y eut une apparition de la sainte Vierge à Talence. Pour commémorer cet événement, une chapelle consacrée « à la Vierge Marie », fut fondée vers 1130 dans le parc Parthenval\footnote{Voir page 90.}, près du ruisseau des Palanquettes. On l'appelait Notre-Dame de Rama ou de la Rame, nom qui lui venait de sa situation dans les bois, sous une ramée, couvert formé de branches entrelacées.

Éléonore d'Aquitaine, fille de Guillaume X, dernier duc d'Aquitaine et comte de Poitiers, paraît avoir créé cette chapelle en 1132\footnote{\textit{Société archéologique de Bordeaux}, t. XXII, p. 3. — BORDES. \textit{Histoire des monuments de Bordeaux}.\\Éléonore d'Aquitaine, reine de France, puis reine d'Angleterre, étant née en 1122, n'avait donc que dix ans quand elle fit élever cet édifice religieux.}.

Notre-Dame de Rama attira tout de suite les pèlerins. Les premières gardiennes de ce saint lieu furent des « Fontevristes », religieuses dépendant de l'abbaye de Fontevrault (Maine-et-Loire). Leur couvent « était placé à une soixantaine de mètres de la chapelle »\footnote{\textit{La Voix de Notre-Dame-de-Talence}, février 1913.}.

Éléonore d'Aquitaine finit ses jours à l'abbaye de Fontevrault le « 31 mai 1204 », si l'on en croit l'inscription gravée sur le socle de son tombeau dans ladite abbaye.

Le souvenir de cette princesse était fidèlement gardé par les religieuses de la même communauté.

« Le nécrologe de Fontevrault contient, en effet, un éloge d'Alienor plutôt exagéré, et les moniales\footnote{Moniales se traduit dans notre langue par \textit{monges} ou \textit{religieuses}.} de Talence devaient le lire tous les ans, au jour anniversaire de sa mort, en se rappelant que la reconnaissance leur faisait un devoir de taire les infidélités de leur bienfaitrice et d'en dire tout le bien que les chroniqueurs oubliaient de raconter\footnote{\textit{L'Aquitaine}, n° 23, 6 juin 1913.}. »

Les Fontevristes quittèrent le couvent du parc de Parthenval avant la fin du XIIIe siècle, « n'y laissant qu'un prieur »\footnote{Dom Reginald BIRON. \textit{Précis de l'histoire religieuse des anciens diocèses de Bordeaux et Bazas}.}. 

Il y avait dans la chapelle de Rama une statue de Marie\footnote{La main de la Vierge ayant été brisée, on l'a modelée en plâtre en lui donnant six doigts. Il existe, paraît-il, d'autres statues de la Vierge avec une main « sexdigitale ».}, tenant le Christ sur ses genoux.

« Une grande affluence de peuple s'y rendait pour honorer la mère de Dieu, dont la tradition publiait l'apparition miraculeuse en cet endroit\footnote{\textit{Dominicale bordelaise}, t. I\ier{}, 9 octobre 1836.}. »

La petite église fut endommagée par les Anglais, et on cessa, pendant la guerre de Cent ans, d'y célébrer le culte\footnote{\textit{Notre-Dame-de-Talence dans la chapelle des Monges au XVIII\ieme{} siècle}.}.

Pendant une période qui dut être assez longue, mais qu'on ne peut exactement déterminer, on n'alla plus en pèlerinage à Notre-Dame de Rama, et le pieux asile, qui avait été témoin des innocentes distractions d'Eléonore d'Aquitaine, « au jour de sa plus tendre jeunesse »\footnote{\textit{L'Aquitaine}, n° 23. 6 juin 1913.}, fut complètement abandonné.

M. Henry Chapotel, « prestre et archiprestre de Cernès », curé de Gradignan, visitant le 23 septembre 1691 les édifices religieux de Talence, se transporta dans la « chapelle ruralle de Notre-Dame de la Rame, vulgairement appellée des Monges, dépendant de l'abbaye de Fontevraux » et la trouva « entièrement ruinée, sans toit et sans porte, tout étant entièrement par terre »\footnote{G, 646. (Archives départementales.)}.

Il est vraisemblable que la chapelle avait été détruite durant une guerre, peut-être celle de la Fronde.

M. Chapotel s'enquit des revenus de Notre-Dame de la Rame. Ils consistaient « en rentes et agrières affermées la somme de cent livres »\footnote{\textit{Ibid}.}.

Le 6 janvier 1730, jour des Rois, des enfants jouant dans les ruines de l'antique chapelle trouvèrent parmi les pierres à demi recouvertes de ronces la statue de la Vierge. Les paroissiens s'empressèrent, dès lors, de « faire rétablir la susdite chapelle et exposer ladite image à la vénération des foules»\footnote{\textit{Chronique bordelaise de 1638 à 1736}. (Archives du château de Caila, appartenant à Mme la comtesse de Galard.) Le document a été transcrit pour les \textit{Archives historiques} par M. Émile Thomas.}.

Le vendredi 12 octobre 1731, la chapelle Notre-Dame de Rama, réparée et ornée\footnote{On y avait peint les armes du roi, suivant les ordres du père Lechat, qui avait été, à cette époque, envoyé en mission aux Monges de Talence par l'abbesse de Fontevrault.} fut bénie par le curé de Léognan, par ordre de Mgr Maniban, archevêque de Bordeaux, et le même jour, on commença à y dire des messes.

Au cours des travaux de reconstruction, on avait trouvé des tombeaux et beaucoup d'ossements dans le terrain, autour de la chapelle, et sur le sol même de cet édifice. En 1730, on voyait encore des restes du couvent de la Rame : des « dessus de croisées » montrant des peintures avec des fleurs de lys.

Un habitant de Bordeaux voulant se rendre, au XVIII\ieme{} siècle, à la chapelle empruntait lé « grand chemin de Bordeaux à Bayonne », et tournait peu après à gauche dans une voie ainsi désignée : « chemin de Bordeaux au Bourdieu de la porte, à la chappelle de N.-D. de Talence et à Agès\footnote{Plan 138. (Archives départementales.)}. » On quittait cette artère pour en prendre une autre — le chemin de Pey-Davant actuel — d'où partait un petit chemin conduisant directement à la chapelle.

On pouvait arriver à la chapelle par le côté opposé, en suivant un chemin ouvert par des marais.

Suivant le plan de 1782, le prieuré de la Rame possédait à cette époque plusieurs fiefs dans le voisinage du lieu où il selevait. L'un de ces biens est ainsi mentionné : « Moulin, étang, ruisseaux, pré et vinière, contenant 1 journal, 19 règes, 4 carreaux, fief du prieuré de la Rame appartenant aux dames religieuses de Fontevreau, formant l'article 1er de la reconnaissance consentie en leur faveur par...\footnote{Plan 138. (Archives départementales.)} ». 

Le moulin dont il est question au début du paragraphe n'était autre que l'ancien moulin de la Lande.

Parmi les personnes présentes à un mariage célébré le 24 janvier 1788, en l'église Saint-Genès, on trouve JeanBaptiste de Roboam, « écuyer, prieur et chapelain de Notre-Dame-de-Talence ».

La Révolution arriva. Le sanctuaire de Notre-Dame-deTalence fut profané. On jeta dans le fossé des Palanquettes la statue de la Vierge. Cette statue, recueillie par de pieux habitants de la paroisse, MM. Castaing, Barron et Mouliney\footnote{\textit{Notre-Dame de Talence dans la chapelle des Monges}.}, fut cachée dans un caveau du presbytère de la chapelle Saint-Pierre.

Le deuxième lot des biens nationaux à Talence, à savoir : « la chapelle Notre-Dame-de-Talence, maison, jardin, vignes et quelques arbres et place vuide vers le couchant »\footnote{Registre des procès-verbaux de la municipalité de Talence. (Archives de la commune.)} fut estimé 4.000 l. et vendu, le 12 messidor an III, à M. Thiac, « constructeur 116, sur le Port », pour la somme de 60.400 livres\footnote{\textit{Documents relatifs à la rente des biens nationaux}.}.

Thiac, prénommé Jean, était constructeur de navires à Bordeaux. Il écrivit le 22 messidor an III aux citoyens composant l'administration du district de Bordeaux pour demander que la municipalité de Talence ne retire de la chapelle que ce qui était porté sur l'inventaire. Il réclamait, en outre, les clés de la chapelle, « attendu qu'il avait payé l'entière adjudication de cet édifice depuis le 15 messidor »\footnote{Registre des procès-verbaux de la municipalité de Talence. (Archives de la commune.)}.

La chapelle fut détruite peu après. Elle avait été particulièrement fréquentée par les marins de Bordeaux et des environs ; ils venaient y faire des voeux. Aussi Notre-Dame de Rama était également appelée Notre-Dame de Bon« Port »\footnote{Dans les registres des procès-verbaux de la municipalité de Talence, sous la date 17 novembre 1793, la chapelle de la Rame est désignée « Notre-Dame de Bon Port ». (Archives de la commune.)\\Des auteurs donnent à Rama ou Rame le sens d'aviron, et c'est ce qui expliquerait, selon eux, l'attrait qu'exerçait sur les navigateurs la chapelle de Talence qu'on appelait aussi Notre-Dame-des-Douleurs (Dom Reginald BIRON. \textit{Précis de l'histoire religieuse des anciens diocèses de Bordeaux et Bazas}), et chapelle des Monges (voir p. 114).}. On y voyait l'image de l'église avec cette légende :

\begin{center}
NOTRE-DAME DE BON-PORT.
\end{center}

\textit{Je fus construitte l'an 1132 et fus détruitte du tempt des guerres, et je fus rétablie l'an 1730 par la charité des fidelles chrétiens ayant opéré nombre de miracles...}

En guise d'ex-voto, des marins déposèrent dans cet asile de la prière les tableaux ci-après désignés et représentant des navires sur la mer en furie : 

Navire \textit{Marie-Elisabeth}, 2 novembre 1742 ;
Navire \textit{Saint-Alexis}, mars 1750 ; 
Navire \textit{Saint-Nazaire}, 1753 ;
Navire \textit{Ville-de-Bergerac}, 1768 ;
Navire \textit{Duc-de-Penthièvre}, 20 mars 1778 ;
Brick \textit{Le Héros}, 11 octobre 1778 ;
Brick \textit{Le Lyon}, 1779 ;
Navire \textit{La Concorde}, 24 décembre 1782 ;
Navire \textit{Les Trois-Frères}, 3 février 1784 ; 
Navire \textit{Le Persévérant}, 1786 ; 
Navire \textit{La Victoire}, 1788 ;
Navire \textit{Le Fils Unique}, 27 octobre 1790 ;

Plus deux navires sans date : \textit{le Voltigeur} et le \textit{Grand-Saint-Jacques}.

Une lettre adressée, le 17 février 1734, par Mouliney, marguillier de la chapelle de Rama, à Mme de Montplaisant, secrétaire de l'abbesse de Fontevrault, se rapporte à des marins venus en pèlerinage à Talence. Voici cette lettre telle qu'elle a été publiée par le père L. Royer dans sa plaquette la « Retrouve » de Notre-Dame-de-Talence :

« MADAME, 

» Comme vous m'avez ordonné de marquer tout ce qui se passe à la chapelle, je prends la liberté de vous écrire pour vous dire que le 7 janvier dernier, il se trouva un vaisseau dans le naufrage, dont il n'aurait jamais réchappé sans un voeu que l'équipage fit à Notre-Dame-du-Bon-Port, lequel voeu a été exécuté hier, 16 du courant, dans ladite chapelle, où je ne manquai pas de me trouver ; il était même nécessaire.

» Ces pauvres gens sont partis de Bourdeaux pour se rendre à la chapelle, pieds nus et en chemise, ayant tous des cierges à la main, chantant tout le long du chemin, les litanies de la sainte Vierge jusqu'à l'entrée de la chapelle, et, en entrant, ont fait entonner le \textit{Te Deum} par le chapelain, qui a été chanté par tout l'équipage et les assistants, qui étaient en grand nombre, et ont fait présent d'un grand tableau à cadre doré, où est représentée la sainte Vierge dans la nue, à qui ils demandent secours, et où est aussi représenté ledit vaisseau dans une tempête affreuse... »

Il y avait également à Notre-Dame de Rama un tableau rappelant un voeu accompli dans cette chapelle par les jurats de Bordeaux, à l'occasion d'une calamité publique. Les magistrats municipaux portaient le catogan. Ce détail fixe l'époque du voeu : la deuxième moitié du XVIII\ieme{} siècle.

Le tableau des jurats\footnote{Il n'avait aucune valeur artistique, mais présentait un réel intérêt historique.} a disparu; les ex-voto des marins sont précieusement conservés dans l'église paroissiale de Talence.

Le lieu où s'élevait Notre-Dame de Rama garda longtemps le nom de quartier de « la Petite Chapelle ».

\section{La chapelle Saint-Pierre}

Elle s'élevait en un lieu dit « les Abideys », à droite de la route de Bayonne, presque à l'angle du chemin Labric\footnote{Chemin des Briques.}. Son emplacement est marqué par une croix sur un plan ancien ; un cimetière la bordait sur le côté regardant Bordeaux\footnote{Plan 138. (Archives départementales.)}.

Tout d'abord, la chapelle Saint-Pierre fut seulement au service d'une confrérie composée de « laboureurs de vignes ». Un acte relatif à un échange fut dressé, le 30 avril 1483, sous réserve des droits de la confrérie de SaintPierre de Talence, à laquelle étaient dus deux deniers « d'esporle et 25 ardits bourdelois de cens par an »\footnote{\textit{Archives historiques de la Gironde}, t. XXXV, p. 417.}.

Sur une copie d'acte délivrée le 8 septembre 1607, on rélève le nom de « Léonnard de Lescarret, prêtre et vicaire de Talence, syndic de la confrérie de Saint-Pierre\footnote{}.
\textit{Ibid}., p. 418.
La chapelle fut réorganisée par le cardinal de Sourdis. Elle devint succursale de l'église Saint-Genès de Talence, un raison de l'étendue de cette paroisse. 

Un important détail doit se glisser ici.

Deux agglomérations de gens ont constitué la commune de Talence. La première s'était formée autour de l'église Saint-Genès, la seconde aux abords de la chapelle Saint-Pierre. Ces deux édifices se trouvaient sur. le chemin de Saint-Jacques. L'intervalle qui les séparait — soit un quart de lieue — se garnit d'habitations et de jardins qui firent en quelque sorte la soudure.

Dans le procès-verbal relatif à sa visite des édifices du culte, à Talence, le 23 septembre 1691, M. Henry Chapotel, l'archiprêtre de Cernès, avait souligné, entre autres détails, que l'église Saint-Pierre était trop petite pour contenir les paroissiens, et qu'elle était dépourvue de sacristie.

L'archevêque de Bordeaux, d'Anglure de Bourlemont, décida, le 1\ier{} juin 1693 : 1° de faire construire une sacristie « le plus tôt qu'il se pourrait et au lieu qui serait jugé le plus convenable » ; 2° de faire agrandir ladite église ou chapelle; « une aisle serait construite du costé du nord quand il y aurait des fonds suffisamment pour cette dépense »\footnote{G, 646 (Archives départementales.)}. Il y avait, à cette époque, plusieurs confréries établies dans l'église Saint-Pierre. L'archevêque de Bordeaux leur prescrivit, le 1\ier{} juin 1693, de lui apporter leurs statuts dans le délai d'un mois ; celles d'entre elles qui n'avaient point de règlements se pourvoiraient devant le prélat pour en avoir\footnote{\textit{Ibid}.}.

Dom Reginald parle de l'érection « d'une nouvelle chapelle à Talence en 1783 »\footnote{\textit{Précis de l'histoire religieuse des anciens diocèses de Bordeaux et Bazas}.}. Il s'agissait, selon nous, d'un nouvel agrandissement de la chapelle Saint-Pierre\footnote{Voir l'article « Peychotte », page 65.}. Les travaux durèrent de juillet 1782 à août 1783.

En 1782, toute la partie de la commune comprise aujourd'hui entre le cours Gambetta, le chemin de Suzon, la mairie et le ruisseau de Talence constituait deux fiefs : l'un de Saint-Pierre, l'autre de la « confrairie de Saint-Pierre-deTalence »\footnote{Plan 138. (Archives départementales.)}.

La chapelle Saint-Pierre, à laquelle était adossé le presbytère de la paroisse, fut, le 28 pluviôse an II, transformée en \textit{Temple de la raison}. Exposons, à ce propos, les termes du paragraphe 4, relatif à la nomenclature des biens nationaux de Talence :

« Plus, l' église Saint-Pierre-de-Talence, maison ci-devant curialle, cour et autres bâtiments attenants, jardin et cimetière, le tout situé au centre de la commune, section Saint-Genès, lesquels effets et église, la commune se réserve pour ses utilités et a, en conséquence, \textit{choisi cette église pour y former le Temple de la raison où l'instruction nationale sera enseignée} »\footnote{Registre des procès-verbaux de la municipalité de Talence. (Archives de la commune.)}.

Le 25 ventôse an III, on décide que des affiches seront apposées pour avertir les habitants et autres citoyens que le jardin du cy-devant presbytère de la commune, situé au couchant du chemin qui y conduit, sera mis aux enchères le 10 germinal prochain ».

La « maison ci-devant curialle » — ou presbytère — fut elle-même adjugée, le 11 thermidor an IV, pour la somme de 6,320 livres, à M. Pierre Pitrey, 6, place Saint-Projet.

En 1802 — un an après le Concordat — la chapelle Saint-Pierre servait d'église paroissiale\footnote{Rappelons que l'ancienne église paroissiale Saint-Genès avait été désaffectée et vendue en l'an III.}. On y plaça la statue de la Vierge et celle-ci, comme on le verra plus loin, fut transférée finalement dans l'église Notre-Dame de Talence.

La chapelle Saint-Pierre n'existe plus, ceci dit pour éviter toute confusion avec le monastère de Saint-Pierre construit, en 1901, chemin de Suzon.

Sous Louis-Philippe, la mairie siégeait dans l'ancien Temple de la Raison\footnote{Auguste BORDES. \textit{Histoire des monuments anciens et modernes de la ville de Bordeaux}.}.

Le presbytère était devenu la maison communale. La mairie y siégeait encore en 1864\footnote{Le Père L. DELPEUCH. \textit{Histoire de Notre-Dame de Talence ou de Rama}.}.

\section{Le prieuré de Bardenac}\footnote{Bardenac ou Bardanac.}

Il s'élevait au point où se touchent les paroisses de Talence, Pessac et Gradignan. Il appartenait, d'après Léo Drouyn, à la paroisse de Talence\footnote{\textit{Bordeaux vers 1450}.}.

Nous avons lu, d'autre part, « Bardanac à Talence »\footnote{Archives départementales de la Gironde, série G.}.

Suivant le \textit{Bulletin de la Société archéologique de Bordeaux}, la maison prieurale et l'hôpital de Notre-Dame de Bardenac étaient dans la paroisse de Pessac. Seuls, l'église prieurale et le cimetière de Bardenac se trouvaient dans Talence\footnote{Tome XXI, année 1896, page 124.}. Toutefois, si l'on consulte les \textit{Registres de la Jurade}, on y voit que c'est l'hôpital de Bardenac qui était dans Talence avec le prieuré\footnote{Tome IV.}.

L'hôpital, créé au XII\ieme{} siècle, devint prieuré dans la suite\footnote{Dom Reginald BIRON. \textit{Précis de l'histoire religieuse des anciens diocèses de Bordeaux et Bazas}.}.

Un acte de 1357 porte qu'à cette date le prieur de Bardenac possédait une maison dans la rue Sainte-Eulalie.

Bardenac était le premier hôpital que rencontraient les pèlerins qui empruntaient la « voie de Gascogne » pour se rendre en Espagne, au pèlerinage de Saint-Jacques-deCompostelle\footnote{L'abbé Baurein. \textit{Variétés bordeloises}.}, si renommé ou moyen âge. Cette voie de Gascogne passait par Bordeaux, Talence, Belin, Salles, Dax, Sordes\footnote{Carte routière dressée d'après le \textit{Codex cornpostellam du XII\ieme{} siècle}}, etc. C'était la route de Bayonne.

Le 25 novembre 1540, on enregistre une reconnaissance pour un maine à Talence, « confrontant au chemin comunau par lequel on va et vient de la ville et citté de Bourdeaux à Sainct-Jaques de Gallice »\footnote{\textit{Inventaire sommaire. Archives de la Gironde}, série G, t. II.}.

Le 8 septembre 1629 est signée une autre reconnaissance pour une vigne sise dans la paroisse de Talence, « au plantier du Branar, confrontant à l'est au grand chemin qui va de Bordeaux à Talence, autrement appelle chemin Roumieu »\footnote{Roumieu ou romieu signifie pèlerin. Chemin de Saint-Jacques ou \textit{camin romiu}.}.

Un paysan picard nommé Manier, se rendant à Compostelle, en septembre 1726, nota sur son carnet de route :

« Départ de Bordeaux. — Le 27 au matin, avons parti de cette ville pour aller à Saint Guenest (Saint-Genès) ; au pont de la Lances (Talence)\footnote{\textit{Société archéologique de Bordeaux}, t. XXI, année 1890, p. 133.}. »

Ce pont était jeté sur le ruisseau de Talence, cours Gambetta, à hauteur de Peychotte.

On prétend que Charlemagne suivit le chemin de Saint-Jacques dit « la Voie du Centre »\footnote{La « voie du Centre » passait par Périgueux, La Réole, Bazas, Mont-de-Marsan, Orthez, Saint-Palais, Ostabat et Saint-Jean-Pied-de-Port.}, quand il partit avec son neveu Roland pour aller combattre les Sarrasins, et qu'il passa, à son retour, par l'autre chemin de Saint-Jacques dit « la Voie de Gascogne ». Ainsi, le grand empereur, ramenant la dépouille mortelle de Roland et celle des autres preux également tués dans l'embuscade de Roncevaux, traversa Talence pour se rendre à Bordeaux, en la basilique Saint-Seurin où, selon la tradition, il déposa l'olifant ou le cor d'ivoire du célèbre paladin.

Un autre grand prince avait, longtemps avant Charlemagne, foulé le sol de Talence.

Clovis, ayant, en 507, à Vouillé, tué de sa main Alaric II et défait ses troupes, avait poursuivi celles-ci en fuite « vers Bourdeaus ». Il rejoignit les fuyards et les extermina dans un terrain limitrophe de Talence\footnote{BORDES. \textit{Histoire des monuments de Bordeaux}.} et voisin du village qui a gardé le nom de Camparian\footnote{Dans l'acte d'acquisition par les maire et jurais de Bordeaux du comté d'Ornon, il est fait mention de la « Prévoté de Camparrian x. (Arch. départ. Intendance, série C, 3660.)} (camp des Ariens). Clovis regagna ensuite Bordeaux pour y passer l'hiver.

Bardenac fut uni, à partir de 1600, au collège de la Madeleine\footnote{Archives départementales de la Gironde, série G.}, dont les bâtiments sont occupés aujourd'hui par le Lycée national. La chapelle de Bardenac fut, depuis lors, desservie par les pères Jésuites.

Suivant une déclaration de revenus des Jésuites faite, le 4 juillet 1692, par le syndic du collège de la Madeleine, Talence rapportait alors « 1.294 livres de rente sur 287 fiefs, tant pour droits de censive que d'agriaires »\footnote{\textit{Société archéologique de Bordeaux}, t. XXI, année 1896, p. 124.}.

Il est question dans une chronique d'un « gros cheval apellé Cadet », qui était né à Bardenac et mourut « de tranchée rouge », au collège des Jésuites, le 7 mars 1742.

Ce cheval, de poil noir bien marqué, était âgé d'environ huit ans\footnote{Archives départementales de la Gironde. (Fonds des Jésuites, carton 39.)}.

Chaque année, le jour de la procession de Saint-Marc, les fidèles de Talence et de Saint-Genès se rencontraient à Bardenac avec les fidèles des églises voisines, et il y avait contestation pour savoir quelle paroisse aurait le premier pas. L'archevêque, soucieux de ménager toutes les susceptibilités, décida que les deux curés de Saint-Genès et de Gradignan marcheraient de front dans le cortège. Un vicaire reçut mission de faire respecter l'ordonnance épiscopale.

Dans leur litige, les ecclésiastiques avaient oublié le fameux précepte : « Les derniers seront les premiers ! »

\section{Les chapelles domestiques}

Le 23 septembre 1691, l'archiprêtre de Cernès, M. Henry Chapotel, visita cinq chapelles domestiques à Talence : celles du sieur de Guionet\footnote{La villa Raba occupe l'emplacement de la maison noble de Guionet (Voir page 51.)}, du sieur Roux, du « sieur Bigot, de M. le président Latresne, et du sieur Pickronneau, converti, ayant acquis la maison du sieur La Roque »\footnote{G, 646. (Archives départementales.)}.

Les quatre premières chapelles étaient pourvues de tous les ornements nécessaires pour la célébration de la sainte « messe » ; celle de Pickronneau était dépourvue « de pierre sacrée, de napes et de tous autres ornements utiles pour l'exercice du culte ».

Le 25 septembre 1726, l'archevêque de Bordeaux accorda à Louis de Gombaut, président à la Cour des Aides, l'autorisation « de faire construire une chapelle domestique dans sa maison de campagne, située dans la paroisse de Talence »\footnote{G. 682. (Archives départementales.)}.

Françoise Labatut, épouse du « sieur président Fossier », avait adressé une requête en vue de faire édifier une chapelle dans son domaine à Talence. Cette requête fut favorablement accueillie le 29 août 1736.

La permission de bâtir une chapelle n'était accordée à la suppliante « qu'autant qu'elle continuerait à s'en rendre digne » par son attention à remplir les conditions prévues en pareil cas, « et par son exactitude à ne pas permettre que d'autre qu'elle, sa famille et les domestiques qui luy seraient nécessaires y entendraient le Saint-Sacrifice de la messe »\footnote{\textit{Ibid}.}. La présidente Fossier devait renvoyer les autres personnes à l'église paroissiale.

Tous lès propriétaires de chapelles domestiques étaient soumis aux mêmes obligations.

\section{L'église de Talence}

En 1817, l'église Saint-Pierre était insuffisante « pour contenir le cinquième des habitants »\footnote{Registre des procès-verbaux de la municipalité de Talence. (Archives de la commune.)}. On songea à bâtir une autre église paroissiale. La première pierre en. fut posée par le fondateur, Mgr d'Aviau, archevêque de Bordeaux, le 12 mars 1821.

L'année suivante, le 24 novembre 1822, le même prélat vint baptiser les deux cloches de la nouvelle église. Les duchesses d'Angoulême et de Berry furent marraines de ces cloches. La cérémonie se fit avec une telle pompe qu'un journal de Bordeaux, en la décrivant, déclara que cette journée serait « célèbre dans les annales de Talence ». 

Le 4 mars 1823, la statue de la Vierge Notre-Dame de Rama — ou Notre-Dame de Talence. — fut transportée solennellement de la chapelle Saint-Pierre dans la nouvelle église. Celle-ci fut bénite et consacrée par l'archevêque, le 1er avril 1823, « en présence de toutes les autorités ecclésiastiques, civiles et militaires de Bordeaux et de la commune de Talence »\footnote{Extrait d'un article écrit par M. Ripollès, curé de Talence, le 20 août 1836, et publié, la même année, dans la Dominicale bordelaise. (Archives municipales de Bordeaux.) M. Ripollès mourut le 13 novembre 1836. L'archevêque de Bordeaux. Magr d'Aviau, en témoignage de la satisfaction qu'il avait éprouvée en consacrant, en 1823, l'église de Talence, dont la construction était due spécialement aux soins et au zèle de M. le curé Ripollès, avait nommé dernier chanoine honoraire. ce (\textit{Tablettes du clergé et des amis de la Religion}, t. III, p. 322.)}.

Cet édifice religieux eut une existence éphémère. En 1840, il menaçait ruine. Etait-ce un défaut de la charpente ? Etait-ce l'instabilité du sol ? Quoi qu'il en soit il fallut l'abandonner en 1842, et bâtir, tout à côté, une autre église : c'est celle que nous voyons actuellement, et qui est renommée par les pèlerinages qu'on y fait toute l'année et, notamment, au mois de mai.

La première pierre de l'église de Talence fut posée par l' archevêque de Bordeaux le 28 mars 1843.

« L'église de Talence, écrivait alors Bernadau, se reconstruit au moyen d'une souscription de 5 centimes Par semaine que chaque habitant a promis de payer au curé, qui dirige fort adroitement cette entreprise, et fait des lotteries du vin que les riches propriétaires donnent en supplément de leur souscription hebdomadaire d'un sou\footnote{\textit{Tablettes}, t. XII.}. »

La consécration de l'église eut lieu le 12 août 1847. Le souvenir de cet événement est perpétué par une plaque de marbre placée dans l'édifice, en face de la chaire, et portant en latin cette inscription :

« L'an du salut 1847 et le 12 du mois d'août, le Souverain Pontife Pie IX, heureusement régnant, M. J.-H.-F. Carros étant curé de la paroisse, et MM. Tulèvre, Roui, Rousseau, Jaquemet, Crespy et Carrier, membres de la fabrique ; ce temple bâti par les offrandes des fidèles a été consacré, au milieu d'un heureux et nombreux concours de peuple, par N.N. S.S. Ferdinand-François-Auguste Donnet, archevêque de Bordeaux, et Clément Villecourt, évêque de La Rochelle ».

L'église appartient au style gréco-rofnain. Le plan comprend un porche précédé d'un portique ; des fonts baptismaux à droite de l'entrée ; à gauche, l'escalier de la tribune ; une nef, une croisière formée du choeur et des chapelles latérales, le sanctuaire principal, une sacristie et un dépôt d'ornements.

L'élévation de la façade se distingue par un perron de six marches. Le portique offre sur deux socles unis deux colonnes cannelées d'ordre ionique, supportant avec leurs antes un entablement denticulaire ajusté d'un fronton sur lequel sont placés une croix et un briquet. Cet entablement est enrichi de plusieurs rosaces répondant dans la frise à l'axe des colonnes.

De chaque côté du portique est creusée une niche renfermant la statue d'un saint. L'entablement principal, dont les extrémités reposent sur un pilastre dorique, présente, dans sa frise, cette dédicace :

\begin{center}
D. p. M. SUB INVOCATIONE\\
B. MARLE VIRGINIS
\end{center}

Au-dessus, dans un encadrement sculpté, la niche où il y a l'image de la Vierge, à qui est consacrée cette église.

On voit encore, palmes la façade, sur et couronnes, deux figures célestes tenant une couronne avec le chiffre de Marie.

L' église Notre-Dame de Talence est l'oeuvre de l'architecte Auguste Bordes. Cet architecte s'est adressé à luimême un compliment en exprimant le regret que les peintures et les autels qui ont été exécutés dans l'église n'aient Pas une « plus grande affinité avec les autres parties du Monument »\footnote{Histoire  des monuments anciens et modernes de la ville de Bordeaux, t. II.}.

D'après le voeu qui lui en avait été exprimé par plusieurs personnes occupées à l'oeuvre de réédification, Bordes avait projeté une église de style gothique ; mais sur le refus du Conseil des bâtiments civils, il avait dû étudier le projet gréco-romain qui fut exécuté.

\section{La croix de Leysotte}

Elle se dresse à l'angle du chemin de Leysotte et de la route de Toulouse, en bordure de cette dernière voie. Elle présente un socle massif en pierre reposant sur deux gradins ; la croix en fer, ouvragée, montre un motif symbolique.

Y a-t-il une inscription sur la face antérieure du socle, celle regardant Bordeaux ? Il est impossible de s'en rendre compte, ce côté de la maçonnerie étant constamment recouvert d'affiches diverses, lesquelles, d'ailleurs, pourraient être, sans inconvénient, placardées autre part.

La croix de Leysotte est indiquée sur la \textit{Carte de Guienne}, de Belleyme\footnote{Dressée en 1775. (Voir page 22.)}. Sans doute devait-elle marquer, du côté de la roule de Toulouse, les limites de la paroisse de Talence ?

Dans un acte du 22 juin 1781, il est fait mention de Etienne Labadie, maître forgeron, habitant « près la Croix de Leyssote, joignant le grand chemin de Bordeaux au Bouscot »\footnote{G, 3146. (Archives départementales.)}.


\section{Les curés de Talence}\footnote{Cette liste est incomplète. Les dates accompagnant les noms sont celles où nous avons trouvé ces ecclésiastiques dans l'exercice de leurs fonctions.}

        \begin{itemize}
        	\item[] Saint-Aignan \dotfill 1676
			\item[] Toland (Jean-Pierre) \dotfill 1684-1709
			\item[] Lacourt \dotfill 1709-1735
			\item[] Dumoulin \dotfill 1735
			\item[] Fortin \dotfill 1740-1788
			\item[] Oscanlan (André-Théophile) \dotfill 1788-1790
			\item[] Toucas-Poyen \dotfill 27 février 1791
			\item[] Le Nègre \dotfill 27 décembre 1791
			\item[] Rippolès \dotfill 1823-1836
			\item[] J.-H.-F. Carros \dotfill 1847
			\item[] Merlin (R. P. oblat) \dotfill 1853-1863
			\item[] Jeanmaire (R. P. oblat) \dotfill 1874
			\item[] Ramadié \dotfill 1892
			\item[] Coubrun \dotfill 1899
			\item[] Royer (R. P. oblat) \dotfill 1901
			\item[] Doreillac (R. P. bénédictin) \dotfill 1905
			\item[] Moureau \dotfill 1913
			\item[] Joanne \dotfill 1922
			\item[] Fort (A.)	\dotfill 1926
        \end{itemize} 
        
\end{document}
        
        
        