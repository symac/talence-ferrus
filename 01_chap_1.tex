%%%% Débute page 20 dans le fichier PDF%%%%
\section{Des origines de Talence jusqu'à nos jours}

\subsection{Étendue de la commune ; ses quartiers}

Talence est une commune privilégiée. Les auteurs la mentionnant dans leurs ouvrages ne tarissent pas d'éloges à son sujet. L'un écrit : « Talence est dans une situation charmante » ; un autre assure que « le territoire de Talence été de tous temps recherché a par les grandes familles de l'antique capitale de la Guienne ».

« La banlieue, écrit Jouannet, n'a point de commune Plus saine, plus agréable ou qui renferme de plus belles maisons de campagne ; plusieurs de ces riches habitations sont citées pour l'élégance des édifices, pour leurs ombrages, leurs jardins et leurs eaux\footnote{\textit{Statistique de la Gironde}, tome II.}. »

Le chroniqueur Bernadau célèbre, lui, sur le mode lyrique, la si séduisante petite cité : 
%%%% Todo : mettre en forme le poème p. 21 %%%%
Pour découvrir jardin charmant, 
Vous parcourez à grands frais la Provence ; 
D'une belle bastide\footnote{Le nom de bastide, qui s'applique ici à la villa Raba, cours Gambetta, est celui que l'on donne aux villas et maisons de plaisance des environs de Marseille.} ici commodément 
Nous jouissons de l'agrément 
En nous promenant à \textit{Talence}. 

Dans les registres des procès-verbaux de la municipalité de Talence, sous la date 1817, on peut lire : 

L'habitude d'un ancien pèlerinage à Notre-Dame de Talence et les agréments des campagnes de notre commune attirent une affluence considérable de personnes religieuses et d'oisifs, les jours de fêtes et dimanches, notamment dans la belle saison\footnote{Archives de la commune.}. »

Arrêtons-là les citations; celles que nous pourrions encore présenter sont, au surplus, du même genre, autrement dit tout à l'avantage de Talence. 

Cette localité s'étend au sud de Bordeaux, sur une plaine légèrement ondulée à l'est et à l'ouest, et traversée par un petit vallon où coule un clair ruisseau : les Malerettes. 

Les limites de Talence sont nettement tracées sur le \textit{Plan de Bordeaux et de sa banlieue}, publié, en 1907, par M. Louis Longueville, géomètre de la ville de Bordeaux. Faisons le tour de Talence en partant de l'angle du boulevard\footnote{Le boulevard de Talence a changé de nom depuis 1919. Le tronçon compris entre la barrière de Pessac et celle de Saint-Genès a été baptisé boulevard George-V ; le reste du boulevard de Talence, allant jusqu'à la barrière de Toulouse, a été dénommé boulevard Victor-Emmanuel-III} et du cours du Maréchal-Galliéni (anciennement chemin de Pessac) : 

Cours du Maréchal-Galliéni jusqu'aux abords du château Haut-Brion, ligne à gauche englobant la Mission Haut-Brion, chemin de Goudats, Villemejoy, Coudourne, route nationale 132 (Bordeaux-Bayonne), la Providence, chemin de Mons, chemin de Gradignan, chemin vicinal n° 7 de Leysotte, la route de Toulouse et la rue de Cauderès jusqu'au boulevard. 

En résumé, une route nationale et une route départementale flanquent le territoire de Talence : la première au levant, allant à Toulouse ; la seconde vers le nord, menant à Arcachon. Au centre de la paroisse passe la grande route Bordeaux-Bayonne, baptisée cours Gambetta dans sa traversée de la commune. 

Au moyen âge, Talence s'étendait beaucoup plus loin vers le nord. Dom Reginald Biron note l'existence du prieuré « Saint-Laurent d'Escures, à Talence »\footnote{\textit{Précis de l'histoire religieuse des anciens diocèses de Bordeaux et Bazas}.} en l'an 1165. 

Les \textit{Rôles gascons} signalent, sous la date 1242, une voie qui se dirigeait vers Bordeaux jusqu'au Petit-Talence, \textit{usque Talansiolam}\footnote{Tome I, page 84.}.

Dom Etienne Du Laura indique le prieuré Saint-Laurent d'Escures comme faisant partie de la paroisse Saint-Genèsde-Talence en 1283\footnote{\textit{Histoire de la Seauve-Maiour}.}.

Dans un acte du 23 février 1632 est mentionnée la vente d'une pièce de vigne « au lieu appelé au Haut-Queyron, dans la paroisse de Talence »\footnote{Leo DROUYN. \textit{Notes manuscrites} (archives municipales de Bordeaux). — La rue du Haut-Queyron, commençant sur le boulevard George-V et aboutissant avenue Jeanne-d'Arc, se trouve maintenant dans Bordeaux, paroisse de Saint-Augustin.}. 

Avant 1865, les rues de Ségur, Duluc et Bertrand-deGoth constituaient les limites administratives de Bordeaux en bordure de la commune de Talence. Après 1865, tout le territoire compris entre les trois rues en question et le boulevard fut incorporé à Bordeaux. Cette ville s'accrut ainsi de ce côté de 35 ha. 40\footnote{\textit{Nomenclature alphabétique des voies de la ville de Bordeaux}, 1\ier{} août 1910.}, au détriment de Talence. 

Les principaux quartiers de Talence sont : Grand-Courneau, Petit-Courneau de Monges, Courneau-d'Ars \footnote{\textit{Cournau} ou \textit{Cornau} signifie : quartier, coin (glossaire gascon. \textit{Archives historiques de la Gironde}, t. XI, et \textit{Dictionnaire provençal-français} de MISTRAL). }, Banquet, Peydavan ou Plume-la-Poule\footnote{\textit{Atlas départemental du Conseil général}, 1888.}, la Médoquine, les Palanquettes, La Croix-de-Leysotte, Pey-Lane. 

Sur la \textit{Carte de Guyenne}, dressée par Belleyme, en 1775, les trois quartiers de Courneau sont ainsi désignés : Cournau-de-Ruhans \footnote{Ce quartier est appelé « Cournau d'Arruan » dans un texte du 27 février 1721. (\textit{Inventaire sommaire des archives de la Gironde}, série G, t. II.) }, Cournau-du-Monge, Cournau-d'Ars. 

Il y eut le 14 juin 1680 une reconnaissance pour une vigne sise à Talence, au lieu dit « au Cornau de Prat et depuis, à Jaubert\footnote{\textit{Inventaire sommaire des archives de la Gironde}, série G, t. II.} ».

En 1692 était au Cornau-d'Ars un bourdieu faisant partie du domaine du prieuré de Saint-James, à Bordeaux\footnote{\textit{Société archéologique de Bordeaux}, t. XXI, p. 124. }. 

Le quartier de Pey-Lane rappelle l'ancien domaine de Peylenne (maison de maître, bâtiments, 2 j. 3/4 enclos ; 5 j. 1/2 vigne), domaine qui appartenait à Chillaud-Desfieux, émigré, et qui fut adjugé, le 17 ventôse an VII, pour la somme de 5.150 livres\footnote{Déduction faite de la valeur de l'usufruit de la veuve Chillaud-Desfieux. (M. MARION, J. BENZACAR, CAUDRILLIER. \textit{Documents relatifs à la vente des Biens nationaux}, t. 1.)}, à M. Duluc, 6, rue des Treilles\footnote{ Rue de Grassi.} à Bordeaux. 

Il y avait autrefois dans la paroisse de Talence « le plantier de Pisselèbre \footnote{Plan n° 1255 (Archives départementales). } » ; il était situé à gauche du « grand chemin royal de Bordeaux à Bayonne ». Dans le plantier de Pisselèbre étaient, entre autres biens : le fief de la chapelle du Puch (aux soeurs Verdale), et les « possessions en maisons, chay, cuvier, cour, puits, jardins et vigne à M. Lamour ».

Il y avait encore, en la section de Saint-Genès, un lieu appelé au Negrot, dans lequel était située « une courrège de vigne » appartenant à la fabrique de la chapelle SaintPierre de Talence, et qui, devenue bien national, fut estimée 1.000 livres le 28 pluviôse an II. 

À la fin de l'ancien régime, la paroisse était « distribuée en huit villages : Cournau de Ruan, C. de Monges, C. Darcs, Coudournes \footnote{Le 11 mai 1774, Pierre Baillet, négociant, fit cession à ses filles Jeanne, épouse Glory, et Elisabeth, épouse d'Egmond, de ses droits de propriété et d'usufruit sur le fief noble de Coudourne, à Talence, moyennant une rente. (Pierre MELLER. \textit{Essais généalogiques}.) }, Paindavan, Pain Lane, Talance \footnote{Le 21 décembre 1584, Jean de Gaufreteau prit à fief nouveau, des Jésuites du couvent de la Madeleine, auquel était uni le prieuré de Saint-Jacques, « un bourdieu situé dans la paroisse Saint-Genès, \textit{appelé à Talence} et anciennement à \textit{Blanchardy} ». (Archives départementales, E, 109-41 ; terrier pour les Jésuites.)} à la chapelle Saint-Pierre et le bourg de Saint-Genès \footnote{L'abbé BAUREIN. Questionnaire.} ». 

À la Révolution, la commune de Talence comptait quatre sections : celles de Saint-Genès, du Courneau, Peydabant et du Moulin-d'Arts\footnote{Registre des procès-verbaux de la municipalité de Talence. (Archives de la commune.)}.

Sous Charles X, la commune de Talence groupait les hameaux ci-après : « Bellard, Cournaud d'Ars, Cournaud de Mage, Grand Courneau, La Capelle, \textit{Lacroix de Laysotte}, la Médoquine, L'Au, Peydavant, Peylanne, Pont d'Ars\footnote{Le Pont d'Ars « section du Moulin d'Ars » en 1783.}, Rabba, Saint-Genès\footnote{\textit{Nomenclature des communes de la Gironde}, 1826. (Archives municipales de Bordeaux.)}. »

Le comté d'Ornon comprenait les paroisses de Gradignan, Léognan, Canéjan, Cestas, Villenave et partie de Martillac, Talence et Bègles 
\footnote{\textit{Société archéologique de Bordeaux}, t. XIII.}. 

Suivant Oreilly, la paroisse de Saint-Genès de Talence, était englobée tout entière dans le comté d'Ornon \footnote{\textit{Histoire de Bordeaux}, t. III (1re partie), p. 671.}.

La baronnie de Veyrines, qui avait son chef-lieu à Mérignac, comprenait les paroisses de Mérignac, Bègles, Illac, Caudéran, Le Bouscat, Pessac et Talence \footnote{L'abbé BAUREIN. \textit{Variétés bordeloises}, t. II, p. 243.}.

En 1637, pour les tailles de Guyenne, Talence payait 1.190 l. 5 s. 4 d. ; deux autres des paroisses de la juridiction de Veyrines, Mérignac et Pessac, versaient de leur côté, à la même date, la première, 2.636 l. 13 s. 4 d., et la seconde, 1.566 l. 18 s. 8 d. \footnote{Archives de M. de Lard. \textit{Fonds Leo Drouyn}, t. XXII, p. 220.}.

Saint-Genès et Talence ont formé à un moment donné deux paroisses distinctes, comme en témoigne le document ci-après daté de 1764, et conservé aux Archives départementales : 

« \textit{Saint-Genès}. — Le terrier de cette paroisse n'avait pas été renouvelé depuis 1475. Les rentes foncières consistaient alors en 4 liv. 10 s. en argent et en agrières au septain tain des fruits sur quelques ténéments. Ces rentes ne paraissent pas avoir jamais été servies. 
%%% Reprendre la mise en forme de la page 25 %%%

» Cette paroisse ne produit aucun revenu casuel ; ce qui ne peut dépendre que de ce que les censives de la ville ne sont point connues depuis près de trois siècles.

» Les possesseurs sont les habitants les plus qualifiés de la ville et les principaux bourgeois qui, pour le moins, prétendent posséder un franc aleu ; plusieurs même se sont formé des directes au préjudice des droits de la ville.

» \textit{Talence}. — Les titres féodaux relatifs à cette paroisse ont disparu dans les incendies de l'Hôtel-de-Ville. La cite est donc vis-à-vis les possesseurs dans le même état que pour la paroisse de Saint-Genès\footnote{OREILLY. \textit{Histoire de Bordeaux}, t. III (1re partie).}. »

En 1785, il y avait à Talence 366 feux, soit environ 1.800 âmes, en comptant, comme il est d'usage, cinq personnes par feu. La population était réduite, en 1823, à « 1.195 individus répartis dans trois cents maisons » \footnote{F.-J. \textit{Le Musée d'Aquitaine}, 1823.}. Cette diminution résultait, à notre avis, des guerres de l'Empire. 

En 1839, le chiffre des habitants atteignait 1.288 ; il n'a fait qu'augmenter depuis lors \footnote{JOUANNET. \textit{Statistique de la Gironde}.}. 

Avant la Révolution, les lettres destinées aux Talençais devaient porter, sous le nom de Talence, la mention \textit{par Bordeaux}. 


\subsection{La justice}

La ville de Bordeaux exerçait la justice dans sa banlieue ; néanmoins le comté d'Ornon — dont dépendait Talence — situé dans l'enclave de cette justice conservait la sienne propre. C'est ce qui détermina les maire et jurats à faire l'acquisition du comté d'Ornon, le 17 décembre 1409\footnote{\textit{Livre des Privilèges}.} des mains de Henri de Lhoët, archevêque d'York, Pour la somme « de 500 marcs sterling d'or de la monnaie d'Angleterre prix qui équivalait, en 1863, à 300.000 livres »\footnote{OREILLY. \textit{Histoire de Bordeaux}.}.

La \textit{Chronique bordelaise} précise bien que cet achat comprenait « tout droit de justice et de seigneurie »\footnote{Cependant, cette question de justice n'était pas nettement tranchée, tout au moins dans les premiers temps de l'acquisition du comté d'Ornon. Ainsi, le château de Thouars conservait encore en 1412, comme on le verra dans un autre chapitre, toute sa juridiction.}. 

Dans une séance tenue le 4 janvier 1416, la jurade eut à s'occuper d'un homme mort à Talence. Elle chargea le lieutenant, le prévôt, le clerc de ville et trois jurats de se rendre à Talence pour enquêter. Voici, du reste, le texte assez curieux de la délibération de la jurade : 

« Premeirament, ordeneren que lo loctenent, lo perbost, clerc, Johan Arostanh, Guilhem Aysselin et Arnaud Fort anguen à Tallanssa per beder l'ome mort\footnote{\textit{Registre de la jurade}.}. » 

Il y a eu à Talence le quartier de « l'Homme-Mort »\footnote{BELLEYME. \textit{Carte de Guyenne}.}. La rue de l'Ormeau-Mort a été ouverte en partie dans ce quartier. « L'Ormeau-Mort » — qui fait penser à un arbre abattu peut-être par la tempête — est donc, à notre avis, une simple altération de « l'Homme-Mort ». Ce dernier nom pouvait avoir son origine dans la découverte, en 1416, du cadavre suspect.

Le tronçon de la rue de l'Ormeau-Mort reliant le boulevard au cours du Maréchal-Galliéni est devenu la rue Gaston-Lespiault.

\subsection{Le ruisseau des Malerettes} 

Le nom des Malerettes\footnote{Il est fait mention du nom des Malerettes dans les Registres des procès-verbaux de la municipalité de Talence, sous l'année 1815. (Archives de la commune.) } — on écrit aussi « Maillerettes » ou « Mulerette » — n'est pas le seul qui ait été donné au ruisseau de Talence. Ce ruisseau figure sur la Carte de Guyenne de Belleyme, mais sans dénomination. L'endroit  où il coupe la route de Toulouse est appelé « Pont d'Ars », et le point où il traverse le chemin de Bègles est indiqué " Pont de Ladaus »\footnote{\textit{Carte de Guyenne}.}.

Sur l'\textit{Atlas départemental du Conseil général}, le ruisseau de Talence est appelé, vers sa source, « ruisseau du Serpent ; après le quartier du Grand-Courneau, il de» vient le « ruisseau d'Ars ». Ayant traversé la route de Toulouse, à la Ferrade, il prend le nom d'« estey de Ladaus ». Il se confond alors avec un autre estey formant plus loin deux branches l'estey de Sainte-Croix et l'estey de : la Moulinasse, allant se jeter tous deux dans la Garonne : le premier, au pont de Brienne ; le second, à la Moulinasse.

Donc, le ruisseau de Talence portait le nom de « ruisseau d'Ars dans sa partie comprise entre le Grand-Courneau » et la route de Toulouse. Pour quelle raison ? Nous allons l'exposer.

Les Romains devenus maîtres de Bordeaux s'occupèrent tout de suite de construire un aqueduc pour amener l'EauBlanche au centre de la cité. Ce cours d'eau abondant vient des Landes, passe à Léognan et va se perdre dans la Garonne, à Courréjean. Il y eut ainsi, durant la domination romaine, une dérivation de l'Eau-Blanche par le moyen de l'aqueduc. Cet ouvrage devait franchir l'Eau-Bourde à Villenave-d'Ornon, puis, s'engageant sur le sol de Talence, ilpassait au-dessus du cours d'eau arrosant cette commune. Les arcs de l'aqueduc donnèrent leur nom au ruisseau de Talence. On l'appelait, en effet, dans les temps anciens, « ruisseau des Arcs » \textit{rivus de Arcubus}.

C'est le savant Élie Vinet qui, en l'an 1552, étant allé se promener du côté de la route de Toulouse, « un jour d'hiver clair et serain, et cerchant là de meilleur air qu'il n'est pas en la ville communément » \footnote{\textit{L'antiquité de Bourdeaus}.}, découvrit le premier les vestiges de l'aqueduc. Il vit un moulin à blé, appelé « le moulin des Arcs »\footnote{Dans une charte de Henry III, du 30 octobre 1242, en faveur du prieur de l'hôpital Saint-Jacques de Bordeaux, il est question d'une étable qui appartenait audit hôpital et avoisinait le moulin des Arcs — \textit{molendinum de Arcubus}. (\textit{Rôles gascons}, t. I, p. 84.)}.

«... En cete valée, écrit Vinet, i avoit des arcs ou arceaus pour conduire l'eau au niveau, ainsi que ces sages anciens savoint qu'il falloit faire pour avoir l'eau bonne et saine ; je pensai que ce moulin avoit prins son nom de ces arcs\footnote{\textit{L'antiquité de Bourdeaus}.}. »

Le ruisseau du Serpent, des Arcs, des Malerettes ou de Talence — pour le présenter sous ses principales dénominations — faisait tourner, au pont de la route de Toulouse, le vieux Moulin d'Ars, dont un quartier à cheval sur Bègles et Talence perpétue le souvenir\footnote{Le Comité des fêtes de Nansouty-quatre-Coins et le Comité de défense des intérêts de ce quartier, reprenant une ancienne. tradition, ont nommé en septembre 1924 une Meunière du Moulin d'Ars et organisé en son honneur des réjouissances populaires. Depuis lors, chaque année, à la même époque, une Meunière du Moulin d'Ars est nommée et fêtée.}.

Le même cours d'eau actionnait un autre moulin de Talence, dit « le Moulin de Lande ».

Dans un acte du 12 décembre 1576, le sieur « Jean Estève, laboureur de la paroisse de Talence, reconnaît tenir féodalement (de Pierre Hazera) une pièce de terre et vigne confrontant des deux côtés à la vigne de Guyraud Delaville, d'un bout à la ruette ou chemin de \textit{Labricq}, vers le nord, et de l'autre, vers le midi, à l'estey commun qui va au moulin de Lande et au moulin d'Artz »\footnote{\textit{Archives historiques de la Gironde}, t. XXXV, p. 418 et 419.}.

Le moulin de Lande se trouvait derrière le Jardin botanique. On y accédait « par un sentier et un chemin venant de Saint Pierre, longeant la propriété de \textit{Plantier} (1623), aujourd'hui chemin de Suzon et impasse des Écoles communales\footnote{Le père ROYER. \textit{Notre-Dame de Talence dans la chapelle des Monges, au XVIII\ieme{} siècle}. (Archives municipales de Bordeaux, n. 2778.)}. Le ruisseau de Talence forme au moins deux chutes après sa traversée du cours Gambetta. La première constitue un des attraits du Jardin botanique. « La roue du moulin de la Lande tournait, écrit le père Royer, sous l'autre chute d'eau, près du chemin Pey Davant, qui s'appelait chemin d'Agès Monjoux \footnote{Ibid.}. » ou de Sur un plan du début du XVIIIe siècle, il y a cette annotation : Pour vérifier le fief de la chappelle de Ramond « Léon, près le molin dars \footnote{Plan 3097. (Archives municipales de Bordeaux.)}. » Ce moulin était détruit lorsque l'abbé Baurein fit établir son Questionnaire \footnote{Cet ouvrage est à la bibliothèque de Bordeaux.}, quelques années avant la Révolution. Quant au moulin de la Lande, il a disparu au commencement du XIXe siècle. Un petit filet d'eau venant des prairies de Monadey, et portant le nom de Palanquettes, traverse en biais le parc de Margaux, le chemin Frédéric-Sévène, près de Peixotto, et se perd dans le ruisseau de Talence, au milieu du parc de Parthenval. Il tire son nom des palanques jetées pour le passage des piétons \footnote{Le père ROYER. \textit{Notre-Dame de Talence dans la chapelle des Monges au XVIII\ieme{} siècle}.}. On comptait trois ponts à Talence : le pont de Coudourne, sur le chemin de Bayonne ; celui de Talence et celui du Moulin d'Ars, sur le chemin de Castres \footnote{L'abbé BAUREIN. \textit{Questionnaire}.}. 

\subsection{Le nom de Talence}

La plus ancienne mention de Talence que nous ayons trouvée remonte à l'an 1077. A cette date, Geoffroy, duc d'Aquitaine, donna par acte au prieuré de Saint-Martin du Mont Judaïque partie des droits qu'il possédait dans diverses localités, notamment à Talence \footnote{\textit{Archives historiques de la Gironde}, t. XLIX.}.

Le 22 août 1182 fut enregistré un accord relatif à un bois du Bouscat, et dans cet accord, il est question d'une maison bâtie « près de Talence ». Le nom est écrit dans la forme latine : \textit{Talanciam}\footnote{\textit{Cartulaire de Saint-Seurin}.}.

Un contrat de vente du 12 mars 1253 signale : \textit{Jehan de Talansa}, moine infirmier\footnote{\textit{Archives historiques de la Gironde}, t. XXVII, p. 200.}.

On enregistra le 20 mars 1274 la reconnaissance en faveur du roi d'Angleterre d'un terrain qui était situé à \textit{Talanssa}\footnote{Manuscrits de Wolfenbüttel.}.

On trouve dans un rôle gascon d'Edouard Ier, daté de 1286, le nom de \textit{Rostandus de Talencia}\footnote{\textit{Archives historiques de la Gironde}, t. XLIV, p. 42. Charles Bémont, de l'Institut, qui a transcrit et communiqué le document, l'a accompagné de cette note : « Talence aujourd'hui quartier de Bordeaux ». Il est étrange qu'une pareille erreur n'ait pas été corrigée ou tout au moin» soulignée par le Comité de direction des Archives historiques de la Gironde.}.

Le 22 mars 1293, les jurats de Bordeaux prêtèrent serment à Philippe-le-Bel, à condition que ce souverain s'engageât à respecter les privilèges et coutumes de la ville de Bordeaux \footnote{\textit{Livre des Bouillons}.}. Parmi eux était \textit{Petrus de Talancia} (Pierre de Talence).

Un bail à fief d'une terre dans la paroisse de Saint-Genès « \textit{a Talansola}, près le chemin de Saint-Jacques », fut passé le 29 avril 1381\footnote{\textit{Inventaire sommaire des archives de la Gironde}, série G, t. II, p. 127}.

Par acte notarié du 14 décembre 1576, le sieur Guyraud Delaville, laboureur de la paroisse de Talence, reconnut « tenir en fief féodalement de Pierre du Hazera, laboureur de la même paroisse, au nom et comme comte de la confrérie \textit{Sainct-Pey dudict Talance} \footnote{La confrérie de Saint-Pierre-de-Talence.} douze règes de terre de vigne, situées en la paroisse de Talence, au lieu dit à Terrefort »\footnote{\textit{Archives historiques de la Gironde}, t. XXXV, p. 419.}.

Sur un rapport du 2 ventôse an III, concernant l'enregistrement du prix des vins, on lit : « Tallance (Graves), le tonneau 1.400 livres\footnote{\textit{Période révolutionnaire. Inventaire sommaire}, t. II, p. 118.} ».

Le nom de Talence a été, on le voit, orthographié de plusieurs manières. Quelle est son origine ?

De doctes écrivains sont d'accord pour déclarer que le mot « tal » appartient à la langue celtique. Ils lui donnent cette signification « la coupe d'un objet quelconque ».

D'autres savants, étudiant un autre mot de nos ancêtres, « tala le traduisent par coupe de forêt », « ». « Tala » aurait encore le sens de « bois »\footnote{\textit{Société archéologique de Bordeaux}, t. XXII, p. 3.}.

Faut-il voir dans la racine du nom de Talence le mol celte « tala ? »

Dans les temps antiques le territoire devenu celui de Talence était couvert de forêts. On voyait surtout des y bois de chênes. Sans doute y venait-on cueillir le gui? On peut se représenter par la pensée cette solennité : les druides munis d'une serpe d'or coupant sur un chêne à la cime élevée la plante sacrée des Celtes.

La dénomination de Talence pourrait venir des forêts profondes recouvrant jadis le sol de cette localité. Toutefois, le mot « tala » n'a pas eu seulement le sens de bois », « coupe de forêt ». Il a signifié aussi « dégâts », « pillage ». 

La première invasion de Burdigala remonte à 276. C'est alors que les Barbares anéantirent les splendeurs de cette cité et de sa banlieue : temples, villas, thermes, Palais-Gallien. L'aqueduc, dont parle Vinet, fut ravagé à cette même époque. Talence pourrait donc aussi tenir son nom des dévastations qu'y commirent les vandales dans la deuxième moitié du III\ieme{} siècle.

Évidemment, nous sommes loin à présent de l'étymologie se rapportant à la coupe de forêt, au bois des druides... Mais les deux hypothèses sont plausibles\footnote{Il y a quelque trente ans, un calembour, assez drôle pour être rapporté ici, fut lancé sur la scène des Bouffes-Bordelais. On jouait \textit{La Belle-Hélène}, l'amusante opérette d'Offenbach, dans laquelle les interprètes, suivant une tradition, se répandent en plaisanteries de leur goût. Achille, cherchant sa lance, quelqu'un lui dit : « Tu ne la verras plus. — Qu'en a-t-on fait? — On en a fait... le boulevard de Talence ! » }.

\subsection{Une forêt royale}

Il y avait en 1284, à Talence, une «forêt royale»\footnote{Manuscrits de Wolfenbüttel.}. Cette forêt était la propriété des souverains anglais, devenus maîtres de la Guyenne à la suite du mariage d'Eléonore de Guyenne avec Henri Plantagenet, en 1152. Henri Plantagenet monta en 1154 sur le trône d'Angleterre.

Les princes anglais de passage ou en résidence à Bordeaux allaient chasser dans la forêt royale de Talence. Où était-elle située exactement ?

Édouard I\ier{}, roi d'Angleterre, vint plusieurs fois en France, notamment en 1273-1274 et de 1286 à 1289\footnote{L'itinéraire du roi Edouard Ier est connu par un travail très sérieux, exécuté d'après des sources anglaises par Henry Gough. (\textit{Rôles gascons}, t. III, p. 13. Introduction.)}.

En 1286, il s'était embarqué à Douvres, accompagné par Robert Burnel, évêque de Bath ou Baa, et Wells, son chancelier muni du grand sceau\footnote{\textit{Rôles gascons}, t. III, p. 9 et 10. Introduction.}. Il était à Bordeaux le 1\ier{} novembre 1286 ; nous l'y trouvons encore le 29 décembre 1286, les 8 janvier, 3 février, 10 mars et 12 et 15 avril 1287.

Le 10 mai 1288, Édouard I\ier{} est de nouveau à Bordeaux : le 21, \textit{apud Genestan}; le 27, à Bordeaux.

\textit{Apud Genestan} se traduit par « à Genest ». Il s'agit peut-être, dit Charles Bémont, de « Genest, commune de Villeneuve-d'Ornon\footnote{Villenave-d'Ornon.}, près de Bordeaux ». Selon nous, cela ne paraît pas douteux. Genest est pris ici pour la paroisse de Sent-Genest « » — plus tard Saint-Genès-de-Talence — que M. Bémont place, par erreur, dans la commune de Villenave-d'Ornon.

Robert Burnel, l'évêque de Bath ou Baa, laissa des traces de son séjour dans notre région. On lit, en effet, dans les \textit{Rôles gascons} :

« Baa (Gironde), bastide royale dont le nom rappelle celui de l'évêque de Bath, Robert Burnel, son fondateur. Le chemin de Saint-Jacques passait au milieu ; des parties de la forêt royale de Bordeaux étaient situées dans le ressort de cette bastide, ainsi que de Camparian. Deux bourgeois sont nommés l'an 1289 Bernard de Cabados en : et Manseau, tailleur d'habits\footnote{\textit{Rôles gascons}, t. III, p. 111. Introduction.}. »

Édouard Ier fut l'hôte de Baa, comme celui de Camparian, où l'on note sa présence le 22 juin 1288\footnote{\textit{Ibid}., t. III, p. 13. Introduction.}.

Sur un dessin « pour servir à une carte de la Guyenne anglaise vers 1307, d'après les \textit{Rôles gascons} », Baa figure un peu au sud-ouest de Bordeaux, à l'est de Bègles, et nord de Gradignan\footnote{\textit{Ibid}., t. III, p. 125. Introduction.}. 

Il y avait dans la partie méridionale de Talence, au lieu dit « le Cornau », une forêt dont on ne peut dire aujourd'hui ni quelle était l'étendue, ni les limites. Une enquête fut ordonnée au XIIIe siècle pour connaître « les bornes de la forêt de Bordeaux ». C'était, croyons-nous, la forêt de Cornau\footnote{\textit{Rôles gascons}, t. I, p. 10, années 1278-1279.}.

Ribadieu parle du Cornau de la Forêt, petit hameau de la commune de Talence.

« Le Prince Noir, écrit-il, allait chasser les bêtes fauves ou mordantes dont les grands bois de Gradignan, de Talence et de Léognan étaient remplis\footnote{\textit{Les châteaux de la Gironde}.}. »

Un plan\footnote{Plan 3078. (Archives municipales de Bordeaux.)} dressé vers 1770 montre en détail le petit hameau cité par Ribadieu ; on y relève, entre autres indications, « le chemin de Bordeaux au village de Forêt et au château de Thouars ».

Des explications qui précèdent, il ressort que la « forêt royale » existant à Talence en 1286 pouvait être la forêt de Bordeaux ou de Cornau, dont on ignorait au XIIIe siècle la superficie et les limites. On peut, par suite, admettre l'opinion du père Royer, à savoir que le château et le domaine de Thouars, à Talence, occuperaient une partie de l'ancienne bastide de Baa\footnote{\textit{Le Démocrate de Talence}, n° du 4 janvier 1925.}.