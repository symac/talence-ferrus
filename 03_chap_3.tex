%%% Commence page 64 du PDF %%%

\section{LES VILLAS CÉLÈBRES}
\subsection{Peychotte}

Le banquier Charles Peixotto acheta le 5 décembre 1765, a Zacharie Goudal, négociant rue Neuve, une vaste propriété, située à Talence, et où il y avait, suivant les termes du contrat de vente, « une maison pour le maître, logement pour le vallet, et autres bâtiments ; chay, cuvier, jardin, parterres, vignes, pré, aubarède ». La pioche renversa les bâtiments, et l'on construisit à la place, en chartreuse, la villa Peixotto, plus connue sous le nom de « Peychotte ».

Primitivement, le corps de logis, qui présente une double façade, se composait seulement d'un rez-de-chaussée ; il est aujourd'hui élevé d'un étage. Charles Marionneau a laissé cette description des deux façades dans leur état ancien :

« Celle du midi est décorée d'un porche ionique formé de quatre colonnes soutenant un fronton semi-circulaire. Sur les côtés de péristyle développent deux ailes terce se minées par des pavillons ; le tout est couronné d'une large corniche et d'une galerie à balustres dérobant le vue des toitures.

» Au milieu de la façade nord se présente en saillie une demi-rotonde percée de trois portes avec impostes cintrées et donnant accès sur un perron\footnote{\textit{Victor Louis.}}. »

Les travaux effectués depuis lors témoignent de l'habileté déployée par l'architecte pour ne pas trop modifier le caractère de l'édifice.

Peychotte serait l'oeuvre de Louis. C'est « sa manière » ; cependant, on ne possède pas de preuves à ce sujet.

Peixotto connaissait Louis ; l'un des premiers, il avait fait des offres à la Jurade pour l'achèvement du Grand-Théâtre, et ce geste dut, vraisemblablement, lui valoir, sinon l'amitié, du moins la gratitude et la sympathie de l'illustre architecte.

Il existe un plan général de la maison de campagne de Peixotto qui porte la date « septembre et octobre 1769 »\footnote{Archives municipales de Bordeaux.}.

Ce document est sans signature. Il est néanmoins intéressant en ce sens qu'il fixe à peu près la date à laquelle on éleva « la villa Peixotto », car elle est ainsi désignée dans un ouvrage publié en 1785.

C'est un bijou architectural. On admire sur la façade nord, entre la corniche et l'extrados des arcades, une délicieuse ornementation sculpturale faite de médaillons soutenus par des guirlandes de fleurs et de feuillage.

Le parc est magnifique avec ses statues en terre cuite, ses pelouses, ses bosquets, ses charmilles, ses allées propices aux douces rêveries.

Un portail en fer séparant deux petits pavillons marque l'entrée principale du domaine.

Un autre pavillon, dans le style du corps de logis, se dresse à l'angle du cours Gambetta et du chemin Frédéric Sévène ; le promeneur ne passe jamais devant cette construction sans la contempler : il chercherait en vain, ailleurs, une pareille merveille.

Ernest Labadie appelle la villa Peixotto « la princière résidence de Talence »\footnote{\textit{La Presse bordelaise pendant la Révolution}.}.

De tous les petits Trianons qui furent construits en banlieue dans la seconde moitié du XVIIIe siècle, celui de Peixotto était le plus agréable et le plus séduisant. On y donna des réjouissances brillantes. On y reçut l'élite de la société bordelaise\footnote{La « Maison Carrée » d'Arlac, située près des sources qui alimentaient Bordeaux jadis, fut aussi la propriété du riche banquier. Cet immeuble élevé, après 1780, par l'architecte Dufart, s'appelait, primitivement, « la maison domaniale de Peixotto ».}. Les artistes du Grand-Théâtre vinrent jouer la comédie.

Le domaine de Peixotto ou « Peychotte » ce — dernier nom figure sur nombre de pièces d'archives — fut aussi désigné « pavillon d'Aranjuez ». Il fut vendu en 1810 à M. de Saint-Guirons. Il est appelé « pavillon de M. Rodrigues, ci-devant Peixotto, dans un texte du 19 mars » 1813\footnote{Registre des procès-verbaux de la municipalité de Talence. (Archives de la commune.)}. Il eut, par la suite, entre autres propriétaires : M. Denis de Thomassin, la baronne de Espelata (20 avril 1858), le docteur Badal.

Peychotte passa, finalement, dans les mains de la famille de Luze.

À l'origine, le domaine de Peychotte s'étendait du chemin Frédéric-Sévène au chemin de Suzon. Le ruisseau des Malerettes le taversait du nord au sud. Un petit pont formé d'une arche élégante était jeté sur ce cours d'eau. Actuellement, le ruisseau limite, à l'est, la magnifique propriété de M. Edouard de Luze.

En juillet 1923, Peychotte servit de cadre à une grande fête de charité offerte par la Mutualité maternelle de Talence et les Camarades de combat, de la même commune. Mme de Luze, présidente de la première de ces Sociétés, et M. de Luze, président d'honneur de la seconde, montrèrent, en mettant leur bien à la disposition des organisateurs de cette kermesse, le vif intérêt qu'ils portent aux oeuvres sociales et patriotiques.

En cette journée, favorisée par un soleil radieux, Bordelais et Talençais goûtèrent le charme de Peixotto. Les promeneurs, sillonnant les allées sous les frais ombrages, suivant le bord fleuri des Malerettes, regardant les belles statues sommairement drapées ornant le centre des avenues, ou reposant leurs yeux sur la verdure, pouvaient se croire en un séjour idéal ; ils vécurent par la pensée les scènes champêtres ou galantes que Watteau — pour ne citer que lui — sut si délicieusement traduire avec sa palette. 

\asterism{}

D'origine juive, Peixotto avait épousé à Londres, en 1762, une demoiselle Mendès Dacosta. Contre celle-ci il intenta, en 1773, une action en nullité de mariage ; il renonça ensuite à cette action pour signifier à son épouse qu'il la répudiait, et il l'assigna pour voir déclarer la répudiation valable.

Le 20 juillet 1779, le Châtelet de Paris renvoya l'affaire, \textit{avant faire droit}, « devant les anciens de la nation établis à Bordeaux »\footnote{Théophile MALVEZIN. \textit{Histoire des Juifs à Bordeaux}.}.

Peixotto se convertit au catholicisme et se fit baptiser en Espagne le 18 avril 1781 ; il eut le roi d'Espagne pour parrain. Les fonctions de parrain furent remplies par dom Jean Dias de la Guerra, doyen de l'église-cathédrale de Siguenza, \textit{au nom de Sa Majesté catholique}\footnote{\textit{Généalogie curieuse et remarquable de M. Peixotto, juif d'origine, chrétien de profession et banquier de Bordeaux}. Plaquette imprimée 1789 chez Aubanel, imprimeur à Avignon, et communiquée par M. Journu. Le manuscrit de cette généalogie était possession de M. en Fortin, curé de Talence. Deux mois après le décès de cet ecclésiastique, Il fut trouvé dans sa bibliothèque et remis à l'imprimeur.}.

La femme de Peixotto mourut peu après.

Le 16 juin 1782, un avis favorable fut donné par les habitants de Talence à la demande de M. Pexoto, pro« posant de faire agrandir la chapelle Saint-Pierre, d'y faire construire des fonts baptismaux et une chapelle entretenue à ses frais, dans laquelle il aurait un banc » \footnote{Registre des baptêmes, mariages et sépultures de Saint-Genès-de-Talence.\\Plusieurs auteurs ont écrit que Peixotto fit reconstruire l'église de Talence ; il ne s'agissait, en réalité, que de la chapelle Saint-Pierre.}.

L'année suivante, le 25 août, on procéda à la bénédiction de la chapelle qu'avait fait bastir M. Pexoto de « Beaulieu, nouvellement converti »\footnote{Registre des baptêmes, mariages et sépultures de Saint-Genès-de-Talence.}. Plusieurs curés et huit messieurs ecclésiastiques de Bordeaux choisis pour leur belle voix » en rendirent «la bénédiction plus éclatante »\footnote{\textit{Ibid}.}.

À la date du 13 février 1790, Peixotto était colonel du régiment des gardes nationales de Talence\footnote{Registre des procès-verbaux de la municipalité de Talence. (Archives de la commune.)}.

En 1791, D. Antonio Ponz, secrétaire du roi d'Espagne, effectuant un voyage en France, remarqua, avant d'arriver à Bordeaux, la villa de Peixotto, à l'entrée de laquelle était en grandes lettres l'inscription « Aranjuez ». Il se renseigna et recueillit tous les détails que nous connaissons au sujet du riche banquier et de sa conversion. Il sut même que Peixotto avait été instruit dans la religion catholique Par l'évêque de Sigüenza.

Antonio Ponz fit la connaissance du banquier, qui le reçut à dîner chez lui.

Peixotto expliqua à son hôte qu'il avait dénommé sa villa « Aranjuez » en mémoire de la belle résidence que le roi d'Espagne, son parrain, possédait dans la ville d'Aranjuez, sur le Tage. Il montra à Antonio Ponz une statue du souverain espagnol; elle était en stuc, mais il avait l'intention « de la perfectionner et de la faire exécuter dans une autre matière »\footnote{D. Antonio PONZ. \textit{Viage fuera de Espana}, 1791.}.

Mis hors la loi, Peixotto fut arrêté dans les premiers jours d'octobre 1793 ; il avait alors cinquante-trois ans, et était domicilié 32, place Dauphine\footnote{Place Gambetta. La maison a conservé le même numéro.}. Après interrogatoire, le 2 novembre, il fut mis en liberté provisoire, sous la garde de quatre sans-culottes.

Le banquier comparut devant la Commission militaire de la Gironde le 26 frimaire an II (16 décembre 1793). L'accusateur public l'accabla.

« Vous avez été, lui dit-il, un spéculateur comme un tas de négociants. Vous avez employé votre argent à orner un bien de campagne\footnote{Le domaine de Talence.} des statues de Henri IV, de Louis XVI et du roi d'Espagne, et vous voulez nous faire croire à votre patriotisme? Vous avez été un égoïste, voilà tout ! 

» — Je n'ai jamais eu la statue de Henri IV, rectifia  Peixotto, mais bien celles de Charles III et du roi d'Espagne\footnote{Cette mise au point de Peixotto n'est pas claire. Il semble établir une distinction entre Charles III et le roi d'Espagne, alors qu'il entendait bien parler de Charles III, fils de Philippe V, qui régna sur le trône d'Espagne de 1759 à 1788.}. Elles ont été, d'ailleurs, démolies... »

L'accusé discuta énergiquement les faits relevés à sa charge. La Commission militaire le condamna à 1.200.000 francs d'amende, soit un million pour la République et 200.000 francs pour les sans-culottes de Bordeaux. On le garda en prison jusqu'au paiement de cette dernière somme; un délai de trois mois lui fut accordé pour le règlement du million restant. L'empressement que le financier avait mis à acheter des biens nationaux, en 1791, fut sans doute une des raisons pour lesquelles il échappa à l'échafaud.

Peixotto fut, en effet, le plus gros acquéreur de biens « nationaux »\footnote{Ernest LABADIE. \textit{La Presse bordelaise pendant la Révolution}.}.

Le 27 septembre 1791, il avait acheté le château de Lormont, domaine des archevêques de Bordeaux, confisqué Par la Révolution. Nous lisons, à ce propos, dans un opuscule :

« Le directoire du district, ouï, etc., adjuge au sieur Pexotto, plus offrant et dernier enchérisseur, le susdit bien tel que l'archevêque de Bordeaux le possédait ou qu'il était en droit de le posséder\footnote{Cet opuscule fut imprimé à Bordeaux 1807 « chez Lavvalle en jeune, imprimeur libraire, rue Sainte-Catherine, 58 ». (Documents particuliers de M. Pinçon, propriétaire actuel du château de Lormont.)}. »

À la suite d'une renonciation de Peixotto, le château de Lormont mains le 27 septembre 1792, dans les du passa, sieur Bourgade.

Peixotto avait également acheté, en 1791, le doyenné de Saint-Seurin\footnote{Bernadau, dans ses \textit{Tablettes}, écrit, sous la date du 20 fructidor an VI (12 juin 1798) : « Le doyenné de Saint-Seurin a pris feu hier. Tout foin qu'on y serroit a été brûlé. Il appartenait au sieur Peixotto. »}, qui était aussi devenu bien national.

Sur le fronton de sa maison de campagne, à Talence, Peixotto avait fait graver ses armoiries surmontées d'une couronne de comte. La municipalité de Talence lui ordonna de supprimer tous ces motifs.

Le banquier, dans un mémoire qu'il présenta en guise de protestation, exposa qu'il était « le chef de la maison de Lévi, si célèbre dans l'Ancien Testament et dans l'histoire politique de tous les empires, reconnu en cette qualité dans toute l'Europe, l'Asie et dans tous les lieux où la nation juive était établie » ; que le roi d'Espagne l'avait lui-même reconnnu pour chef de la maison de Lévi ; que la marque de la maison de Lévi a toujours été un \textit{pedreal} représentant les douze tribus auxquelles elle présidait; que c'est ce \textit{pedreal} qu'il avait fait sculpter sur le fronton de sa résidence de Talence ; qu'il était prêt à faire enlever la couronne surmontant le \textit{pedreal}, mais que celui-ci ne pouvait être considéré comme des armoiries\footnote{Dans son ouvrage \textit{Victor Louis}, Charles Marionneau prétend que le mémoire de Peixotto, au sujet de ses armoiries, est aux Archives départementales. Ce document n'a pu nous être communiqué. Les extraits que nous en avons cités sont tirés de l'\textit{Histoire des juifs à Bordeaux}, de M. Théophile Malvezin.}.

Malgré ses explications, Peixotto dut faire disparaître non seulement la couronne, mais encore tous les détails de sculpture visibles sur le fronton de sa villa.

Le banquier avait orné la chapelle Saint-Pierre d'un tableau où l'on voyait la sainte Vierge sur une nuée, tenant l'enfant Jésus dans ses bras; le roi d'Espagne lui présentait Paul Peixotto qui était à droite, l'épée au côté. En ruban sortait de la bouche de la sainte Vierge ; on y lisait : « Etant de ma famille, il est juste qu'il me soit présenté par le roi d'Espagne\footnote{Théophile MALVEZIN. \textit{Histoire des Juifs à Bordeaux}.} ».

Ce tableau, qui avait mécontenté le curé Fortin, de Talence, fut enlevé par ordre de l'archevêque de Bordeaux\footnote{\textit{Généalogie curieuse et remarquable de M. Peixotto}, etc.}. 

Le 24 brumaire an V, le Bureau central ayant besoin d'un chêne, pour le planter comme Arbre de la Liberté, en demanda un, en ces termes, « au citoyen Peixotto, sur les Fossés » : 

« Nous avons besoin d'un arbre chêne, symbole de la liberté, pour remplacer celui que des malveillants ont abbattu sur la place Nationale. Vous en avez un, dit-on, dans votre \textit{bois de Talence}, d'une tige et d'une venue assez belle, et d'un âge à espérer qu'étant replanté, il prendra racine, poussera des feuilles nouvelles au printems prochain, et affectera agréablement le coeur et les yeux des bons patriotes et des vrais amis de la liberté.

» Veuillez bien nous donner par écrit la permission de le faire, déraciner avec précaution, et de le faire emporter. Nous nous plaisons à croire que vous l'accorderés et que vous me la ferez parvenir le plus tôt que vous pourrés\footnote{\textit{Période révolutionnaire. Inventaire sommaire}, t. III, p. 129}. »

Peixotto subit de gros revers de fortune. Il mourut le 23 messidor an XIII (12 juillet 1805), en sa demeure « au ci-devant doyené, allée d'Amour, fauxbourg Saint-Seurin »\footnote{Registre des décès. (Archives municipales de Bordeaux.)}. Son acte de décès nous apprend qu'il avait quatre prénoms : Samuel, Charles, Joseph et Paul, qu'il était natif de Bordeaux, « âgé de soixante-cinq ans ou environ, ci-devant banquier, veuf de Sara Mendès Dacosta »\footnote{\textit{Ibid}.}.

Partie du chemin Frédéric-Sévène s'est appelée autrefois ruelle de Pechote ».

\subsection{Raba}

« Talence est le nom d'un agréable bien de campagne situé dans une commune ainsi appelée, à l'entrée des graves Bordeaux. »

Ainsi s'exprime Bernadau dans une relation de la visite qu'il fit à la villa Raba, en l'an XII (1803).

Les Raba, d'origine portugaise, appartenaient à la religion juive. Ils étaient cinq frères : Abraham-Henrique Raba aîné, négociant ; Jacob-Henrique, médecin ; Moïse-Antoine-Rodrigue, négociant ; Aaron-Henrique, négociant; Gabriel-Salomon-Henrique, négociant.

Le livre des \textit{Autographes}\footnote{\textit{Archives historiques de la Gironde}, t. XXX.} fait mention de François Raba junior, né à Bordeaux, en 1742, possesseur d'une grosse fortune et chef d'une importante maison de commerce. Les jurats bordelais le reçurent, comme syndic de la nation portugaise, le 10 avril 1788.

François Raba ne serait-il pas un des cinq frères Raba cités ci-dessus ? Nous inclinons vers l'affirmative.

La villa Raba s'élève dans la partie sud de Talence, à gauche du cours Gambetta, en allant vers Gradignan.

En cet endroit — primitivement appelé Coudourne — il y avait une maison noble dite « la maison Guionnet »\footnote{Suivant une tradition, Henri IV aurait couché dans cette maison l'avant-veille de la bataille de Coutras, en 1587. L'abbé Baurein rapporte qu'un « chemin couvert » allait du château Guionnet au château d'Agès. (\textit{Questionnaire}.)} qui fut achetée sous Louis XVI par les Raba. Ceux-ci la démolirent et firent bâtir à la place la villa qui porte leur nom.

Cette villa fut meublée avec un grand luxe. On y établit à côté une salle de concert ; les tapisseries de cette salle disparaissaient sous une profusion de tableaux. Le jardin fut peuplé de statues.

Le domaine de Raba était un véritable temple de l'art. Sur le piédestal d'une statue de l'Amour, placée dans une niche, on lisait ce distique bien connu : 

%%% TODO : vérifier la mise en forme %%%
\begin{center}
Qui que tu sois, voici ton maître, Il l'est, le fut ou le doit être.
\end{center}

Des allées ombreuses permettaient de circuler en tous sens dans le parc, où l'on voyait, outre les oeuvres sculpturales, des volières, des pièces d'eau, des charmilles, un grand salon de verdure », des parterres idéalement fleuris, un labyrinthe, le pavillon des muses. En sortant du labyrinthe, on arrivait dans un bosquet à l'entrée duquel était ce quatrain : 


\begin{center}
Sous ce berceau délicieux,\\
L'amant auprès de sa bergère\\
Peut souvent oublier la terre,\\
Quand il voit le ciel dans ses yeux.
\end{center}

Le jardin était, de plus, agrémenté d'installations dans le goût de Trianon, figurant un moulin, la cabane de l'Enfant prodigue, une bergerie, la Petite Charité, le « Temps découvrant la Vérité », etc.

Des animaux domestiques et sauvages étaient représentés sur les pelouses, dans les bosquets. La vue des bêtes féroces — factices — n'était pas, parfois, sans causer quelque émotion aux visiteurs non avertis.

Le domaine des Raba était le « Chantilly bordelais ». Les étrangers s'y rendaient en foule ; ils y étaient fort aimablement reçus par les propriétaires qui se plaisaient, d'ailleurs, à faire admirer toutes les oeuvres d'art, tous les bibelots de valeur, toutes les curiosités réunies chez eux.

Le Parlement de Bordeaux se rendit en corps à Raba. Des personnages remarquables y furent accueillis. On cite parmi eux :

Dupaty, président à mortier au Parlement de Bordeaux, auteur de plusieurs ouvrages, entre autres des \textit{Lettres sur l'Italie}, qui eurent une grande vogue.

François de Neufchâteau (Nicolas-Louis), homme politique et littérateur, membre de l'Académie française. l'âge de douze ans, François de Neufchâteau fit des vers qui lui valurent les encouragements de Voltaire.

Raynal, historien et philosophe, qui fréquenta les salons d'Helvétius, d'Holbach et de Mme Geoffrin. Ayant reçu les ordres, il avait quitté l'état ecclésiastique pour entrer comme rédacteur au \textit{Mercure de France}.

Léonard (Nicolas-Germain) poète élégiaque, originaire de La Guadeloupe.

Maréchal (P. Sylvain), écrivain, né à Paris. Chaud partisan de la Révolution, il fut un des chantres de la liberté et de la déesse \textit{Raison}. Maréchal était particulièrement lié avec l'astronome Lalande.

Beaumarchais, qui se fit une grande réputation dans le monde par ses factums, qui eurent un succès prodigieux, et par ses pièces de théâtre : le \textit{Barbier de Séville} et le \textit{Mariage de Figaro}.

Pour un des coins mystérieux de Raba, Beaumarchais écrivit cette dédicace : 

\begin{center}
Sur cet agréable gazon,\\
Jeune beauté qui se repose\\
Quelquefois perdra la raison,\\
Et plus souvent bien autre chose. 
\end{center}

\asterism{}

Pendant la disette qui sévit fin 1792 et en 1793, des souscriptions furent ouvertes. Les frères Raba versèrent 20.000 francs\footnote{Leur ex-coreligionnaire Peixotto fut moins généreux dans cette circonstance. Il versa seulement... 800 francs.}.

Dans la nuit du 6 au 7 octobre 1793, les frères Raba furent appréhendés et emprisonnés.

L'aîné, Abraham-Henrique, avait alors soixante-quatre ans ; ses autres frères soixante ans, cinquante-huit ans, cinquante-six ans et cinquante-deux ans. Leur domicile commun à Bordeaux était « 67, Fossés de Ville »\footnote{Portion du cours Victor-Hugo comprise aujourd'hui entre la rue Saint-James et la rue Sainte-Catherine.}.

Ils comparurent ensemble devant le tribunal révolutionnaire le 9 brumaire an II (30 octobre 1793). Il résulte des pièces du procès qu'ils avaient été arrêtés comme \textit{suspects d'accaparement, d'agiotage et d'avoir fourni des sommes considérables pour la force départementale}.

Lacombe leur demanda s'ils n'avaient pas été partisans de la Commission populaire.

Gabriel-Salomon-Henrique Raba répondit :

« Mon frère le médecin et moi étions à la campagne lors de la formation de cette Commission, et nous n'avons pu, par conséquent, adhérer à ses actes. Pour moi, je fais ma résidence à Talence, et j'y suis commandant de la garde nationale de cette commune. »

Le tribunal estima que les frères Rabà avaient encouragé la Commission populaire et participé à ses crimes ; il considéra, néanmoins, qu'ils avaient donné depuis le commencement de la Révolution des preuves de patriotisme ; qu'ils n'avaient fait aucune de ces opérations qui avaient tant contribué à discréditer les assignats et à augmenter d'une manière effrayante les prix des denrées de première nécessité.

Les frères Raba furent condamnés solidairement à une amende de 500.000 livres, dont 400.000 pour les frais de l'armée révolutionnaire, payables dans trois mois, et 100.000 envers les sans-culottes de toutes les sections de la ville, payables en quinze jours. On les remit sur-le-champ en liberté.

Le 13 juillet 1794, les frères Raba offrirent un dîner dans leur domaine, à Talence. Ils furent dénoncés pour ce fait. 

En son nom et au nom de ses frères, Gabriel-Salomon-Henrique Raba écrivit, le 17 juillet 1794, à Garnier, le nouveau conventionnel envoyé à Bordeaux, une lettre dont voici les passages essentiels :

« CITOYEN REPRÉSENTANT, 

» Deux fois dans les séances du Club national, nous avons été dénoncés pour avoir donné à dîner à quelquesuns de nos concitoyens, et deux fois les faits y ont été dénaturés. La vérité et l'estime de nos concitoyens nous sont très chères pour ne pas chercher à dissiper les impressions fâcheuses que l'on a gravées dans le coeur du citoyen Jullien... témoim'avaient 

» Depuis plus d'un mois, quelques amis m'avaient témoigné le désir de venir passer la journée dans notre campagne. L'envie de célébrer les victoires de nos armées nous fournit l'occasion de. les rassembler quintidi dernier avec quelques autres citoyens, excellents républicains, voisins de notre campagne. 

» L'on a dit que le dîner était riche et somptueux ; l'on nous a calomniés. Quelques vases de fleurs naturelles en étaient le plus bel ornement ; il fut d'un seul service. Le nombre des mets n'était même pas proportionné à celui des convives, et il ne s'y trouvait aucune de ces productions rares et chères qui faisaient tous les mérites des repas d'autrefois... 

» Il ne s'y trouvait d'autre argenterie qu'un huilier et environ dix-huit couverts d'argent. Depuis longtemps, nous avons porté le reste à la Monnaie.

» Je peux donc vous assurer que ce fut un vrai repas à la sans-culotte dont le luxe, la vanité et tout discours politique y fut banni ; la frugalité, la fraternité et la gaîté républicaine en firent tous les frais ; l'on but quelques »toasts patriotiques, l'on chanta des cantiques républicains, et à huit heures et demie, tout le monde s'était retiré. On dit qu'il y avait cent convives : c'est une calomnie...

» La table n'était que d'environ 40 personnes, dont 10 enfants et 10 femmes.

» Ce repas est le premier que nous ayons donné depuis plus d'un an, et nous défions de prouver que depuis cette époque aucun homme en place ait dîné chez nous, excepté le citoyen Roman\footnote{Roman était commissaire des guerres.}, une seule fois...

» Une remarque importante qui détruit tous les soupçons, c'est que les citoyens Plénaud, Lacombe\footnote{Lacombe (Jean-Baptiste), trente-trois ans, instituteur, demeurant à Bordeaux, place Nationale, président de la Commission militaire de la Gironde (Tribunal révolutionnaire).}, Rey\footnote{Rey (Jean), trente-trois ans, capitaine au 10e régiment de chasseurs, logé au ci-devant séminaire, membre du Tribunal révolutionnaire.} et Hennuyer\footnote{Hennuyer, secrétaire de Lacombe.} étaient invités verbalement quinze jours avant la réorganisation de la Municipalité et du Club national, réorganisation que nous ignorions absolument.

» On nous accuse d'avoir laissé ignorer à chacun des convives le nom de ceux avec lesquels il devait se trouver chez nous. La plupart des billets d'invitation démentent cette inculpation.

» Quant à notre bien de campagne, il doit être plutôt considéré comme un jardin public que comme une propriété particulière ; en effet, loin d'être le rendez-vous de l'intrigue, les sans-culottes s'y rendent en foule les jours de fête, et vont s'y récréer des fatigues de la décade, et si, quelquefois, nous avons projeté de le dénaturer, nous n'en avons été empêchés que par le regret d'ôter à nos concitoyens cette jouissance\footnote{Archives départementales de la Gironde, série L.}.

À leur lettre, les frères Raba avaient annexé le tableau suivant :

« Citoyens invités par billet ou de parole : 

» Marguerié\footnote{Marguerié (Antoine), quarante-cinq ans, marchand à La Réole, membre du Tribunal révolutionnaire.}, Thomas, Morel\footnote{Morel (Jacques), quarante-trois ans, doreur, rue Saint-Martin, 33, membre du Tribunal révolutionnaire.}, Cardoze, Parmentier\footnote{Parmentier (Jean-Charles), vingt-cinq ans, comédien, 19, rue Bouhaut, membre du Tribunal révolutionnaire.}, Giffrey\footnote{Giffrey (François-Gauthier), quarante et un ans, rue du Cahernan, 7, greffier en chef de la police correctionnelle.}, Robles, Plénaud, Jay, Pereyre et son épouse, voisin de campagne ; Veissière aîné, du district voisin de la Ville.

» Ces onze citoyens ne vinrent pas.

» Nombre de ceux qui se sont trouvés à dîner :

» Le Commandant Joyeux, cultivateur, voisin de la campagne ; Roman ; Morel, dentiste ; Bermingan ; Lorrendo. Marchand, de Pessac ; Dassas\footnote{Dassas ou Darsas. Il s'agit, croyons-nous, de Barsac (Guillaume), trente-trois ans, commis, rue Doidy, 32, membre du Tribunal révolutionnaire. Ne connaissant probablement pas ce personnage, Gabriel-Salomon-Henrique Raba dut estropier son nom.} ; Rey ; Hennuyer, le maire de Talence ; le maire de Pessac ; Mandavy, de Gradignan ; Lacombe, président ; Baylle-Huscard, commis de la maison, et un autre Huscard, son camarade.

» Une partie de ces citoyens étaient accompagnés de leurs femmes, de leurs enfants, ou de quelque ami qu'ils invitèrent eux-mêmes. De sorte que, y compris les femmes. les enfants et 5 de la famille Raba, nous n'étions en tout que 40 à 42 personnes à dîner.

» Les citoyens Raba n'ayant pas le plaisir de connaître Jullien, étant persuadés de celui que sa présence aurait fait à leur société, avaient prié les citoyens Lorrendo et Bermingan, leurs amis, de l'y amener. »

L'affaire n'eut aucune suite fâcheuse pour les Raba. Seul, « Raba l'Amériquain »\footnote{Le plus jeune des frères Raba, Gabriel-Salomon-Henrique, était surnommé « Raba l'Américain », parce qu'il avait séjourné à Port-auPrince (capitale de la république d'Haïti). Note de M. Cardozo de Bethencourt.)} fut arrêté pendant quelques heures et traduit devant le Comité révolutionnaire de surveillance. Il dit être âgé de cinquante-quatre ans, et natif de Bragance (Portugal)\footnote{\textit{Annales de la Terreur à Bordeaux. Tribunal révolutionnaire}, t. XXX, p. 145 bis.}. Il précisa que Lacombe, président du Tribunal militaire, avait été invité avec toute sa famille. D'autre part, les cinq frères avaient chargé Roman d'inviter en leur nom le général Beguinot\footnote{Simple soldat sous l'ancien régime, Beguinot ou Béguignot était général à la Révolution. On l'avait surnommé « ventre d'argent » parce qu'il portait une plaque de ce métal à l'abdomen où il avait reçu une blessure qui n'avait pu se cicatriser. Beguinot reçut le titre de comte sous l'empire et fut nommé sénateur.}, qui n'était pas connu d'eux.

On lit au dos du procès-verbal de l'interrogatoire de « Raba l'Amériquain » : « \textit{Renvoyé} chez lui en liberté ».

Il est possible que le festin des Raba n'ait pas été « magnifique », contrairement à l'opinion émise par M. E. Labadie\footnote{\textit{La Presse bordelaise pendant la Révolution}, p. 135.} ; mais il n'en est pas moins vrai que la fête avait été organisée en l'honneur des Jacobins bordelais puisque tous les membres du Tribunal révolutionnaire y avaient été conviés.

Au 13 juillet 1794, la Terreur sévissait avec le plus d'intensité. On s'explique, au demeurant, que les Raba, toujours menacés parce que très riches, aient voulu se ménager la sympathie des terroristes, dont le règne devait finir, du reste, quatorze jours plus tard (9 thermidor).

\asterism{}

Durant un séjour qu'ils firent à Bordeaux en 1796, les citoyens Duplaa et Forcade, d'Orthez; leurs femmes et leurs amis Lapeyre et Engelhseim allèrent voir « le beau bien de Raba »\footnote{Le manuscrit est conservé aux Archives municipales de Bordeaux. La visite des citoyens d'Orthez à Raba eut lieu entre le 9 ventôse (28 février) et le 10 germinal an IV (30 mars 1796).} ; ils s'étaient procuré un billet pour franchir le seuil du domaine.

L'une des personnes ci-dessus désignées lit un récit de sa visite à Raba. On en lira ci-après d'intéressants extraits :

«... Nous primes la droite de la maison; à côté, nous commençames de voir deux grandes et jolies cages à oiseaux ; il ni avait dans l'une que des lapins privés, l'autre étoit vide, sans doute à cause de la rigueur de la saison. Deux écriteaux exprimaient l'emploi auquel ces deux cages étaient destinées. A une petite distance de là, un arbre superbe excita notre curiosité ; nous en approchâmes, nous assîmes même sur des bans qui l'entouraient et nous y lûmes cet écriteau : 

\begin{center}
Oui le voici, cet orme heureux\\
Où ma Louise a reçu mes voeux.
\end{center}

» Nous vîmes ensuite le réservoir, qui est très beau ; mais le temps qui était très sombre — même il tombait quelque goutte de pluie — empêcha que nous vissions du poisson. Nous étions là à peut-être six pas de la maison, sur le derrière. Nous nous y rendîmes sur le perron. Que d'objets se présentèrent à notre vue ! Un parterre magnifique ; au bout est un cabinet de muses et sur lequel on voit la renommée annonçant avec sa trompette la victoire. Nous promenâmes le parterre, où des ouvriers étaient occupés et nous fûmes au cabinet des muses où nos citoyennes furent frappées d'admiration des jolies choses qu'il y avoit. Après y avoir examiné et reexaminé les belles choses qui y sont, nous sortîmes pour aller au labyrinthe, mais
sans verdure et les haies sans feuille. Nous pouvons bous vanter de ne pas nous y être égarés. Nous en sortîmes après y avoir longtemps couru et fûmes promener dans un petit bois, à droite du labyrinthe.

» Forcade, Lapeyre et la citoyenne Chinette firent au plus courir dans une allée qu'il y a; mais quelle fut notre surprise d'entendre la dite citoyenne crier et revenir à bous. Nous ne fûmes pas longtemps sans savoir le motif, et qu'un léopard en était la cause. Elle crut qu'elle l'avoit à ses trousses pour la dévorer, mais rassurée bientôt en apprenant que ce n'étoit là que la forme, elle fut tranquille. Nous fîmes là une petite pose et rîmes bien de la peur de la citoyenne Chinette.

» Nous continuâmes notre promenade jusqu'au bout de l'allée qui est d'une longueur immense et en lisant toutes ces inscriptions qui sont attachées à des arbres ; elles sont analogues au séjour délicieux de la retraite et du délassement. Arrivés au bout, nous prîmes là une autre allée à gauche qui nous conduisit à un moulin à vent ; mais avant d'y arriver nous visitâmes quelques petites cabanes des habitants du bois, entre autres celle de la petite charité, c'est-à-dire un mannequin qui, un chapeau à la main, demande la charité...

» Nos citoyennes n'osaient pas entrer d'abord dans aucune des cabanes, mais à la fin elles s'y habituèrent et les parcouraient toutes.

» Au moulin, nos citoyennes furent contentes et rirent bien de voir le meunier et la meunière, chacun à une fenêtre; de loin, elles les auraient pris pour des personnes en vie, mais ce n'est que des mannequins vêtus.

» La vue est assez belle et porte particulièrement sur deux petits prés : dans l'un, on voit la forme d'une vache couchée, et dans l'autre, des animaux propres aux charrois du moulin.

» Nous continuâmes à parcourir le bois, et a nous arrêter dans toutes les cabanes où il y avait particulièrement des écriteaux. Nous arrivâmes à celle de l'Enfant prodigue, où la pluie nous obligea d'entrer... »

De retour au logis, ils visitent les appartements : 

« Que de belles choses ! Là était un petit cabinet de curiosités et tout à fait intéressantes pour notre citoyenne qui n'avait assez de ses yeux pour en examiner toutes les beautés. A côté de celui-là, il y en avait un autre, et encore plus beau. En un mot, nous restâmes plus d'une heure dans l'intérieur de la maison, et toujours occupés à voir des objets qui vraiment frappaient la vue. La pluie cessa dans l'intervalle, et nous profitâmes du retour du beau temps pour promener dans le jardin. La nature ne nous offrit que des objets artistement arrangés. Ici, nous ne devons pas passer sous silence un mouvement de la citoyenne Chinette. Etant à examiner et voyant une poule croupie lui jette une pierre en s'écriant avec cette vivacité qui lui est ordinaire : « Es-tu comme le reste, immobile ou une poule réellement ? » Sa fuite nous fît bien rire, et elle fut surprise de sa surprise. 

» Après avoir bien couru dans le jardin et confié nos noms sur un arbre d'une allée attenante audit jardin, nous fîmes la partie d'aller au Pont-de-la-Maye voir là une manufacture d'indienne qu'il y a... » 

Sous l'empire, la villa Raba connut la grande vogue qu'elle avait eue au crépuscule de l'ancien régime.

Bernadau qui apprécia tout le charme de Raba, en 1803, nota qu'une centaine de voitures stationnaient à la porte de ce domaine quand il s'y présenta.

Se rendant à Bayonne, lors des guerres d'Espagne, Napoléon s'arrêta pendant quelques jours à Bordeaux où il était arrivé le 4 avril 1808. Le samedi 9 avril, accompagné par sa garde d'honneur à cheval, Napoléon fit une »promenade à Pessac, puis à Talence. Il visita Raba. Après s être reposé un court moment dans les jardins de cette belle résidence, il revint à Bordeaux « par le chemin de Bayonne ».

Joséphine, ayant rejoint l'empereur à Bordeaux, voulut aller voir le domaine de Raba. Elle s'y rendit le 14 avril, ainsi qu'il résulte de l'article ci-après :

« Bordeaux, 15 avril. S. M l'impératrice-reine est allée hier matin, escortée d'un détachement de la garde d'honneur à cheval, visiter à Talence l'agréable maison de campagne de M. Raba, que l'empereur avait déjà honorée de sa présence. S. M. s'est promenée dans les jardins, et est entrée dans les appartements, où les propriétaires de ce lieu de plaisance ont eu l'honneur de la recevoir\footnote{\textit{Le Moniteur}, n° du 21 avril 1808.}. » 

Joséphine était accompagnée par Mmes de MontmorencyMatignon et Maret, dames du Palais, et Gazam et Fauchet. Elle resta plus d'une heure à Raba. Très sensible à l'accueil empressé de ses hôtes, elle promit de leur envoyer, en remerciement, un buste de Napoléon en porcelaine de Sèvres.

Charles Saunier assure que les frères Raba comptaient parmi leurs correspondants de la Martinique la famille de la Pagerie, et qu'ils connaissaient, par conséquent, Joséphine de Beauharnais.

« Quand, devenue impératrice, elle passa à Bordeaux, écrit-il, c'est chez eux, en leur château de Talence, \textit{qu'elle descendit} »\footnote{\textit{Les Villes d'art célèbres. Bordeaux}.}. Elle leur fit une simple visite, voilà la vérité.

Le 14 avril 1823, Raba reçut la duchesse d'Angoulême\footnote{\textit{Le Mémorial bordelais}. La duchesse d'Angoulême, fille de Louis XVI et belle-fille de Charles X, était d'un caractère énergique. Elle était, suivant le mot de Napoléon, « le seul homme de sa famille ».}.

Ayant repris la route de Bordeaux, la duchesse s'arrêta un instant à l'église de Talence où elle fut saluée par le curé de la paroisse, M. Ripolles. Celui-ci prononça une allocution. La princesse, frappée par son accent, lui demanda s'il était espagnol. 

« Je suis né Espagnol, répondit le curé, mais fils adoptif de la France, et par conséquent de S. A. R. ».

La villa Raba fut longtemps à la mode ; on la visitait encore en 1840. Après la mort de « Raba l'Amériquain », qui paraît avoir été son principal fondateur, la villa déclina.

H. Ribadieu écrivait en 1856, à propos de Raba :

«... Cette splendide demeure, qui s'éleva sons le règne de Louis XVI, et qui a conservé à peine un reste de sa première magnificence\footnote{\textit{Les châteaux de la Gironde}, p. 587}.»

Ribadieu exagérait. Encore de nos jours, le domaine de Raba est digne d'admiration.

Le portail d'entrée en fer forgé est encadré de pilastres monumentaux flanqués eux-mêmes de portillons en quart de cercle. Dans une sorte de médaillon, surmontant la grille, on aperçoit, dorée, la lettre R.

Une longue avenue d'ormeaux conduit à la villa qui présente un corps de logis principal à proportion de château. Il y-a un salon de musique isolé dans la verdure.

De belles allées traversent la propriété où l'on voit parterres, boulingrins, bosquets, grottes, pièces d'eau, statues, motifs décoratifs. C'est un enchantement des yeux.

Les propriétaires actuels, MM. Albert Ellissen et son fils Robert, ont redonné à Raba — en admettant que son attrait ait été diminué — sa splendeur primitive. Ils ont respecté la villa et ont fait édifier à proximité un logis qui est masqué par des arbres. Le domaine de Raba est d'ailleurs tellement en recul du cours Gambetta qu'on ne peut, du portail d'entrée, s'en faire une idée exacte. Il disparait presque complètement dans les frondaisons. Cependant on devine qu'il y a, là, une délicieuse résidence.

Les Raba dépensaient annuellement 15.000 francs\footnote{15.000 francs sous le premier Empire représenteraient aujourd'hui environ 100.000 francs.} pour l'entretien de leur bien. Ils étaient heureux d'y séjourner et d'y voir surtout des promeneurs.