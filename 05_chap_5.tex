%%%% Commence page 110 du PDF

\section{LES ÉDIFICES RELIGIEUX}

\subsection{L'ancienne église Saint-Genès de Talence}
La paroisse Saint-Genès-de-Talence dépendait de l'archiprêtré de Cernès. Le chef-lieu de cet archiprêtré était la paroisse de Saint-Pierre-de-Gradignan\footnote{Le curé de Gradignan avait le titre d'archiprêtre de Cernès.}.

La paroisse Saint-Genès-de-Talence comprenait, outre l'église matrice : la chapelle Notre-Dame-de-la-Rame, le prieuré de Bardenac et la chapelle Saint-Pierre.

L'église matrice Saint-Genès-de-Talence — Gleysa Sent Genest — était située « à l'ouest du chemin de Saint-Genès\footnote{Rue de Saint-Genès.}, entre la rue des Treuils et les maisons qui sont en face de la rue Duluc, autrefois rue Vigneron »\footnote{Léo DROUYN. \textit{Bordeaux vers 1450}.\\
	Selon Charles Chauliac, l'église Saint-Genès, dont on voyait encore les ruines au commencement du XIXe siècle, était construite « sur l'emplacement actuel de la maison de la rue Saint-Genès portant le numéro 203, entre les rues Solférino et de Ségur ». (\textit{Étude sur les croix des carrefours de Bordeaux}, 1881.)} . Sa position à cet endroit est certifiée par un « plan géométrique de partie du plantier de Saint-Genez »\footnote{Plan n° 1250. (Archives départementales.)}. Pierrugues, sur son plan de 1819, la place sur le même point, après la rue de Ségur, en face de la « rue du Luc ». Un chemin la séparait du territoire de Sainte-Eulalie.

L'église Saint-Genès existait au XIII\ieme{} siècle, comme l'atteste un acte de 1287, par lequel Amanieu d'Escure donna aux religieuses de Sainte-Claire « une rente sur des vignes situées entre Saint-Nicolas et Saint-Genès\footnote{Léo DROUYN. \textit{Bordeaux vers 1450}.} ». Elle était « d'une passable grandeur, mais pas assés considérable pour le nombre des parroissiens »\footnote{Abbé BAUREIN. \textit{Questionnaire}.}. Sa structure n'offrait rien de remarquable.

Il y avait deux autels collatéraux : l'un était dédié à la Sainte Vierge, l'autre à Saint Barthelemi. Il y avait deux clochers, un petit, à côté du choeur\footnote{Ce petit clocher menaçait « ruine prochaine » en 1788. (Cahier de visite, G. 646. (Archives départementales.)}, et un plus grand.

L'église avait naturellement son cimetière, dont il est fait mention dans des actes de 1341\footnote{\textit{Société archéologique de Bordeaux}, t. XXXIX.}.

À la fin de l'année 1545, ou tout au début de 1546, le Parlement avait rendu un arrêt condamnant plusieurs locataires d'une maison située rue « Bouau » à payer à Guilhem de la Ville et Pierre du Hazera, « scindics des fabriqueurs de l'église parroissiale de Saint-Genès-de-Talence, la somme de soixante dix livres dix sols cinq deniers parisis ». Les locataires, parmi lesquels était Gratienne Mosson, devaient, en outre, évacuer l'immeuble.

Le 12 février 1546, un huissier fut chargé de faire exécuter l'arrêt de la cour. Cet huissier, accompagné de Guilhem de la Ville et Pierre du Hazera, se transporta le mercredi 6 avril 1546 « avant Pasques », dans la maison en question et parlant aux locataires leur demanda « pourquoi ils n'avaient pas vuidé »\footnote{Archives départementales, G, 3146.}.

En 1677, Sainct-Aignan, prêtre, vicaire perpétuel de Saint-Genès-de-Talence, enregistra un bref portant concession d'indulgences à ladite paroisse et daté du 19 octobre 1676\footnote{Actes pontificaux. Série G, 525. (Archives départementales.)}.

Suivant une commission de Louis d'Anglure de Bourlemont, archevêque de Bordeaux, primat d'Aquitaine, en date du 19 juin 1691, Henry Chapotel, prêtre et archiprêtre de Cernès, curé de Gradignan, visita, le 23 septembre 1691, l'église Saint-Genès-de-Talence et la chapelle SaintPierre; il rendit compte de l'état de ces deux édifices, ainsi que du mobilier qu'ils contenaient\footnote{G. 646. Archives départementales.}.

Ayant lu le procès-verbal de visite de M. Chapotel, l'archevêque de Bordeaux décida, le 1er juin 1693, en ce qui concerne Saint-Genès-de-Talence, que les « pierres sacrées des autels de Notre-Dame et de Saint-Barthelemy seraient avancées de quatre doigts sur le devant desdits autels ».

D'autre part, le cimetière « tourné du costé du nord » serait fermé d'une muraille, et l'on placerait des grilles aux deux entrées\footnote{\textit{Ibid}.}.

Du 1\ier{} décembre 1766 au 6 décembre 1767 « le grand ouvrié de l'église Saint-Genès-de-Talance » — ou trésorier de la fabrique — fut André Dubos, dit « Baron », qui eut pour successeur Pierre Prévôt\footnote{Livre des comptes de la paroisse Saint-Genès de Talance 1730 à 1793, folio 153. (Archives départementales.)}.

André Dubos rendit compte « de la recette et depance par luy faite en laditte calitté, en présence et du consentement de messire de Gourgue de Touars, présidant en la première chambre des enquêtes du Parlement de Bordeaux, et syndic honoraire de la paroisse » ; de M. Fourtin, curé de Saint-Genès; Jean Perrens, syndic honoraire, et des propriétaires et habitants de la commune « convoqués, et assemblés au son de la cloche en la manière accoutumée »\footnote{\textit{Ibid}.}.

Le 13 janvier 1788, Simon Langoiran, vicaire général du diocèse, visita l'église Saint-Genès. On relève sur son procès-verbal\footnote{Cahier de visite G. 646. (Archives départementales.)} les observations suivantes :

Registre des baptêmes, mariages et sépultures : « Ils sont très mal écrits parce que M. le curé veut transcrire tous les actes, et que sa main tremblante ne lui permet pas de le faire comme il serait nécessaire\footnote{Il s'agissait de l'abbé Fortin, « écrivain de plusieurs documents paroissiaux si difficiles à déchiffrer, même de son temps, paraît-il, que l'archevêque de Bordeaux fil paraître une ordonnance pour lui interdire la rédaction des registres de sa paroisse... ce qui fut d'ailleurs la cause qu'ils ne furent plus tenus depuis cette époque ».\\C'est la \textit{Société archéologique de Bordeaux} (t. XXXIX), qui souligne ce détail, inexact du reste, car les archives communales possèdent le registre des mariages, naissances et décès de la paroisse Saint-Genès-de-Talence allant de 1786 à 1792 (G G, 26, in-4°, 184 feuillets).}. »

Cimetière : « En bon état, à la réserve de vingt brasses de mur du côté du couchant qu'il faudrait, faire. »

Reste net, charges acquittées : « Les charges excèdent les revenus ».

Le même procès-verbal fait savoir qu'il n'y avait point de maître d'école, de maîtresse d'école, de sage-femme, d'hôpitaux, de monastères d'hommes et de filles, de « commanderies et lieux en dependans ».

Enfin, on lit sur le même document :

« Chapelles publiques hors de l'église : il y en a une à Bardanac où se rendent le jour de Saint-Marc les processions de Talence, Gradignan et Pessac ».

« Objets et fondations : Il y a trois fondations pour les pauvres : la première de 12 livres de rente laissée par M. Tolan, ancien curé de Talence ; la seconde de 1.200 l. de capital, laissée par M. Dufau, grand chantre, laquelle somme produit une rente de 60 livres, exactement payée ; la troisième de 1.200 l. de capital, donnée par Mme la Présidente Fossier ; cette rente est « arriérée », en sorte qu'il est dû 400 l.

» Il y a aussi : 1° une messe haute pour M. Tolland, ancien curé de Talence ; 2e six messes basses dont l'honoraire n'a pas été payé depuis plus de cinquante ans, quoiqu'il y ait des fonds existants sur lesquels ledit honoraire est établi ; 3° une messe pour Mme de Chauvignac ; 4° une pour Me Petit, bourgeois ; 5° une autre dont l'honoraire est payé par Me Bobérat. »

Un historien, l'abbé Baurein, fut vicaire à Talence pendant deux années. Les actes du registre des baptêmes, mariages et sépultures de la paroisse Saint-Genès-de-Talence, durant les années 1738 et 1739 sont, en effet, signés « Baurein, vicaire »\footnote{L'abbé Baurein, originaire de Bordeaux, y est mort en mai 1790, rue du Hâ. On lui doit les Variétés bordeloises, ouvrage en plusieurs volumes publiés de 1784 à 1786.}.

Le 28 pluviôse an II, la municipalité et les membres du Conseil de la commune de Talence arrêtent, « en conséquence d'une lettre des administrateurs de Bordeaux du 22 du même mois, qu'il sera fait état des biens nationaux provenant de diverses suppressions des corps et communautés, et qui demeurent invendus »\footnote{Registre des procès-verbaux de la municipalité de Talence. (Archives de la commune.)}. En tête de la liste de ces biens figure le lot suivant : « Eglise Saint-Genès, cimetière, maison attenante, cour, jardin et vignes, estimé 20.000 livres »\footnote{\textit{Ibid}.}.

Il y eut une autre estimation du même lot, le 18 pluviôse an III ; elle atteignait celle-ci 20.250 livres\footnote{M. MARION, J. BENZACAR, CAUDRILLIER : \textit{Documents relatifs à la vente des biens nationaux}, t. I\ier{}.}.

Le 12 messidor an III, M. Rolland, négociant rue Baut, 273, fut déclaré adjudicataire, pour la somme de 110.000 livres, de l'église Saint-Genès, du cimetière, d'une maison attenante au cimetière et de 3 journaux, 27 règes, le tout appartenant à la fabrique\footnote{M. MARION, J. BENZACAR, CAUDRILLIER : \textit{Documents relatifs à la vente des biens nationaux}, t. I\ier{}.}. M. Rolland reçut les clés de l'église le 15 thermidor an III.

Un autre bien de la même fabrique (28 r. vigne à Talence) mis aux enchères le 12 messidor an III et estimé 1.000 livres, fut adjugé à M. Sandre, rentier, 44, place Nationale, pour 1.600 livres.

Le vocable de l'église Saint-Genès rappelait soit saint Genès qui fut histrion à Rome ; soit saint Genès d'Arles\footnote{Le Bordelais Paulin, évêque de Nole, auteur de \textit{Lettres} et \textit{poésies latines}, a écrit aussi une \textit{Histoire du martyre de saint Genès d'Arles}.}, greffier public à Arles, au quatrième siècle; soit encore saint Genès qui fut évêque de Clermont.

Le premier, l'histrion, jouait un jour devant Dioclétien une parodie indécente des cérémonies du christianisme, lorsqu'il fut frappé d'une vision intérieure et se convertit à la foi nouvelle. Il subit le supplice du chevalet en 286 ou 303.

Saint Genès, le greffier, avait été chargé de transcrire un édit de persécution contre les chrétiens, signé à Arles par Maximien. Il s'y était refusé, en sa qualité de catéchumène et avait pris la fuite. Arrêté, il fut décapité. L'église célèbre le 25 août la fête de saint Genès.

Cette fête concerne les deux martyrs : l'histrion romain et le greffier d'Arles.

Quant au troisième saint Genès, l'évêque de Clermont, qui mourut vers 662, sa fête a lieu le 3 juin.

\subsection{La croix de Saint-Genès}

Elle se dressait au carrefour formé par la jonction de la route de Bayonne (cours de l'Argonne) et de la rue de Saint-Genès\footnote{Ces deux voies s'appelaient, vers le milieu du dix-huitième siècle : « Chemin royal de Bordeaux à Bayonne » et « chemin de la porte SainteEulalie à Saint-Genez ». Plan n° 1250. (Archives départementales.)}.

M. Chauliac pense que la première croix placée au carrefour était probablement celle du cimetière de Saint-Genès\footnote{\textit{Étude sur les croix des carrefours de Bordeaux}.}.

À notre avis, cette croix n'avait rien de commun avec le cimetière qui, bordant l'église Saint-Genès, se trouvait, par conséquent, à une petite distance du carrefour. La croix de Saint-Genès était une des croix qui jalonnaient le chemin de Saint-Jacques (aujourd'hui cours de l'Argonne dans une partie de sa traversée de Bordeaux, et cours Gambetta dans sa traversée de Talence).

Dans un texte de l'an 1400, il est fait mention d'une nouvelle croix à Saint-Genès, sur le grand chemin de Saint-Jacques\footnote{Léo DROUYN. \textit{Bordeaux vers 1450}.}. On observera qu'il s'agit d'une « nouvelle croix » ; ainsi il y en avait eu une autre, précédemment, sur le même chemin.

La croix de Saint-Genès menaçait ruine à la fin du XVIe siècle ; on la reconstruisit en 1606.

Un contrat fut passé le 22 mai 1606 entre, d'une part. Nicolas Dupuy et Arnaud de Lagrave, laboureurs et tous deux fabriciens de la paroisse Saint-Genès, et, d'autre part, Pierre de Coustances, maître-maçon. Le contrat stipulait que la croix devait être élevée sur cinq degrés « de bonne pierre dure de Bouchet ou de Rauzan ».

Pendant la Révolution, le croisillon fut détruit et le chapiteau mutilé ; une croix en fonte ouvragée remplaça plus tard la croix de pierre.

Cette réparation fut peut-être l'oeuvre du citoyen Babot, qui écrivait le 23 nivôse an IX : 

« Je déclare avoir racommodés la Crois de St-jenès par ordre du citoyen Bonnefous et qu'il m'a payer ledit ouvrage dont je te tiens quitte\footnote{Registre des délibérations de la municipalité de Talence. (Archives de la commune.)}. »

En 1902, la croix de Saint-Genès fut portée dans un petit enclos voisin, aménagé dans le domaine de la Solitude de Nazareth ; on substitua alors à la croix en fonte une croix en pierre.

Entouré de murs sur trois côtés, le monument est protégé sur le quatrième côté — celui regardant la place SaintGenès — par une grille en fer. Il présente une colonne hexagonale. A mi-hauteur est sculptée une statuette de saint, la tête nimbée. Cette figure est apparemment celle de l'histrion de Rome, ou celle du greffier d'Arles, ou bien encore celle de saint Genès, l'évêque de Clermont.

Ce pieux personnage porte une longue barbe et est coiffé d'un chaperon. Il tient à la main une banderole avec inscription devenue illisible.

La partie supérieure du fut est ornée de quatre têtes d'anges formant chapiteau. Entre les têtes existent de petits cartouches sur deux desquels on lit le monogramme du Christ. Le millésime 1606 est également gravé sur un des cartouches; il surplombe la statuette de saint Genès\footnote{Pour certains, celle figure serait tout simplement celle de saint Jacques.}.

On conserve aux Archives municipales de Bordeaux une remarquable aquarelle d'un artiste bordelais, M. Edmond Fontan, représentant la croix de Saint-Genès en 1900, par conséquent à l'époque où elle se dressait encore sur son emplacement primitif.

\subsection{Notre-Dame de Rama}

%%%% la note de cette page renvoie à une page à mettre à jour
Suivant une tradition, il y eut une apparition de la sainte Vierge à Talence. Pour commémorer cet événement, une chapelle consacrée « à la Vierge Marie », fut fondée vers 1130 dans le parc Parthenval\footnote{Voir page 90.}, près du ruisseau des Palanquettes. On l'appelait Notre-Dame de Rama ou de la Rame, nom qui lui venait de sa situation dans les bois, sous une ramée, couvert formé de branches entrelacées.

Éléonore d'Aquitaine, fille de Guillaume X, dernier duc d'Aquitaine et comte de Poitiers, paraît avoir créé cette chapelle en 1132\footnote{\textit{Société archéologique de Bordeaux}, t. XXII, p. 3. — BORDES. \textit{Histoire des monuments de Bordeaux}.\\Éléonore d'Aquitaine, reine de France, puis reine d'Angleterre, étant née en 1122, n'avait donc que dix ans quand elle fit élever cet édifice religieux.}.

Notre-Dame de Rama attira tout de suite les pèlerins. Les premières gardiennes de ce saint lieu furent des « Fontevristes », religieuses dépendant de l'abbaye de Fontevrault (Maine-et-Loire). Leur couvent « était placé à une soixantaine de mètres de la chapelle »\footnote{\textit{La Voix de Notre-Dame-de-Talence}, février 1913.}.

Éléonore d'Aquitaine finit ses jours à l'abbaye de Fontevrault le « 31 mai 1204 », si l'on en croit l'inscription gravée sur le socle de son tombeau dans ladite abbaye.

Le souvenir de cette princesse était fidèlement gardé par les religieuses de la même communauté.

« Le nécrologe de Fontevrault contient, en effet, un éloge d'Alienor plutôt exagéré, et les moniales\footnote{Moniales se traduit dans notre langue par \textit{monges} ou \textit{religieuses}.} de Talence devaient le lire tous les ans, au jour anniversaire de sa mort, en se rappelant que la reconnaissance leur faisait un devoir de taire les infidélités de leur bienfaitrice et d'en dire tout le bien que les chroniqueurs oubliaient de raconter\footnote{\textit{L'Aquitaine}, n° 23, 6 juin 1913.}. »

Les Fontevristes quittèrent le couvent du parc de Parthenval avant la fin du XIIIe siècle, « n'y laissant qu'un prieur »\footnote{Dom Reginald BIRON. \textit{Précis de l'histoire religieuse des anciens diocèses de Bordeaux et Bazas}.}. 

Il y avait dans la chapelle de Rama une statue de Marie\footnote{La main de la Vierge ayant été brisée, on l'a modelée en plâtre en lui donnant six doigts. Il existe, paraît-il, d'autres statues de la Vierge avec une main « sexdigitale ».}, tenant le Christ sur ses genoux.

« Une grande affluence de peuple s'y rendait pour honorer la mère de Dieu, dont la tradition publiait l'apparition miraculeuse en cet endroit\footnote{\textit{Dominicale bordelaise}, t. I\ier{}, 9 octobre 1836.}. »

La petite église fut endommagée par les Anglais, et on cessa, pendant la guerre de Cent ans, d'y célébrer le culte\footnote{\textit{Notre-Dame-de-Talence dans la chapelle des Monges au XVIII\ieme{} siècle}.}.

Pendant une période qui dut être assez longue, mais qu'on ne peut exactement déterminer, on n'alla plus en pèlerinage à Notre-Dame de Rama, et le pieux asile, qui avait été témoin des innocentes distractions d'Eléonore d'Aquitaine, « au jour de sa plus tendre jeunesse »\footnote{\textit{L'Aquitaine}, n° 23. 6 juin 1913.}, fut complètement abandonné.

M. Henry Chapotel, « prestre et archiprestre de Cernès », curé de Gradignan, visitant le 23 septembre 1691 les édifices religieux de Talence, se transporta dans la « chapelle ruralle de Notre-Dame de la Rame, vulgairement appellée des Monges, dépendant de l'abbaye de Fontevraux » et la trouva « entièrement ruinée, sans toit et sans porte, tout étant entièrement par terre »\footnote{G, 646. (Archives départementales.)}.

Il est vraisemblable que la chapelle avait été détruite durant une guerre, peut-être celle de la Fronde.

M. Chapotel s'enquit des revenus de Notre-Dame de la Rame. Ils consistaient « en rentes et agrières affermées la somme de cent livres »\footnote{\textit{Ibid}.}.

Le 6 janvier 1730, jour des Rois, des enfants jouant dans les ruines de l'antique chapelle trouvèrent parmi les pierres à demi recouvertes de ronces la statue de la Vierge. Les paroissiens s'empressèrent, dès lors, de « faire rétablir la susdite chapelle et exposer ladite image à la vénération des foules»\footnote{\textit{Chronique bordelaise de 1638 à 1736}. (Archives du château de Caila, appartenant à Mme la comtesse de Galard.) Le document a été transcrit pour les \textit{Archives historiques} par M. Émile Thomas.}.

Le vendredi 12 octobre 1731, la chapelle Notre-Dame de Rama, réparée et ornée\footnote{On y avait peint les armes du roi, suivant les ordres du père Lechat, qui avait été, à cette époque, envoyé en mission aux Monges de Talence par l'abbesse de Fontevrault.} fut bénie par le curé de Léognan, par ordre de Mgr Maniban, archevêque de Bordeaux, et le même jour, on commença à y dire des messes.

Au cours des travaux de reconstruction, on avait trouvé des tombeaux et beaucoup d'ossements dans le terrain, autour de la chapelle, et sur le sol même de cet édifice. En 1730, on voyait encore des restes du couvent de la Rame : des « dessus de croisées » montrant des peintures avec des fleurs de lys.

Un habitant de Bordeaux voulant se rendre, au XVIII\ieme{} siècle, à la chapelle empruntait lé « grand chemin de Bordeaux à Bayonne », et tournait peu après à gauche dans une voie ainsi désignée : « chemin de Bordeaux au Bourdieu de la porte, à la chappelle de N.-D. de Talence et à Agès\footnote{Plan 138. (Archives départementales.)}. » On quittait cette artère pour en prendre une autre — le chemin de Pey-Davant actuel — d'où partait un petit chemin conduisant directement à la chapelle.

On pouvait arriver à la chapelle par le côté opposé, en suivant un chemin ouvert par des marais.

Suivant le plan de 1782, le prieuré de la Rame possédait à cette époque plusieurs fiefs dans le voisinage du lieu où il selevait. L'un de ces biens est ainsi mentionné : « Moulin, étang, ruisseaux, pré et vinière, contenant 1 journal, 19 règes, 4 carreaux, fief du prieuré de la Rame appartenant aux dames religieuses de Fontevreau, formant l'article 1er de la reconnaissance consentie en leur faveur par...\footnote{Plan 138. (Archives départementales.)} ». 

Le moulin dont il est question au début du paragraphe n'était autre que l'ancien moulin de la Lande.

Parmi les personnes présentes à un mariage célébré le 24 janvier 1788, en l'église Saint-Genès, on trouve JeanBaptiste de Roboam, « écuyer, prieur et chapelain de Notre-Dame-de-Talence ».

La Révolution arriva. Le sanctuaire de Notre-Dame-deTalence fut profané. On jeta dans le fossé des Palanquettes la statue de la Vierge. Cette statue, recueillie par de pieux habitants de la paroisse, MM. Castaing, Barron et Mouliney\footnote{\textit{Notre-Dame de Talence dans la chapelle des Monges}.}, fut cachée dans un caveau du presbytère de la chapelle Saint-Pierre.

Le deuxième lot des biens nationaux à Talence, à savoir : « la chapelle Notre-Dame-de-Talence, maison, jardin, vignes et quelques arbres et place vuide vers le couchant »\footnote{Registre des procès-verbaux de la municipalité de Talence. (Archives de la commune.)} fut estimé 4.000 l. et vendu, le 12 messidor an III, à M. Thiac, « constructeur 116, sur le Port », pour la somme de 60.400 livres\footnote{\textit{Documents relatifs à la rente des biens nationaux}.}.

Thiac, prénommé Jean, était constructeur de navires à Bordeaux. Il écrivit le 22 messidor an III aux citoyens composant l'administration du district de Bordeaux pour demander que la municipalité de Talence ne retire de la chapelle que ce qui était porté sur l'inventaire. Il réclamait, en outre, les clés de la chapelle, « attendu qu'il avait payé l'entière adjudication de cet édifice depuis le 15 messidor »\footnote{Registre des procès-verbaux de la municipalité de Talence. (Archives de la commune.)}.

La chapelle fut détruite peu après. Elle avait été particulièrement fréquentée par les marins de Bordeaux et des environs ; ils venaient y faire des voeux. Aussi Notre-Dame de Rama était également appelée Notre-Dame de Bon« Port »\footnote{Dans les registres des procès-verbaux de la municipalité de Talence, sous la date 17 novembre 1793, la chapelle de la Rame est désignée « Notre-Dame de Bon Port ». (Archives de la commune.)\\Des auteurs donnent à Rama ou Rame le sens d'aviron, et c'est ce qui expliquerait, selon eux, l'attrait qu'exerçait sur les navigateurs la chapelle de Talence qu'on appelait aussi Notre-Dame-des-Douleurs (Dom Reginald BIRON. \textit{Précis de l'histoire religieuse des anciens diocèses de Bordeaux et Bazas}), et chapelle des Monges (voir p. 114).}. On y voyait l'image de l'église avec cette légende :

\begin{center}
NOTRE-DAME DE BON-PORT.
\end{center}

\textit{Je fus construitte l'an 1132 et fus détruitte du tempt des guerres, et je fus rétablie l'an 1730 par la charité des fidelles chrétiens ayant opéré nombre de miracles...}

En guise d'ex-voto, des marins déposèrent dans cet asile de la prière les tableaux ci-après désignés et représentant des navires sur la mer en furie : 

Navire \textit{Marie-Elisabeth}, 2 novembre 1742 ;
Navire \textit{Saint-Alexis}, mars 1750 ; 
Navire \textit{Saint-Nazaire}, 1753 ;
Navire \textit{Ville-de-Bergerac}, 1768 ;
Navire \textit{Duc-de-Penthièvre}, 20 mars 1778 ;
Brick \textit{Le Héros}, 11 octobre 1778 ;
Brick \textit{Le Lyon}, 1779 ;
Navire \textit{La Concorde}, 24 décembre 1782 ;
Navire \textit{Les Trois-Frères}, 3 février 1784 ; 
Navire \textit{Le Persévérant}, 1786 ; 
Navire \textit{La Victoire}, 1788 ;
Navire \textit{Le Fils Unique}, 27 octobre 1790 ;

Plus deux navires sans date : \textit{le Voltigeur} et le \textit{Grand-Saint-Jacques}.

Une lettre adressée, le 17 février 1734, par Mouliney, marguillier de la chapelle de Rama, à Mme de Montplaisant, secrétaire de l'abbesse de Fontevrault, se rapporte à des marins venus en pèlerinage à Talence. Voici cette lettre telle qu'elle a été publiée par le père L. Royer dans sa plaquette la « Retrouve » de Notre-Dame-de-Talence :

« MADAME, 

» Comme vous m'avez ordonné de marquer tout ce qui se passe à la chapelle, je prends la liberté de vous écrire pour vous dire que le 7 janvier dernier, il se trouva un vaisseau dans le naufrage, dont il n'aurait jamais réchappé sans un voeu que l'équipage fit à Notre-Dame-du-Bon-Port, lequel voeu a été exécuté hier, 16 du courant, dans ladite chapelle, où je ne manquai pas de me trouver ; il était même nécessaire.

» Ces pauvres gens sont partis de Bourdeaux pour se rendre à la chapelle, pieds nus et en chemise, ayant tous des cierges à la main, chantant tout le long du chemin, les litanies de la sainte Vierge jusqu'à l'entrée de la chapelle, et, en entrant, ont fait entonner le \textit{Te Deum} par le chapelain, qui a été chanté par tout l'équipage et les assistants, qui étaient en grand nombre, et ont fait présent d'un grand tableau à cadre doré, où est représentée la sainte Vierge dans la nue, à qui ils demandent secours, et où est aussi représenté ledit vaisseau dans une tempête affreuse... »

Il y avait également à Notre-Dame de Rama un tableau rappelant un voeu accompli dans cette chapelle par les jurats de Bordeaux, à l'occasion d'une calamité publique. Les magistrats municipaux portaient le catogan. Ce détail fixe l'époque du voeu : la deuxième moitié du XVIII\ieme{} siècle.

Le tableau des jurats\footnote{Il n'avait aucune valeur artistique, mais présentait un réel intérêt historique.} a disparu; les ex-voto des marins sont précieusement conservés dans l'église paroissiale de Talence.

Le lieu où s'élevait Notre-Dame de Rama garda longtemps le nom de quartier de « la Petite Chapelle ».

\subsection{La chapelle Saint-Pierre}

Elle s'élevait en un lieu dit « les Abideys », à droite de la route de Bayonne, presque à l'angle du chemin Labric\footnote{Chemin des Briques.}. Son emplacement est marqué par une croix sur un plan ancien ; un cimetière la bordait sur le côté regardant Bordeaux\footnote{Plan 138. (Archives départementales.)}.

Tout d'abord, la chapelle Saint-Pierre fut seulement au service d'une confrérie composée de « laboureurs de vignes ». Un acte relatif à un échange fut dressé, le 30 avril 1483, sous réserve des droits de la confrérie de SaintPierre de Talence, à laquelle étaient dus deux deniers « d'esporle et 25 ardits bourdelois de cens par an »\footnote{\textit{Archives historiques de la Gironde}, t. XXXV, p. 417.}.

Sur une copie d'acte délivrée le 8 septembre 1607, on rélève le nom de « Léonnard de Lescarret, prêtre et vicaire de Talence, syndic de la confrérie de Saint-Pierre\footnote{}.
\textit{Ibid}., p. 418.
La chapelle fut réorganisée par le cardinal de Sourdis. Elle devint succursale de l'église Saint-Genès de Talence, un raison de l'étendue de cette paroisse. 

Un important détail doit se glisser ici.

Deux agglomérations de gens ont constitué la commune de Talence. La première s'était formée autour de l'église Saint-Genès, la seconde aux abords de la chapelle Saint-Pierre. Ces deux édifices se trouvaient sur. le chemin de Saint-Jacques. L'intervalle qui les séparait — soit un quart de lieue — se garnit d'habitations et de jardins qui firent en quelque sorte la soudure.

Dans le procès-verbal relatif à sa visite des édifices du culte, à Talence, le 23 septembre 1691, M. Henry Chapotel, l'archiprêtre de Cernès, avait souligné, entre autres détails, que l'église Saint-Pierre était trop petite pour contenir les paroissiens, et qu'elle était dépourvue de sacristie.

L'archevêque de Bordeaux, d'Anglure de Bourlemont, décida, le 1\ier{} juin 1693 : 1° de faire construire une sacristie « le plus tôt qu'il se pourrait et au lieu qui serait jugé le plus convenable » ; 2° de faire agrandir ladite église ou chapelle; « une aisle serait construite du costé du nord quand il y aurait des fonds suffisamment pour cette dépense »\footnote{G, 646 (Archives départementales.)}. Il y avait, à cette époque, plusieurs confréries établies dans l'église Saint-Pierre. L'archevêque de Bordeaux leur prescrivit, le 1\ier{} juin 1693, de lui apporter leurs statuts dans le délai d'un mois ; celles d'entre elles qui n'avaient point de règlements se pourvoiraient devant le prélat pour en avoir\footnote{\textit{Ibid}.}.

Dom Reginald parle de l'érection « d'une nouvelle chapelle à Talence en 1783 »\footnote{\textit{Précis de l'histoire religieuse des anciens diocèses de Bordeaux et Bazas}.}. Il s'agissait, selon nous, d'un nouvel agrandissement de la chapelle Saint-Pierre\footnote{Voir l'article « Peychotte », page 65.}. Les travaux durèrent de juillet 1782 à août 1783.

En 1782, toute la partie de la commune comprise aujourd'hui entre le cours Gambetta, le chemin de Suzon, la mairie et le ruisseau de Talence constituait deux fiefs : l'un de Saint-Pierre, l'autre de la « confrairie de Saint-Pierre-deTalence »\footnote{Plan 138. (Archives départementales.)}.

La chapelle Saint-Pierre, à laquelle était adossé le presbytère de la paroisse, fut, le 28 pluviôse an II, transformée en \textit{Temple de la raison}. Exposons, à ce propos, les termes du paragraphe 4, relatif à la nomenclature des biens nationaux de Talence :

« Plus, l' église Saint-Pierre-de-Talence, maison ci-devant curialle, cour et autres bâtiments attenants, jardin et cimetière, le tout situé au centre de la commune, section Saint-Genès, lesquels effets et église, la commune se réserve pour ses utilités et a, en conséquence, \textit{choisi cette église pour y former le Temple de la raison où l'instruction nationale sera enseignée} »\footnote{Registre des procès-verbaux de la municipalité de Talence. (Archives de la commune.)}.

Le 25 ventôse an III, on décide que des affiches seront apposées pour avertir les habitants et autres citoyens que le jardin du cy-devant presbytère de la commune, situé au couchant du chemin qui y conduit, sera mis aux enchères le 10 germinal prochain ».

La « maison ci-devant curialle » — ou presbytère — fut elle-même adjugée, le 11 thermidor an IV, pour la somme de 6,320 livres, à M. Pierre Pitrey, 6, place Saint-Projet.

En 1802 — un an après le Concordat — la chapelle Saint-Pierre servait d'église paroissiale\footnote{Rappelons que l'ancienne église paroissiale Saint-Genès avait été désaffectée et vendue en l'an III.}. On y plaça la statue de la Vierge et celle-ci, comme on le verra plus loin, fut transférée finalement dans l'église Notre-Dame de Talence.

La chapelle Saint-Pierre n'existe plus, ceci dit pour éviter toute confusion avec le monastère de Saint-Pierre construit, en 1901, chemin de Suzon.

Sous Louis-Philippe, la mairie siégeait dans l'ancien Temple de la Raison\footnote{Auguste BORDES. \textit{Histoire des monuments anciens et modernes de la ville de Bordeaux}.}.

Le presbytère était devenu la maison communale. La mairie y siégeait encore en 1864\footnote{Le Père L. DELPEUCH. \textit{Histoire de Notre-Dame de Talence ou de Rama}.}.

\subsection{Le prieuré de Bardenac}\footnote{Bardenac ou Bardanac.}

Il s'élevait au point où se touchent les paroisses de Talence, Pessac et Gradignan. Il appartenait, d'après Léo Drouyn, à la paroisse de Talence\footnote{\textit{Bordeaux vers 1450}.}.

Nous avons lu, d'autre part, « Bardanac à Talence »\footnote{Archives départementales de la Gironde, série G.}.

Suivant le \textit{Bulletin de la Société archéologique de Bordeaux}, la maison prieurale et l'hôpital de Notre-Dame de Bardenac étaient dans la paroisse de Pessac. Seuls, l'église prieurale et le cimetière de Bardenac se trouvaient dans Talence\footnote{Tome XXI, année 1896, page 124.}. Toutefois, si l'on consulte les \textit{Registres de la Jurade}, on y voit que c'est l'hôpital de Bardenac qui était dans Talence avec le prieuré\footnote{Tome IV.}.

L'hôpital, créé au XII\ieme{} siècle, devint prieuré dans la suite\footnote{Dom Reginald BIRON. \textit{Précis de l'histoire religieuse des anciens diocèses de Bordeaux et Bazas}.}.

Un acte de 1357 porte qu'à cette date le prieur de Bardenac possédait une maison dans la rue Sainte-Eulalie.

Bardenac était le premier hôpital que rencontraient les pèlerins qui empruntaient la « voie de Gascogne » pour se rendre en Espagne, au pèlerinage de Saint-Jacques-deCompostelle\footnote{L'abbé Baurein. \textit{Variétés bordeloises}.}, si renommé ou moyen âge. Cette voie de Gascogne passait par Bordeaux, Talence, Belin, Salles, Dax, Sordes\footnote{Carte routière dressée d'après le \textit{Codex cornpostellam du XII\ieme{} siècle}}, etc. C'était la route de Bayonne.

Le 25 novembre 1540, on enregistre une reconnaissance pour un maine à Talence, « confrontant au chemin comunau par lequel on va et vient de la ville et citté de Bourdeaux à Sainct-Jaques de Gallice »\footnote{\textit{Inventaire sommaire. Archives de la Gironde}, série G, t. II.}.

Le 8 septembre 1629 est signée une autre reconnaissance pour une vigne sise dans la paroisse de Talence, « au plantier du Branar, confrontant à l'est au grand chemin qui va de Bordeaux à Talence, autrement appelle chemin Roumieu »\footnote{Roumieu ou romieu signifie pèlerin. Chemin de Saint-Jacques ou \textit{camin romiu}.}.

Un paysan picard nommé Manier, se rendant à Compostelle, en septembre 1726, nota sur son carnet de route :

« Départ de Bordeaux. — Le 27 au matin, avons parti de cette ville pour aller à Saint Guenest (Saint-Genès) ; au pont de la Lances (Talence)\footnote{\textit{Société archéologique de Bordeaux}, t. XXI, année 1890, p. 133.}. »

Ce pont était jeté sur le ruisseau de Talence, cours Gambetta, à hauteur de Peychotte.

On prétend que Charlemagne suivit le chemin de Saint-Jacques dit « la Voie du Centre »\footnote{La « voie du Centre » passait par Périgueux, La Réole, Bazas, Mont-de-Marsan, Orthez, Saint-Palais, Ostabat et Saint-Jean-Pied-de-Port.}, quand il partit avec son neveu Roland pour aller combattre les Sarrasins, et qu'il passa, à son retour, par l'autre chemin de Saint-Jacques dit « la Voie de Gascogne ». Ainsi, le grand empereur, ramenant la dépouille mortelle de Roland et celle des autres preux également tués dans l'embuscade de Roncevaux, traversa Talence pour se rendre à Bordeaux, en la basilique Saint-Seurin où, selon la tradition, il déposa l'olifant ou le cor d'ivoire du célèbre paladin.

Un autre grand prince avait, longtemps avant Charlemagne, foulé le sol de Talence.

Clovis, ayant, en 507, à Vouillé, tué de sa main Alaric II et défait ses troupes, avait poursuivi celles-ci en fuite « vers Bourdeaus ». Il rejoignit les fuyards et les extermina dans un terrain limitrophe de Talence\footnote{BORDES. \textit{Histoire des monuments de Bordeaux}.} et voisin du village qui a gardé le nom de Camparian\footnote{Dans l'acte d'acquisition par les maire et jurais de Bordeaux du comté d'Ornon, il est fait mention de la « Prévoté de Camparrian x. (Arch. départ. Intendance, série C, 3660.)} (camp des Ariens). Clovis regagna ensuite Bordeaux pour y passer l'hiver.

Bardenac fut uni, à partir de 1600, au collège de la Madeleine\footnote{Archives départementales de la Gironde, série G.}, dont les bâtiments sont occupés aujourd'hui par le Lycée national. La chapelle de Bardenac fut, depuis lors, desservie par les pères Jésuites.

Suivant une déclaration de revenus des Jésuites faite, le 4 juillet 1692, par le syndic du collège de la Madeleine, Talence rapportait alors « 1.294 livres de rente sur 287 fiefs, tant pour droits de censive que d'agriaires »\footnote{\textit{Société archéologique de Bordeaux}, t. XXI, année 1896, p. 124.}.

Il est question dans une chronique d'un « gros cheval apellé Cadet », qui était né à Bardenac et mourut « de tranchée rouge », au collège des Jésuites, le 7 mars 1742.

Ce cheval, de poil noir bien marqué, était âgé d'environ huit ans\footnote{Archives départementales de la Gironde. (Fonds des Jésuites, carton 39.)}.

Chaque année, le jour de la procession de Saint-Marc, les fidèles de Talence et de Saint-Genès se rencontraient à Bardenac avec les fidèles des églises voisines, et il y avait contestation pour savoir quelle paroisse aurait le premier pas. L'archevêque, soucieux de ménager toutes les susceptibilités, décida que les deux curés de Saint-Genès et de Gradignan marcheraient de front dans le cortège. Un vicaire reçut mission de faire respecter l'ordonnance épiscopale.

Dans leur litige, les ecclésiastiques avaient oublié le fameux précepte : « Les derniers seront les premiers ! »

\subsection{Les chapelles domestiques}

Le 23 septembre 1691, l'archiprêtre de Cernès, M. Henry Chapotel, visita cinq chapelles domestiques à Talence : celles du sieur de Guionet\footnote{La villa Raba occupe l'emplacement de la maison noble de Guionet (Voir page 51.)}, du sieur Roux, du « sieur Bigot, de M. le président Latresne, et du sieur Pickronneau, converti, ayant acquis la maison du sieur La Roque »\footnote{G, 646. (Archives départementales.)}.

Les quatre premières chapelles étaient pourvues de tous les ornements nécessaires pour la célébration de la sainte « messe » ; celle de Pickronneau était dépourvue « de pierre sacrée, de napes et de tous autres ornements utiles pour l'exercice du culte ».

Le 25 septembre 1726, l'archevêque de Bordeaux accorda à Louis de Gombaut, président à la Cour des Aides, l'autorisation « de faire construire une chapelle domestique dans sa maison de campagne, située dans la paroisse de Talence »\footnote{G. 682. (Archives départementales.)}.

Françoise Labatut, épouse du « sieur président Fossier », avait adressé une requête en vue de faire édifier une chapelle dans son domaine à Talence. Cette requête fut favorablement accueillie le 29 août 1736.

La permission de bâtir une chapelle n'était accordée à la suppliante « qu'autant qu'elle continuerait à s'en rendre digne » par son attention à remplir les conditions prévues en pareil cas, « et par son exactitude à ne pas permettre que d'autre qu'elle, sa famille et les domestiques qui luy seraient nécessaires y entendraient le Saint-Sacrifice de la messe »\footnote{\textit{Ibid}.}. La présidente Fossier devait renvoyer les autres personnes à l'église paroissiale.

Tous lès propriétaires de chapelles domestiques étaient soumis aux mêmes obligations.

\subsection{L'église de Talence}

En 1817, l'église Saint-Pierre était insuffisante « pour contenir le cinquième des habitants »\footnote{Registre des procès-verbaux de la municipalité de Talence. (Archives de la commune.)}. On songea à bâtir une autre église paroissiale. La première pierre en. fut posée par le fondateur, Mgr d'Aviau, archevêque de Bordeaux, le 12 mars 1821.

L'année suivante, le 24 novembre 1822, le même prélat vint baptiser les deux cloches de la nouvelle église. Les duchesses d'Angoulême et de Berry furent marraines de ces cloches. La cérémonie se fit avec une telle pompe qu'un journal de Bordeaux, en la décrivant, déclara que cette journée serait « célèbre dans les annales de Talence ». 

Le 4 mars 1823, la statue de la Vierge Notre-Dame de Rama — ou Notre-Dame de Talence. — fut transportée solennellement de la chapelle Saint-Pierre dans la nouvelle église. Celle-ci fut bénite et consacrée par l'archevêque, le 1er avril 1823, « en présence de toutes les autorités ecclésiastiques, civiles et militaires de Bordeaux et de la commune de Talence »\footnote{Extrait d'un article écrit par M. Ripollès, curé de Talence, le 20 août 1836, et publié, la même année, dans la Dominicale bordelaise. (Archives municipales de Bordeaux.) M. Ripollès mourut le 13 novembre 1836. L'archevêque de Bordeaux. Magr d'Aviau, en témoignage de la satisfaction qu'il avait éprouvée en consacrant, en 1823, l'église de Talence, dont la construction était due spécialement aux soins et au zèle de M. le curé Ripollès, avait nommé dernier chanoine honoraire. ce (\textit{Tablettes du clergé et des amis de la Religion}, t. III, p. 322.)}.

Cet édifice religieux eut une existence éphémère. En 1840, il menaçait ruine. Etait-ce un défaut de la charpente ? Etait-ce l'instabilité du sol ? Quoi qu'il en soit il fallut l'abandonner en 1842, et bâtir, tout à côté, une autre église : c'est celle que nous voyons actuellement, et qui est renommée par les pèlerinages qu'on y fait toute l'année et, notamment, au mois de mai.

La première pierre de l'église de Talence fut posée par l' archevêque de Bordeaux le 28 mars 1843.

« L'église de Talence, écrivait alors Bernadau, se reconstruit au moyen d'une souscription de 5 centimes Par semaine que chaque habitant a promis de payer au curé, qui dirige fort adroitement cette entreprise, et fait des lotteries du vin que les riches propriétaires donnent en supplément de leur souscription hebdomadaire d'un sou\footnote{\textit{Tablettes}, t. XII.}. »

La consécration de l'église eut lieu le 12 août 1847. Le souvenir de cet événement est perpétué par une plaque de marbre placée dans l'édifice, en face de la chaire, et portant en latin cette inscription :

« L'an du salut 1847 et le 12 du mois d'août, le Souverain Pontife Pie IX, heureusement régnant, M. J.-H.-F. Carros étant curé de la paroisse, et MM. Tulèvre, Roui, Rousseau, Jaquemet, Crespy et Carrier, membres de la fabrique ; ce temple bâti par les offrandes des fidèles a été consacré, au milieu d'un heureux et nombreux concours de peuple, par N.N. S.S. Ferdinand-François-Auguste Donnet, archevêque de Bordeaux, et Clément Villecourt, évêque de La Rochelle ».

L'église appartient au style gréco-rofnain. Le plan comprend un porche précédé d'un portique ; des fonts baptismaux à droite de l'entrée ; à gauche, l'escalier de la tribune ; une nef, une croisière formée du choeur et des chapelles latérales, le sanctuaire principal, une sacristie et un dépôt d'ornements.

L'élévation de la façade se distingue par un perron de six marches. Le portique offre sur deux socles unis deux colonnes cannelées d'ordre ionique, supportant avec leurs antes un entablement denticulaire ajusté d'un fronton sur lequel sont placés une croix et un briquet. Cet entablement est enrichi de plusieurs rosaces répondant dans la frise à l'axe des colonnes.

De chaque côté du portique est creusée une niche renfermant la statue d'un saint. L'entablement principal, dont les extrémités reposent sur un pilastre dorique, présente, dans sa frise, cette dédicace :

\begin{center}
D. p. M. SUB INVOCATIONE\\
B. MARLE VIRGINIS
\end{center}

Au-dessus, dans un encadrement sculpté, la niche où il y a l'image de la Vierge, à qui est consacrée cette église.

On voit encore, palmes la façade, sur et couronnes, deux figures célestes tenant une couronne avec le chiffre de Marie.

L' église Notre-Dame de Talence est l'oeuvre de l'architecte Auguste Bordes. Cet architecte s'est adressé à luimême un compliment en exprimant le regret que les peintures et les autels qui ont été exécutés dans l'église n'aient Pas une « plus grande affinité avec les autres parties du Monument »\footnote{Histoire  des monuments anciens et modernes de la ville de Bordeaux, t. II.}.

D'après le voeu qui lui en avait été exprimé par plusieurs personnes occupées à l'oeuvre de réédification, Bordes avait projeté une église de style gothique ; mais sur le refus du Conseil des bâtiments civils, il avait dû étudier le projet gréco-romain qui fut exécuté.

\subsection{La croix de Leysotte}

Elle se dresse à l'angle du chemin de Leysotte et de la route de Toulouse, en bordure de cette dernière voie. Elle présente un socle massif en pierre reposant sur deux gradins ; la croix en fer, ouvragée, montre un motif symbolique.

Y a-t-il une inscription sur la face antérieure du socle, celle regardant Bordeaux ? Il est impossible de s'en rendre compte, ce côté de la maçonnerie étant constamment recouvert d'affiches diverses, lesquelles, d'ailleurs, pourraient être, sans inconvénient, placardées autre part.

La croix de Leysotte est indiquée sur la \textit{Carte de Guienne}, de Belleyme\footnote{Dressée en 1775. (Voir page 22.)}. Sans doute devait-elle marquer, du côté de la roule de Toulouse, les limites de la paroisse de Talence ?

Dans un acte du 22 juin 1781, il est fait mention de Etienne Labadie, maître forgeron, habitant « près la Croix de Leyssote, joignant le grand chemin de Bordeaux au Bouscot »\footnote{G, 3146. (Archives départementales.)}.


\subsection{Les curés de Talence}\footnote{Cette liste est incomplète. Les dates accompagnant les noms sont celles où nous avons trouvé ces ecclésiastiques dans l'exercice de leurs fonctions.}

%%%% TODO : liste à mettre en forme
SAINT-AIGNAN .................. TOLAND (Jean-Pierre)......... DUMOULIN. LACOURT FORTIN OSCANLAN (André-Théophile)... février LE NÈGRE...... 27 décembre 27 TOUCAS-POYEN............... RIPPOLÈS ....................... J.-H.-F. CARROS . 1676 1684-1700 1709-1735 1735 1740-1788 1788-1790 1791 1791 1823-1836 1847 1853-1863 1874 1892 MERLIN (R. P. oblat) JEANMAIRE (R. P. oblat) RAMADIÉ COUBRUN ............................................1899 1901 ROYER (R. P. oblat) DOREILLAC (R. P. bénédictin) . . 1905 1913 MOUREAU ....... JOANNE...... (A.)......... FORT 1922 1926 départementales.)  