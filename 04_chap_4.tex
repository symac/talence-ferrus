%%% Commence page 89 du PDF
\section{AUTRES DOMAINES REMARQUABLES}

\subsection{Le château de Talence (autrefois "des Flottes")}

Il s'élève au milieu d'une plaine, à l'est du Petit Lycée, entre l'avenue de Mégret et la rue du Général-Bordas. L'avenue de Belligny, bordée de marronniers, conduit à l'entrée du château.

Cet édifice imposant fut construit au XVII\ieme{} siècle. Il est précédé d'une vaste cour garnie de platanes. Il présente un corps de logis (premier étage et mansardes) flanqué de deux ailes, le tout recouvert d'ardoises. Un perron règne entre les ailes et donne accès dans un vestibule dont les visiteurs admirent la magnifique architecture.

Devant la façade sud du château s'étend un jardin anglais entouré d'un grand parc entrecoupé de nombreuses allées.

Le château des Flottes compta parmi ses propriétaires, au XVIII\ieme{} siècle, M. de Mercier et M. de Saint-Cric.

Pierre Saincric, « écuyer, citoyen de Bordeaux », avait demandé à l'archevêque l'autorisation de faire construire « une chapelle dans sa maison de campagne, située dans la paroisse de Talence, distante d'une demi-lieue de l'église paroissiale ». Sa requête fut examinée et l'autorisation accordée le 25 août 1734\footnote{G, 682. (Archives départementales.)}.

Le 18 août 1743 fut célébré, en l'église Saint-Genès, le mariage de « Pierre de Saincric de Malvirade, fils de Pierre de Saincric, écuyer, et de Jeanne Dumay ».

Le château des Flottes\footnote{Il est également désigné « château des Flottes » sur le plan cadastral de Talence établi en 1847.} est mentionné sous ce nom dans un acte de vente du 22 mars 1748. Il n'avait pas les privilèges d'une maison noble, et il payait des redevances aux jurais de Bordeaux « seigneurs de Talence », et à la fabrique de Saint-Michel.

Le domaine fut saisi comme bien national à la Révolution et acheté par le nommé Duluc, domicilié rue des Treilles. Ce dernier le revendit au sieur « de Clerc », ancien receveur général à Bordeaux.

En 1827, le domaine de M. Declerc était « remarquable par ses plantations en arbres étrangers et indigènes »\footnote{\textit{Guide de l'étranger à Bordeaux}.}.

En 1835, le bien passa, dans les mains de Espeletta. Il fut acheté, le 25 janvier 1849, par J.-B. de Mégret de Belligny, qui le restaura, planta dans le parc des arbres exotiques, renouvela les vignes et mourut laissant à son fils aîné — Jean Santiago de Mégret de Belligny\footnote{Il était né à Santiago-de-Cuba.} — le soin de poursuivre ses projets d'embellissement du château.

Armateur à Bordeaux de 1857 à 1875, trois fois élu maire de Talence, J.-B. de Mégret de Belligny consacrait aux lettres tous ses loisirs. L'Académie de Bordeaux l'accueillit le 30 novembre 1865 « comme poète et auteur dramatique ».

M. de Mégret de Belligny a composé une partie de son oeuvre dans sa résidence de Talence. Une atmosphère calme et pure, des ombrages délicieux, de riants paysages, le gazouillis des oiseaux, le murmure des forêts, firent agréablement vibrer sa muse.

Parmi ses dernières productions littéraires, signalons un petit poème \textit{Sous l'Yeuse}\footnote{Ces pièces ont été publiées dans les \textit{Actes de l'Académie de Bordeaux}.}, portant la date « Château-Talence, 14 mai 1904 » et une \textit{Epître à l'Académie}\footnote{Ces pièces ont été publiées dans les \textit{Actes de l'Académie de Bordeaux}.} datée, du « Château de Talence 3 juin 1904 ».

%%% TODO : pour les références 5 et 6, à grouper cf. p. 91
L' ancien nom de château des Flottes était remplacé par ceux de Château-Talence ou Château de Talence.

M. de Mégret de Belligny mourut en 1905. Lors de la séance de l'Académie de Bordeaux, le 9 novembre 1905, le président rappela le décès et les obsèques de M. de Mégret de Belligny et, interprétant les sentiments de l'Académie, il fit, en quelques paroles émues, l'éloge que la modestie du défunt n'avait pas permis de prononcer sur sa tombe. L'orateur loua « les dons brillants du poète et l'affabilité de l'homme privé et de l'académicien ».

Quelques mois auparavant, les membres de l'Académie de Bordeaux avaient marqué par une manifestation de cordiale sympathie les quarante ans de présence de M. de Mégret de Belligny dans cette savante Compagnie.
 
Les « anciens » du pays — entre autres M. Claude Chambon — ont gardé le souvenir de l'ancien maire de Talence. Ils le revoient se promener à cheval; il était coiffé d'un chapeau haut-de-forme gris. Les enfants sortant de l'école chrétienne alors, il n'y avait d'autres le en pas saluaient, et il ne manquait pas de leur jeter des pièces de monnaie.
 
Armes de la famille de Mégret : « D'azur à trois besants d'argent, au chef d'or chargé d'une tête de lion arrachée de gueules, timbrée d'une couronne de comte\footnote{\textit{Annuaire du Tout-Sud-Ouest illustré}, 1907-1908.}.»

En 1923, la grande salle à manger du château servit de cadre au banquet annuel de l'Union des Républicains démocrates de Talence, qui réunissait une centaine de convives. M. Ed. de Luze présidait, entouré de MM. E. Frouin, député de la Gironde ; Henri de Mégret ; Bost, directeur du \textit{Démocrate de Talence} ; Toulet, conseiller municipal de Bordeaux ; Doussou, secrétaire général du groupe de l'Union des démocrates ; Aurousseau, président des Camarades de Combat, etc. A l'issue du repas, les convives se répandirent dans le parc et devisèrent aimablement sous les rameaux de chênes séculaires.

M. de Mégret, le propriétaire actuel du château de Talence, est le petit-fils de l'ancien membre de l'Académie des sciences, belles-lettres et arts de Bordeaux. 

\subsection{Le château Bonnefont}

Vers 1805, Balguerie junior devint propriétaire du domaine Bonnefont, situé sur le ruisseau de Talence, à droite du cours Gambetta, à hauteur de « Peychotte ». Il y avait dans ce domaine « un superbe jardin anglais » qui avait fait l'admiration du grand architecte Louis. Il « n'en avait jamais vu d'aussi beau, de son propre aveu »\footnote{\textit{Guide de l'étranger à Bordeaux}, 1827.}. 

Balguerie fit construire à Bonnefont, dans le style empire, une importante villa, qu'il meubla luxueusement. Il y avait, dans l'immeuble, entre autres belles choses, un trumeau montrant M. et Mme Balguerie, M. et Mme Tarteiron, M. Lawton et sa fiancée, Mlle Balguerie\footnote{Renseignements communiqués par M. Meaudre de Lapouyade, qui a fait une reproduction de ce trumeau.}.

Un collaborateur du \textit{Musée d'Aquitaine}, F. J., vantant, en 1823, la splendeur des villas de Talence, mettait au premier plan « la maison Balguerie ». « Vous y reconnaissez sans doute, écrivait-il, la main de l'art, mais d'un art délicat qui ne soumet pas tout à la règle et à l'équerre, et qui ne touche aux beautés naturelles que pour les faire valoir\footnote{\textit{Musée d'Aquitaine}, t. II et III, p. 111.}. » Une promenade dans le domaine Balguerie laissait le visiteur sous le charme. Les jardins, les parterres fleuris, les allées ombreuses, les bouquets d'arbres, les eaux courantes, les magnifiques réservoirs, tout contribuait à faire de cette demeure un séjour idéal.

La maison Balguerie passa, aux environs de 1830, dans les mains de M. Christian Gaden, négociant et vice-consul de Mecklembourg à Bordeaux.

M. Christian Gaden, né à Pormpow (grand-duché de Mecklembourg-Schwerin), vers 1777, avait fondé, en 1803, la maison de commerce bien connue sous la firme C. Ga« den et Klipsch ».

Le château de Bonnefont fut ensuite habité par le fils, Hermann Gaden, puis par le petit-fils, Charles Gaden, qui fut adjoint au maire de Bordeaux, administrateur des hospices civils et membre de la Chambre de commerce.

Édouard Guillon visita la « villa Gaden » — il l'appelle ainsi — en 1865. Il résuma ainsi son impression : « La maison est belle et meublée avec luxe ; il y a des tapisseries qui ont une grande valeur \footnote{\textit{Les châteaux historiques et vinicoles de la Gironde}.}. »

La « villa Gaden » ne fut appelée ainsi que pendant un certain temps. Les annuaires signalent, effectivement, que Charles Gaden, qui avait une maison à Bordeaux, rue de la Course, 109, était propriétaire « du château Bonnefont à Talence ».

Le domaine — d'une étendue de 9 hectares — a été acheté, il y a quelques années, par M. Paul-Armand Beaumartin. Le nom de Bonnefont lui vient, prétend-on, du ruisseau de Talence : « Bonne-Font » pour bonne fontaine, bonne source.

\subsection{Parthenval}

Suivant un usage assez courant dans la banlieue bordelaise, les propriétaires donnent à leurs résidences des noms de fantaisie ou qui rappellent soit des souvenirs-d'histoire locale, soit de pieuses légendes. M. le baron de la Touche d'Avrilly, devenu propriétaire du lieu où s'élevait l'ancienne chapelle des Monges, dénomma ce lieu « Parthenval » (Vallée de la Vierge).

Le domaine est limité par le chemin de Suzon, le chemin Peydavant, le chemin Frédéric - Sévène et la propriété Peixotto. Sur un pilastre de la porte d'entrée (77, chemin Frédéric-Sévène), on lit cette inscription, peinte par un amateur : « Parteneval ».

Dans la propriété la Touche d'Avrilly était encore, en 1872, un chêne auquel M. le comte de Sarrau, qui le vit alors, donnait « six siècles d'existence ». Un creux qui s'était produit dans le tronc de l'arbre, à une petite hauteur du sol, était garni d'une statuette de la Vierge. Par qui cette statuette avait-elle été déposée dans le creux du chêne ? Peut-être par des pèlerins.

L'arbre faisait autrefois partie d'une forêt qui devait longer — ou qui traversait — le chemin de Saint-Jacques, (cours Gambetta). Fut-il respecté à cause de l'image de la Vierge ?

Quoiqu'il en soit, le chêne est mort de vieillesse, et on dut l'abattre, car par sa chute menaçante, il était devenu un danger.

M. le comte de Sarrau nous a écrit à ce sujet :

« Quand on l'abattit, me disait un vieux vigneron de l'en¬droit, les cloches sonnèrent, et les bûcherons qui le débitèrent ont chanté des chants funèbres. Légende sans doute, mais touchante croyance. »

\subsection{Un pavillon Louis XVI}

Le chemin Roul prend naissance cours Gambetta, presque en face du chemin Frédéric-Sévène. Suivons le chemin Roul. Nous trouvons bientôt à gauche « Béthanie, maison médicale et de repos », et à droite la propriété Holagray, qui s'étend jusqu'au chemin Vieille-Tour. Une partie de cette propriété présente un vaste champ au milieu duquel se dresse un petit pavillon de forme circulaire avec balustrade entourant le faîte. Une porte précédée de quatre marches et regardant le chemin Roul, donne accès dans le local. Il y a une porte semblable du côté opposé. Entre les portes il y a des fenêtres, et au-dessus des unes et des autres on remarque d'élégantes sculptures montrant des guirlandes de fruits, de fleurs ou de feuillage. Ce pavillon est un joyau du style Louis XVI.

\subsection{Le château Larroque}

Un peu après avoir dépassé le chemin Vieille-Tour, le chemin Roul fait un coude brusque à gauche, tandis que s'ouvre, sur son prolongement direct, le chemin Bernos. Dans la fourche des deux voies, un établissement champêtre avec cette enseigne fixée sur un arbre : « Au rendezvous des chasseurs. »

Le chemin Bernos, d'abord d'une largeur ordinaire, se rétrécît soudain ; il se trouve alors comme canalisé entre deux murs. C'est sur ce chemin qu'est situé, à l'extrémitéouest de Talence, le château Larroque. Il fut bâti au XVIIIe siècle par la famille de Larroque. C'est un corps de logis rectangulaire, assez simple et élevé d'un étage, avec pavillon central de forme quadrangulaire et recouvert d'ardoises. Ce pavillon compte deux étages surmontés de mansardes.

Devant le château est une cour fermée par une grille et des servitudes. En avant de cette grille, par conséquent de l'autre côté du chemin Bernos, il y a une belle garenne composée d'arbres de différentes espèces.

Le château Larroque appartenait à M. Bernos en 1866\footnote{GUILLON. \textit{Les châteaux historiques et vinicoles de la Gironde}.}. Il est actuellement la propriété de Mme Chambrelent, qui l'a quitté depuis quelques années pour aller habiter avenue Suffren, 33, à Paris.

L'immeuble a été réparé en partie. Il donne l'impression d'être totalement abandonné. Il est à vendre ou à louer, d'après la personne qui en garde les clés et qui a bien voulu nous le faire visiter\footnote{Le 2 juillet 1926.}.

\subsection{Château La Tour}

Ce domaine appartient à M. Victor Cousteau. Il est voisin de Chollet et s'ouvre sur le chemin Vieille-Tour. De l'autre côté du chemin, en face de la propriété La Tour, il y avait autrefois un immeuble appelé « Rostan ». À l'origine, les biens de Rostaing de la Tour devaient englober une grande partie de ce quartier de Talence. 

\subsection{Le château de Salles}

Il est situé chemin Banquey, 42, dans un joli cadre de verdure qui en dérobe la vue aux passants. C'est une construction moderne recouverte en ardoises. Le château de Salles a appartenu à M. Eugène Olibet, qui l'a laissé à ses descendants.

\subsection{Le château Henri II}

Il s'élève cours Galliéni, 86. Il appartient à M. Louis Faget, conseiller général, maire de Targon.

Ce château doit naturellement son nom à son style. Il fut bâti il y a une cinquantaine d'années. Les salons, les autres pièces, les couloirs sont garnis de tableaux, de bronzes, de marbres, de riches tapisseries, d'objets rares et de valeur. C'est un véritable musée d'art que M. Faget présente aux personnes qu'il reçoit chez lui.

Derrière le château est un vaste jardin tiré au cordeau, bien entretenu. Le propriétaire y avait réuni, il y a deux ou trois ans, une incomparable collection de camélias.

\subsection{Le château Maucamp}

Il est situé sur le chemin de Thouars, à gauche, à 500 mètres environ du château de Thouars. Sur le fronton de la grille d'entrée on lit : « Maucamp 202.» Une vaste pelouse plantée d'arbres et bordée, sur un côté, par des communs, précède le corps de logis, élevé d'un étage.

Maucamp n'est pas le nom d'un château, mais celui d'un domaine de vingt-trois hectares. Les titres de propriété font savoir que Maucamp appartenait, en 1792, à M. Laclotte. Il fut vendu, en 1793, à M. Duvergier ; en 1810, à M. Maillères ; en 1826, à M. Brunet ; en 1831, à M. Sapène ; en 1836, à M. Chaine ; en 1837, à M. Tulèvre, et le 13 mai 1856, à M. Anduze. Au décès de ce dernier, le domaine fut mis en vente sur licitation et adjugé, le 13 juillet 1858, à M. François Cuzol, qui avait épousé Mlle Elisabeth Claire Anduze. M. Cuzol en fit un grand vignoble qu'il exploita pendant près de cinquante ans, sous l'appellation de « Château Maucamp (Haut-Talence) ».

Le 2 avril 1920, M. Etienne Canac acheta Maucamp aux héritiers de M. Cuzol.

« La vigne n'existait plus dans le domaine, nous a dit M. Canac, et actuellement, il n'y a que des prairies ou des terres en friche qui servent de pâturage à un troupeau de quinze vaches laitières. »

\subsection{Le château de la Médoquine}

On a donné cette désignation à une vaste construction carrée, élevée de deux étages, située près de la route de Pessac, et dominant le quartier et l'établissement dit « la Médoquine ».

Dans une délibération relative à la voirie en 1813, il est question d'un « point au lieu appellé la Médoquine et l'encoigneure ou bien de M. Salles donnant sur le chemin de Pessac... »\footnote{Registre des procès-verbaux de la municipalité de Talence. (Archives de la commune.)}.

C'est la première mention que nous ayons trouvée de la Médoquine.

Le nom de ce quartier figure dans la \textit{Nomenctature des communes de la Gironde}, publiée en 1826\footnote{Archives municipales de Bordeaux.}.

Quelle est l'origine du nom de la Médoquine ? On pense qu'une femme originaire du Médoc a dû, tout au début du XIXe siècle, acheter et habiter le château qui a conservé son nom. Sans doute, en bon français, il eut fallu dire \textit{château de la Médocaine}, mais le patois du pays fît de Médocaine Médoquine, et ce dernier nom prévalut.

Le château de la Médoquine a compté parmi ses propriétaires M. Pédron, puis Mme Martin, sa fille, qui l'a revendu en 1866.

\subsection{Le château La Mission-Haut-Brion}

Ce domaine est situé en bordure de la route de Bordeaux à Pessac ; le vignoble s'étend sur une haute plaine graveleuse. Un portail d'entrée s'ouvre sur le chemin situé entre la voie ferrée et l'établissement de la Médoquine. Il y a sur le portail, au-dessous d'une croix, cette inscription :

\begin{center}
1757\\
CHATEAU\\
LA MISSION HAUT-BRION
\end{center}

Dans un livre intitulé \textit{Bordeaux et ses vins}\footnote{Ouvrage de Ch. COOKS et Ed. FERET.} et publié en 1898, on lit :

« Le vignoble de La Mission-Haut-Brion paraît avoir été fondé par une congrégation religieuse dont il porte le nom. »

Certes, le domaine de La Mission rappelle bien la congrégation de Lazaristes ou prêtres de la Mission\footnote{La congrégation des Lazaristes ou prêtres de la Mission fut fondée en 1625 par saint Vincent de Paul. Le nom de Lazaristes lui fut donné parce qu'elle avait été établie dans une maison qui avait appartenu à l'ordre de Saint-Lazare. }, qui en fut propriétaire ; mais la création du vignoble est antérieure à l'installation de ces religieux à Bordeaux.

Au milieu du dix-septième siècle, Antoine de Gourgues, premier président au Parlement de Bordeaux, était propriétaire de biens au Haut-Brion. Sa veuve, Olive de Lestonnac, par testament du 30 mars 1650, réserva une somme de 15.000 francs, dont le revenu devait être utilisé pour des missions dans le diocèse. Le 6 octobre 1654, Henry de Béthune, archevêque de Bordeaux, désigna Jean de Fonteneil, directeur général de la congrégation des prêtres du clergé, pour recueillir le legs et en accomplir les conditions.

Plus de neuf ans s'écoulèrent au bout desquels Jean de Fonteneil entra en possession de l'héritage.

Par acte passé le 26 janvier 1664, à l'archevêché, et retenu par Me Poitevin, notaire royal, Catherine de Mullet, veuve de Pierre de Lestonnac, héritière de la présidente de Gourgues, comme tutrice de ses enfants, s'engageait à payer, pour le service des missions, 17.000 au lieu de 15.000 livres. Par le même acte, elle cédait au directeur général de la congrégation des prêtres du clergé la « maiterie d'Haubrion située ez la paroisse de Talance... et une chambre basse à loger les vallets au fond, et en suitte un grand chay cuvier garni d'un fouloir en pierre de taille... et vingt-deux journaux de vigne... et cinq autres ou environ de bois tailhis, le tout ez Aubrion et aux environs du dit chay, confrontant du nord au grand chemin qui va de Bourdeaux à l'esglize de Pessac.. »\footnote{Archives départementales, G, 996.}.

%%% TODO : la note précédente doit être 17 bis à la p. 100 du doc d'origine %%%%
La congrégation des prêtres du clergé fut remplacée, en 1682, par une congrégation de Lazaristes ou prêtres de la Mission.

D'abord, deux prêtres de la Mission, MM. René Simon et Julien Guiot, furent envoyés par M. Joly, supérieur général des Lazaristes, « pour recevoir et exécuter les ordres de Mgr Louis d'Anglure de Bourlemont, primat d'Aquitaine, touchant leur établissement dans son séminaire ». Peu après, le 10 novembre 1682, M. Simon prit possession de la propriété du \textit{Hautbrion}, suivant les formalités alors en usage. En présence d'un notaire et de quatre témoins, il entra dans la maison et les chambres, dont il ouvrit et ferma les portes ; dans les vignes où il rompit plusieurs branches de sarment ; dans le bois où il cassa également diverses branches d'arbres, « le tout en signe de la vraie possession réelle, actuelle et corporelle »\footnote{L. BERTRAND. \textit{Séminaires de Bordeaux et de Bazas}, t. I, p. 289 et suiv.}.

Les Lazaristes \textit{étant entrés dans les vignes} n'eurent donc pas à les planter.

Dès leur installation Haut-Brion, les Lazaristes s'ocau cupèrent de mettre leur vignoble en valeur. Durant les années 1683 et 1684, ils plantèrent en vignes les cinq journaux de « bois tailhis » dont il est fait mention dans l'acte du 26 janvier 1664. Le revenu des Pères barbiches, comme on les appelait, parce qu'ils portaient une petite barbe à l'imitation de Vincent de Paul, fondateur de l'ordre, se montait bon an mal an, à 4.000 livres, ce qui était, pour le temps, un appréciable bénéfice.

Le 26 août 1698 eut lieu la bénédiction de la chapelle " de Haut-Brion » destinée aux prêtres et séminaristes de la Mission.

Ces religieux allaient en mission à l'étranger pour y répandre le christianisme et s'y livrer à l'instruction des jeunes clercs. Au cours de leurs voyages, — joignant l'utile à l'agréable — ils s'occupaient du placement de leurs vins.

Le produit de la Mission Haut-Brion — grand premier cru de graves de Bordeaux — était servi notamment sur des tables épiscopales.

Dans un document portant les dates 1754-1755, il est question d'un « bien, dans Talence, assez connu par son fameux nom de Haut-Brion », taxé 2.000 l. de vingtième, dans lequel M. le Président de La Tresne avait une portion\footnote{Archives historiques de la Gironde, t. XLIV, p. 365.\\Le marquis de La Tresne était président à mortier au Parlement de Bordeaux.}.

En 1791, les biens des Lazaristes furent saisis. Leur grand séminaire de la rue du Palais-Gallien\footnote{Aujourd'hui l'hôtel des postes.} abrita les conventionnels Tallien et Ysabeau, venus à Bordeaux pour régénérer la ville.

Le bien de Haut-Brion « à Talence et Pessac », maison de maître, bâtiments d'exploitation, 45 j. 24 r. vigne, appartenant à la congrégation de la Mission, estimé 100.000 livres, fut vendu, le 14 novembre 1792, au sieur Vaillant, pour la somme de 302.000 livres\footnote{M. MARION, J. BENZACAR, CAUDRILLIER. \textit{Documents relatifs à la vente des biens nationaux}, t. I, p. 630.}.

Le jeudi 15 novembre 1792, il fut procédé par-devant les administrateurs du Directoire du district de Bordeaux à l'adjudication de 31 tonneaux 3 barriques de vin rouge fin, de la récolte 1792 ; de 13 barriques de second vin rouge de cette même année, et de 2 barriques de vin rouge vieux, le tout récolté sur le domaine des ci-devant missionnaires de Bordeaux, situé au quartier du Haut Brion, paroisse de Talence.

La vente se fit en neuf lots.

Le premier lot (5 tonneaux) fut adjugé pour 645 livres le tonneau; le deuxième (5 tonneaux), 660 livres; les troisième et quatrième lots (5 tonneaux chacun), 665 livres le tonneau; le cinquième (6 tonneaux), 665 livres le tonneau ; le sixième ?

Le septième lot (3 barriques de second vin), 350 livres les trois barriques; le huitième (8 barriques de vin « treuillis » et 1 barrique de premier vin, fûté,) 425 livres prix total; le neuvième (2 barriques de vin vieux), 395 livres prix total\footnote{22}.

\asterism{}

Une controverse s'est élevée récemment à propos du Haut-Brion.

Le Haut-Brion est-il un lieu dit, comme le Tondu par exemple ? Est-ce le nom d'une propriété particulière à l'instar du Pape Clément ? La question nous ayant été soumise, nous avons répondu qu'à notre avis le Haut-Brion est un lieu dit 23. A l'appui de notre thèse, nous avons cité un texte mentionnant que durant les années 1425 et 1436, des esporles 24 furent consenties en faveur de Jeanne Monadey, citoyenne de Bordeaux, femme de Gaillard d'Arsac, damoiseau, notamment pour des biens sis dans les graves de Bordeaux à « Haut-Mont, alias au Brion ».

En l'année 1698, il fut question de baux à loyer ou à ferme des maisons et biens dépendant du séminaire de la Mission dans « les paroisses de Talence et Pessac » au lieu dit « à Haut-Brion ».

Il est difficile aujourd'hui de préciser le périmètre de ce lieu. Dans le \textit{Dictionnaire des postes et télégraphes} (année 1898), indiquant les noms de toutes les communes et des localités les plus importantes de la France, on lit : « Haut-Brion (Gironde) 25 habitants, commune de Pessac ».

Mais anciennement, le Haut-Brion s'étendait sur plusieurs paroisses. On a vu, d'après la vente de la Mission historique de Bordeaux (supplément) 1909. Le docteur G. Martin, qui a transcrit le document, dit que celui-ci est aux Archives départementales, sous la cote a q. 844 ». Or, ce document est introuvable. Nous n'avons pu, par suite, réparer une omission du docteur Martin portant sur la quantité et le prix du sixième lot. 23. Voir La Petite Gironde, n° du 27 avril 1926. 24. Esporles : ce qu'on donnait ou offrait au seigneur pour obtenir de lui l'investiture de quelque fief. 22. RevueHaut-Brion, que le domaine portant ce nom était situé dans les paroisses de Pessac et de Talence. Le rapport d'estimation en date du 31 janvier 1791 — conservé aux Archives départementales — précise : « Un bien de campagne appellé Haut-Brion, situé paroisse de Talence et Pessac, ayant été possédé par le grand séminaire ». On relève ce paragraphe dans la nomenclature des biens de deuxième origine à Talence : « Vigneras, père d'émigré, maison et 16 j. vigne du cru du Haut-Brion, estimé 24.180 l. 25. » Nous avons lu d'autre part : « Fumel. Domaine de Haut-Brion : 228 j. vignes sur lesquels 15 j. dans Bordeaux et 18 j. dans Talence 26. » D'où il ressort nettement qu'à la Révolution, le HautBrion dépendait au moins de trois communes. Au surplus, au cours de l'assemblée générale de la Société archéologique (juillet 1926), un des membres les plus distingués de cette société, M. P. Trial, a communiqué, au sujet du Haut-Brion, le résultat de ses recherches faites, aux Archives départementales, parmi les pièces des XVe, XVIe, XVIIe et XVIIIe siècles. Suivant M. Trial, « le nom de Haut-Brion s'appliquait autrefois non seulement à une maison noble, mais aussi à un lieu dit qui s'étendait vraisemblablement à toute la partie surélevée des graves de Bordeaux ou hautes graves 27 ». La Mission Haut-Brion passa, en 1821, aux mains d'un riche colon de la Nouvelle-Orléans : Célestin Chiapella. 25. M. MARION, J. BENZACAR, CAUDRILLIER. des biens nationaux, 1911. 26. Ibid. 27. Liberté du Sud-Ouest, n° du 18 Documents relatifs à la vente juillet 1926.Celui-ci laissa le domaine à son fils Jérôme Chiapella, qui accrut encore la renommée du domaine. La récolte avait atteint 41 tonneaux en 1835. On trouve dans le Nouveau guide de l'étranger à Bordeaux pour 1856 les détails ci-après : Vins rouges de Graves : 1er cru La Mission (Talence) 900 à 1.000 francs par tonneau. En 1861, le jury de l'Exposition de Londres décerna une médaile d'or à M. J. Chiapella pour l'envoi de ses bouteilles. Cette haute récompense témoignait de la grande faveur dont jouissaient les vins de la Mission au delà de la Manche, il y a quatre-vingts ans. Leur excellente réputation ne fait, du reste, que s'accroître chez nos amis anglais. Les Américains de New-York et de la Nouvelle-Orléans appréciaient fort, à la fin du XIXe siècle et jusqu'à ces temps derniers encore, La Mission Haut-Brion. Ils sont à cette heure au régime sec... officiellement du moins! Le commerce a payé à M. J. Chiapella 3.000 francs le tonneau une portion de la récolte de 1865. Le château La Mission Haut-Brion devint, en 1903, la Propriété de M. Victor Coustau. Il appartient aujourd'hui à M. F. Woltner, conseiller du commerce extérieur. Le mercredi 15 septembre 1926, M. Woltner, avec la meilleure bonne grâce, nous a fait les honneurs de son domaine. Nous étions particulièrement désireux de voir la vieille chapelle des Lazaristes. Nous avons vu, réunies dans ce pieux asile, de curieuses pièces religieuses très anciennes pour la plupart. Au plafond, des dates sont inscrites, en chiffres d'or. Ce sont les années fameuses qui ont produit les récoltes le plus justement réputées. Citons quelques-unes de ces dates qui garnissent, du reste, le plafond de la chapelle : 1847, 1848, 1858, 1864, 1869, 1870, 1875, 1877, 1895, 1899, 1914, 1916, 1918.Sur une table est un gros registre où apposent leurs signatures les visiteurs du château de la Mission Haut-Brion. Nous y avons relevé les noms de personnalités françaises et étrangères, appartenant au commerce, à l'industrie, aux arts, aux lettres, à la politique. Des commentaires flatteurs accompagnent souvent les signatures. Le dimanche 20 juin 1926, à l'occasion du troisième Congrès national des conseillers du commerce extérieur, M. Woltner a offert une garden-party dans son château La Mission Haut-Brion. L'histoire nous apprend que Rabelais, fin gourmet, appréciait vivement les vins des Graves. Danton aussi. Ce nectar n'était pas seulement agréable à son palais : il favorisaif sa vigueur et son éloquence. Le maréchal duc de Richelieu, gouverneur de la Guienne, connu par ses défauts, autant que par ses qualités, était grand amateur de bonne chère et de vins capiteux. Il disait un jour, parlant de la Mission Haut-Brion, son cru de prédilection : « Si Dieu défendait de boire, aurait-il fait ce vin si bon ? » 28. Momplaisir Au 82, chemin Peydavant se trouve le domaine de Momplaisir. Il est ainsi désigné sur le plan cadastral (1847) « A Montagne ». 28. Le Sud-Ouest de la France, 1925, p. 148. 29. Archives particulières de Mme Lortau.La façade du château présente deux étages au-dessus du rez-de-chaussée ; elle est ornée de refends, surmontée d'un fronton, et flanquée de deux tourelles hexagones en saillie sur la ligne de la façade. Détail curieux : une des tourelles est coiffée d'un toit conique de forme très élancée; l'autre tourelle supporte une toiture en mansardes. Toutes deux sont couvertes en ardoises. Les première actes notariés conservés par la propriétaire actuelle, Mme Lortau, datent du 25 germinal an XII. Momplaisir a compté beaucoup de propriétaires parmi lesquels le citoyen Louis Cluchet, habitant Bordeaux, et le citoyen Jouis, fontainier de la ville, demeurant à la Font de l'Or, sur le port, 145. Le 17 janvier 1763, il y eut un accord entre les jurats et. le sieur Jouis par lequel ce dernier devait « donner par écrit le secret et le détail du mécanisme de la machine hidraulique qu'il avait imaginée pour le service des fontaines qui étaient sur le port » 30. La Font de l'Or, située quai de la Grave, alimentait quatre fontaines placées sur les quais; elle tenait sa désignation non point de la qualité de ses eaux, comme on pourrait le croire, mais de. « l'abondance de sa source » 31. Après le citoyen Jouis, Momplaisir appartint successivement : Au comte Chastenet de Puységur, lieutenant général des armées du roi, gentilhomme de sa chambre 32; à. M. Lanusse ; à M. Rataboul (ce dernier morcela le domaine et le réduisit à trois hectares, tel qu'il est de nos jours) ; à M. Sercam, prêtre; à M. Bethezet. 30. Inventaire sommaire des registres de la Jurade, t. VI. 31. Bernadeau. Le viographe bordelais. 32. Il s'agit vraisemblablement de Charles-Jacques-Louis-Maxime de Chastenet, comte de Puységur, qui était maréchal de camp en 1815 et fut nommé lieutenant général honoraire en 1826.Le 19 mai 1917, M. Lortau lit l'acquisition de Momplaisir. On conte que le fontainier Jouis, à une époque troublée par un changement de gouvernement, a apporté l'or de la ville dans un souterrain reliant « Momplaisir à l'église de Talence.» N'ajoutant aucun crédit à celte histoire, les propriétaires de Momplaisir n'ont jamais cherché à découvrir. l'entrée desdits souterrains. Au surplus, en admettant qu'on y aurait déposé de l'or, il ne put y être que momentanément caché. 33. Le Mémorial bordelais.