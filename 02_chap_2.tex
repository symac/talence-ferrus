%%%%% Dans le fichier PDF, paragraphe 2 qui commence à la vue 37 %%%%%

\section{Châteaux historiques}
\subsection{Château du Prince Noir}

Dans la légende d'une gravure représentant un plan de Talence, et publiée à la fin de la plaquette \textit{Notre-Dame-de-Talence dans la chapelle des Monges}, du père Royer, on lit : 

%%%% TODO : voir comment centrer comme sur le fichier source
\begin{center}
7. Château où est né le Prince Noir.
8. Habitation dudit prince. 
\end{center}

Le numéro 7 se rapporte à l'édifice qui s'élève vers le nord de Talence, en bordure de la ligne du chemin de fer, côté gauche, en allant vers Bayonne, et que l'on désigne communément sous le nom de château du Prince Noir.

Le numéro 8 concerne le château de Thouars.

Les recherches effectuées dans les divers dépôts d'archives n'ont pas permis de découvrir un document relatif à la naissance, à Talence, du fils célèbre d'Édouard III. D'autre part, les ouvrages biographiques sont muets en ce qui touche le lieu où vit le jour le Prince Noir.

Suivant une thèse, signée J. Moisant, le Prince Noir naquit le 15 juin 1330, dans le château royal de Woodstock, près d'Oxford, d'où son surnom : Edward de Woodstock\footnote{\textit{Le Prince Noir}. (Bibliothèque municipale de Bordeaux.)}. Il débarqua à Bordeaux le 20 septembre 1355\footnote{\textit{Livre des Coutumes}, t. V, p. 439.}. Lé lendemain, en présence de plusieurs seigneurs de la Guyenne, de chanoines, de nobles et de bourgeois de Bordeaux, le prince de Galles fit lire au maire, aux jurats et au peuple de la ville, convoqués dans la nef de SaintAndré, les lettres patentes par lesquelles Edouard III, son père, l'avait nommé, le 10 juillet précédent, son lieutenant en Guyenne et dans le royaume de France.

Le savant abbé Baurein écrivait en 1785 : 

« On prétend que le château de Thouars, à Talence, était le lieu où les rois d'Angleterre allaient prendre le plaisir de la chasse dans le temps où ils faisaient quelque séjour à Bordeaux, et que ce fait nous a été transmis par une tradition ancienne\footnote{\textit{Les Variétés bordeloises}.}. »

L'abbé Baurein n'a pas connu le document attestant qu'il y eut, à l'époque de la domination anglaise, une « forêt royale » à Talence. D'où sa prudente réserve.

Cela n'est pas douteux, les souverains d'Angleterre, accompagnés par les seigneurs de leur cour, chassaient dans la forêt en question, et il est également certain qu'un pavillon à leur intention devait s'élever sur le territoire de Talence.

Ce pavillon était-il bâti sur l'emplacement où nous voyons la construction connue sous le nom de « château du Prince Noir? » C'est possible.

Contradiction peu banale : alors que le père Royer fait naître le Prince Noir dans le manoir qui porte son nom, Edouard Guillon fait édifier ce même manoir par le prince lui-même\footnote{\textit{Les châteaux historiques et vinicoles de la Gironde}.}.

Le château du Prince Noir était appelé anciennement le « château de Brama » D'où vient cette dénomination ? A-t-elle un rapport quelconque avec le mot bramer, crier, en parlant du cerf ? L'hypothèse est soutenable, car le fils d'Édouard III et ses invités durent courir le cerf aux environs du pavillon isolé à l'entrée des landes.

Ce pavillon passa aux successeurs du prince de Galles ; nous le trouvons, au XV\ieme{} siècle, en possession\todo{texte original : posssesion} de la famille Roustaing. Les membres de cette famille ajoutèrent à leur nom celui de leur domaine. L'historien du Haillan\footnote{Bernard de Girard, seigneur du Haillan, historiographe de Charles IX et d'Henri III, né à Bordeaux en 1535, mort à Paris en 1610.} cite Arnaud de Roustaing « seigneur de Brama ». Un Jean de Roustaing était, en 1500, « seigneur de Brama et citoyen de Bordeaux ».

Les Roustaing démolirent le pavillon de chasse et élevèrent le château actuel.

Sur la \textit{Carte} de Belleyme, il y a, à côté du dessin figurant le château, l'indication « Rostain le Brana ».

Il y avait un « fief de la maison de Rostaing dans le » plantier de Pisselèbre\footnote{Plan 1255. (Archives départementales.)}.

\asterism{}

Vers la fin du règne de Louis XVI, le château passa dans les mains de M. Bruno ou Bruneau. L'abbé Baurein, qui le visita alors, a laissé une description des armoiries qui s'y trouvaient :

« On aperçoit, dit-il, sur une porte de la maison noble de Roustaing (c'est ainsi qu'il est écrit dans les anciens titres), des armoiries portant trois écussons ; celui du milieu est surmonté d'un casque. Ces armoiries désignent des alliances et quoiqu'elles soyent très antiques, elles sont très distinctes. À chacun des côtés sont les chiffres suivants R A R. Au-dessus de la même porte sont deux figures dont l'une représente un homme et l'autre semble représenter la figure d'une femme.

» On voit sur les piliers de l'escalier de la même maison deux lions, tenant chacun à la griffe un écusson. Sur l'un est représenté trois fleurs de lys barrées d'un côté, et sur l'autre côté du même écusson, un soleil et trois étoiles au-dessous.

» L'écusson du second lion représente de chaque côté une sirenne et un lion\footnote{\textit{Questionnaire}.}. »

Baurein ajoute, parlant de cette maison qu'il jugeait très ancienne :

« Il est possible que c'est le chef-lieu du double-fief de Roustaing, possédé tant par M. Roberdeau que par M. Bruneau et que ce premier, quoique plus considérable, n'est qu'un démembrement du second\footnote{\textit{Ibid}.} ».

L'édifice est dénommé \textit{château Bruno} sur le plan cadastral de Talence.

Sa désignation de « château du Prince Noir » serait, dit-on, récente. Le père Royer la situe vers 1850\footnote{\textit{Le Démocrate de Talence}, n° du 4 janvier 1925.}.

Nous pouvons avancer que cette désignation remonte au moins au début du XIX\ieme{} siècle. Félix Annoni\footnote{Félix Annoni, peintre d'histoire, d'origine italienne, né à Bordeaux vers 1786, mort dans la même ville vers 1850.} exécuta, \textit{en 1812}, des dessins de monuments divers de la Gironde « détruits en totalité ou en partie », et parmi ces édifices il y avait « le château du Prince Noir, à Talence »\footnote{\textit{Société archéologique de Bordeaux}, t. III, année 1876, p. 47. Les dessins en question furent trouvés chez un brocanteur par MM. Braquehaye et Piganeau, membres de ladite Société.}.

D'autre part, dans un ouvrage qu'il publia en 1835, Henri-Charles Guilhe, parlant du même prince, note que « sa maison de plaisance et de chasse se voyait naguère entre Talence et Pessac »\footnote{\textit{Études sur l'histoire de Bordeaux, de l'Aquitaine et de la Guienne}.}.

En 1819, le château fut acheté par M. Grant qui l'habita un certain temps, puis l'abandonna pour revenir y mourir vingt ans après. Dans l'intervalle, l'immeuble avait été loué à un Lyonnais qui y établit une fabrique de toiles cirées.

En 1862, les créanciers du propriétaire vendirent le domaine à M. Jouassin. Ce dernier ne l'avait acheté que pour le revendre ; il le distribua pour un morcellement, et " fit abattre l'allée qui allait du château à l'église »\footnote{\textit{Les châteaux historiques et vinicoles de la Gironde}.}.

%%%% TODO : graphie de mademoiselle p. 41 %%%%
Le château appartient maintenant à Mlle Vialatte, qui en a fait une maison de santé pour les malades et les vieillards.

Il présente deux corps de logis rectangulaires, posés Parallèlement, séparés par une cour et reliés, à l'est, par un mur ; à l'ouest, par une galerie à laquelle on arrive par un large escalier de pierre prenant naissance dans la cour. Un seul corps de logis offre, aujourd'hui, un intérêt architectural : celui qui est au midi. Il est surmonté d'un pignon et flanqué d'une tourelle à toit conique avec une flèche supportant le léopard britannique. Les murs, percés de fenêtres à lierre, de tapissés endroits, meneaux, et, par montrent des sculptures du XVI\ieme{} siècle.

L'autre corps de logis, couvert d'une toiture ordinaire, a été transformé en chais.

Dans le jardin potager, il y a un puits très ancien à côté duquel est l'entrée d'un souterrain dont on ne connaît pas la sortie, nul n'ayant osé parcourir ce sombre couloir dans toute sa longueur...

Ce souterrain aboutirait à un château, du côté de Gradignan. Il passe tout près de l'église de Talence. Les hommes qui s'y sont engagés n'ont jamais dépassé le point à hauteur de l'église, faute de pouvoir s'éclairer. «Leurs bougies s'éteignaient », paraît-il, quand ils voulaient pousser plus loin leur visite souterraine.

Le promeneur arrêté un instant devant le château du Prince Noir, ou plutôt devant l'édifice qui a remplacé le pavillon de chasse, se prend, malgré lui, à réfléchir. Il voit, par la pensée, le fameux Prince Noir — ainsi nommé à cause de la couleur de son armure qui — vainquit à Poitiers le roi de France et le conduisit, prisonnier, au doyenné de Saint-André\footnote{Le doyenné de Saint-André s'élevait sur le côté septentrional de la cathédrale, sur une partie du terrain de la place Pierre-Laffitte. Il a été appelé à tort « abbaye » par le chroniqueur Froissart, Élie Vinet et la plupart des écrivains venus après eux.
	
	Richard II, roi d'Angleterre, fils du Prince Noir, naquit au doyenné de Saint-André. (Renseignements fournis par la Bibliothèque de Londres.)}. Plus tard, le connétable Duguesclin, fait lui aussi prisonnier par le Prince Noir, fut enfermé dans le même doyenné...

Le prince de Galles actuel a traversé plusieurs fois Bordeaux, se rendant à Biarritz. Peut-être, durant l'un de ses séjours en France, tiendra-t-il à visiter Talence, où demeure toujours le souvenir de son ancêtre le Prince Noir, de qui David Hume a dit très justement : « Il laissa une mémoire immortalisée par de grands exploits, par de grandes vertus et par une vie sans tache. »

\subsection{La Tour de Rostaing}

L'abbé Baurein parle d'une maison noble de Roustaing distincte de celle du château de Brama ou du Prince Noir, et qu'il appelle \textit{la Tour de Rostaing}. Le manoir habité par cette famille était connu à Talence sous le nom de la Vieille Tour petit il l'origine, de n'avait qu'une été, sorte à — fort — et il s'élevait à proximité de la ligne de l'ancien chemin de fer de La Teste. S'il est vrai que le château du Prince Noir ne passa qu'au XV\ieme{} siècle aux Roustaing, devenus depuis lors seigneurs de Brama, comme l'a écrit Édouard Guillon\footnote{\textit{Les châteaux historiques et vinicoles de la Gironde}.} il s'ensuit que ces personnages se fixèrent à Talence bien après les Rostaing « seigneurs de la Tour ». Et dans ces conditions, ce serait à cette dernière famille qu'appartiendrait \textit{Arostanh de Talanssa}\footnote{Arostanh de Talanssa — Guillaume Rostang — était maire de Bordeaux en 1229; il prenait le litre de donzet, de damoiseau ou d'écuyer.} — ou \textit{Rustanhdus de Thalancia}, — qui fut l'un des quatorze commissaires désignés le 20 décembre 1261 par le prince Édouard\footnote{Ce prince devint roi d'Angleterre sous le nom d'Edouard 1er. Il régna de 1272 à 1307 ; son respect des libertés parlementaires lui valut d'être surnommé « le Justinien britannique ».}, fils de Henri III, pour juger les questions litigieuses pendantes entre le prince et la commune, notamment au sujet « des alluvions et de padouens de Bordeaux »\footnote{\textit{Livre des Bouillons}, tome 1, p. 365.\\Les padouens ou vacants étaient des lieux de pacage.}.

Des titres d'octobre 1547 font mention de Louis Rostaing, écuyer, seigneur de la Tour.

Dans un acte relatif à une reconnaissance féodale en date du 2 septembre 1602, il est question d'un chemin commun ou ruette conduisant de la chapelle de Talence. à la Tour de Roustaing\footnote{\textit{Archives historiques de la Gironde}, t. XXXV, p. 420.}. Il s'agit de la « ruette de Labricq »\footnote{Chemin des Briques.}, précise le même document.

La tour de Rostaing existait encore en 1785. On la trouve ainsi mentionnée sur le \textit{Questionnaire} de l'abbé Baurein : « La Tour de Rostin, à Mad. Saige ». Mais depuis un certain temps déjà, on avait construit dans le domaine un grand bâtiment recouvert à la Mansard, et n'ayant ni pavillons ni tourelles. Partie de l'ancienne propriété de la famille des Rostaing, seigneurs de la Tour, était passée dans les mains de M. de Cholet, « procureur du roi en l'amirauté de Guienne ». Le 8 décembre 1782, M. de Cholet fut « unanimement choisi et nommé grand sindic de la paroisse de Talence » pour présider à ses délibérations\footnote{Livre des comptes de la paroisse 1730-1793. (Archives départementales.)}.

Les Chollet étaient fixés depuis longtemps à Talence. Le 6 février 1714, Pierre de Chollet, receveur des décimes, s'était marié, en l'église Saint-Genès, avec Marie David\footnote{G G, 15, Archives communales.}.

Sous la Révolution, l'immeuble devint bien national, mais ne fut pas vendu.

Le 8 thermidor an II, les sieurs Boireau, officier municipal, et Capeyron, notable, furent nommés commissaires à l'effet de se transporter « à Cholet pour faire le recensement des vins, barriques neuves, de vidange, vieilles et mairain, tant vieux que neufs ou travaillé »\footnote{Registre des procès-verbaux de la commune de Talence. (Archives de la mairie.)}.

Ledit « état fait » devait être envoyé aux administrateurs du district, lesquels l'avaient demandé par lettre du 6 thermidor. Quelques jours plus tard, on vendit « à la chaleur des enchères et extinction de trois feux, deux tonneaux, deux barriques trois-quarts vin provenant du bien de Chollet »\footnote{\textit{Ibid}.}.

En 1804, Napoléon I\ier{} fit de Cholet le chef-lieu d'une sénatorerie et en pourvut la même année de Pérignon, maréchal de France\footnote{Originaire de Grenade (Haute-Garonne), de Pérignon s'était distingué pendant les guerres de la Révolution contre les armées espagnoles. Napoléon I\ier{} l'avait compris dans la première promotion des maréchaux, le 18 mai 1804, et nommé comte 1808. Rallié aux Bourbons en 1814, de Pérignon fut nommé pair de France et reçut, en 1817, le titre de marquis. Il mourut l'année suivante.}. Le maréchal échangea bientôt cette résidence contre un domaine aux environs de Paris.

Redevenu propriétaire de Cholet, l'Etat le revendit sous la Restauration.

M. Cayron était, en 1858, propriétaire de Cholet, qui appartint, plus tard, à M. Wustenberg, puis à M. Jouassin.

Le domaine abrite à présent une pouponnière. Sur la porte d'entrée, qui s'ouvre chemin de la Vieille-Tour, en lace de la rue Camille-Pelletan, on lit : 

%%%% TODO : voir pour en faire un paragraphe uni %%%%
\begin{center}
121\\DOMAINE DE CHOLET\\HOSPICES CIVILS DE BORDEAUX\\MAISON MATERNELLE
\end{center}

\subsection{Château de Thouars}

Comme nous l'avons dit dans le premier chapitre, il est vraisemblable que le château et le domaine de Thouars occupent une partie du terrain de l'ancienne bastide de Baâ, construite dans la forêt royale »\footnote{Voir page 34.}. Mais sur l'em« placement du château, il y eut, semble-t-il, primitivement, un pavillon servant de rendez-vous de chasse — toujours aux rois d'Angleterre.

Guillon signale que ce lieu de rendez-vous « fut vendu ou donné en fief aux d'Agès, \textit{dans le XIV\ieme{} siècle} »\footnote{\textit{Les châteaux historiques et vinicoles de la Gironde}.}. Erreur. 

Au début du XV\ieme{} siècle, Bernard de Lesparre était « seigneur de La Barde et de Thouars »\footnote{\textit{Archives historiques de la Gironde}, t. XXXIV, p. 292.\\Bernard de Lesparre, amiral anglais, à la tête de dix navires, livra bataille en Gironde à la flotte de l'amiral du roi de France, et remporta la victoire. Il fit plusieurs prises sur lesquelles des capitaines anglais prétendirent avoir des droits, bien qu'ils n'eussent pas assisté au combat, Il s'ensuivit un procès qui fut jugé en 1407, par Henry, évêque de Bath et de Wells, et par Richard de Grey. (Public Record Office, chancery Miscellanea 25, 6, n° 8.)}. Ce personnage, par testament daté du 11 avril 1412, « donna et laissa par droit et titre d'institution à Lanselot de Lesparre, son fils naturel et légitime, le lieu et seigneurie de Toars, avec toutes ses dépendances et toute la juridiction \textit{mera, mixta et imperi}\footnote{Droit de haute, moyenne et basse justice.}, appartenant audit lieu de Toars »\footnote{\textit{Archives historiques de la Gironde}, t. XXXIV, p. 292.}.

Par la suite, Thouars cessa pendant un certain temps d'être une seigneurie. Un acte du 24 juillet 1481, passé devant Arnaldo de Guissamelhano, fait savoir que « la maison de Thouars et domaines en dépendant étaient tenus en toute roture »\footnote{Série D D, 4. (Archives municipales de Bordeaux.)}. Il en fut ainsi jusqu'en 1505, époque à laquelle la famille Dagès prit possession du domaine. Il appartenait précédemment à « noble François Vacquier, citoyen de Bordeaux »\footnote{Archives départementales de la Gironde (minutes de Jacques Devaux, 199-1). Le membre de phrase cité est reproduit d'après la \textit{Société archéologique de Bordeaux}, t. IV, année 1877.}.

C'est le 24 juillet 1505 que Pierre Dagès devint propriétaire de la maison de Thouars. L'acte, passé devant le notaire Pierre de Bosco, porte « l'anoblissement de Lamothe de Thouars, ses appartenances et dépendances, fait par la ville audit Dagès, seigneur de Saint-Magne, à la charge de prestation, hommage et dénombrement du tout »\footnote{Série D D, 4. (Archives municipales de Bordeaux.)}.

Le seigneur de Thouars était donc vassal des « seigneurs jurats » de Bordeaux. En signe d'hommage, il devait leur offrir, tous les vingt-neuf ans, un épervier volant. Le même hommage devait être rendu, suivant l'acte constitutif de nobilité du 24 juillet 1505, à chaque mutation de propriétaire.

Le château présentait un massif quadrilatère, flanqué de quatre grosses tours, couronnées de créneaux et de mâchicoulis ; une porte ogivale s'ouvrait dans la courtine reliant les deux tours en façade. Les remparts étaient crénelés et fortifiés, comme les tours. Un large fossé entourait le château, qui passait pour une. des plus belles places fortes de la région\footnote{Le peintre d'histoire Félix Annoni a laissé une très intéressante reconstitution par le dessin du château de Thouars, tel qu'il était au XVIe siècle et tel qu'on pouvait le voir encore vers 1745. Ce dessin a été fait sur les lieux, d'après la tradition locale, les traces existantes et d'anciennes vues de l'édifice. (\textit{Musée d'Aquitaine}, t. II-III, p. 197.)}.

Un Thibaud d'Agès réussit dans plusieurs missions officielles. Il accueillit dans son château Mme de Nevers chargée de recevoir à Bordeaux Eléonore d'Autriche\footnote{Éléonore d'Autriche, fille de l'archiduc Philippe d'Autriche et de Jeanne de Castille, était la soeur aînée de Charles-Quint. Mariée en 1519 avec Emmanuel le Grand, roi de Portugal, elle était devenue veuve deux ans plus lard; son mariage avec le roi de France fut célébré à Mont-de-Marsan le 4 juillet 1530.}, devenue reine de France par son mariage avec François I\ier{}.

La reine fit son entrée à Bordeaux le 11 juillet 1530. On lit au début de la relation de cette solennité :

« Mme de Nevers vint de Touars disner à Bordeaulx, accompagnée de Madame dé la Trémoille, la mareschalle de la Marche, la grande seneschalle, des dames de Mirebeau et aultres, en nombre XXXIIII, toutes vestues les principales de veloux cramoysy, les aultres de satin cramoisy, chacune ouvrées et diaprées à qui mieulx mieulx\footnote{\textit{Archives historiques de la Gironde}, t. IV, p. 157.}. »

La reine arriva à Bordeaux avec les deux enfants de François I\ier{}, qui avaient été laissés en otage à Madrid, et pour la rançon desquels la ville de Bordeaux avait prêté 10.000 écus.

%%%% TODO : reprendre abbréviation mme p. 48 %%%
Après la réception, Mme de Nevers conduisit Eléonore d'Autriche au logis du premier président. Elle lui tint compagnie jusqu'à la chute du jour, puis « alla prendre messieurs les enfants pour les mener à Thouars, à Madame\footnote{\textit{Archives historiques de la Gironde}, t. IV, p. 160.}, belle-soeur de François I\ier{}.

« Le roy, sur le soir, arriva audit Bordeaulx, et le lendemain la royne partit pour aller visiter Madame, qui estoit malade\footnote{\textit{Ibid}., p. 161.} ».

Ainsi se trouvèrent réunis à Thouars la reine, les enfants de François I\ier{}, Madame, la dame de Nevers et les dames de leur suite.

Dans des textes de 1551 et 1554, un Pierre d'Agès est qualifié « seigneur de Saint-Maigne, Thouars et La MotteSaint-Sulpice »\footnote{DUCOURNEAU. \textit{La Guienne monumentale}.}.

Dans un arrêt du Parlement du 1\ier{} décembre 1553, il est fait mention de René d'Agès, seigneur de Thouars, « eschanson du roy »\footnote{Archives départementales. B, 68, portefeuille. (Note de M. Michelot.) — Echanson, officier qui servait à boire à un grand personnage.}.

%%%% Note 40 bis dans la page 48 à reprendre tel quel %%%%
À la requête du procureur général du roi, le fils aîné du seigneur de Thouars, poursuivi pour « crime d'hérésie », fit défaut devant la Cour du Parlement, le 4 avril 1556, avant Pâques. La Cour décida de continuer les poursuites. L'arrêt porte les signatures ci-après : de Fauguerolles, Baulon, de Vergoing, de Alès, de Massei, de Mung, de Pomiers, de Mérignac\footnote{Archives départementales. B, 104. (Note de M. Michelot.)}.

\asterism{}

Au printemps de 1565, Charles IX, venant de séjourner dans le Languedoc, suivit les rives de la Garonne, se dirigeant vers Bordeaux. L'accompagnaient : sa mère Catherine de Médicis ; Jeanne d'Albret, reine de Navarre ; Henri de Bourbon (futur Henri IV), alors âgé de douze ans, et plusieurs personnages de la cour.

Les jurats avaient décidé de faire au roi de France une réception grandiose. Les préparatifs n'étant pas terminés au moment où Charles IX arrivait aux portes de Bordeaux, le monarque fut invité à retarder son entrée de quelques jours ; il acquiesça à ce désir, et c'est ainsi qu'il fut, pendant près d'une semaine, l'hôte de René d'Agès.

Voici ce que rapporte à ce sujet Abel Jouan\footnote{Charles IX qualifiait ainsi Abel Jouan : « sommelier en nostre cuisine de bouche. »}, qui suivait la cour en qualité d'historiographe :

« Le mardi, troisième jour d'avril, le roy alla disner et coucher à Toars qui est un petit château à une lieue de ladite ville. (Bordeaux) ; auquel lieu le roi séjourna six jours pendant lesquels l'entrée de ladite ville se préparait\footnote{\textit{Recueil et discours du voyage du Roy Charles IX}, 1565. \textit{Pièces fugitives pour servir à l'Histoire de France}. Tome I\ier{}.}. »

Catherine de Médicis reçut au château de Thouars les autorités de la ville.

Charles IX se livra au plaisir de la chasse dans les forêts des environs ; le lit de sa chambre était « en bois de noyer, à quatre quenouilles, tout sculpté de fleurs de lys »\footnote{Édouard GUILLON. \textit{Les châteaux historiques et vinicoles de la Gironde}.}.

Mais suivons la chronique d'Abel Jouan : « Le roy partit de Toars le lundi 9° jour dudict mois d'avril pour aller dîner à Frands\footnote{Château de Franc, à Bègles, près de la Garonne, sur un des bras de l'Eau-Bourde appelé « l'Estey de Franc ».}, qui est une belle petite maison ; et après dîner s'en alla embarquer sur la Garonne, en un batteau que les maire et jurats de Bordeaux lui envoyèrent, et s'en alla descendre au-dessous du Chasteau-Trompette qui est un fort chasteau qui fait le coin de la ville, sur le bord du port. »

Des notes historiques et géographiques complétant le texte d'Abel Jouan font savoir que celui-ci ignorait dans quelle localité se trouvait le château de Thouars.

« La carte du Bourdelois de Delisle, et celle de la direction de Bourdeaux de Nolin, écrivait l'historiographe, ne donnant point la position de Toars, ni d'aucun lieu qui en approche, il faut attendre que quelqu'un plus exact et plus curieux nous l'apprenne. En attendant, hazardonsen la position : long. 16 d. 57 m. ; lat. 44 d. 53 m. »

René d'Agès avait des accointances avec les protestants, qui faisaient alors un effort considérable pour gagner de nouveaux adeptes à leurs idées. Il leur Ouvrit « toute grande sa maison de Thouars, leur permettant d'y faire le prêche à différentes reprises »\footnote{A. GAILLARD. \textit{La Baronne de Saint-Magne}, t. I, p. 140.}. Le Parlement de Bordeaux commença contre René d'Agès une instruction criminelle, mais ce seigneur, ancien échanson du roi, eut des protecteurs puissants, et tout se termina au mieux de ses intérêts. Il fit son testament le 16 avril 1572.

En 1575, François d'Agés, fils aîné de René d'Agès, était en procès. 

Le 24 juillet 1612, on procède à l'inventaire des meubles, titres et papiers délaissés par le baron François d'Agès « dans la maison noble de Thouars, sise paroisse de Talence, banlieue de Bordeaux »\footnote{Léo DROUYN. Notes manuscrites.}.

Le 14 septembre 1634, on baptisa dans l'église SaintGenès-de-Talence François de Ruat, qui eut pour parrain « François d'Agès, baron et seigneur de Thouars »\footnote{G G, 3 (Archives communales).}.

Le 20 octobre 1638 eut lieu, dans la même église, le baptême d'Eléonor d'Agès, qui fut chevalier, seigneur « baron de Saint-Magne, Thouars, Vibrac et autres places. » Il était né au château de Thouars.

Le 27 juillet 1657, Antoine de Cazenove, écuyer, seigneur de Lérisson, épousa, par contrat retenu par Me Lavail, notaire royal, « damoiselle Marguerite d'Agés »\footnote{BOURROUSSE DE LAFFORE. \textit{Nobiliaire de Guienne et de Gascogne}, t. III, p. 221.} dont René d'Agès avait été l'ancêtre paternel.

Nous entrons alors dans une période assez mouvementée. Les créanciers des d'Agès avaient fait saisir et vendre la seigneurie de Thouars. Mais celte vente n'était pas définitive. Les d'Agès reviennent par la force dans leur ancien domaine. Les jurats bordelais envoient des soldats occuper le château et rétablir l'ordre. Une héritière, Suzanne de Lavergne, fit à son tour saisir le domaine. Enfin il y eut un procès qui dura jusqu'en 1672\footnote{Notes communiquées par M. le comte Aurélien de Sarrau.}.


\asterism{}

Dans le document ci-après est rappelé le droit des maire et jurats sur le domaine de Thouars :

« Le sieur procureur scindic a remis une ordonnance rendue par les commissaires establis pour le domaine du roy, portant enjonction. aux pocesseurs des fief et rentes qu'ils possèdent dans le païs bourdelois d'avoir à donner leur adveu et dénombrement, qui a esté donné à M. Demons, conseiller et commissaire aux requêtes du palaix, pour la maison noble de Touars, et d'autant que ladite maison est mouvante de la ville en la Conté d'Ornon, attant a requis acte de la remise de ladite ordonnance, et qu'il luy soit permis de se pourvoir pour deffendre le droit que MM. les maire et jurats ont sur ladite maison en qualité de comtes d'Ornon, ce quy lui a esté permis\footnote{Registre de la Jurade du 1\ier{} août 1672 au 1\ier{} août 1673. (Archives municipales.)}. » 

Par acte du 22 mars 1692, la seigneurie de Thouars passa dans les mains de Jacques Demons, « conseiller du roi en la cour de Parlement de Bordeaux, et commissaire aux requettes du Palais\footnote{Série D D, carton 155. (Archives municipales de Bordeaux.)} »

Le 28 novembre 1705, un mariage fut célébré dans la chapelle de M. Démons, au château de Thouars.

La mort de Jacques Demons, seigneur de la maison noble de Thouars, « décédé dans sa maison, sur les Fossés du Chapeau Rouge », fut notifiée le 4 mars 1720\footnote{Série G. Bénéficiers de Saint-Michel, p. 271.}.

Par contrat du 8 février 1771, Thouars devint le bien de « Messire Michel-Joseph de Gourgues, président au Parlement de Bordeaux. » Il fut acheté peu après par le président de Lalanne\footnote{Le président de Lalanne mourut le 14 juillet 1774.}, qui eut pour exécuteur testamentaire Léonard-Casimir Le Comte.

Léonard-Casimir Lecomte, chevalier, captal de Latresne, baron de Calamiac, seigneur de la maison noble de Thouars et autres lieux\footnote{Entre autres de la maison noble de Port-Neuf à Camblanes. (Georges BOUCHON : \textit{Histoire d'une imprimerie bordelaise}, 1600-1900.)}, maréchal des camps et armées du roi, demeurant en son hôtel rue Porte-Dijeaux, paroisse Puy-Paulin, fut saisi féodalement à la requête du procureur du roi, le 21 avril 1775, pour n'avoir pas rendu au souverain l'hommage de la maison noble de Thouars. Il fut assigné devant le sénéchal de Guienne « aux fins de se voir condamner de payer es-mains du receveur des droits et devoirs seigneuriaux de Bordeaux le montant des lods et ventes résultant de l'acquisition faite par feu M. le président de Lalanne de la maison noble de Thouars ».

Or, le bureau des finances du domaine déclara par jugement que ladite maison, « située dans la paroisse de Tallance », n'était pas du domaine de sa majesté. La saisie féodale faite à la requête du procureur du roi, au préjudice du captal de Latresne, le 25 avril 1775, était donc nulle et non avenue. Le jugement fut notifié à l'intéressé le 11 juillet 1777.

Mais si l'hommage n'était pas dû au roi, il devait être rendu aux seigneurs jurats de Bordeaux en leur qualité de comtes d'Ornon. La cérémonie se déroula le 12 juillet 1777, dans la Maison commune.

Léonard-Casimir Lecomte, captal de La Tresne, se présente devant les jurats parmi lesquels « Messire MarcAntoine, marquis de Verteuil, maréchal des camps et armées du roi, seigneur de la Rame et autres lieux ».

Le captal reconnaît tenir Thouars et ses dépendances, s'étendant sur les paroisses de Talence, Gradignan et Villenave, dans les juridiction et comté d'Ornon, sous la redevance d'un épervier volant d'hommage tous les vingtneuf ans. N'ayant pu se procurer l'épervier, il prie les seigneurs jurats d'accepter, au lieu et place de l'oiseau, une paire de gants blancs brodés.

Sur réponse affirmative, le captal « ayant mis un genou à terre, tête découverte, sans épée ni éperons, les mains jointes dans celles du marquis de Verteuil », promet et jure d'être « bon et loyal vassal, de tout son pouvoir deffendre, garder et soutenir les droits, biens et honneurs des seigneurs maire, lieutenant de maire et jurats, à cause du Comté d'Ornon »\footnote{Archives municipales, D D, 4.}.

Ayant accompli tous autres actes de fidélité au cas requis, le captal offre une paire de gants blancs brodés au marquis de Verteuil, qui les accepte sans entendre dénaturer pour l'avenir la prestation habituelle : l'épervier volant.

Le marquis reçoit comme vassal de la ville Léonard-Casimir Lecomte, qui devra fournir le dénombrement de la maison de Lamotte de Thouars, domaines, fiefs, cens, rentes, appartenances et dépendances, dans le délai de quarante jours prescrits par les dispositions de droit et de coutume. Il est spécifié, en outre, que le captal, ses successeurs ou ayants cause devront renouveler le même hommage tous les vingt-neuf ans ou à mutation de propriétaire. C'était le rappel de l'acte d'anoblissement de la maison de Thouars du 24 juillet 1505.

Le marquis de La Tresne mourut le 31 octobre 1782. Dans la déclaration de succession faite par GuillaumeMarie Lecomte, « Chevalier de Malthe », héritier du marquis de La Tresne, on lit : 

«... Plus et finalement la tierce de la maison noble située dans les Graves de Bordeaux, appellée Thouars, paroisse de Talence, que j'estime la somme de cinquante mille livres\footnote{\textit{Archives historiques de la Gironde}, t. XLV.}. »

Par contrat en date du 9 mai 1788, M. Tarteiron de Saint-Remi acheta le château de Thouars à M. de Galatheau, écuyer à Bordeaux\footnote{Documents particuliers.}. Le 28 juin 1788, M. Tarteiron de Saint-Remi, étant à Bordeaux, écrit à M. de Galatheau :

« M. Brun me dit hier de votre part que vous toucheriez avec plaisir 25.000 livres sur les 50 que je vous dois sur le bien de Thouars, à la décharge de M. Leconte...\footnote{\textit{Ibid}.}. »

Sans doute « M. Leconte » ou plutôt M. Lecomte avaitil emprunté de l'argent à M. de Galatheau et lui avait-il donné en gage le château ?

Le 15 mai 1789, M. Tarteiron de Saint-Remi, retour du Languedoc, où il avait laissé sa femme, écrit à M. de Galatheau :

« ... Me voilà présentement hors de l'année du retrait ... de Thouars ; personne n'a voulu échanger beaucoup contre rien\footnote{Documents particuliers.}... ». 

À tort ou à raison, M. Tarteiron dénigrait le château de Thouars dont il était désormais définitivement propriétaire, aucune opposition — le retrait — n'ayant été faite, dans le délai d'un an, à la vente de ce château. 

\asterism{}

Au début de la Révolution, les paysans envahirent le château de Thouars, s'y livrèrent au pillage et firent un autodafé du lit où Charles IX avait couché. Ce lit avait été jusqu'alors conservé avec soin ; on allait le voir par curiosité comme on va contempler aujourd'hui, au château de Pau, la carapace de tortue qui servit de berceau à Henri IV.

M. Tarteiron, qui était négociant à Bordeaux, rue Courbin, fut arrêté par mesure de sûreté générale, vers la fin de 1793\footnote{Le 29 novembre 1793, Peyrend d'Herval apposait les scellés au domicile de Tarteiron ; ces scellés furent levés le 26 décembre 1793.\\Jean Tarteiron, alors âgé de soixante et un ans, était natif de Gaud (Hérault).}. Mme Tarteiron donna 58.000 francs pour racheter tête de son mari. Avec cette somme, elle fit remettre à Lacombe un billet écrit de sa main et résumant comme suit la position du négociant.

«... Sa campagne de Thouars, que connaît le citoyen Morel\footnote{Morel était membre du Tribunal révolutionnaire.}, est d'un terrain très ingrat et coûte beaucoup plus d'entretien que de revenu. Cependant, bien loin de faire diminuer les travaux, il les fait suivre et augmenter de tout son pouvoir ; il en a le certificat le plus honorable de la commune de Talence\footnote{\textit{Annales de la Terreur à Bordeaux. Le Tribunal révolutionnaire}. T. XIV, p. 126 et 127.}. »

Tarteiron comparut le 15 février 1794 devant le tribunal révolutionnaire.

« La Commission militaire, dit un extrait du jugement, convaincue que l'accusé, toujours occupé du bonheur de ses concitoyens, a fait travailler les terres les plus arides, qu'il n'a jamais cessé d'estimer les habitants de la campagne comme ses frères, avec lesquels il a partagé sa fortune ; que cette même fortune ne peut être mal envisagée par un tribunal révolutionnaire qui n'examine que la vertu, etc., ordonne qu'il sera sur le champ mis en liberté ».

Le jugement est du 27 pluviôse an II (15 février 1794). L'acquittement du négociant avait été naturellement obtenu grâce aux 58.000 francs versés à Lacombe par l'épouse de Tarteiron. Celui-ci, antérieurement à son arrestation, avait donné « 94.000 francs pour la République ».

Pour la commodité des habitants de Talence et de Villenave-d'Ornon, voisins de domaine de Thouars, il avait été demandé l'ouverture d'une route sur une partie de ce domaine.

Le 23 vendémiaire an IV (15 octobre 1795), les maire, officiers municipaux et membres du Conseil général de Talence, réunis au lieu habituel de leur séance, prirent connaissance d'une lettre présentée par le citoyen « Jean Tarteiron jeune, propriétaire du bien appelé Lamothe de Thouars »\footnote{Registre des procès-verbaux de la municipalité de Talence.}, en date du 10 vendémiaire, accompagnée d'une carte topographique signée par lui \textit{ne varietur}, et également signée par la municipalité de Villenave. 

Jean Tarteiron s'engageait à ouvrir le chemin désiré, dans les conditions indiquées sur le plan.

La municipalité de Talence, jugeant ce plan « favorable au bien public », en autorisa l'exécution, en spécifiant qu'on n'exigerait pas de Jean Tarteiron « d'autre chemin que celui porté sur ladite carte »\footnote{Registre des procès-verbaux de la municipalité de Talence.}.

Jean Tarteiron, le père, mourut dans le château de Thouars en 1822 et il y fut inhumé. Son tombeau a été respecté.

Au début du XIX\ieme{} siècle, le château était en très mauvais état. La courtine où se trouvait la porte d'entrée avait disparu, ainsi que les créneaux qui la couronnaient; les tours étaient coiffées d'un chapeau en charpente. Bref, l'ancienne résidence seigneuriale — et royale — « ne méritait pas un regard »\footnote{\textit{Musée d'Aquitaine}, p. 193.}. Cela n'empêcha pas la duchesse d'Angoulême d'aller visiter Thouars le 17 avril 1823\footnote{\textit{Mémorial bordelais}.}. Elle s'y rendit avec sa garde d'honneur.

En 1835, Mme Tarteiron fit relever les tours du château. Cette dame mourut en 1859, et son domaine devint la propriété de M. Balguerie.

Thouars a été restauré par Charles Durand qui en a fait, suivant l'expression du comte Aurélien de Sarrau, « un château mi-gothique, mi-moderne, ce qui constitue un amalgame bizarre ».

Mme la marquise du Vivier était, il y a peu de temps encore, propriétaire de Thouars qui appartient maintenant à M. Th. V. d'Ornellas.

Le 22 mai 1926, M. d'Ornellas a bien voulu nous montrer la chambre qu'occupèrent Charles X, puis Henri IV\footnote{Ce prince, venu à Thouars en 1565 avec Charles IX, y aurait séjourné de nouveau en 1587.}. Cette pièce, de forme ronde, est située au premier étage d'une des deux tours : celle de droite pour le spectateur. Elle est meublée, mais il n'y reste aucun objet, aucun ornement de l'époque des souverains qui y logèrent. Guillon y trouva, en 1866, des restes de tentures en riches étoffes, qui pouvaient bien être, à l'en croire, « contemporaines du Béarnais »\footnote{\textit{Les châteaux historiques et vinicoles de la Gironde}.}.

Guillon nota, en outre, qu'il y avait dans la même demeure un billard rajusté, et sur lequel Henri IV « s'était amusé à faire quelques parties »\footnote{\textit{Ibid}.}. Ce billard n'existe plus.

Sur la façade sud du château, on remarque de jolis petits détails de sculptures, entre autres un joueur de mandoline et un joueur de cornemuse. On a fixé dans le mur (côté ouest) une plaque montrant les armes d'un ancien personnage, figurées par un casque de chevalier, avec, au-dessous, le scorpion et le léopard britannique.

Le domaine de Thouars est séparé par un chemin des communes de Gradignan et Villenave-d'Ornon. Il a une superficie d'environ 100 hectares, soit quatre fois le Parc-Bordelais.

Un jardin à la française s'étend devant la façade nord du château ; un jardin analogue va être aménagé devant la façade sud.

Des quatre tours rondes qui. marquaient le quadrilatère du château de Thouars, il n'en reste plus que deux, incorporées au nouvel édifice. On voyait encore, il y a quelque soixante ans, des vestiges des deux tours détruites. Leur emplacement est recouvert à présent d'un tapis de verdure. On peut néanmoins se figurer par l'esprit le château de Thouars tel qu'il était au moment où il dut remplacer la fameuse bastide de Baâ, un des symboles de la puissance anglaise en Guienne au XIII\ieme{} siècle. 

\subsection{Château Monadey}

La famille Monadey, une des plus remarquables familles de Bordeaux au moyen âge, posséda, indépendamment de la seigneurie de Castets-en-Dorthe, un domaine à « Talence »\footnote{Léo DROUYN. \textit{Notes manuscrites}.}.

Cette famille tenait son nom de la fabrique de la monnaie à Bordeaux, dont elle avait été chargée vers le XI\ieme{} siècle. Le mot gascon « monadey » signifie monnayeur.

L'abbé Baurein écrit à propos de Monadey, de Talence : « C'étoit la maison de campagne d'un très ancien citoyen de Bordeaux dont il est fort question dans des titres du XIIIe siècle et suivans\footnote{\textit{Variétés bordeloises}, t. II, p. 300.}.»

Il y a eu un Raimond Moneder qui fut maire de Bordeaux en 1230, 1234 et 1238\footnote{J.-A. BRUTAILS. \textit{Maires et curés de Bordeaux}.}.

Est-ce ce seigneur qui eut le premier une maison de campagne à Talence? Est-ce l'un de ses descendants? Nous inclinons plutôt vers la première hypothèse.

Il n'est resté aucun vestige de la première maison de Monadey. Sur son emplacement, on bâtit, au XVII\ieme{} siècle, l'immeuble actuel. C'est un vaste édifice rectangulaire à deux étages. Il n'a ni tourelles ni pavillons. La porte centrale est surmontée d'un fronton; la toiture en ardoises est conique.

En 1751 la résidence de Monadey était la propriété d'Abraham Gradis, fils de David Gradis\footnote{David Gradis, issu d'une famille portugaise chassée de sa patrie Par l'Inquisition, tint une des premières places parmi ces armateurs qui, dans la première moitié du XVIII\ieme{} siècle, s'employèrent à développer les relations commerciales de Bordeaux avec les colonies.}, lequel avait fondé à Bordeaux, en 1728, l'importante maison connue sous la firme David Gradis et fils.

Le 10 février 1751, Abraham Gradis fit hommage au roi pour la terre de Monadey\footnote{Pierre MELLER. \textit{Armorial du Bordelais}, t. II, p. 179.}. Il s'y rendait le vendredi soir de chaque semaine. « Il s'y adonnait à une passion qu'il avait en commun avec le marquis de La Galissonnière : l'arboriculture et le jardinage. Il n'est, en effet, question que de plantes et de graines dans les lettres à lui adressées par le vainqueur de l'amiral Byng\footnote{Jean DE MAUPASSANT. \textit{Un grand armateur de Bordeaux : Abraham Gradis}, 1699-1780.}. »

Au début de l'année 1759, le général duc de Lorges\footnote{Louis de Durfort-Duras, duc de Lorges, né en 1714, était petit-fils du maréchal de Lorges, neveu de Turenne. Il prit part à de nombreuses batailles, entre autres à celle de. Fontenoy et contribua puissamment au succès de la journée. Il mourut après 1772.} envoyé en Guienne, où il commandait sous l'autorité du maréchal duc de Richelieu, s'installa avec toute sa suite dans la maison noble de Monadey, à Talence. Il avait demandé à Abraham Gradis de lui prêter ce domaine. L'armateur s'était empressé d'acquiescer au désir du duc dé Lorges. Toutefois, en février 1759, il écrivait à son oncle Moïse, faisant allusion à la présence du général et de ses gens à Monadey : « C'est quelque embarras\footnote{Jean DE MAUPASSANT. \textit{Un grand armateur de Bordeaux : Abraham Gradis}, 1699-1780.}. »

Abraham Gradis connaissait tous les gouverneurs ou commandants militaires de la Guienne ; il comptait parmi les intimes du maréchal duc de Richelieu et du marquis de Narbonne, adjoint au maréchal dans son commandement.

Plus tard, le duc de Lorges déclarait en présentant Abraham Gradis au duc de Choiseul, à Versailles : « C'est l'homme à qui j'ai le plus d'obligation, et qui m'a rendu les plus grands services, et qui en a rendu à tout le monde. »

Abraham Gradis mourut octogénaire à Bordeaux le 17 juillet 1780. Il fit des libéralités aux pauvres catholiques des paroisses Sainte-Eulalie et de Talence.

Un peu avant la Révolution, la terre de Monadey avait été achetée par Antoine de Léglise. Ce personnage figura, en 1789, sur « la liste des membres de l'ordre de la noblesse, présidée par messire Marc-Antoine du Périer, conseiller du roi, premier baron, grand sénéchal de Guienne »\footnote{OREILLY. \textit{Histoire de Bordeaux}, I\iere{} partie, 4, p. 468.}. Lors de la réunion de la noblesse en 1789, Antoine de Léglise « s'y qualifia seigneur de Monadey et autres lieux\footnote{GUILLON. \textit{Les châteaux historiques et vinicoles de la Gironde}.} » Les de Léglise étaient, en effet, seigneurs de Tardes, Pinsac, Monadey, Vigourous, le Port et Saint-Pey-d'Aurillac\footnote{Pierre MELLER. \textit{Armorial du Bordelais}, t. II, p. 342.}.

Au XIX\ieme{} siècle, Monadey devint la propriété de M. Roul, qui laissa ce bien à sa fille. Celle-ci ayant épousé le comte Lemercier, partit avec lui pour la Saintonge. Le domaine « de Monadey, abandonné de ses maîtres, était en vente en 1866 »\footnote{GUILLON. \textit{Les châteaux historiques et vinicoles de la Gironde}.}. Il fut aliéné en 1870\footnote{Note communiquée par M. le comte Aurélien de Sarrau.}.

Il appartient depuis une quinzaine d'années à Mme veuve Ichon, qui le tenait de M. Bourdeille. 

\end{document}