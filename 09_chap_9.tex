%%%% Débute page 166 dans le fichier PDF%%%%

\documentclass[a4paper,11pt]{book}
\usepackage[T1]{fontenc}
\usepackage[utf8]{inputenc}
\usepackage{lmodern}
\usepackage[french]{babel}
\usepackage{todonotes}

\newcommand{\asterism}{\bigskip\par\noindent\parbox{\linewidth}{\centering\large{*}\\[-4pt]{*}\hskip 0.75em{*}}\bigskip\par}%

\begin{document}
\title{Histoire de Talence}
\author{Maurice Ferrus}
\frontmatter
\maketitle

\mainmatter{}

\chapter{Viographie - Archéologie}
\section{L'ancien chemin de fer de La Teste}

La grande ligne ferrée Bordeaux-Irun, avec embranchement à Lamothe sur Arcachon, traverse la commune de Talence.

En 1921 a été mise en exploitation la ligne du chemin de fer de ceinture, reliant la gare du Midi à celle de Saint-Louis (Médoc). L'établissement de cette voie ferrée a amené la création de la gare « La Médoquine-Talence ».

Il y avait eu déjà une station de « La Médoquine » au temps du premier chemin de fer de Bordeaux à La Teste.

La gare de départ de ce chemin de fer se trouvait entre les rues de Pessac, des Treuils et de Ségur, sur l'emplacement occupé par le Conseil de guerre, la caserne Boudet et le château d'eau.

Talence n'était pas tête de ligne, mais il ne s'en fallait que de quelques mètres — tout au plus l'intervalle de la rue de Ségur, laquelle, rappelons-le, séparait autrefois une partie du territoire de Bordeaux de celui de Talence\footnote{Plan du projet d'annexion à la ville de Bordeaux du territoire des communes limitrophes. (Archives municipales. Dossier 30 D. 2.) Voir page 21. \textbf{TODO : mettre à jour le renvoie}}.

La première pierre de la gare du chemin de fer de La Teste — le quatrième construit en France — fut posée solennellement, en août 1839, par le duc d'Orléans, fils aîné de Louis-Philippe, qui se trouvait à Bordeaux avec la duchesse, son épouse\footnote{Article de Gustave \textsc{LABAT} inséré dans l'ouvrage \textit{Le Centenaire du Lycée de Bordeaux}.}. L'inauguration de la ligne eut lieu le 6 juillet 1841. On comptait dix-neuf stations ; la première était celle de « La Médoquine ».

Un ami du Parnasse — J.-B. Couvé — composa un petit poème en effectuant le voyage de Bordeaux à La Teste. 

Voici un échantillon de sa lyre :

\leftskip=3cm

\begin{small}
\noindent
Nous marchons sur un viaduc\\
Après la Médoquine.\\
De Monsieur de Vergez\footnote{M. de Vergez, ingénieur, était le constructeur de la ligne ferrée de La Teste ; il eut comme collaborateur Alphand, alors aspirant ingénieur. Paris dut plus tard à Alphand des transformations remarquables.} le truc\\
Est fameux, j'imagine.\\
C'est un beau morceau,\\
Il me place haut.\\
Je règne sur l'espace,\\
Je me fais l'effet\\
D'un roi, d'un préfet\\
Qu'on laisse encore eu place...\\
\end{small}

\leftskip=0cm

Couvé faisait allusion au passage du train sur un viaduc de 920 mètres établi après le Haut-Brion.

Le chemin de fer de La Teste devint, le 27 juillet 1853, propriété de la « Compagnie des chemins de fer dû Midi et du canal latéral à la Garonne\footnote{Tel était le nom primitif de la Compagnie du Midi.} ». Le 8 mars 1854 fut proposée la suppression des « garages ou ports secs » de la ligne de La Teste, entre autres La Médoquine. Celte
proposition fut adoptée.

La Compagnie du Midi s'empressa de construire une nouvelle gare de La Médoquine en créant le chemin de fer de ceinture.

La gare « La Médoquine-Talence » — pour l'appeler par son véritable nom — est dotée d'un service de factage et de camionnage qui, desservant les quartiers très éloignés de Bordeaux-Saint-Jean, présente pour le public le plus grand intérêt.

De février à décembre 1921, ladite gare n'a pas expédié moins de 5.500 tonnes.

Les nonagénaires de La Médoquine qui, dans leurs jeunes années, regardaient passer le petit chemin de fer de La Teste, voient circuler maintenant sur le même point, à
90 à l'heure et dans un fracas de tonnerre, les grands express européens.

La ligne du chemin de fer de ceinture a été ouverte : 1° au service normal des marchandises le 18 février 1921; 2° au service des voyageurs le 14 mars de la même année.

\section{les voies romaines}

Plusieurs routes romaines partaient de Bordeaux allant dans différentes directions. Deux d'entre elles conduisaient l'une à La Teste par Croix d'Hins, l'autre à Bayonne par Salles. Ces longues chaussées aux robustes empierrements et dont on voit encore, par endroits, quelques vestiges, furent suivies par les armées de l'empire, les fonctionnaires, les voyageurs.

Les chemins de Saint-Jacques-de-Compostelle furent plus tard, et pour la plupart, semble-t-il, établis sur des routes romaines. Les pèlerins s'étant arrêtés à Bordeaux, rue du Mirail, à l'hôpital Saint-Jacques, et, reprenant leur voyage, tournaient rue des Augustins, gagnaient la rue Sainte-Catherine\footnote{Il y a trois ou quatre ans, des ouvriers occupés à des travaux de terrassements, près de la porte d'Aquitaine, mirent à jour dans l'axe de la rue Sainte-Catherine, à environ 1 m. 50 de profondeur, un sol pavé. On se trouvait en présence — ce fut l'avis des gens compétents — d'un fragment de route romaine.} et s'engageaient route de Bayonne, (cours de l'Argonne); arrivés au carrefour Saint-Genès, ils prenaient « le grant chemyn romyeu » (cours Gambetta)\footnote{Voir page 123 (note 72) et page 133. \textbf{TODO : voir le renvoi}}, le suivaient jusqu'au chemin Roul, pénétraient dans ce chemin qu'ils parcouraient dans son entier, laissant à droite le village du Grand Courneau de Ruhan, et ils atteignaient une route désignée encore sur les plans « chemin de la voie romaine ».

La rue Sainte-Catherine, le cours de l'Argonne, le cours Gambetta et le chemin Roul constituaient apparemment un tronçon de la route Bordeaux-Bayonne, par Salles, tracée par les vainqueurs de la Gaule, et cette route passait donc sur le sol où devait se former le coeur même de Talence.

L'autre voie romaine Bordeaux-La Teste par Croix-d'Hins, plus au nord, sillonnait également le futur territoire de Talence, car elle suivait à peu près le tracé de la ligne ferrée Bordeaux-Arcachon.

\section{À travers les rues de Talence}

Les nouvelles municipalités, à quelque parti qu'elles appartiennent, n'ont, en général, dès leur installation, qu'un souci : glorifier tout de suite leurs idoles ou leurs amis défunts. On change alors les plaques de rues. On supprime des noms sentant le terroir ou rappelant des traditions ou des faits d'histoire locale et on les remplace par des noms de personnages politiques.

C'est là un très mauvais système. Certes, nous n'entrerons pas dans le détail des inconvénients multiples que provoquent les perturbations viographiques. Nous ferons seulement observer que lorsqu'on s'engage dans cette voie — le mot est de circonstance — on ne peut prévoir où l'on s'arrêtera.

Supposons une municipalité socialiste arrivant au pouvoir : elle modifie les plaques indicatrices dans le sens de ses opinions. C'est un sentiment très naturel.

Qu'une municipalité modérée survienne : elle s'empressera de faire disparaître les appellations socialistes. C'est encore humain.

Qu'une municipalité royaliste soit élue : elle fera revivre les noms qui lui sont chers au détriment de tous les autres.

Il importe donc d'agir avec beaucoup de circonspection quand on touche aux noms de rues consacrés par l'usage. D'autre part, on ne devrait donner à celles-ci que les noms
d'hommes politiques dont la valeur personnelle, l'autorité morale et surtout les services rendus au pays imposent un respect unanime.

De cette manière, on éviterait l'engrenage. On ne mécontenterait personne. On ne froisserait aucune susceptibilité. On n'éveillerait aucune jalousie.

Les municipalités de couleurs différentes qui se succéderaient n'auraient pas à user de représailles vis-à-vis les unes des autres.

Il y a, au surplus, un moyen de tout concilier, du moins dans une certaine mesure. C'est de laisser une fois pour toutes les noms existants, quels qu'ils soient, et de réserver les noms nouveaux pour les voies nouvelles. C'est ce qu'a compris, d'ailleurs, la municipalité de Talence qui, dans sa séance du 12 juillet 1926, a baptisé comme suit les rues et places créées dans les lotissements du Haut-Brion, des Templiers, de Toulouse et Dunoyer : 

« Lotissement du Haut-Brion : Rue A, portera le nom rue Marcel-Sembat ; rue B, rue Louis-Blanc ; rue C, rue Ader ; rue D, rue Étienne-Dolet; la place, place du 11-Novembre.

» Lotissement des Templiers : La rue sera dénommée rue Vergniaud ; la place, place Camille-Desmoulins.

» Lotissement de Toulouse : La voie n° 1 prendra le nom de rue Général-André ; la voie 2, rue Docteur-Dupeux; la voie 3, rue Mattéotti ; la voie 4, rue Ernest-Renan ; la voie
5, rue Jules-Guesde.

» Lotissement Dunoyer : La voie n° 1 sera dénommée rue Calixte-Camelle; la voie 2, rue Paul-Louis-Courrier; la voie 3, rue Léon-Bourgeois ; la place, place du 1\textsuperscript{er}-Mai ».

La place du 11-Novembre est la plus heureuse de ces dénominations, car c'est la date qui vit la fin du plus grand cataclysme que l'histoire ait jamais enregistré.

Et maintenant, passons à une petite étude sur la viographie talençaise qui avait naturellement sa place toute marquée dans cette monographie.

\section{Bontemps (chemin)}
Ce nom est apparemment celui de Bontemps-Dubarry qui, pour raison de santé, donna sa démission de membre du Conseil municipal de Talence le 15 avril 1815\footnote{Registre des procès-verbaux de la municipalité de Talence (Archives de la commune.)}.

\section{Cauderès (rue de)}

Elle perpétue la mémoire d'un enfant de Talence, Jean Baptiste Cauderès, prêtre du diocèse de Bordeaux.

En 1790, Cauderès avait trente-six ans ; des textes de cette époque font suivre son nom de la mention : « ci-devant curé de Canéjan depuis 1780. » Il fut déclaré émigré le 24 nivôse an II (13 janvier 1794), et rentra en France après la Terreur. Mgr d'Aviau, archevêque de Bordeaux, le nomma chanoine honoraire le 27 juin 1803.

Cauderès avait publié en 1783 un \textit{Éloge du comte d'Estaing}\footnote{Charles-Hector d'Estaing, amiral français, se signala par des succès contre les Anglais sur terre et sur mer pendant la guerre d'Amérique, prit Saint-Vincent et l'île de la Grenade ; monta sur l'échafaud, parce que noble, en 1794.}.

Seul, le côté est de la rue de Cauderès appartient à Talence ; le côté ouest fait partie de Bordeaux.

\section{Charles-Laterrade (rue)}

Né à Bordeaux en 1818, Charles Laterrade fut agriculteur, littérateur et homme politique. Membre du Conseil municipal de Talence le 11 mai 1871, il fut élu deux fois maire de cette commune, où il créa une bibliothèque populaire\footnote{Il y a une dizaine d'années, la Bibliothèque populaire de Talence fut supprimée; les livres en furent répartis entre les différentes écoles de la localité. (Notes de la mairie.)} et organisa des cours d'adultes. Il fut révoqué, après le 24 mai, par l'Ordre moral.

Laterrade fut conseiller général du quatrième canton ; il mourut le 22 juin 1876.

\section{Clément-Thomas (rue)}

Né à Libourne le 31 décembre 1809, Clément Thomas fut nommé représentant du peuple en 1848.

Déporté après le coup d'Etat de 1852, il rentra à Paris lors de la proclamation de la troisième République. Le 4 novembre 1870, il devint commandant en chef de la première armée, des gardes nationales de la Seine. Il démissionna le 14 février 1871. Après l'insurrection du 18 mars, il fut arrêté et fusillé, séance tenante, sans le moindre simulacre de jugement. L'État a élevé au Père-Lachaise, à Paris, un monument à la mémoire de Clément Thomas et du général Lecomte, fusillé, dans les mêmes conditions que
lui, quelques jours plus tard.

\section{Colonel-Moll (rue du)}

Le lieutenant-colonel Moll commandait le territoire du Tchad quand il fut tué, le 9 novembre 1910, au cours d'une action contre les forces du sultan du Massalit. Cet officier supérieur, doublé d'un administrateur de premier ordre, avait pour devise : « Je commande, donc je suis responsable. »

\section{Cronstadt (avenue de)}

Cronstadt, port militaire et commercial de la Russie, au fond du golfe de Finlande. Le tsar Alexandre III y reçut, en 1891, une escadre française. Ce fut le point de départ de l'ancienne alliance franco-russe.

Le 13 octobre 1893 eut lieu, à Toulon, la réception d'une escadre russe placée sous les ordres du contre-amiral Avellan. Cet événement ne fit que développer les sentiments d'amitié qui unissaient les peuples français et russe.

\section{De Mons (chemin)}

Cette voie reliant le cours Gambetta au chemin de Gradignan\footnote{\textit{Plan de Bordeaux et de sa banlieue} par Louis \textsc{LONGUEVILLE}.} porte le nom d'une famille noble. Il est fait mention sur le plan\footnote{Ce plan n° 3097 est aux Archives municipales de Bordeaux.} d'un quartier de Talence, au début du du XVIIIe siècle, de la « vigne de Monsieur de Mons ».

D'autre part, on relève sur une liste des moulins établis sur le ruisseau de Talence « le moulin d'Arz à Mons »\footnote{\textit{Registre du Clerc de ville}. (Archives municipales de Bordeaux.)}.

Le 11 juin 1594, il y eut deux reconnaissances de fonds situés à Talence et relevant des maisons nobles de Langon et de Mons\footnote{\textit{Inventaire sommaire. Registre de la Jurade}, t. XI.\\Le même ouvrage nous apprend que les 25 juin et 26 juillet 1536, il y avait eu cinq reconnaissances, « en faveur du seigneur de la maison noble de Langon, de fonds situés à Talence ».}.

Le 3 décembre 1692, le sieur Rivière fit l'acquisition « d'un bois de haute futaye, appartenant à M. de Mons, conseiller, situé en la paroisse de Talence, \textit{en graves}\footnote{\textit{Inventaire sommaire. Registre de la Jurade}, t. IX.} ».

Il s'agissait de M. Démons, conseiller au Parlement, devenu propriétaire de la seigneurie de Thouars, le 22 mars 1692.

\section{D'Espeleta (avenue)}

À l'origine, le domaine de Peixotto s'étendait du chemin Frédéric-Sévène au chemin de Suzon\footnote{Plan 138 (Archives départementales). On lit sur ce plan, levé en juin 1782 : « Emplacement de la maison, appent, chay, cuvier et écuries qui existaient lors de la reconnaissance consentie par M. de Jegun, le 12 avril 1758, devant Laville, notaire à Bordeaux. Le tout démoli par le s\textsuperscript{r} Pexotte ».}, qui est la dernière voie que l'on trouve, à gauche, avant d'arriver avenue d'Espeleta, en partant de l'église de Talence.

Le baron Espeleta, propriétaire de ce domaine, fit don à la commune du terrain sur lequel on a construit la mairie et créé la belle esplanade y conduisant. Le baron avait consenti cette donation à la condition que son nom fût donné à ladite esplanade\footnote{Notes de la mairie.}. Voilà pourquoi il y a, à Talence, une avenue d'Espeleta.

\section{Edison (rue)}
Cette voie a pris le nom du grand physicien américain, inventeur de nombreux appareils électriques, notamment du phonographe. Précédemment, c'était la rue du Prince-Noir.

Le souvenir du fils célèbre d'Édouard III est rappelé, en dehors du château, par l'enseigne \textit{Usine du Prince noir}, peinte — en lettres noires, dans flèche blanche, sur fond noir — sur le mur, à l'angle du chemin de Roustaing et du cours Gambetta.

\section{Émile-Zola (rue)}

Cette rue était dénommée antérieurement « chemin des Montagnes », par rapport à une élévation de terre qui l'avoisinait et sur laquelle passait la ligne du chemin de fer de Bordeaux à La Teste\footnote{Nous avons vu dans notre enfance des talus qui étaient des vestiges de ces « montagnes ».}.

\section{François-Coppée et de l'Union (rues)}

Ces rues ont remplacé la « rue du Bois-de-Boulogne », et la « petite rue du Bois-de-Boulogne », qui rappelaient la salle de danse dont il a été question dans un précédent chapitre. Il est regrettable, à notre avis, que cette jolie désignation de Bois-de-Boulogne ait complètement disparu de Talence.

\section{Gambetta (cours)}

Ce cours constitue l'artère principale de Talence, les deux autres grandes voies, le cours du Maréchal-Galliéni et la route de Toulouse, n'appartenant que par un côté à la commune.

Le cours Gambetta est la voie la plus animée, la plus commerçante. On y trouve les écoles primaires, la poste, des établissements divers. Il passe devant l'église et devant la mairie; il est ensuite borde par les magnifiques domaines qui font l'incomparable charme de Talence.

À partir du chemin Banquey jusqu'à l'église de Talence, une ligne de beaux peupliers se dressait de chaque côté du cours Gambetta. Un petit fossé courait devant les maisons. Sur les instances des propriétaires, les peupliers furent abattus, les fossés comblés, et l'on créa les trottoirs.

\section{Maréchal-Galliéni (cours du)}

C'est l'ancien chemin de Pessac. Le côté des numéros pairs appartient à Talence ; celui des numéros impairs dépend de Bordeaux.

Le Conseil municipal de Talence eut, le premier, la pensée de baptiser du nom de Galliéni le chemin de Pessac. Les édiles bordelais approuvèrent cette désignation le 12 novembre 1920. Par suite, le chemin de Pessac devint le cours Général-Galliéni.

Le 12 avril 1921, la Chambre des députés adoptait sans débat, à mains levées et à l'unanimité, une proposition de loi tendant à conférer, à titre posthume, la dignité de maréchal de France au général Galliéni.

Du fait de ce vote, les plaques indicatrices « cours Général-Galliéni » furent remplacées par celles de « cours du Maréchal-Galliéni ».

Galliéni mourut le 27 mai 1916, dans la maison de retraite et de convalescence des soeurs franciscaines, rue Maurepas, 29, à Versailles. On l'a justement appelé « l'animateur de la victoire de la Marne ».

\section{Pierre-Curie (rue)}

Pierre Curie naquit à Paris le 15 mai 1859. Chef des travaux de physique à l'École de physique et de chimie industrielle de Paris en 1882, il fut ensuite professeur de physique générale à l'École de physique et de chimie, puis à la Sorbonne, en 1904. On lui doit, entre autres découvertes, celle du radium. Sa femme collabora activement à ses nombreux travaux. La moitié du prix Nobel pour les sciences fut attribuée à M. et M\textsuperscript{me} Curie en 1904.

L'année suivante, Pierre Curie entrait à l'Académie des sciences.

Ce grand savant fut écrasé par une charrette alors qu'il traversait la place Dauphine, à Paris, le 18 avril 1906.

\section{Rocambole (impasse)}

Il y a eu un conseiller municipal de Talence appelé Gilbert et surnommé Rocambole\footnote{Rocambole est le héros d'un des romans du fameux conteur Ponson du Terrail, qui mourut à Bordeaux, rue de Pessac.}. Ce conseiller habitait dans la voie à laquelle son sobriquet a été donné.

\section{Roul (chemin)}

Il commence cours Gambetta et aboutit à la limite de la commune, vers Pessac.

Propriétaire du château de Monadey, Roul fut élu maire de Talence le 22 avril 1825. Député de la Gironde, il obtint du Parlement, en 1845, une somme de 5 millions pour achever le pavage de la route allant de Bordeaux à Bayonne par les grandes landes.

Roul finit ses jours à Monadey à l'âge de quatre-vingt-quatre ans. Il avait exprimé le désir d'être inhumé dans une chapelle qu'il avait fait édifier dans son domaine; mais on ne tint pas compte de ce désir.

La dépouille mortelle de l'ancien député fut donc portée au cimetière de la commune.

Le plan cadastral de Talence fut terminé en 1847\footnote{Depuis lors, il n'a pas été établi d'autre plan cadastral de Talence.}, le baron Sers étant préfet de la Gironde, et Roul, maire de Talence.

\section{Roustaing (chemin de)}

Ce chemin conserve le souvenir des deux familles qui ont tant marqué dans l'histoire de Talence : les Roustaing, seigneurs de Brama, et les Rostaing, seigneurs de La Tour.

\section{Taillade (chemin de la)}

Ce chemin porte le nom d'une famille qui avait des propriétés dans le quartier. Il est appelé « ruelle Taillard » sur le tableau d'assemblage du plan cadastral 1847, et « ruelle Taillade » sur une des divisions du même plan.

\section{Vieille-Tour (chemin de la)}

En 1813, c'était le « chemin n° 5 de 3\textsuperscript{e} classe »\footnote{Registre des procès-verbaux de la municipalité de Talence. (Archives de la commune.)}. Partant du lieu appelé La Médoquine, il passait « devant le domaine de M. le maréchal Perignon, Latour, Clamageran. Sandré » et allait « jusqu'au village du Grand-Cournau ». Il était long de 900 mètres.

Ce chemin s'appelait « chemin du Courneau à la Médoquine » en 1847\footnote{Plan cadastral.}. On le baptisa, par la suite, chemin de la Vieille-Tour, peut-être parce qu'il restait encore quelques vestiges de l'ancienne Tour des Rostaing ?

\section{Visitandines (rue des)}

On appelle Visitandines un ordre de femmes institué en 1610, à Annecy, par saint François de Sales et la baronne de Chantal, en mémoire « de la visite que la Vierge fit à sainte Elisabeth quelques jours après l'Annonciation ». Cet ordre, dont la règle est peu sévère, fut approuvé par Urbain VIII, en 1626, et se répandit bientôt en France, en Italie, en Allemagne, en Pologne.

Les Visitandines étaient aussi dénommées « Religieuses de la Visitation ».

Un groupe de ces religieuses s'établit à Talence. Il devait occuper un emplacement situé dans l'angle formé par la rencontre du cours Gambetta et du chemin de la Taillade. Cet endroit est, d'ailleurs, ainsi indiqué sur le plan cadastral de Talence : « La Visitendine. »

À la Révolution, le domaine des Visitandines devint propriété de la Nation, et fut vendu, le 13 janvier 1791, pour la somme de 31.400 livres, à M. Peynado fils, négociant, Fossés de ville\footnote{Cours Victor-Hugo, à Bordeaux.}. Il comprenait « un bien de campagne, maison, chai, cuvier, 20 j. 17 r. vigne et terre »\footnote{M. \textsc{MARION}, J. \textsc{BENZACAR}, \textsc{CAUDRILLIER}. \textit{Documents relatifs à vente des biens nationaux}, t. I.}.

La rue des Visitandines est devenue depuis peu rue Émile-Combes.

\asterism

Les noms d'hommes politiques tiennent une place importante dans la viographie talençaise. On y trouve, le cours Gambetta et les rues Adolphe-Thiers, Jules-Simon, Carnot, Jules-Ferry, Ludovic-Trarieux, Émile-Combes, Paul-Bert, Charles-Floquet, Maurice-Berteaux, de Freycinet, Jean-Jaurès, Waldeck-Rousseau, Félix-Faure, Camille Pelletan. Deluns-Montaud, Émile-Loubet, René-Goblet.

Les écrivains sont représentés par les rues Voltaire, Victor-Hugo, Émile-Zola, de Balzac, Elisée-Reclus, Auguste-Comte, François-Coppée.

Les rues Pasteur, Marcelin-Berthelot, Pierre-Curie et Edison symbolisent la science.

La Révolution est rappelée par les rues Mirabeau, Danton et Robespierre.

Deux voies portent des dates célèbres : la rue du XIV-Juillet (commémoration de la prise de la Bastille en 1789), et la rue du IV-Septembre (proclamation de la troisième République en 1870).

Les militaires sont aussi à l'honneur. Outre le cours du Maréchal-Galliéni, il y a les rues Hoche, Marceau, Chanzy, Margueritte, Bourbaki, Bordas.

Le chemin Bayard évoque sans doute la mémoire du Chevalier sans peur et sans reproche ?

La rue Denfert-Rochereau glorifie le colonel qui défendit Belfort, et la rue Rouget-de-l'Isle perpétue le nom de l'officier du génie auteur de \textit{La Marseillaise}.

L'éminent économiste Léon Say a également son nom sur une plaque de rue.

Et voici des dénominations d'inspiration républicaine : rues de l'Égalité, de la Liberté, de la Fraternité, de la Vérité, de la Paix, de l'Union, de l'Espérance ; place de la Concorde. Enregistrons aussi la rue de la République.

Les rues de Coulmiers et de Bazeilles portent des noms de batailles. Celles d'Alsace et de Lorraine ont été aussi baptisées en souvenir des deux provinces arrachées à la France en 1871 et qui ont été réintégrées dans notre territoire en 1918.

Des sentiments poétiques ont fait naître ces charmantes appellations : rues de la Fauvette, de la Charmille, de laPrairie, du Porte-Bonheur (dans celte dernière artère devait fleurir le muguet).

Des voies tiennent leur nom du quartier qu'elles traversent : chemins de Banquey, de la Médoquine ; rues de Peydavant et du Haut-Carré. D'autres ont pris le nom d'anciens châteaux auxquels elles conduisent : chemin de Thouars, de Raba, etc.

La rue Pey-Bouquey se rapporte à un bien dont il est question dans un acte de 1532, et qui était situé « esgrabes de bourdeaux, au plantier de \textit{pey Coucquey}, autrement au Haumont.

Dans une esporle de 1618, dont il existe une copie datée de 1691, le même plantier est ainsi désigné \textit{puch boucquey}\footnote{Arch. départ., G, 1001. (Renseignements communiqués par M. Trial, membre de la Société archéologique).\\Dans le langage du pays, pey a le sens de Pierre (Pey Berland), et \textit{puch} ou \textit{puy} celui d'élévation du sol (Puy Paulin).}.

Plusieurs artères rappellent les noms d'hommes qui ont présidé, en tant que maires, aux destinées de la commune : Roul, Charles-Laterrade, de Mégret, Frédéric-Sévène, Jacques Juillac.

En donnant à une rue le nom de Eugène-Olibet\footnote{Honoré-Jean Olibet et son fils Eugène ont fondé, après 1862, l'usine de Talence.}, on a rendu un hommage légitime au commerce et à l'industrie.

Certaines voies portent des noms de propriétaires. Une autre a un nom bizarre : c'est la rue du Pont-projeté.

\asterism 

Trois lignes de tramways desservent Talence par la route de Toulouse, le cours Gambetta et le cours du Maréchal-Galliéni. Depuis un an environ, les trams de la banlieue empruntent le réseau urbain. En sorte que les voyageurs venant de Léognan, Gradignan ou Gazinet sont directement conduits, s'ils le désirent, au centre même de Bordeaux; avant cette pénétration, les voyageurs devaient changer de voitures aux barrières, ce qui était parfois fort désagréable.

Un coup d'oeil dans un \textit{Annuaire} de 1840 nous a permis de savoir comment on allait, il y a 86 ans, de Bordeaux à Talence. Des voitures omnibus dites « les Bordelaises » partaient de la place de la Comédie et se rendaient à Talence en passant par la place d'Aquitaine (de la Victoire). Le prix du voyage était de 45 centimes, soit 15 centimes de la Comédie à la place d'Aquitaine, et 30 centimes d'Aquitaine à Talence.

D'autres voitures omnibus dites « les Françaises » se rendaient de la rue Gobineau à la barrière du Moulin-d'Ars, en passant par la place des Capucins. Le prix du trajet était de 30 centimes : soit 15 centimes de la rue Gobineau aux Capucins, et 15 centimes des Capucins à la barrière du Moulin-d'Ars.

Une entreprise de voitures brédoises assurait le service de Gradignan. On s'embarquait place d'Aquitaine. On lit, à ce propos, dans l'\textit{Annuaire} de 1840 :

« Il part tous les jours à 7 heures, 11 heures du matin et 3 heures de l'après-midi une voiture pour Talence et Gradignan, et revenant à 9 heures, 2 heures et 4 heures\footnote[25 bis]{TODO 25 BIS : Cet \textit{Annuaire} est aux archives municipales de Bordeaux.}. »

Sous la rubrique « Entreprises diverses » on trouve le service de Pessac. Trois départs avaient lieu tous les jours pour cette destination : à 9 heures du matin, à midi et à 4 heures du soir, en hiver; à 8 heures du matin, à midi et à 5 heures du soir en été. Dans la belle saison, un autre départ avait lieu le dimanche, à 2 heures de l'après-midi. La voiture stationnait au coin de la rue de Berry, près de l'hôpital.

En 1865, des voitures de la Compagnie générale des Omnibus effectuaient le transport des voyageurs de la place extérieure d'Aquitaine à Talence. Le service durait de 7 heures du matin à 7 heures du soir\footnote{Charles \textsc{COCKS}. \textit{Guide de l'étranger à Bordeaux}.}. Le tarif était toujours de 30 centimes.

En 1876, même point de départ et même prix. Il existait alors une autre ligne de voitures reliant le cours du XXX-Juillet (devant la maison n° 1) à la barrière Saint-Genès. Le coût du voyage était de 20 centimes.

\section{Découvertes archéologiques}
\subsection{Deux tombes Jumelles}

Un coffre à double cuve — ou deux tombes jumelles — paraissant appartenir à la fin de l'ère mérovingienne, fut découvert, en 1875, par M. Delfortrie\footnote{\textit{Société arehéologique de Bordeaux}, t. II, année 1875.}  dans le jardin de la maison 59, cours Gambetta. Ce tombeau servait d'auge ! Les gens qui lui avaient donné cette destination ignoraient évidemment tout l'intérêt qu'il présente au point de vue de l'archéologie.

Suivant les renseignements recueillis, le propriétaire de l'immeuble avait trouvé le coffre à la place qu'il occupait. Les habitants du quartier « l'avaient toujours vu là où il était ». On estima que le tombeau avait dû être trouvé sur les lieux ou à une petite distance. Comme il a un poids énorme, il est peu probable, en effet, que l'on se soit donné la peine de le transporter de loin. La pierre ne présente aucune inscription.

Au cours d'une de ses séances en 1878, la Société d'archéologie adressa des remerciements à M. Juillac, propriétaire à Talence, qui avait offert gracieusement, à cette société, le coffre à double cuve\footnote{\textit{Société archéologique de Bordeaux}, t. V, année 1878.}.

Le tombeau est déposé au Musée des Antiques, rue Mably.

\subsection{Une stèle funéraire}

Il y a eu au château de Thouars une stèle funéraire atti-
que\footnote{Le monument a été découvert par le père Royer qui l'a signalé, en février 1913, à M. le comte Aurélien de Sarrau, lequel l'a visité en compagnie de M. l'abbé Moureau, curé de Talence.}, en marbre pentélique, dans un bel état de conservation. Pendant longtemps, cette stèle resta reléguée en un hangar du château. Elle se dressa ensuite sous un berceau de verdure, contre la margelle d'un puits couvert au ras du sol.

Voici le sujet sculpté sur la face principale, la seule ornementée :

Deux femmes s'abordent et se donnent la main. Elles se présentent de profil. Toutes deux portent une sorte de manteau passé sous le bras droit, et dégageant l'épaule. Le pan du vêtement s'enroule autour de l'autre bras, fait bourrelet à la ceinture, puis retombe.

Les deux femmes se serrent la main en signe d'adieu, vraisemblablement.

Cette scène est encadrée par deux pilastres terminés par des chapiteaux plats supportant une voûte faite de trois arcs. De chaque côté de l'archivolte, il y a une petite rosace en relief.

Au-dessus du cintre, dans la frise, apparaît une inscription grecque que l'on peut traduire ainsi :

\leftskip=3cm

\begin{small}
\noindent
Zozimè, fille de Kallinikos, milésienne,\\
Femme de Phocion d'Otrynè.
\end{small}

\leftskip=0cm

Cette inscription révèle le mariage d'un Athénien avec une étrangère. De telles unions n'étaient pas admises au temps d'Athènes libre, ou elles étaient très rares. Elles se multiplièrent quand la Grèce devint province romaine. On en conclut que la stèle de Talence remonte à la fin du second siècle ou du premier siècle avant Jésus-Christ.

Le monument est couronné d'un fronton triangulaire montrant, dans le tympan, une rosace de même style que les deux autres, mais de plus grande dimension. Il a 1 m. 20 de hauteur, 0 m. 50 de largeur au fronton, 48 cm. 5 au corps, et 12 cm. 5 d'épaisseur\footnote{\textit{L'Aquitaine}, n° du 21 février 1913.}.

À propos de cette stèle, M. Paul Fournier, professeur à la Faculté des lettres de Bordeaux, a écrit :

« À la banalité du sujet, des visages, des poses, on reconnaît le produit d'un atelier sur le chemin du cimetière, où les parents de la morte ont fait leur choix, le lendemain de l'enterrement\footnote{\textit{Revue des Études anciennes}, n° de juillet-septembre 1913.}. »

Comment cette pierre funéraire se trouvait-elle à Talence ? M. Paul Fournier assure qu'elle ne « peut passer pour un témoin de notre histoire locale ». A-t-elle été apportée sur nos quais par un bateau qui l'aurait reçue d'une barque à laquelle elle aurait servi de lest ?

L'armateur Balguerie junior a été propriétaire du château de Thouars. Il est possible qu'un de ses navires ait ramené la pierre à Bordeaux dans les conditions indiquées.

L'armateur, ayant eu son attention attirée sur la stèle, se serait empressé de faire transporter celle-ci dans son domaine à Talence. Il n'y aurait plus songé par la suite, puisqu'elle fut retrouvée, dans un hangar, par M\textsuperscript{me} la marquise du Vivier, qui la fit elle-même placer contre la margelle du puits.


Pour M. le comte Aurélien de Sarrau, la stèle a dû être exécutée et sculptée sur place ; « elle devait, dit-il, surmonter quelque sarcophage ». A l'appui de son opinion, M. de Sarrau souligne que, dans le même lieu, se trouve un grand sarcophage\footnote{Ce cercueil de pierre a les dimensions suivantes : longueur, 2 m. 10; largeur, 0 m. 74; hauteur, 0 m. 56, profondeur (intérieur), 0m. 44; épaisseur, 0 m. 12.} de marbre blanc veiné de bleu — marbre des Pyrénées — dans lequel fleurissent, actuellement, graminées et plantes diverses. Il fait observer, en outre, qu'au temps d'Ausone, la plupart des serviteurs étaient Grecs.

Quoi qu'il en soit, la stèle du château de Thouars « porte à trois le nombre des inscriptions grecques de Bordeaux ou de sa banlieue »\footnote{Paul \textsc{COURTEAULT}. \textit{Revue historique de Bordeaux}. 1913.}. Elle a été emportée par le propriétaire qui a vendu le domaine à M. d'Ornellas.

\subsection{Curieuses margelles de puits}

Il y a à Talence trois margelles de puits en pierre avec écussons des XVII\textsuperscript{e} et XVIII\textsuperscript{e} siècles.

La première de ces margelles se trouve dans le domaine de Capdaurat, entre le chemin de Pey-Bouquey et celui de la Vieille-Tour. Elle est formée par un renflement circulaire portant un écusson de forme ancienne, avec une croix et la date 1731.

La seconde margelle est à l'extrémité de la vigne de Capdaurat, près du chemin de la Vieille-Tour. De même forme que la précédente, elle est cependant d'une facture plus soignée ; sur son renflement se détache un écusson portant, gravés, le monogramme du Christ avec croix et le millésime 1699.

La troisième margelle orne la cour de la propriété du Castel, qui est en bordure du chemin Roul et appartient à M. Holagray. Elle présente « deux moulures bombées, en forme de doucine droite et renversée, séparées par une bande ornée d'un écusson contourné, de forme moderne »\footnote{Ces margelles ont fait l'objet d'une communication do M. l'abbé Royer (séance du 9 juillet 1920 de la Société Archéologique).}.

\subsection{Monnaies romaines}

En février 1857, on fit, au Noviciat des frères des Ecoles
chrétiennes, à Talence, cours Gambetta, 124\footnote{Le Noviciat est indiqué sur l'\textit{Atlas départemental du Conseil général}, cours Gambetta, à gauche, un peu après le pont du chemin de fer.} une importante trouvaille touchant la science numismatique. En creusant un trou dans une allée du jardin pour en extraire du sable, on découvrit, à 70 centimètres de profondeur, un lot de pièces romaines en bronze, de moyen et petit module, à l'effigie des derniers empereurs romains. Il n'y avait autour aucun vestige de bois pouvant faire présumer que les pièces avaient été placées dans une caisse. On s'était borné sans doute, avant de les enfouir, à les envelopper dans un petit sac qui aura été détruit par les siècles. L'absence complète de monnaies françaises parmi celles qui ont été mises à jour « autoriserait à croire que le dépôt remonte aux invasions barbares »\footnote{\textit{La Guienne}, n\textsuperscript{os} des 15 et 22 février et 1\textsuperscript{er} mars 1857.}.

Toutes les pièces, parfaitement conservées, « étaient réunies et collées ensemble au moyen d'oxide »\footnote{\textit{Ibid}.}.

La plus vieille est à l'effigie de Postumus, qui fut proclamé empereur dans les Gaules au début de l'an 261. En 267, après avoir vaincu le tyran Lélien, près de Mayence, Postumus fut massacré par ses soldats pour n'avoir pas voulu leur livrer le pillage de cette ville\footnote{\textit{L'art de vérifier les dates}.}.

La moins ancienne des pièces de monnaie rappelle Constantin II dit « le Jeune » né à Arles le 1\textsuperscript{er} mars 316, et proclamé auguste et empereur l'an 337, après la mort de son père, le grand Constantin.

Constantin II périt, l'an 340, dans une embuscade que lui dressèrent les généraux de son frère Constant, près d'Aquilée\footnote{\textit{L'art de vérifier les dates}.}.

\subsection{L'aqueduc romain}

En 1826, l'attention de Billaudel, ingénieur des ponts et chaussées, fut attirée par la découverte d'une portion d'aqueduc, au Pont d'Ars, dans une sablière appartenant à un pharmacien nommé Cazenave.

Une commission fut chargée d'aller reconnaître cette portion d'aqueduc. Elle comprenait MM. Blanc-Dutrouilh, secrétaire général; Billaudel, l'ingénieur ; Durand, architecte; Lartigue, pharmacien, et Jouannet, conservateur des antiquités du département.

Cette commission fit un rapport\footnote{Ce rapport a été publié dans les \textit{Actes de l'Académie des sciences, bettes-lettres et arts de Bordeaux} (années 1825-1887).} très intéressant.

L'aqueduc, rencontré à six pieds de profondeur, était de forme rectangulaire, construit en béton et présentait tous les caractères des constructions romaines de ce genre. On en découvrit d'autres fragments qui permirent d'établir que cet ouvrage était tantôt rampant, tantôt souterrain, tantôt porté au-dessus du sol, suivant les ondulations du terrain.

Quand l'aqueduc courait au-dessus du sol, il était porté ou sur un mur ou sur des arcades. C'est sur des arcades qu'il franchissait le ruisseau dé Talence ou des Malerettes. La commission estima que « les noms de Pont d'Ars et de Courneau d'Ars que conservent, sur la rive droite des Malerettes, deux endroits éloignés l'un de l'autre de cinq cents toises, semblent rappeler le souvenir de ces arcades. »

MM. Blanc-Dutrouilh, Billaudel et leurs collègues s'occupèrent de rechercher la direction de l'aqueduc. De la sablière du pharmacien Cazenave, par conséquent du Pont d'Ars, il allait en droite ligne sur Bordeaux.

Cet aqueduc n'était autre que celui dont il a été question au premier chapitre et qui conduisait l'Eau Blanche au cœur de Burdigala. Élie Vinet, on s'en souvient, en avait, le premier, reconnu les vestiges en 1552.

L'ouvrage avait été édifié sous Tibère. On peut en voir des morceaux au musée des Antiques, rue Mably, entre autres, une pierre ayant 43 centimètres de largeur à l'intérieur et 67 centimètres de hauteur.

Des restes peu connus du même aqueduc subsistent près du village de Sarcignan\footnote{J.-A. \textsc{BRUTAILS}. \textit{Guide illustré dans Bordeaux et les environs}, 1906.}.

De sa promenade au Moulin d'Ars en 1552, Élie Vinet avait rapporté un fragment de l'aqueduc consistant en « un tuïau de terre cuite d'environ demi pié de diamètre, rompu par les deux bouts, et aiant encore de longueur bien près de pié et demi »\footnote{\textit{L'antiquité de Bourdeaus}.}. Il avait fait présent de ce tuyau « à maistre Joseph de La Chassagne, conseiller du roi en la court du Parlement de Bourdeaus, homme fort studieux et grand admirateur d'antiquité »\footnote{\textit{Ibid}.}.

On lit dans un article intitulé « Talence » et publié dans le \textit{Musée d'Aquitaine} (année 1823) :

« Nous avons reconnu dans le sud de Talence, du côté du château de Salle, quelques pieds d'un aqueduc rampant, que nous nous proposons de suivre pour reconnaître sa direction ».

L'auteur de cet article a mal situé le château de Salles ou plutôt le \textit{Bien de Salles}\footnote{Plan cadastral de Talence.}, qui est en quelque sorte au centre de Talence, exactement à gauche du cours Gambetta, un peu avant le quartier de Banquey.

D'autre part, on ne sait s'il donna suite à son projet de s'assurer de la direction de l'aqueduc rampant. En tout cas, la commission nommée en 1826 paraît avoir mis les choses au point en ce qui concerne l'aqueduc de Tibère qui passait, au Pont d'Ars, sur le territoire de Talence.

\subsection{Une inscription latine}

Dans le sol où fut ouverte la rue Duluc — laquelle séparait autrefois Bordeaux de Talence — on a trouvé « une pierre funéraire, celle d'un Espagnol, avec inscription latine ». Cette découverte, suivant le comte de Sarrau, indique que peut-être il y a eu là un domaine gallo-romain.

\subsection{Peintures décoratives au château du Prince-Noir}

M. Clavé, étant propriétaire du château du Prince Noir, fit faire, un peu avant l'année 1888, des réparations intérieures à la partie du château reconstruite vers le XVII\textsuperscript{e} siècle. En effectuant le travail, les ouvriers trouvèrent, sous du papier de tenture, une décoration qui était destinée à un petit oratoire du premier étage.

M. Augier a cru pouvoir attribuer ces restes de peintures
décoratives à la fin du XVII\textsuperscript{e} siècle\footnote{Séance de la Société archéologique de Bordeaux du 13 janvier 1888.}. Elle consistent dans l'ornementation d'un contre-rétable constitué par deux pilastres cannelés supportant une corniche; la peinture a suppléé à la sculpture dans toutes les moulures. Sur les côtés formant la saillie des pilastres, il y a de jolies arabesques avec des têtes d'anges, se détachant sur un fond noir. Des filets dorés venaient égayer les tons rouges, bruns, verts et bleus dont sont couverts les fonds et les surfaces planes. Le plafond est un lambris, sur lequel le peintre décorateur a tracé des divisions pour des caissons dans lesquels un ornement varié a été peint de différentes couleurs sur un fond. Au centre, le Saint Esprit dans des nuages avec des têtes d'anges.

Des réparations antérieures avaient fait disparaître le reste des décorations sur le mur.

Ces peintures, quoique n'offrant pas beaucoup d'intérêt, devaient être néanmoins, estimait-on, conservées et restaurées.

\subsection{Une médaille aragonaise}

Quand on creusa les fondations de la maison 199, rue de Saint-Genès, on trouva beaucoup d'ossements, des crânes dolichocéphales, des pièces de monnaie d'Henri, roi d'Angleterre ; de Philippe - Auguste, des doubles - tournois Louis XIII, une médaille aragonaise\footnote{Notes de M. le comte de Sarrau.}. La maison en question s'élève sur une partie de l'emplacement de l'ancien cimetière de Saint-Genès, où devaient être inhumés, non seulement les habitants de la paroisse, mais encore les pèlerins décédés en arrivant aux portes de Bordeaux.

La médaille aragonaise indique bien le retour de voyageurs venant d'Espagne.
        
\end{document}
        
        
        