%%%% Débute page 166 dans le fichier PDF%%%%

\documentclass[a4paper,11pt]{book}
\usepackage[T1]{fontenc}
\usepackage[utf8]{inputenc}
\usepackage{lmodern}
\usepackage[french]{babel}
\usepackage{todonotes}
\usepackage{booktabs}

\newcommand{\asterism}{\bigskip\par\noindent\parbox{\linewidth}{\centering\large{*}\\[-4pt]{*}\hskip 0.75em{*}}\bigskip\par}%

\begin{document}
\title{Histoire de Talence}
\author{Maurice Ferrus}
\frontmatter
\maketitle

\mainmatter{}

\section{En l'honneur des morts glorieux}
\subsection{Le cimetière américain}

Un cimetière américain a été créé sur le territoire de Talence. On y a inhumé les restes des soldats du général Pershing morts, dans les formations sanitaires, des suites de blessures reçues à l'ennemi.

Nous avons assisté, le 30 mai 1920, à la manifestation patriotique organisée, dans ce cimetière, à l'occasion du Mémorial Day. À côté des autorités françaises étaient réunis MM. Théodore Jaeckel, consul des États-Unis; Atlee, consul d'Angleterre; le chapelain Hewetson; Mathews, du service des cimetières américains; Smith Port, secrétaire de l'Y.~M.~C.~A. La colonie américaine de Bordeaux avait fait déposer dans la nécropole une gerbe de roses avec cette inscription en anglais : « À la mémoire des soldats américains tombés pour la patrie. »

Chaque année, le 30 mai, une pieuse cérémonie se déroule, avec le même, rite, dans l'\textit{American cimetery}.

\subsection{Le cimetière communal}

La ville de Talence a été fortement éprouvée par la guerre 1914-1918. Combien de ses enfants sont morts au champ d'honneur! L'âme de ces braves plane sur la commune; elle y entretient la notion du devoir, l'amour de la liberté, le culte du patriotisme.

Talence a élevé au cimetière un monument à ses fils tombés pour la plus noble des causes.

Le monument, d'une sobre architecture, présente une pyramide sur piédestal, entouré de gradins. Il porte cette inscription : « Aux enfants de Talence morts pour la patrie »

La face antérieure de la pyramide montre une palme; sur la face opposée, il y a un glaive, la lame tournée vers le sol. Cette arme, dans sa position, doit symboliser la paix. Un flambeau renversé orne chacune des deux autres faces de la pyramide. C'est sans doute l'image de la vie éteinte en pleine ardeur ?

Les noms des héros garnissent le piédestal, formant comme un palmarès sublime. Quatre urnes, que supportent des socles, encadrent le mausolée commémoratif.

L'inauguration de ce monument a eu lieu le dimanche 1\textsuperscript{er} juillet 1923, au cours d'une émouvante solennité.

Le cortège officiel se forma devant la mairie et se rendit à la nécropole par le chemin de Suzon. Marchaient en tête la batterie La Talençaise et la Lyre talençaise, jouant des marches funèbres. Venaient ensuite les sociétés locales avec leurs drapeaux ou bannières : Les Enfants de Talence (gymnastes), le bataillon féminin des Libellules, Les Camarades de combat, La Sauvegarde du Poilu, Les Vétérans de 1870-1871, l'orphéon l'Avenir.

Les élèves du lycée de Talence et ceux des écoles communales fermaient le cortège avec M. Jean Iriquin, maire de Talence, qu'accompagnaient M. Boucanus, adjoint, et plusieurs autres membres de la municipalité.

Au cimetière étaient réunis, au milieu d'une affluence recueillie, les représentants du gouvernement, ceux de l'armée, les délégués des corps élus, les consuls des puissances amies et alliées. Les morts glorieux de Talence reçurent, ce jour-là, l'hommage suprême de la France et des nations qui luttèrent à nos côtés pour le triomphe du droit, de la justice et de la civilisation.

M. Iriquin salua les braves disparus, s'inclina devant les familles endeuillées, puis, s'adressant à la jeunesse, il dit :

« N'oubliez pas que notre nation était condamnée à disparaître à tout jamais, si les héros dont le nom est maintenant gravé sur ce monument n'avaient constitué de leurs poitrines une barrière à l'envahisseur. »

Le maire, d'une voix émue, donna lecture de la liste des fils de Talence tués à l'ennemi.

À l'appel de chaque nom, les élèves des écoles répondaient en choeur : « Mort pour la France ! »

Le secrétaire général de la préfecture évoqua les premières heures du conflit mondial :

« La France, souligna-t-il, a manifesté son sincère désir de paix quand, en août 1914, elle faisait reculer ses troupes de 10 kilomètres et laissait ses frontières ouvertes à l'ennemi ; mais lorsque la guerre nous fut imposée, ce fut avec un mâle courage que tous les Français volèrent au salut de la patrie\footnote{\textit{La Petite Gironde}, n° du 2 juillet 1923.}.»

Et le représentant du gouvernement conclut en faisant appel à l'union indispensable de tous les citoyens pour permettre à la France de se relever, de panser ses blessures et de poursuivre ses glorieuses destinées.

\subsection{Liste des enfants de Talence morts pour la patrie et dont les noms figurent sur le monument commémoratif}

\begin{tabular}{@{}|l|l|}
\toprule
NOMS ET PRÉNOMS & RÉGIMENT\\
\midrule
AGO Louis-Manuel\dotfill & 10\textsuperscript{e} huss. \\
AGO Gabriel\dotfill  & 7\textsuperscript{•} colon.\\
ADAM Albert-Victor\dotfill  & 144\textsuperscript{•} R. I.\\
ALICOT Gaston\dotfill & 220e R. I. \\
AVEZ André-François\dotfill & 160\textsuperscript{e} R. I. \\
AMEAU Jean\dotfill & 142\textsuperscript{e} R. I. \\
ANDIRAN Henri\dotfill & 19\textsuperscript{e} corps. \\
AYMARD Joseph\dotfill & 53\textsuperscript{e} R. I. \\
ARMELLA Henri-André\dotfill & 155\textsuperscript{e} R. I. \\
AUDIGNON Pascal\dotfill & 1\textsuperscript{er} R.A.C. \\
AUGER Louis-Paul\dotfill & 56\textsuperscript{e} R. I. \\
ARTIGUEBIELLE Emile\dotfill & 35\textsuperscript{e} R. I. \\
AGUERRE Jean\dotfill & 43\textsuperscript{e} R. I. \\
BERNARD Jean-Emman.\dotfill & 20\textsuperscript{e} R. I. \\
BERNARD Raymond\dotfill & 417\textsuperscript{e} R. I. \\
BOURBON Marcel\dotfill & 108\textsuperscript{e} R. I. \\
BOULNOIS Auguste\dotfill & 7e colon. \\
BARRÈRE Félicien\dotfill & 58e R.A.C. \\
BARBÉ Jean-Marie\dotfill & 16e ch.àp. \\
BARRÈRE Jean-Marie\dotfill & 16e ch.àp. \\
BARANDON Pierre\dotfill & 7e R. I. \\
BARRÉRE Maurice\dotfill & 212\textsuperscript{e} R. I. \\
BARRIÈRE Jules\dotfill & 119\textsuperscript{e} R. I. \\
BAUER Jean-Georges\dotfill & 127\textsuperscript{e} R. I. \\
BEAUMONT Simon\dotfill & 299\textsuperscript{e} R. I. \\
BÉCÈDE Jacques-Urbain\dotfill & 140e terr. \\
BELHADE Gaston-Louis\dotfill & 18e sect \\
BELLEGARDE Frédéric\dotfill & 272e R. \\
\midrule
\multicolumn{2}{|l|}{L'ouvrage de Mauris Ferrus contient qui s'étend des pages 223 à 227} \\
\bottomrule
\end{tabular}
\end{document}