\chapter*{Avant-propos}

La commune de Talence est une des plus peuplées de la Gironde. En exceptant Bordeaux elle n'a plus devant elle que Libourne, Bègles et Caudéran. Encore l'écart n'est-il pas bien grand entre le nombre des habitants de la plus importante de ces trois villes — Libourne — et celui de Talence. Cette dernière ville compte, en effet, près de 17 000 âmes, d'après les derniers recensements, et Libourne en groupe 18.000 tout au plus.

Il y a à Talence des usines, des fabriques, des ateliers où l'on travaille activement. À la sortie de ces divers établissements — surtout le soir — la cité s'anime comme par le fait d'une baguette magique. Un gros mouvement règne dans les principales artères, où se déverse le flot des Talençais rentrant de Bordeaux, où ils ont leurs occupations.

Trois lignes de tramways desservent la localité. La construction du chemin de fer de ceinture a amené la création de la station « Talence-La Médoquine », qui était devenue nécessaire vu l'augmentation croissante du trafic.

Talence n'est pas seulement un centre industriel et commercial. C'est aussi un foyer scientifique. Le petit lycée est une annexe du lycée de Bordeaux. Le Jardin botanique dépendant de la Faculté de médecine et de pharmacie renferme des collections de tout premier ordre.

Il y a, rue Henri-Brisson, une école primaire supérieure intercommunale, fréquentée par de nombreux élèves. Une école primaire supérieure de filles est ouverte.

Talence possède un observatoire, une école d'infirmières à Bagatelle, une pouponnière à Cholet, un hôpital militaire.

C'est, d'autre part, une ville riche en souvenirs historiques. On y peut admirer de beaux domaines, survivances de siècles lointains ou constructions modernes.

Et que de coins délicieux à Talence ! Un bol d'air pur est une chose exquise. Il y en a à Talence plus que partout ailleurs. Les Bordelais ne l'ignorent pas. Aussi affluent-ils le dimanche vers cette partie si riante de la banlieue, où ils peuvent respirer à pleins poumons, où ils flânent tout à leur aise et sont ainsi plus dispos pour reprendre, le lendemain, la tâche quotidienne. Il faut les voir monter dans les tramways, à la barrière de Saint-Genès, en particulier. C'est la ruée sur les voitures.

L'hippodrome est connu de tous les amateurs de courses de chevaux. Les réunions qu'on y donne sont très suivies. Les notabilités du turf n'en manquent pas une ! Les élégantes viennent y montrer — pour les lancer — les modes nouvelles.

« Que de choses dans un menuet ! » disait le fameux danseur Vestris. Que d'attraits dans Talence ! dirons-nous de notre côté. Sans compter qu'on y récolte des vins de graves renommés dans le monde entier.

Chose inimaginable, il n'existait aucune monographie de cette ville. Alors que des cités moins importantes et surtout moins intéressantes du département avaient leur histoire, Talence attendait encore la sienne. Il y avait là une lacune vraiment regrettable. Nous avons essayé de la combler.

Nous présentons donc aujourd'hui au public une Histoire de Talence. Nous l'avons faite aussi complète que possible. Certes, nous n'avons pas écrit là des pages définitives ; loin de nous semblable prétention. Nous avons voulu simplement faire oeuvre utile en brossant un tableau fidèle de la localité qui nous est chère, en dévoilant les beautés qu'elle recèle, en rappelant les événements dont elle fut le théâtre, en faisant, enfin, revivre l'éclat et la gloire de son passé.

Combien de Talençais savent qu'ils ont eu, chez eux, sous les princes anglais, une bastide ?

Combien savent qu'ils possèdent une oeuvre inestimable de Louis, le génial architecte du Grand-Théâtre ?

Combien savent que la route de Bayonne était un des chemins conduisant au célèbre pèlerinage de Compostelle et que Charlemagne l'emprunta pour revenir d'Espagne ?

Combien savent que Charles IX, accompagné de toute la cour de France, fut, pendant une semaine, l'hôte du château de Thouars ?

Combien savent que la villa Raba fut appelée le « Chantilly bordelais », que le Parlement s'y rendit en corps, qu'un banquet y fut offert, en pleine Terreur, aux membres du Tribunal révolutionnaire ?

Combien savent que Napoléon Ier tint à visiter Talence, lors de son passage à Bordeaux en 1808 ?

Certains pensent que les enfants ont beaucoup de peine à apprendre l'histoire, qu'elle est, en outre, susceptible de fausser leur esprit, et que, par suite, on perd du temps à la leur enseigner. Lors d'un Congrès du Syndicat national des instituteurs, un orateur ne proposait-il pas d'abandonner Clio ? Sans doute voulait-il se singulariser par l'inattendu de sa proposition comme le grotesque bouffon qui plantait naguère le « drapeau dans le fumier ».

Nous estimons, au contraire, qu'il faut développer l'enseignement de l'histoire, et en cela, du reste, nous sommes parfaitement d'accord avec l'immense majorité des
éducateurs de la jeunesse.

Oui, il faut intéresser les enfants — voire les hommes — aux paysages qui les entourent. On doit leur conter tout ce qui a trait aux maisons qu'ils habitent, au clocher natal, au manoir voisin, au vieux cimetière, aux pans de murs couverts de mousse ou qu'enguirlande le lierre capricieux.

L'Histoire est la science du souvenir. Elle est, pour qui sait ou qui veut la comprendre, une grande leçon. Elle montre, étape par étape, l'évolution humaine, la physionomie des différents pays. Elle fait connaître la vie, les moeurs, le caractère des individus. Elle note la distinction entre les races, entre les peuples. Elle souligne les vertus, les défauts de chacun d'eux. Elle exalte les nobles sentiments. Elle crée le patriotisme. Et c'est ainsi qu'en août 1914, les fils de France, en un élan sublime, inoubliable, se dressèrent pour barrer la route à l'envahisseur.

Ne l'oublions pas. Le culte de la petite patrie donne l'amour de la grande. Aimons donc l'Histoire, quelque opinion qu'aient d'elle plusieurs hurluberlus. Maintenons nos traditions. Respectons nos héros. Vénérons nos gloires nationales. L'âme d'hier formera l'âme de demain !

M. F.